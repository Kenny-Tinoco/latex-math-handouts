\section{Desarrollo}
Probablemente conozca la ecuación $x^2 + y^2 = z^2$ por el teorema de Pitágoras, el cual describe la relación entre los lados $x,y,z$ con respecto al ángulo recto.
Si los lados son todos enteros positivos, estos forman una \textbf{tripla pitagórica} o \textbf{terna pitagórica}.
Por ejemplo, $(x,y,z) = (3,4,5)$ es una terna pitagórica, asi como $(5,12,13)$ es también una terna pitagórica.
También, cuando un triángulo rectángulo tiene longitudes enteras, lo llamaremos un triángulo pitagórico.

\begin{lemma}
    Sean $x,y,z$ una terna pitagórica y $d = \mcd{x}{y,z}$.
    Entonces,
    \begin{enumerate}
        \item [i)] Se tiene $d = \mcd x y = \mcd y z = \mcd z x$.
        \item [ii)] Si escribimos $x = d x_1$, $y = d y_1$ y $z = d z_1$, entonces $x_1, y_1, z_1$ es también una terna pitagórica.
    \end{enumerate}
\end{lemma}

\begin{definition.box}{Terna pitagórica primitiva}{}
    Se dice que una terna pitagórica es primitiva si $\mcd{x}{y,z} = 1$ o equivalentemente cualquiera de los $\mcd x y$, $\mcd y z$ o $\mcd z x$ es 1.
\end{definition.box}

\begin{lemma}
    Sea $x,y,z$ una terna pitagórica primitiva, entonces, entre $x,y,z$ hay exactamente un número par, que bien puede ser $x$ o $y$.
\end{lemma}

\begin{theorem.box}{}{}
    Sea $x,y,z$ una terna pitagórica primitiva de enteros positivos, con $y$ par.
    Entonces, existen enteros coprimos no negativos $m,n$ con $m > n$ talque
    \begin{align*}
        x = m^2 - n^2,\quad y = 2mn,\quad z = m^2 + n^2.
    \end{align*}
\end{theorem.box}

La ecuación de Fermat o el último teorema de Fermat fue propuesta por el matemático francés Pierre de Fermat.
En matemáticas, es un teorema famoso por su dificultad y el proceso de demostración de este teorema ha llevado a muchos descubrimientos importantes tanto en álgebra como en análisis.

Mientras estudiaba la obra del antiguo matemático griego Diofánto,
Fermat escribió en el margen de su copia de un libro de Diofánto
\begin{center}
    ``La ecuación a $a^n + b^n = c^n$ no tiene raíces enteras positivas para $n \geq 3$.
    He descubierto una prueba verdaderamente hermosa pero este margen es demasiado pequeño para contenerla.''
\end{center}
Durante más de 350 años, muchos matemáticos intentaron demostrar la afirmación de Fermat o refutarla encontrando algún contraejemplo.
En junio de 1993, Andrew Wiles, un matemático inglés de la Universidad de Princeton, afirmó que había demostrado el teorema.
El 25 de octubre del siguiente año, Wiles envió una prueba revisada a tres colegas, después de que sus colegas la juzgaran completa, Wiles publicó su prueba.
Los siguientes teoremas son ecuaciones de la forma del teorema de Fermat.

\begin{theorem.box}{}{}
    La ecuación $x^4 + y^4 = z^2$ no tiene soluciones enteras positivas.
\end{theorem.box}

\begin{theorem.box}{}{}
    La ecuación $x^4 - y^4 = z^2$ no tiene soluciones enteras positivas.
\end{theorem.box}

\begin{theorem.box}{}{}
    La ecuación $x^4 + y^2 = z^2$ no tiene soluciones enteras positivas.
\end{theorem.box}

\begin{theorem.box}{}{}
    La ecuación $x^4 - 4y^2 = z^2$ no tiene soluciones enteras positivas.
\end{theorem.box}

\begin{theorem.box}{}{}
    La ecuación $x^4 + 4y^2 = z^2$ no tiene soluciones enteras positivas.
\end{theorem.box}



\subsection{Ejercicios y problemas}
Ejercicios y problemas para el autoestudio.

\begin{exercise}
    Resolver las siguientes ecuaciones sobre los números naturales
    \begin{multicols}{2}
        \begin{enumerate}
            \item $15y^4 - z^2 - 2y^2 z + 1 = 0$
            \item $x^4 + y^4 - x^2 y^2 - z^2 - 2xyz = 0$
            \item $x^4 y^4 - 2x^2 y^2 + 1 = x^4 + y^4$
            \item $x^8 + y^8 - 3x^{4}y^4 = 625$
            \item $y^4 = 168x^4 + 338x^3 y + y^2$
        \end{enumerate}
    \end{multicols}
\end{exercise}

\begin{exercise}
    Resolver las siguientes ecuaciones sobre los números naturales
    \begin{multicols}{2}
        \begin{enumerate}
            \item $2x^4 + 2x^2 + 1 = z^4$
            \item $-14x^2 + y^4 = 49$
            \item $3x^4 + y^4 - 102x^2 + 2061 = 0$
            \item $3x^4 - 4x^2 + 1 = y^4$
            \item $x^4 + 4x^3 + 297x^2 + 4x + 1 = y^4$
        \end{enumerate}
    \end{multicols}
\end{exercise}

\begin{problem}
    Demostrar que el radio de la circunferencia inscrita en un triángulo pitagórico de lados enteros es siempre un número natural.
\end{problem}

\begin{problem}
    Demostrar que el área de un triángulo pitagórico no puede ser un cuadrado perfecto.
\end{problem}

\begin{problem}
    Hallar todas las soluciones en enteros positivos para el sistema de ecuaciones
    \[
        \begin{cases}
            a^2 + b^2 = c^2\\
            b^2 + c^2 = d^2.
        \end{cases}
    \]
\end{problem}

\begin{problem}
    Probar que la ecuación $p = x^2 + y^2$ para un primo $p$ de la forma $p = 4k + 1$ con $k$ entero, tiene una y
    solamente una solución en los enteros sin tomar en cuenta las permutaciones de $x,y$.
    (Pista: Usar ternas pitagóricas y luego descenso infinito)
\end{problem}