\section{Desarrollo}

\begin{definition.tcb}{Ecuaciones Diofánticas}{}
    Se llama \textbf{ecuación diofántica} o \textbf{ecuación diofantina} a cualquier ecuación polinomial con coeficientes enteros cuya
    solución se restringe únicamente a aquellos valores enteros que la satisfacen.
\end{definition.tcb}

Es decir, una expresión de la forma
\[
    a_1 x_1 + a_2 x_2 + \cdots + a_n x_n = 0, \quad \text{con} \quad 1 \leq i \leq n,\ a_i \in \Z.
\]
donde la $n$-upla de números enteros $(r_1, r_2, \ldots, r_n)$ hace la igualdad.
Es claro que una ecuación diofántica puede tener una o más $n$-upla que hagan la igualdad.

\begin{definition.tcb}{}{}
    A una $n$-upla, con $n$ entero, que satisface una ecuación diofántica, se le llama \textbf{solución} de la ecuación.
    Una ecuación diofántica con una o más soluciones se llama ecuación \textbf{soluble}, así también, una ecuación diofántica sin
    soluciones se llama ecuación \textbf{insoluble} o \textbf{irresoluble}.
\end{definition.tcb}

Una ecuación diofántica se dice que tiene una familia de soluciones cuando un conjunto de estas, o todas sus soluciones,
puede ser expresada en función de un o más parámetros enteros.
Como por ejemplo, la ecuación $x^2 + 2y^2 = z^2$ vemos tiene soluciones $(-2, 0, 2)$, $(-1, 2, 3)$, $(2, 4, 6)$, $\ldots$,
las cuales podemos expresar como
\[
    \begin{cases}
        x = r^2 - 2\\
        y = 2r\\
        z = r^2 + 2
    \end{cases}
\]
donde $r$ es un número entero sin ninguna restricción, lo cual hace que la ecuación diofántica tenga infinita soluciones de esta forma.


\subsection{Ejercicios y problemas}

Ejercicios y problemas para el autoestudio.

\begin{exercise}
    Hallar la cantidad de parejas de enteros positivos $x_1, x_2$ tales que $x_1 \cdot x_2 = 25 \cdot 15^3$.
\end{exercise}

\begin{exercise}
    Probar que la ecuación $x^2 + 2y^2 = z^2$ tiene como solución a $x = a^2 - 2b^2$, $y = 2ab$ y $z = a^2 + 2b^2$ donde $a, b \in \Z$.
\end{exercise}

\begin{exercise}
    Probar que la ecuación $x^3 = 2y^3$ no tiene soluciones enteras.
\end{exercise}