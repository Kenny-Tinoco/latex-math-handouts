\newpage
\section{Plan de clase}

\subsection{¿Qué?}

Mostrar el uso de los principios de Inducción Matemática y Descenso al infinito de Fermat (DIF).
Dando a conocer en qué consiste el DIF y su utilizad en las ecuaciones diofánticas insolubles.
Dar un repaso del principio de Inducción y mostra cómo este implica otros principios como el DIF y el buen orden, asi
como su equivalencia.

Si $n_0$ es el menor entero positivo para el cual $P(n_0)$ es cierto, entonces $P(n)$ es falso para todo $n<n_0$.

\subsection{¿Cómo?}


\newpage
\subsection{Comentarios}

Preguntas claves: ¿me entendieron?
¿me salté algún tema?
¿se dio tiempo suficiente para pensar los problemas?
¿participaron?
¿problemas muy fáciles o muy difíciles, demasiados o muy pocos?
¿mis explicaciones/ejemplos fueron suficientes o buenos?

\foreach \x in {1,...,24}{
    \myhrule{8.5}
}

