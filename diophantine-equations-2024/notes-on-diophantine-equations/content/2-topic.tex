\section{Introducción a las Ecuaciones Diofánticas}

La resolución de ecuaciones es un tema que se presenta en secundaria desde los primeros niveles.
Paralelamente se van estudiando los diferentes conjuntos numéricos a saber: el conjunto de números Naturales $(\N)$,
el conjunto de números enteros $(\Z)$, el conjunto de números racionales $(\Q)$ y el conjunto de número Reales $(\R)$.

Además, se repasan conceptos elementales de teoría de números como son: máximo común divisor, mínimo común múltiplo, números primos, etc.
En este curso estudiaremos los métodos de resolución de un tipo especial de ecuaciones llamadas \textbf{Ecuaciones Diofánticas.}

\subsection{Definiciones}

\begin{definition.box}{Ecuación Diofántica}{}
    Se llama {ecuación diofántica} o {ecuación diofantina} a cualquier ecuación polinomial con coeficientes enteros cuya
    solución se restringe únicamente a aquellos valores enteros que la satisfacen.
\end{definition.box}

Es decir, una expresión de la forma
\[
    a_1 x_1 + a_2 x_2 + \cdots + a_n x_n = b, \quad \text{con} \quad 1 \leq i \leq n,\ a_i, b \in \Z.
\]
Donde la $n$-upla de números enteros $(r_1, r_2, \ldots, r_n)$ hace la igualdad.
Es claro que una ecuación diofántica puede tener una o más $n$-upla que hagan la igualdad.

\begin{definition.box}{}{}
    A una $n$-upla, con $n$ entero, que satisface una ecuación diofántica, se le llama {solución} de la ecuación.
    Una ecuación diofántica con una o más soluciones se llama ecuación {soluble}, así también, una ecuación diofántica sin
    soluciones se llama ecuación {insoluble} o {irresoluble}.
\end{definition.box}

Una ecuación diofántica se dice que tiene una familia de soluciones cuando un conjunto de estas, o todas, puede ser expresada en función de uno o más parámetros enteros.
Como por ejemplo la ecuación $x^2 + 2y^2 = z^2$, vemos que tiene soluciones $(-2, 0, 2)$, $(-1, 2, 3)$, $(2, 4, 6)$, $\ldots$, las cuales podemos expresar como
\[
    \begin{cases}
        x = r^2 - 2\\
        y = 2r\\
        z = r^2 + 2
    \end{cases}
\]
donde $r$ es un número entero sin ninguna restricción, por lo que la ecuación diofántica tiene infinitas soluciones de esta forma.

Ahora, nos centraremos en el conjunto de métodos elementales para la resolución de Ecuaciones Diofánticas, a lo largo del curso
haremos uso de estos métodos, tales como factorización, identidades algebraicas, desigualdades, parametrización y congruencias.