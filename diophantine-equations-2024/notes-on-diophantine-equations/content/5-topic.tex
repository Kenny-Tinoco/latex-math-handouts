\section{Método de parametrización}

En muchas situaciones, las soluciones enteras a una ecuación diofántica pueden ser expresadas de forma paramétrica, dónde dichos parámetros son variables enteras.

El conjunto de soluciones de algunas ecuaciones diofánticas podría tener múltiples representaciones paramétricas.
Para la mayoría de ecuaciones diofánticas, no es posible encontrar todas las soluciones explícitamente.
En muchos casos, el método paramétrico proporciona una prueba de la existencia de infinitas soluciones.

\begin{example}
    Hallar todas las soluciones enteras $(m,n)$ que satisfacen la ecuación
    \[
        12m - 5n = 97 - mn.
    \]
\end{example}
\begin{solution}
    Transformando la ecuación convenientemente
    \begin{multline*}
        12m + mn = 97 + 5n \iff
        m(12 + n) = 97 + 5n\\ \iff
        m = \frac{97 + 5n}{12 + n} \iff
        m = 5 + \frac{37}{n + 12}.
    \end{multline*}
    Consideremos la variable $t = \frac{37}{n + 12}$ con lo cual $m = t + 5$, como $m$ es entero necesariamente $t$
    también lo es.
    Es fácil ver $n = \frac{37}{t} - 12$, ya que $n$ es entero, entonces $t$ debe ser divisor de 37.
    Con esta información, es claro que los posibles valores son $t \in \{\pm 1, \pm 37 \}$.
    Finalmente, las soluciones de la ecuación están dadas por $(m,n) \in \{(4, -49), (6, 25), (-32, -13), (42, -11)\}$.
\end{solution}

Como vimos en el ejemplo anterior, la idea es expresar las variables $m$ y $n$ en términos de un parámetro $t$,
reduciendo el problema a solo encontrar los posibles valores de $t$, y puesto de que estamos trabajando en enteros
y que hay ciertas condiciones con respecto a esta variable, la pudimos acotar fácilmente.

\subsubsection{Ejercicios y problemas}

Ejercicios y problemas para el autoestudio.

\begin{exercise}
    Encuentra las soluciones de la siguiente ecuación diofántica
    \[
        2(x + y) = xy + 9.
    \]
\end{exercise}

\begin{exercise}
    Demostrar que la ecuación $x^2 + y^2 - z^2 - x + 3y - z - 4 = 0$ posee infinitas soluciones en los números enteros.
\end{exercise}

\begin{exercise}
    Determinar los números enteros $x$ que verifican que $x^4 + 2$ es múltiplo de $x + 2$.
\end{exercise}

\begin{exercise}
    Dado tres números enteros positivos $x,y,z$ hallar el valor de su producto sabiendo que cumplen
    \[
        x + 2y = z \quad \text{y} \quad x^2 - 4y^2 + z^2 = 310.
    \]
\end{exercise}

\begin{exercise}
    Encontrar todas las soluciones enteras $(x,y)$ de la ecuación
    \[
        p(x + y) = xy,
    \]
    donde $p$ es un número primo.
\end{exercise}

\begin{exercise}
    Hallar todas las triplas $(x,y,z)$ de enteros positivos tales que
    \[
        \frac{1}{x} + \frac{1}{y} = \frac{1}{z}.
    \]
\end{exercise}

\begin{exercise}
    Probar que existen infinitas triplas $(x,y,z)$ de enteros tales que
    \[
        x^3 + y^3 + z^3 = x^2 + y^2 + z^2.
    \]
\end{exercise}

\begin{problem}
    Probar que si $a,b,c$ son enteros positivos tales que $\dfrac{1}{a} + \dfrac{1}{b} = \dfrac{1}{c}$, entonces
    \begin{itemize}
        \item $a + b$ es un cuadrado perfecto.
        \item $a^2 + b^2 + c^2$ es un cuadrado perfecto.
    \end{itemize}
\end{problem}