\section{Ecuaciones de Pell}

Hasta el momento solo hemos abordado ecuaciones cuadráticas, o grado par, por medio de las ternas pitagóricas y
las ecuaciones de Fermat.
Sin embargo, existe tipos de ecuaciones diofánticas cuadráticas muy especiales, conocidas como las ecuaciones de Pell.

\begin{definition.box}{Ecuación de Pell}{}
    Una ecuación diofántica de la forma
    \[
        x^2 - dy^2 = 1, \ \text{con} \ x,y \in \Z,
    \]
    donde $d \in \N$ no es cuadrado perfecto, es conocida como la ecuación de Pell.
\end{definition.box}

¿Por qué decimos que $d$ no es un cuadrado perfecto?
Porque si esto fuera así, se formaría una diferencia de cuadrados, lo cual puede resolverse fácilmente asignado divisores a cada factor.
De esta definición obtenemos.

\begin{definition.box}{Ecuación de tipo Pell}{}
    Una ecuación de la forma $x^2 - dy^2 = a$ para un entero $a$ usualmente la llamaremos como ecuación del Pell.
\end{definition.box}

De manera general, cualquier ecuación diofántica de dos variables puede reducirse a una ecuación del tipo Pell.
Por lo cual esta ecuación diofántica resulta muy útil, con esto dicho la pregunta que surge es ¿cómo se resuelve una ecuación del tipo Pell?
Resulta que estas ecuaciones tiene un tipo de solución muy sencillas de comprender, ya que a partir de una solución concreta pueden generarse una infinidad de soluciones.
Similar a lo que ocurre con las soluciones de las ecuaciones diofánticas lineales.
No obstante, la demostración de esto requiere conocer otros resultados en teoría de números, este escrito no abordará estos aspectos, sin embargo, se invita al lector investigarlo por su cuenta.

\begin{definition.box}{}{}
    El conjugado del número $z = x + y\sqrt{d}$ es definido por $\overline{z} = x - y \sqrt{d}$, y su módulo está dado por $N(z) = z \overline{z} = x^2 - dy^2 \in \Z$.
\end{definition.box}

\begin{theorem.box}{}{}
    Para la ecuación de Pell $x^2 - dy^2 = 1$ si $(x_1, y_1)$ es la solución más pequeña, y se tiene una solución $(x_k, y_k)$, entonces
    \begin{align*}
        x_k + d y_k = (x_1 + dy_1)^r, \quad \text{para algún}\ r\in \positiveSet{\Z}.
    \end{align*}
\end{theorem.box}

\begin{theorem.box}{}{}
    La ecuación de Pell $x^2 - dy^2 = 1$ tiene infinitas soluciones enteras no negativas y si $(x_1, y_1)$ es la solución más pequeña,
    la solución general está dada por
    \begin{align*}
        x_n = \dfrac{(x_1 + y_1 d)^n + (x_1 - y_1 d)^n}{2}, \quad
        y_n = \dfrac{(x_1 + y_1 d)^n - (x_1 - y_1 d)^n}{2\sqrt{d}}
    \end{align*}
    con $n$ un natural.
\end{theorem.box}

Para la ecuación $x^2 - dy^2 = -1$, la situación es similar.
Si $(x_1, y_1)$ es la solución minima, entonces la solución $(x_n, y_n)$ tiene la misma forma del teorema anterior para $n$ impar.

¿Qué podemos decir del conjunto de soluciones para la ecuación $x^2 - dy^2 = a$ con $a \neq 1$?
A diferencia de la ecuación de Pell, la ecuación $x^2 - dy^2 = a$ no necesariamente tiene soluciones.
Sin embargo, si esta tiene una solución minima, entonces tiene una infinidad de soluciones.

\begin{theorem.box}{}{}
    La ecuación de tipo Pell $x^2 - dy^2 = a$ si tiene a $(x_1, y_1)$ como la solución más pequeña,
    entonces tiene una cantidad infinita de soluciones.
\end{theorem.box}

En las siguientes clases abordaremos un método que nos permite encontrar las soluciones de una ecuación de Pell.
Esta herramienta es conocida como fracciones continuas, se invita al lector investigar este tema.

\subsection{Ejercicios y problemas}
Ejercicios y problemas para el autoestudio.

\begin{exercise}
    Verificar que si $(x_1, y_1)$ es la solución mínima de la ecuación de Pell $x^2 - dy^2 = 1$, entonces

    \begin{align*}
        x_n = \dfrac{(x_1 + y_1 d)^n + (x_1 - y_1 d)^n}{2}, \quad
        y_n = \dfrac{(x_1 + y_1 d)^n - (x_1 - y_1 d)^n}{2\sqrt{d}}
    \end{align*}
\end{exercise}

\begin{exercise}
    Hallar todas las soluciones enteras positivas de la ecuación $x^2 - 2y^2 = 1$.
\end{exercise}

\begin{exercise}
    Demostrar que la ecuación $x^2 - dy^2 = -1$ no tienes soluciones cuando $d \modulo{3}{4}$.
\end{exercise}

\begin{problem}
    Resolver en enteros la ecuación $x^2 + y^2 - 1 = 4xy$.
\end{problem}

\begin{problem}
    Demostrar que la ecuación $x^2 - dy^2 = -1$ no tiene solución si $d$ es divisible por un primo de la forma $4k + 3$.
\end{problem}

\begin{problem}
    Hallar todas las soluciones $x,y$ para la ecuación $x^2 - 3y^2 = 6$.
\end{problem}

\begin{problem}
    Hallar todas las soluciones $x,y$ para la ecuación $x^2 - 7y^2 = 2$.
\end{problem}

\begin{problem}
    Hallar todas las soluciones $x,y$ para la ecuación $x^2 - 7y^2 = 1$.
\end{problem}

\begin{problem}
    Demostrar que si la diferencia de dos cubos consecutivos es $n^{2}$ con $n \in \N$, entonces $2n - 1$ es también un cuadrado.
\end{problem}