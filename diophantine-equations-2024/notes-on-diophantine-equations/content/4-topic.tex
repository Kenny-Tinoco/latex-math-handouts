\section{Métodos de desigualdades}

Una de las estrategias de resolución es utilizar las desigualdades.
La idea principal es reducir la cantidad de casos mediante el uso de las inecuaciones.

Antes de iniciar el estudio de este método, recordemos algunas propiedades de las desigualdades numéricas.
Los axiomas elementales sobre desigualdades son los siguientes

\begin{enumerate}
    \item Dado un número real $x$, se tiene que $x > 0$, $x = 0$ o $x < 0$.
    \item Si $a > 0$ y $b > 0$, entonces $a + b > 0$ y $ab > 0$.
    \item Si $a > b$, entonces $a + c > b + c$.
\end{enumerate}
Todas las demás desigualdades se derivan de estos axiomas.
Como por ejemplo
\begin{enumerate}
    \item Si $a > b$ y $c < 0$, entonces $ac < bc$.
    \item Si $0 < a < 1$, entonces $a^2 < a$.
    \item Si $|a| > 1$, entonces $a^2 > a$.
\end{enumerate}

Cuando se tiene una desigualdad de una variable, los pasos comunes son perecidos a los de resolver una ecuación líneal.
Como por ejemplo,
\begin{multicols}{2}
    \begin{enumerate}
        \item Remover denominadores.
        \item Remover paréntesis y corchetes.
        \item Mover términos para combinarlos.
        \item Mover términos semejantes.
        \item Normalizar coeficientes.
    \end{enumerate}
\end{multicols}

\begin{example}
    Dado que $2(x - 2) - 3(4x - 1) = 9(1 - x)$ y $y < x + 9$, comparar las cantidades $\dfrac{y}{\pi}$ y $\dfrac{10}{31}y$.
\end{example}


Ahora, vemos los siguientes teoremas.

\begin{theorem.box}{}{}
    Si $x$ es un número real, entonces $x^2 \geq 0$.
    La igualdad se da si y solo si $x = 0$.
\end{theorem.box}

\begin{theorem.box}{Desigualdad Media Aritmética - Geometríca}{}
    Dado $n$ números reales positivos $x_1$, $x_2$, $\ldots$, $x_n$ se tiene que
    \[
        \dfrac{x_1 + x_2 + \cdots + x_n}{n} \geq \sqrt[n]{x_1 x_2 \cdots x_n}.
    \]
    Donde la igualdad se da cuando $x_1 = x_2 = \cdots = x_n$.
\end{theorem.box}

\begin{theorem.box}{Desigualdad Cauchy-Schwartz}{}
    Dado los reales positivos $a_1, a_2, \ldots, a_n$ y $b_1, b_2, \ldots, b_n$ se tiene que
    \[
        (a_1^2 + a_2^2 + \cdots + a_n^2)(b_1^2 + b_2^2 + \cdots + b_n^2) \geq (a_1 b_1 + a_2 b_2 + \cdots + a_n b_n)^2.
    \]
    Donde la igualdad se da si $b_i = k a_i$ para todo $a_i \neq 0$ con $i = 1, 2, \ldots, n$
\end{theorem.box}

Finalmente, se proponen unos ejercicios para prácticar estas técnicas.



\subsection{Ejercicios y problemas}

Ejercicios y problemas para el autoestudio.

\begin{exercise}
    Sabiendo que $ac < 0$, argumentar cuántas de las siguiente desiguadades pueden ser verdaderas.
    \[
        \frac{a}{c} < 0,\quad ac^2 < 0,\quad a^2 c < 0,\quad c^3 a < 0,\quad ca^3 < 0.
    \]
\end{exercise}

\begin{exercise}
    Hay cuatro afirmaciones como se indica a continuación:
    \begin{itemize}
        \item[(i)] Cuando $0 < x < 1$, entonces $\dfrac{1}{1 + x} < 1 - x + x^2$;
        \item[(ii)] Cuando $0 < x < 1$, entonces $\dfrac{1}{1 + x} > 1 - x + x^2$;
        \item[(iii)] Cuando $-1 < x < 0$, entonces $\dfrac{1}{1 + x} < 1 - x + x^2$;
        \item[(iv)] Cuando $-1 < x < 0$, entonces $\dfrac{1}{1 + x} > 1 - x + x^2$.
    \end{itemize}
    Dar como respuesta las afirmaciones correctas.
\end{exercise}

\begin{exercise}
    Dado los números reales $a,b$.
    Si $a = \dfrac{x + 3}{4}$, $b = \dfrac{2x + 1}{3}$ y $b < \dfrac{7}{3} < 2a$, hallar el rango de valores para $x$.
\end{exercise}

\begin{exercise}
    Dado que $a < b < c < 0$ ordenar de manera descendiente los números $\dfrac{a}{b + c}$, $\dfrac{b}{c + a}$ y $\dfrac{c}{a + b}$.
\end{exercise}

\begin{exercise}
    Probar que para reales no negativos $x,y,z$ se cumple que
    \[
        a^2 + b^2 + c^2 \geq ab + bc + ca.
    \]
\end{exercise}

\begin{exercise}
    Sean $a,b,c$ reales no negativos, probar que
    \[
        (a + b)(b + c)(c + a) \geq 8abc.
    \]
\end{exercise}

\begin{exercise}
    Sean $a,b,c > 0$ probar que
    \[
        \frac{a^3}{bc} + \frac{b^3}{ca} + \frac{c^3}{ab} \geq a + b + c.
    \]
\end{exercise}

\begin{exercise}
    Sean $a,b,c$ reales positivos tales que $abc = 1$.
    Probar que
    \[
        a^2 + b^2 + c^2 \geq a + b + c.
    \]
\end{exercise}

\begin{exercise}
    Hallar todas las duplas positivas $(x,y)$ tal que $x^3 - y^3 = xy + 61$.
\end{exercise}

\begin{exercise}
    Resolver la siguiente ecuación en los enteros positivos $x,y,z$
    \[
        \frac{1}{x} + \frac{1}{y} + \frac{1}{z} = \frac{3}{5}.
    \]
\end{exercise}

\begin{exercise}
    Hallar todas las soluciones enteras $(x,y)$ de $x^3 + y^3 = (x + y)^2$.
\end{exercise}

\begin{exercise}
    Determinar todos los números enteros positivos $(x,y,z)$ que sean solución de
    \[
        \left(1 + \dfrac{1}{x}\right)\left(1 + \dfrac{1}{y}\right)\left(1 + \dfrac{1}{z}\right) = 2.
    \]
\end{exercise}

\begin{exercise}
    Hallar todas las soluciones en enteros de la ecuación
    \[
        x^3 + (x + 1)^3 + (x + 2)^3 + \cdots + (x + 7)^2 = y^3.
    \]
\end{exercise}

\begin{exercise}
    Determinar todos los números enteros positivos $(x,y,z)$ que sean solución de
    \[
        xy + yz + zx - xyz = 2
    \]
\end{exercise}

\begin{exercise}
    Halle todas las soluciones $(w,x,y,z)$ de enteros positivos tales que
    \[
        x^2 + y^2 + z^2 + 2xy + 2x(z - 1) + 2y(z + 1) = w^2
    \]
\end{exercise}

\begin{exercise}
    Determinar todos los números enteros positivos $(x,y,z)$ que sean solución de
    \[
        (x + y)^2 + 3x + y + 1 = z^2
    \]
\end{exercise}

\begin{exercise}
    Determine todas las parejas de enteros $(x, y)$ tales que
    \[
        1 + 2^x + 2^{2x + 1} = y^2.
    \]
\end{exercise}

\begin{problem}
    Encuentra todos los pares de números positivos $(a, b)$ tales que $ab^2 + b + 7$ divide a $a^2 b + a + b$.
\end{problem}

\begin{exercise}
    Si $a,b,c$ son positivos y se sabe que
    \[
        \frac{c}{a + b} < \frac{a}{b + c} < \frac{b}{a + c},
    \]
    escribir los números $a,b,c$ en orden descendiente.
\end{exercise}

\begin{exercise}
    Resolver la siguiente ecuación en enteros positivos
    \[
        3(xy + yz + zx) = 4xyz.
    \]
\end{exercise}

\begin{problem}
    Resolver en enteros distintos la ecuación
    \[
        x^2 + y^2 + z^2 + w^2 = 3(x + y + z + w).
    \]
\end{problem}