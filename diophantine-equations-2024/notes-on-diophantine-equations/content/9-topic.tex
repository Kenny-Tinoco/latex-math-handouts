\section{Tripla pitagóricas}

Probablemente conozca la ecuación $x^2 + y^2 = z^2$ por el teorema de Pitágoras, el cual describe la relación entre los lados $x,y,z$ con respecto al ángulo recto.
Si los lados son todos enteros positivos, estos forman una \textbf{tripla pitagórica} o \textbf{terna pitagórica}.
Por ejemplo, $(x,y,z) = (3,4,5)$ es una terna pitagórica, asi como $(5,12,13)$ es también una terna pitagórica.
También, cuando un triángulo rectángulo tiene longitudes enteras, lo llamaremos un triángulo pitagórico.

\begin{lemma}
    Sean $x,y,z$ una terna pitagórica y $d = \mcd{x}{y,z}$.
    Entonces,
    \begin{enumerate}
        \item [i)] Se tiene $d = \mcd x y = \mcd y z = \mcd z x$.
        \item [ii)] Si escribimos $x = d x_1$, $y = d y_1$ y $z = d z_1$, entonces $x_1, y_1, z_1$ es también una terna pitagórica.
    \end{enumerate}
\end{lemma}

\begin{definition.box}{Terna pitagórica primitiva}{}
    Se dice que una terna pitagórica es primitiva si $\mcd{x}{y,z} = 1$ o equivalentemente cualquiera de los $\mcd x y$, $\mcd y z$ o $\mcd z x$ es 1.
\end{definition.box}

\begin{lemma}
    Sea $x,y,z$ una terna pitagórica primitiva, entonces, entre $x,y,z$ hay exactamente un número par, que bien puede ser $x$ o $y$.
\end{lemma}

\begin{theorem.box}{}{}
    Sea $x,y,z$ una terna pitagórica primitiva de enteros positivos, con $y$ par.
    Entonces, existen enteros coprimos no negativos $m,n$ con $m > n$ talque
    \begin{align*}
        x = m^2 - n^2,\quad y = 2mn,\quad z = m^2 + n^2.
    \end{align*}
\end{theorem.box}
La mayoría de los problemas que implican ternas pitagóricas se reducen a utilizar las ternas ya conocidas o bien,
demostrar que una ecuación no tiene soluciones porque su forma es de una ecuación de Fermat, luego veremos algunos teoremas sobre este aspecto.
Por lo cual es importante conocer algunas ternas pitagóricas, en el cuadro 1 se muestran las primeras diez ternas
pitagóricas con lo cual se invita al lector tenerlas presente.
\begin{table}
    \centering
    \begin{tabular}{|c|c|c||c|c|}
        \hline
        x & y & z & m & n\\
        \hline\hline
        3  &  4 &  5 & 2 & 1\\
        5  & 12 & 13 & 3 & 2\\
        15 &  8 & 17 & 4 & 1\\
        7  & 24 & 25 & 4 & 3\\
        21 & 20 & 29 & 5 & 2\\
        9  & 40 & 41 & 5 & 4\\
        35 & 12 & 37 & 6 & 1\\
        11 & 60 & 61 & 6 & 5\\
        45 & 28 & 53 & 7 & 2\\
        33 & 56 & 65 & 7 & 4\\
        \hline
    \end{tabular}
    \caption{Las primeras ternas pitagóricas primitivas.}
\end{table}

\begin{lemma}
    No existen dos enteros positivos tal que la suma y la diferencia de sus cuadrados también son cuadrados.
\end{lemma}
\begin{proof}
    El lema es equivalente a demostrar que el sistema de ecuaciones
    \[
        \begin{cases}
            x^2 + y^2 = z^2\\
            x^2 - y^2 = w^2
        \end{cases}
    \]
    es insoluble en los enteros positivos.
    Asumamos por absurdo que dicho sistema tiene soluciones en los enteros positivos y consideremos el par $(x,y)$ tal que
    $(x^2 + y^2)$ es mínimo, luego, es fácil ver que $\mcd x y = 1$.
    Sumando las dos ecuaciones obtenemos que $2x^2 = z^2 + w^2$ por lo cual $z$ y $w$ tienen la misma paridad.
    De aquí obtenemos que $(z + w)$ y $(z - w)$ son ambos pares, con esto podemos escribir
    \begin{align*}
        2x^2 = z^2 + w^2 \implies x^2 = \left(\frac{z + w}{2}\right)^2 + \left(\frac{z - w}{2}\right)^2
    \end{align*}
    la ecuación pitagórica de la derecha cumple con $\mcd{x}{\frac{z + w}{2}, \frac{z - w}{2}} = 1$, en caso contrario, es decir
    $\mcd{x}{\frac{z + w}{2}, \frac{z - w}{2}} = d \geq 2$ tendríamos que $d \mid x$ y $d \mid \frac{z + w}{2} + \frac{z - w}{2} = z$
    que por el sistema original obtenemos $d \mid y$ lo cual contradice que $\mcd x y = 1$.
    Por tanto, existen enteros positivos $m,n$ tales que
    \[
        \frac{z + w}{2} = m^2 - n^2, \quad \frac{z - w}{2} = 2mn,
    \]
    o bien
    \[
        \frac{z + w}{2} = 2mn, \quad \frac{z - w}{2} = m^2 - n^2.
    \]
    Puesto que $2y^2 = z^2 - w^2$, en cualquiera de los dos casos anteriores tenemos que $2y^2 = 2(m^2 - n^2)\cdot 4mn$ y por tanto
    $y^2 = 4mn(m^2 - n^2)$.
    De esto obtenemos que $y = 2k$, luego $k^2 = mn(m + n)(m - n)$.
    Ya que $m,n$ son primos relativos $(m + n)$ es impar, así los enteros $m, n, (m + n), (m - n)$ son todos primos relativos dos a dos.
    Por lo cual, necesariamente $m = a^2, n = b^2, (m + n) = c^2, (m - n) = d^2$.
    Pero de aquí que $a^2 + b^2 = c^2$ y $a^2 - b^2 = d^2$ es decir $(a,b,c,d)$ también es solución al sistema original, sin embargo
    \[
        a^2 + b^2 = m + n < 4mn(m^2 - n^2) = y^2 < x^2 + y^2,
    \]
    lo cual contradice que $x^2 + y^2$ sea mínimo.
\end{proof}

La ecuación de Fermat o el último teorema de Fermat fue propuesta por el matemático francés Pierre de Fermat.
En matemáticas, es un teorema famoso por su dificultad y el proceso de demostración de este teorema ha llevado a muchos descubrimientos importantes tanto en álgebra como en análisis.

Mientras estudiaba la obra del antiguo matemático griego Diofánto,
Fermat escribió en el margen de su copia de un libro de Diofánto:
\begin{center}
    ``La ecuación a $a^n + b^n = c^n$ no tiene raíces enteras positivas para $n \geq 3$.
    He descubierto una prueba verdaderamente hermosa pero este margen es demasiado pequeño para contenerla.''
\end{center}
Durante más de 350 años, muchos matemáticos intentaron demostrar la afirmación de Fermat o refutarla encontrando algún contraejemplo.
En junio de 1993, Andrew Wiles, un matemático inglés de la Universidad de Princeton, afirmó que había demostrado el teorema.
El 25 de octubre del siguiente año, Wiles envió una prueba revisada a tres colegas, después de que sus colegas la juzgaran completa, Wiles publicó su prueba.

Los siguientes teoremas son ecuaciones de la forma del teorema de Fermat.

\begin{theorem.box}{}{}
    La ecuación $x^4 + y^4 = z^2$ no tiene soluciones enteras positivas.
\end{theorem.box}

\begin{theorem.box}{}{}
    La ecuación $x^4 - y^4 = z^2$ no tiene soluciones enteras positivas.
\end{theorem.box}

\begin{theorem.box}{}{}
    La ecuación $x^4 + y^2 = z^2$ no tiene soluciones enteras positivas.
\end{theorem.box}

\begin{theorem.box}{}{}
    La ecuación $x^4 - 4y^2 = z^2$ no tiene soluciones enteras positivas.
\end{theorem.box}

\begin{theorem.box}{}{}
    La ecuación $x^4 + 4y^2 = z^2$ no tiene soluciones enteras positivas.
\end{theorem.box}

\begin{example}
    Hallar los números $x \in \Z$ tales que $81 = (2x - x^2)(x^2 -2x + 2)$.
\end{example}

\begin{solution}
    Operando sobre la expresión, se obtiene.
    \begin{align*}
    (2x - x^2)(x^2 -2x + 2) &= 81 \\
    -(x^2 - 2x)(x^2 -2x + 2) &= 9^2 \\
    -(x^2 - 2x + 1 - 1)(x^2 -2x + 1 + 1) &= 3^4\\
    -\left[(x - 1)^2 - 1\right]\left[(x - 1)^2 + 1\right] &= 3^4\\
    1 - (x - 1)^4 &= 3^4\\
    3^4 + (x - 1)^4 &= 1^2
    \end{align*}
    Dicha expresión no puede tener soluciones enteras positivas, ya que esta tiene la forma del teorema 2.1, por lo cual no existe un entero $x$ tal que se cumpla la ecuación.
\end{solution}



\subsection{Ejercicios y problemas}
Ejercicios y problemas para el autoestudio.

\begin{exercise}
    Resolver las siguientes ecuaciones sobre los números naturales
    \begin{multicols}{2}
        \begin{enumerate}
            \item $15y^4 - z^2 - 2y^2 z + 1 = 0$
            \item $x^4 + y^4 - x^2 y^2 - z^2 - 2xyz = 0$
            \item $x^4 y^4 - 2x^2 y^2 + 1 = x^4 + y^4$
            \item $x^8 + y^8 - 3x^{4}y^4 = 625$
            \item $y^4 = 168x^4 + 338x^3 y + y^2$
        \end{enumerate}
    \end{multicols}
\end{exercise}

\begin{exercise}
    Resolver las siguientes ecuaciones sobre los números naturales
    \begin{multicols}{2}
        \begin{enumerate}
            \item $2x^4 + 2x^2 + 1 = z^4$
            \item $-14x^2 + y^4 = 49$
            \item $3x^4 + y^4 - 102x^2 + 2061 = 0$
            \item $3x^4 - 4x^2 + 1 = y^4$
            \item $x^4 + 4x^3 + 297x^2 + 4x + 1 = y^4$
        \end{enumerate}
    \end{multicols}
\end{exercise}

\begin{problem}
    Demostrar que el radio de la circunferencia inscrita en un triángulo pitagórico de lados enteros es siempre un número natural.
\end{problem}

\begin{problem}
    Demostrar que el área de un triángulo pitagórico no puede ser un cuadrado perfecto.
\end{problem}

\begin{problem}
    Hallar todas las soluciones en enteros positivos para el sistema de ecuaciones
    \[
        \begin{cases}
            a^2 + b^2 = c^2\\
            b^2 + c^2 = d^2.
        \end{cases}
    \]
\end{problem}

\begin{problem}
    Probar que la ecuación $p = x^2 + y^2$ para un primo $p$ de la forma $p = 4k + 1$ con $k$ entero, tiene una y
    solamente una solución en los enteros sin tomar en cuenta las permutaciones de $x,y$.
    (Pista: Usar ternas pitagóricas y luego descenso infinito)
\end{problem}

\begin{problem}
    Sea $\Delta ABC$ un triángulo no equilátero con lados de longitudes enteras.
    Sea $D$ y $E$ los puntos medios de $BC$ y $CA$, respectivamente, y sea $G$ el baricentro de $\Delta ABC$.
    Suponga que $D,C,E, G$ son cíclicos.
    Hallar el menos perimetro posible de $\Delta ABC$.
\end{problem}

\begin{problem}
    Hallar todos los $x,y \in \N$ tales que $x^2 + y^2 = 2017(x - y)$.
\end{problem}

\begin{problem}
    Resolver la ecuación diofántica $x^{-2} + y^{-2} = z^{-2}.$
\end{problem}