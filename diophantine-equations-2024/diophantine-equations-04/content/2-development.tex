\section{Desarrollo}

La congruencia en enteros es una poderosa herramienta en la solución de ecuaciones diofánticas, usualmente aplicaremos
este método para probar que ciertas ecuaciones son insolubles o bien deducir condiciones que las soluciones deben cumplir.
Ahora, vemos algunas definiciones.

\begin{definition.box}{Divisibilidad}{}
    Si $a$ y $b$ son enteros, se dice que $a$ divide a $b$ o que $b$ es múltiplo de $a$ si $b = aq$ para algún entero $q$,
    y se denota por $a \mid b$.
\end{definition.box}

\begin{definition.box}{Congruencias}{}
    Dados dos enteros $a$, $b$ y un entero positivo $m$, decimos que $a$ es congruente con $b$ módulo $m$ si $(a - b)$ es múltiplo de $m$.
    En este caso escribimos $a \modulo{b}{m}$.
\end{definition.box}
Es decir, tenemos $a \modulo{b}{m} \iff (a - b) \mid m$.
Con esto, podemos demostrar el siguiente teorema.

\begin{theorem.box}{Propiedades de Congrencia}{}
    Sean los enteros $a,b,c,d$ y $m \geq 1$.
    \begin{enumerate}
        \item Si $a \modulo{c}{m}$ y $c \modulo{d}{m}$, entonces $a \modulo{d}{m}$.
        \item Si $a \modulo{c}{m}$ y $b \modulo{d}{m}$, entonces $ab \modulo{cd}{m}$.
        \item Si $a \modulo{c}{m}$, entonces $a^n \modulo{c^n}{m}$ para todo entero positivo $n$.
        \item Si $ab \modulo{bc}{m}$, entonces $a \modulo{c}{\frac{m}{d}}$ donde $d = mcd(b,m)$.
    \end{enumerate}
\end{theorem.box}

\begin{example}
    Hallar el resto cuando $6^{1987}$ es dividido por 37.
\end{example}
\begin{solution}
    Como $6^2 = 36 = 37 - 1$ es claro que $6^2 \modulo{-1}{37}$.
    Asi al considerar $6^{1987} = 6\cdot 6^{1986} = 6\cdot (6^2)^{993}$ tenemos que $6^{1987} \equiv 6(-1)^{993} \equiv -6 \modulo{31}{37}$.
    Por lo cual, el resto de la división de $6^{1987}$ por 37 es 31.
\end{solution}

\begin{definition.box}{Función phi de Euler}{}
    Para cualquier entero $n$ se denota a $\varphi(n)$ como cantidad de números de coprimos menores a $n$.
\end{definition.box}

\begin{theorem.box}{Teorema de Euler}{}
    Si $a$ y $n$ son dos enteros positivos primos relativos entre si, entonces $a^{\varphi(n)} \modulo{1}{n}$.
\end{theorem.box}

\begin{theorem.box}{Pequeño teorema de Fermat}{}
    Si $p$ es primo y $a$ es un entero primo relativo con $p$, entonces $a^{p - 1} \modulo{1}{p}$.
\end{theorem.box}

Es claro que el teorema de Fermat es un caso concreto del teorema de Euler, de la propia definición de números primos
se deduce que un primo $p$ tiene exactamente $(p - 1)$ primos relativos, luego el resultado es evidente.

De estas definiciones y teoremas se obtienen ciertos restos especiales, conocidos como restos potenciales respecto a
un módulo, quizás los más famosos serían los restos cuadráticos, veamos algunos de ellos.
\begin{align*}
    &x^2 \modulo{0,1}{3}     && x^2 \modulo{0,1,4}{8}    && x^3 \modulo{0,\pm 1}{7}\\[2mm]
    &x^2 \modulo{0,1}{4}     && x^2 \modulo{0,1,4,9}{16} && x^4 \modulo{0,1}{16}\\[1.6mm]
    &x^2 \modulo{0,\pm 1}{5} && x^3 \modulo{0,\pm 1}{9}  && x^5 \modulo{0,\pm 1}{11}
\end{align*}
Estos restos pueden ser probados fácilmente, algunos son resultados inmediatos de teoremas como Fermat o Euler, asi mismo,
se puede considerar el conjunto de residuos de cada módulo y luego investigar el comportamiento de las potencias.
Veamos un ejemplo.

\begin{example}
    Probar que para todo entero $x$ se tiene $x^2 \modulo{0,1}{4}$.
\end{example}
\begin{solution}
    Cuando $x = 2x_1$, entonces $x^2 = 4x_1^2 \modulo{0}{4}$.
    Ahora, es claro que $\varphi(4) = 2$, aplicando Euler tenemos $x^2 \modulo{1}{4}$.
\end{solution}

Se deja como ejercicio al lector probar los demás restos potenciales.

\begin{example}
    Hallar todos los enteros $x,y$ tal que $15x^2 - 7y^2 = 9$.
\end{example}
\begin{solution}[1]
    Analizando la ecuación en módulo 3 tenemos que $-7y^2 \modulo{0}{3}$, es claro que $y$ es múltiplo de 3, es decir $y = 3y_1$.
    Por tanto,
    \[
        15x^2 - 7(9y_1^2) = 9 \iff 5x^2 - 7(3y_1^2) = 3
    \]
    donde en módulo 3 obtenemos que $x$ también es múltiplo 3, es decir $x = 3x_1$.
    Esto es,
    \[
        5(9x_1^2) - 7(3y_1^2) = 3 \iff 5(3x_1^2) - 7y_1^2 = 1 \iff 15x_1^2 - 7y_1^2 = 1.
    \]
    Al analizar la última ecuación en módulo 3 obtenemos que $y_1^2 \modulo{-1}{3}$, lo cual no es resto cuadrático en módulo 3.
    Luego, la ecuación no tiene soluciones.
\end{solution}

\begin{solution}[2]
    Analizando en módulo 5 tenemos que $-7y^2 \modulo{1}{5}$ por lo cual\footnote{¿Podés justificar por qué?} $y^2 \modulo{3}{5}$,
    este resto no es posible en módulo 5.
    Luego, la ecuación no tiene soluciones enteras.
\end{solution}


\subsection{Ejercicios y problemas}

Ejercicios y problemas para el autoestudio.

\begin{exercise}
    Demostrar que 7 divide a $3^{2n + 1} + 2^{n + 2}$ para todo natural $n$.
\end{exercise}

\begin{exercise}
    Hallar los restos cuadráticos en módulo 13.
\end{exercise}

\begin{exercise}
    Hallar los enteros positivos $a,b$ tales que $a^2 - 3b^2 = 8$.
\end{exercise}

\begin{exercise}
    ¿Existen enteros positivos $x,y$ tal que $x^3 = 2^y + 15$?
\end{exercise}

\begin{exercise}
    Probar que la ecuación $x^2 + 3xy - 2y^2 = 122$ no tiene soluciones enteras
\end{exercise}

\begin{exercise}
    Demostrar que la ecuación $x^2 - 7y = 3$ no tiene soluciones enteras.
\end{exercise}

\begin{exercise}
    Demostrar que no hay enteros para los cuales $800000007 = x^2 + y^2 + z^2$.
\end{exercise}

\begin{exercise}
    Hallar las soluciones enteras de la ecuación $x^2 - 5y^2 = 2$.
\end{exercise}

\begin{exercise}
    Si se cumple que $n \modulo{4}{9}$, probar que la ecuación $x^3 + y^3 + z^3 = n$ no tiene soluciones enteras.
\end{exercise}

\begin{exercise}
    Hallar las soluciones de enteros tal que $a^3 + 2b^3 + 4c^3 = 9d^3$.
\end{exercise}

\begin{exercise}
    Encontrar todas las soluciones $(p,q)$ de números primos tales que
    \[
        p^3 - q^5 = (p + q)^2.
    \]
\end{exercise}

\begin{problem}
    Probar que la ecuación $(x + 1)^2 + (x + 2)^2 + \cdots + (x + 2001)^2 = y^2$ no es soluble en enteros $x,y$.
\end{problem}

\begin{problem}
    Probar que la ecuación $(x + 1)^2 + (x + 2)^2 + \cdots + (x + 99)^2 = y^z$ no tiene soluciones enteras $(x,y,z)$ con $z > 1$.
\end{problem}

\begin{problem}
    Determinar todos los primos $p$ para los cuales el sistema de ecuaciones
    \[
        \begin{cases}
            p + 1 = 2x^2\\
            p^2 + 1 = 2y^2
        \end{cases}
    \]
    tiene soluciones enteras $(x,y)$.
\end{problem}

\begin{problem}
    Probar que la ecuación $x^5 - y^2 = 4$ es insoluble en los enteros.
\end{problem}

\begin{problem}
    Si $n$ es un entero positivo tal que la ecuación $x^3 - 3xy^2 + y^3 = n$ tiene tres soluciones enteras $(x,y)$.
    Probar que la ecuación es insoluble cuando $n = 2891$.
\end{problem}

\begin{problem}
    Determinar las posibles soluciones enteras no negativas $(x_1, x_2, \ldots, x_{14})$ de la ecuación $x_1^4 + x_2^4 + \cdots + x_{14}^4 = 15999$.
\end{problem}