\section{Desarrollo}

La congruencia en enteros es una poderosa herramienta en la solución de ecuaciones diofánticas, usualmente aplicaremos
este método para probar que ciertas ecuaciones son insolubles o bien deducir condiciones que las soluciones deben cumplir.
Ahora, vemos algunas definiciones.

\begin{definition}[Divisibilidad]
    Si $a$ y $b$ son enteros, se dice que $a$ divide a $b$ o que $b$ es múltiplo de $a$ si $b = aq$ para algún entero $q$,
    y se denota por $a \mid b$.
\end{definition}

\begin{definition}[Congruencias]
    Dados dos enteros $a$, $b$ y un entero positivo $m$, decimos que $a$ es congruente con $b$ módulo $m$ si $(a - b)$ es múltiplo de $m$.
    En este caso escribimos $a \modulo{b}{m}$.
\end{definition}

\begin{theorem}[Propiedades de Congrencia]
    Sean los enteros $a,b,c,d,m$ con $m \geq 1$.
    \begin{enumerate}
        \item Si $a \modulo{c}{m}$ y $c \modulo{d}{m}$, entonces $a \modulo{d}{m}$.
        \item Si $a \modulo{c}{m}$ y $b \modulo{d}{m}$, entonces $ab \modulo{cd}{m}$.
        \item Si $a \modulo{c}{m}$, entonces $a^n \modulo{c^n}{m}$ para todo entero positivo $n$.
        \item Si $ab \modulo{bc}{m}$, entonces $a \modulo{c}{\frac{m}{d}}$ donde $d = mcd(b,m)$.
    \end{enumerate}
\end{theorem}

\begin{theorem}[Pequeño teorema de Fermat]
    Si $p$ es primo y $a$ es un entero primo relativo con $p$, entonces $a^{p - 1} \modulo{1}{p}$.
\end{theorem}

\begin{theorem}[Teorema de Euler]
    Si $a$ y $n$ son dos enteros positivos, entonces
    \[
        a^{\varphi(n)} \modulo{1}{n}.
    \]
    Donde $\varphi(n)$ representa\footnote{Se conoce a $\varphi(x)$ como la función phi de Euler.} la cantidad de primos relativos menores a $n$.
\end{theorem}

Es claro que el teorema de Fermat es un caso concreto del teorema de Euler, puesto que un primo $p$ tiene exactamente
$(p - 1)$ primos relativos, por la propia definición de números primos, el resultado es evidente.

De estas definiciones y teoremas se obtienen los siguientes resultados.
\begin{multicols}{2}
    \begin{itemize}
        \item $x^2 \modulo{0,1}{3}$
        \item $x^2 \modulo{0,1}{4}$
        \item $x^2 \modulo{0,1,4}{8}$
        \item $x^2 \modulo{0,1,4,9}{16}$
        \item $x^3 \modulo{0,\pm 1}{9}$
        \item $x^4 \modulo{0,1}{16}$
        \item $x^5 \modulo{0,\pm 1}{11}$
    \end{itemize}
\end{multicols}
Estos resultados pueden ser probados fácilmente, algunos son resultados inmediatos de teoremas como Fermat o Euler.
O bien, se puede considerar el conjunto de residuos de cada módulo y luego investigar el comportamiento de las potencias.
En cualquier caso, se deja al lector el ejercicios de probar estos resultados.

\subsection{Ejercicios y problemas}

Ejercicios y problemas para el autoestudio.

\begin{exercise}
    Si se cumple que $n \modulo{4}{9}$, probar que la ecuación $x^3 + y^3 + z^3 = n$ no tiene soluciones enteras.
\end{exercise}

\begin{exercise}
    Probar que la ecuación $(x + 1)^2 + (x + 2)^2 + \cdots + (x + 2001)^2 = y^2$ no es soluble en enteros $x,y$.
\end{exercise}

\begin{exercise}
    Encontrar todas las soluciones $(p,q)$ de números primos tales que $p^3 - q^5 = (p + q)^2$.
\end{exercise}

\begin{exercise}
    Determinar todos los primos $p$ para los cuales el sistema de ecuaciones
    \[
        \begin{cases}
            p + 1 = 2x^2\\
            p^2 + 1 = 2y^2
        \end{cases}
    \]
    tiene soluciones enteras $(x,y)$.
\end{exercise}

\begin{exercise}
    Probar que la ecuación $(x + 1)^2 + (x + 2)^2 + \cdots + (x + 99)^2 = y^z$ no tiene soluciones enteras $(x,y,z)$ con $z > 1$.
\end{exercise}

\begin{exercise}
    Probar que la ecuación $x^5 - y^2 = 4$ es insoluble en los enteros.
\end{exercise}

\begin{exercise}
    Si $n$ es un entero positivo tal que la ecuación $x^3 - 3xy^2 + y^3 = n$ tiene tres soluciones enteras $(x,y)$.
    Probar que la ecuación es insoluble cuando $n = 2891$.
\end{exercise}

\begin{exercise}
    Determinar las posibles soluciones enteras no negativas $(x_1, x_2, \ldots, x_{14})$ de la ecuación $x_1^4 + x_2^4 + \cdots + x_{14}^4 = 15999$.
\end{exercise}