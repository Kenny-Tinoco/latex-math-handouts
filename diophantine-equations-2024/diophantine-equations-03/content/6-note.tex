\newpage
\section{Notas de clase}

\subsection{Qué}

Mostrar cómo usar el método de parametrización para resolver ecuaciones diofánticas con las cuales no se puede, o es más
fácil de usar, los métodos de factorización y desigualdades.

Repasar los temas de factorización y desigualdades con una pequeña lista de problemas.

\subsection{Cómo}

\subsubsection{Problemas para motivar el tema}
Ejercicio 1.2 (ejemplo).

Ejercicio 1.9, 1.11 y 1.13 (Para los estudiantes).

Ejercicio 1.10 (ejemplo).

\subsubsection{Ejemplos}
Los ejercicios 1.1 y 1.3.

\subsubsection{Contraejemplos}
Actualmente, no conozco una manera adecuada de desarrollar contraejemplos, y tampoco una perspectiva clara de como aplicarlos.

\subsubsection{Cómo ser más claro al explicar}
Por el momento, para lograr claridad solo se me ocurre ser minucioso y detallista con los ejemplos que se desarrollen.

\subsubsection{Problemas retadores pero posibles de resolver}
Los ejercicios 1.4 y 1.7.

Los ejercicios 1.8 y 1.12.

\subsubsection{Cantidad de tiempo por actividad}

10 min: Problema para motivar el tema.

15 min: Primer ejemplo práctico (5 min para intentarlo)

15 min: Segundo ejemplo práctico (5 min para intentarlo)

15 min: Primer problema retador (10 min para la solución)

\ 5 min: de descanso

30 min: Ejercicios para estudiantes (10 min por ejercicios)

\ * min: Dar las ideas generales de los últimos problemas retadores, formar dos grupos y a trabajar en la pizarra.


\subsection{Evaluación}
\\
\vspace{10mm}\\
\hrule\\
\vspace{6mm}
\ \hrule\\
\vspace{6mm}
\ \hrule\\
\vspace{6mm}
\ \hrule\\
\vspace{6mm}
\ \hrule\\
\vspace{6mm}
\ \hrule\\
\vspace{6mm}
\ \hrule\\
\vspace{6mm}
\ \hrule\\
\vspace{6mm}
\ \hrule\\
\vspace{6mm}
\ \hrule\\