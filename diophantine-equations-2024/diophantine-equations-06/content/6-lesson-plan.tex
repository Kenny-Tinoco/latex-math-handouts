\newpage
\section{Plan de clase}

\subsection{¿Qué?}

Dar a los estudiantes una serie de ejercicios para que practiquen los conceptos aprendidos.



\subsection{¿Cómo?}

\begin{activity}[][5 min]
    Dar a conocer que la asignación consiste en elegir 3 ejercicios de los propuestos y solucionarlos,
    las soluciones deben de traerlas el próximo encuentro.
\end{activity}

\par\vspace{2mm}
Nivel aproximado de los estudiantes, donde cero representa un nivel elemental y diez un nivel avanzado.
\begin{center}
    \begin{tabular}{|l|c|}
        \hline
        Alumno & Puntaje\\\hline\hline
        Bradley & 5\\\hline
        Brisa & 4\\\hline
        Daniela & 2 \\\hline
        Diego & 8 \\\hline
        Fabiana & 7 \\\hline
        Félix & 8 \\\hline
        Gerald & 5\\\hline
        Melvin & 6\\\hline
        Nahomí & 5 \\\hline
        Nathaly & 4 \\\hline
        Sharloth & 3 \\\hline
        William & 5 \\\hline
    \end{tabular}
\end{center}

De acuerdo a la tabla anterior pedir a los estudiantes que respondan las siguientes preguntas, priorizar los estudiantes
con menor puntaje, en caso de que un estudiante no pueda responder, pasar al siguiente estudiante hasta que la pregunta sea contestada.

\begin{activity}[][5 min]
    ¿Qué es una ecuación diofántica?
    ¿Qué es una ecuación insoluble y soluble?
\end{activity}

\begin{activity}[][5 min]
    ¿Dar 5 ejemplos ecuaciones diofánticas?
\end{activity}

\begin{activity}[][5 min]
    Explicar el método de factorización para la resolución de ecuaciones diofánticas.
\end{activity}

\begin{activity}[][5 min]
    ¿Cuáles son las posibles relaciones que pueden haber respecto a dos números reales?
\end{activity}

\begin{activity}[][5 min]
    ¿Qué es el método de desigualdades para la resolución de ED?
\end{activity}

\begin{activity}[][5 min]
    ¿Explicar las desigualdades de las medias?
    Dar un ejemplo.
\end{activity}

\begin{activity}[][5 min]
    ¿Método de parametrización?
    ¿qué relación tiene entre las familias de soluciones de una ED?
\end{activity}

\begin{activity}[][5 min]
    ¿Cuál es la diferencia entre el método de parametrización y factorización?
\end{activity}

\begin{activity}[][5 min]
    ¿Qué es la función phi ($\varphi$) de Euler?
    Dar un ejemplo, ¿qué pasa con los números primos y la función de Euler?
    ¿Cuánto es $\varphi(100)$?
\end{activity}

\begin{activity}[][5 min]
    ¿Qué es la aritmética modular?
    ¿Quién la introdujo?
\end{activity}

\begin{activity}[][5 min]
    ¿Qué es una congruencia?
    Dar dos ejemplos.
\end{activity}

\begin{activity}[][5 min]
    ¿Qué es una congruencia potencial?
    Dar un ejemplo y su demostración.
    ¿Siempre hay patrones en una congruencia potencial?
\end{activity}

\begin{activity}[][5 min]
    ¿Qué es la inducción matemática?
    ¿Podés dar una analogía de esta técnica?
\end{activity}

\begin{activity}[][5 min]
    ¿Qué es el método de descenso a infinito?
\end{activity}

\begin{activity}[][5 min]
    ¿Qué se puede concluir si tenemos un subconjunto de los números reales con respecto a un máximo y mínimo?
\end{activity}

\begin{exercise}
    Calcular los siguientes resultados:
    \begin{enumerate}
        \item $7 + 13 \modulo{x}{5}$
        \item $15 - 8 \modulo{x}{7}$
        \item $4 \times 6 \modulo{x}{11}$
        \item $3^5 \modulo{x}{13}$
        \item $2x + 5 \modulo{11}{13}$
        \item Hallar todos los $x$ tales que $x^2 \modulo{1}{17}$.
    \end{enumerate}
\end{exercise}

\begin{activity}[][5 min]
    ¿Qué es una ED líneal?
\end{activity}



\newpage
\subsection{Comentarios}
Preguntas claves: ¿me entendieron?
¿me salté algún tema?
¿di tiempo suficiente para pensar los problemas?
¿participaron?
¿problemas muy fáciles o muy difíciles, demasiados o muy pocos?
¿las explicaciones/ejemplos fueron suficientes y buenos?

\foreach \x in {1,...,25}{
    \myhrule{8.4}
}