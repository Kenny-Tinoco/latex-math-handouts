\section{Refutaciones y contraejemplos}

En matemáticas siempre se pregunta qué pasa con el recíproco de una implicación.
Se enuncia un teorema que tiene la clásica estructura $P\implies Q$ y nos preguntamos:
\begin{center}
    ``¿Será cierto también que $Q\implies P$, su recíproco?''
\end{center}
Muy a menudo, ocurre que el recíproco no es cierto.
Pero ¿cómo se prueba que una implicación es falsa?
Por medio de un contraejemplo.
Un contraejemplo es un caso en que las premisas son ciertas, pero la conclusión es falsa.
Exhibir un contraejemplo es refutar una implicación.
Pongamos como ejemplo el conocido teorema de que toda función derivable en un punto es continua.
La demostración consta de una línea; si $f(x)$ es derivable en $x = a$, entonces:
\[
    asfdasdf
\]
Ahora bien, ¿es cierto el recíproco, que toda función continua es derivable?
No, y el contraejemplo lo constituye la función $f(x) = |x|$ en $x = 0$.
Los siguientes cálculos lo prueban:
\[
    vasdf
\]
En (1) se demuestra que el valor absoluto es continuo;
en (2), que no es derivable, ya que los límites laterales de la derivada son distintos.
Esto es un contraejemplo de que las funciones continuas son derivables.
Obsérvese que basta dar un solo contraejemplo para invalidar la implicación entera.

Por último, por amor del desafío intelectual, hay autores que presentan problemas que aparentemente contienen algún tipo de trampa.
Una vez más, se resuelven a base de imaginación y lógica.
A continuación reproducimos un conocido problema de esta categoría (tomado del excelente libro ¡Ajá!, de Martin Gardner [Gar89]).

\begin{problem}
    En su lecho de muerte un beduino reparte la herencia a sus tres hijos.
    El hombre comunica a sus hijos que les deja 11 camellos que han de repartirse como sigue:
    $\frac{1}{2}$ de los camellos para el mayor, $\frac{1}{4}$ para el mediano y $\frac{1}{6}$ para el pequeño.
    Sin embargo, no saben cómo llevar a cabo el reparto, ya que ni 2, ni 4, ni 6 dividen a 11.
    Piden ayuda a un sabio, quien acude servicial a la casa del beduino en camello.
    Tras escuchar a los hijos, decide regalarles su camello para facilitar el reparto.
    Entonces ahora hay 12 camellos.
    Al mayor le corresponden 6, al mediano 3 y al menor 2.
    En total, se han repartido 11 camellos.
    El sabio toma su camello otra vez y vuelve a su casa.
    ¿Cómo pudo hacer el reparto?
\end{problem}

