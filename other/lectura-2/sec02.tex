\section{¿Qué es una demostración?}

El problema es que no se enseña a escribir textos matemáticos.
Como mucho, se exponen demostraciones delante de los alumnos, pero luego no se les exige que lo hagan bien ellos mismos hasta el último detalle y con rigurosidad.
Una demostración en matemáticas es el texto más común.
Empezaremos por revisar la estructura de una demostración y sus tipos.
Una vez aprendida la estructura de esta pieza básica de la escritura matemática, entraremos a fondo en otras cuestiones más generales tales como las refutaciones, la escritura de problemas, los principios generales de escritura y la corrección lingüística.

Una demostración o prueba está compuesta de tres partes: premisas, razonamiento y consecuencia.
Las premisas se llaman también hipótesis y la consecuencia, tesis o conclusión.
La relación entre las premisas, el razonamiento y la consecuencia es que siempre que las premisas sean ciertas y exista un razonamiento lógicamente correcto, entonces la consecuencia es cierta.
Un enunciado que relacione un conjunto de premisas y una consecuencia se llama teorema.
Probar o demostrar un teorema consiste en proporcionar un razonamiento lógicamente correcto que una premisas y conclusión.
Por ejemplo, veamos el siguiente teorema.

\begin{theorem}
    Sean $a,b,c$ tres enteros, donde $a,b \neq 0$.
    Entonces, si $a$ divide a $b$ y $b$ divide a $c$, se sigue que $a$ divide a $c$.
\end{theorem}

El teorema 1 establece que la divisibilidad es una propiedad transitiva.
La primera línea
\begin{center}
    Sean $a,b,c$ tres enteros, donde $a,b \neq 0$.
\end{center}
delimita el alcance del teorema.
Proclamamos la transitividad de una propiedad de los números enteros, pero no de otros conjuntos de números.
Esto es una cuestión de precisión.
La segunda línea
\begin{center}
    Entonces, si $a$ divide a $b$ y $b$ divide a $c$, se sigue que $a$ divide a $c$.
\end{center}
es el núcleo del enunciado del teorema, esto es, donde reside la sustancia lógica del teorema.
Las premisas son $a$ divide a $b$ y $b$ divide a $c$; la consecuencia, $a$ divide a $c$.
El teorema establece que si es cierto que $a$ divide a $b$ y $b$ divide a $c$, entonces es cierto también que $a$ divide a $c$.
Otros teoremas pueden tener una estructura lógica menos evidente.
Por ejemplo:
\begin{theorem}
    Sea $p$ un número primo y $a,b$ dos números enteros.
    Si $p$ divide a $a\cdot b$, entonces o bien $p$ divide a $a$ o bien $p$ divide a $b$.
\end{theorem}
La consecuencia del teorema 2 consiste en dos enunciados, $p$ divide a $a$ y $p$ divide a $b$, y la consecuencia establece que uno de los dos o ambos es cierto.
En este ejemplo, la estructura lógica del teorema es más complicada.
He aquí otro ejemplo de teorema.
\begin{theorem}
    Sea $n \geq 1$ un número natural.
    La siguiente fórmula es cierta para todo n:
    \[
        1 + 2 + \cdots + n = \frac{n(n + 1)}{2}.
    \]
\end{theorem}
Aquí la estructura del teorema 3 es
\begin{center}
    Si $n\geq 1$ es un número natural,\\ entonces $1 + 2 + \cdots + n = \frac{n(n + 1)}{2}$ es una igualdad cierta.
\end{center}
a pesar de que en la redacción literal del teorema 3 la palabra “entonces” no aparece.
En general, la estructura de un teorema es