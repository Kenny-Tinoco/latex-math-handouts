Con la técnica de sustitución vista en las sesiones pasadas se pueden resolver los siguientes problemas, con ciertas
evaluaciones y manipuleos es posible hallar sus soluciones fácilmente, además estos han aparecido en olimpiadas internacionales.
Por favor, intentá resolver estos problemas por tu cuenta.

\begin{prob-without-section}[India, 2010]
    Encontrar todas las funciones $f: \R \to \R$ que satisfacen
    \[
        f(x + y) + xy = f(x)f(y),\ \text{para todo}\ x,y \in \R.
    \]
\end{prob-without-section}

\begin{prob-without-section}[IMO, 2002]
    Hallar todas las funciones $f: \R \to \R$ tales que, para cualesquiera $x,y,u,v$ reales, se cumple
    \[
        \left[f(x) + f(y)\right]\left[f(u) + f(v)\right] = f(xu - yv).
    \]
\end{prob-without-section}

\begin{prob-without-section}[Korea, 2000]
    Hallar todas las funciones $f : \R \to \R$ que satisfacen
    \[
        f(x^2 - y^2) = (x - y)\left(f(x) + f(y)\right),\ \text{para todo}\ x,y \in \R.
    \]
\end{prob-without-section}

\section{Propieades de funciones}

Veamos algunas propiedades importantes sobre las funciones.
\begin{definition.box}{Inyectiva}{}
    Una función $f : A \to B$ es \textbf{inyectiva} si para los números $x_1, x_2 \in A$ con $f(x_1) = f(x_2)$, entonces se tiene $x_1 = x_2$.
    Es decir, valores de entrada distintos tienen salidas distintas.
\end{definition.box}

\begin{definition.box}{Sobreyectiva}{}
    Una función $f : A \to B$ es \textbf{sobreyectiva} si para todo $y \in B$ existe un $x \in A$ tal que $f(x) = y$.
\end{definition.box}

\begin{definition.box}{Biyectiva}{}
    Una función $f : A \to B$ es \textbf{biyectiva} si es inyectiva y sobreyectiva a la vez.
\end{definition.box}

\begin{definition.box}{Creciente}{}
    Una función $f : A \to B$ es \textbf{creciente} si para todo $(x_1 \leq x_2) \in A$ se tiene que $f(x_1) \leq f(x_2)$.
    En el caso de que la igualdad no se cumpla se dice que es estrictamente creciente.
\end{definition.box}

\begin{definition.box}{Decreciente}{}
    Una función $f : A \to B$ es \textbf{decreciente} si para todo $(x_1 \leq x_2) \in A$ se tiene que $f(x_2) \leq f(x_1)$.
    En el caso de que la igualdad no se cumpla se dice que es estrictamente decreciente.
\end{definition.box}

\begin{definition.box}{Monótona}{}
    Una función $f : A \to B$ es \textbf{monótona} si es creciente o decreciente.
    Además, se dice que es \textbf{estrictamente monótona} si es estrictamente creciente o estrictamente decreciente.
\end{definition.box}

\begin{exercise}
    Hallar todas las funciones $f : \Z \to \Z$ con $f(0) = 1$ que satisfacen
    \[
        f(f(n)) = f(f(n + 2) + 2) = n,
    \]
    para todo entero $n$.
\end{exercise}

\begin{exercise}
    Encontrar todas las funciones $f : \Z \to \Z$ tales que
    \[
        f(x + f(y)) = f(x) + y,\ \text{para todo}\ x,y \in \Z
    \]
\end{exercise}

\begin{exercise}
    Hallar todas las funciones $f,g : \R \to \R$ tales que $g$ es inyectiva y
    \[
        f(g(x) + y) = g(x + f(y)),\ \text{para todo}\ x,y \in \R.
    \]
\end{exercise}

\begin{exercise}
    Hallar todas las funciones $f$ de reales a reales tales que
    \[
        f(f(x) + y) = 2x + f(f(y) - x)
    \]
    para todo $x,y \in \R$.
\end{exercise}

\begin{exercise}
    Hallar todas las funciones $f : \R \to \R$ tal que para cualesquieras $x,y \in \R$,
    \[
        (y + 1)f(x)  f(x f(y) + f(x + y)) = y.
    \]
\end{exercise}

\begin{exercise}
    Sea $f: (0, \infty) \to \R$ una función tal que
    \begin{enumerate}
        \item[i)] $f$ es estrictamente decreciente,
        \item[ii)] $f(x) > - \frac{1}{x}$ para todo $x > 0$ y
        \item[iii)] $f(x)f\left(f(x) + \frac{1}{x}\right) = 1$ para todo $x > 0$.
    \end{enumerate}
    Hallar $f(1)$.
\end{exercise}

\begin{exercise}
    Hallar todas las funciones $f : \R \to \R$ tal que
    \[
        f\left(f(x)^2 + f(y)\right) = xf(x) + y
    \]
    para todo $x,y \in \R$.
\end{exercise}


\section{Problemas}

\begin{prob-without-section}[Lista corta IMO, 1988]
    Sea $f: \N \to \N$ una función que cumple
    \[
        f(f(m) + f(n)) = m + n,\ \text{para todos}\ m,n.
    \]
    Hallar los posibles valores de $f(1988)$.
\end{prob-without-section}

\begin{prob-without-section}[Lista corta IMO, 2002]
    Encontrar todas las funciones $f : \R \to \R$ tales que
    \[
        f(f(x) + y) = 2x + f(f(y) - x),\ \text{para todos}\ x,y \in \R.
    \]
\end{prob-without-section}

\begin{prob-without-section}[Ibero, 1993]
    Encontrar todas las funciones estrictamente crecientes $f : \N \to \N$ que satisfacen
    \[
        f(n f(m)) = m^2 f(mn),\ \text{para todos}\ m,n \in \R.
    \]
\end{prob-without-section}

\begin{prob-without-section}[Italia, 1999]
    Encontrar todas las funciones estrictamente monótonas $f : \R \to \R$ tal que
    \[
        f(x + f(y)) = f(x) + y,\ \text{para}\ x,y \in \R.
    \]
\end{prob-without-section}