\newcommand{\cvec}[1]{{\mathbf #1}}
\newcommand{\rvec}[1]{\vec{\mathbf #1}}
\newcommand{\ihat}{\hat{\textbf{\i}}}
\newcommand{\jhat}{\hat{\textbf{\j}}}
\newcommand{\khat}{\hat{\textbf{k}}}
\newcommand{\minor}{{\rm minor}}
\newcommand{\trace}{{\rm trace}}
\newcommand{\spn}{{\rm Span}}
\newcommand{\rem}{{\rm rem}}
\newcommand{\ran}{{\rm range}}
\newcommand{\range}{{\rm range}}
\newcommand{\mdiv}{{\rm div}}
\newcommand{\proj}{{\rm proj}}
\newcommand{\R}{\mathbb{R}}
\newcommand{\N}{\mathbb{N}}
\newcommand{\Q}{\mathbb{Q}}
\newcommand{\Z}{\mathbb{Z}}
\newcommand{\C}{\mathbb{C}}
\newcommand{\ZP}{\mathbb{Z^+}}
\newcommand{\ZN}{\mathbb{Z^-}}

\newcommand{\<}{\langle}
\renewcommand{\>}{\rangle}

\newcommand{\attn}[1]{\textbf{#1}}
\renewcommand{\emptyset}{\varnothing}

\newcommand{\modIn}[1]{\mbox{ (mód }#1\mbox{)}}
\newcommand{\fullModIn}[2]{\equiv #1 \mbox{ (mód }#2\mbox{)}}

\newcommand{\MDIF}{\textit{(MDIF)}}
\newcommand{\WellOrderingPrinciple}{\textit{Principio del Buen Orden}}

\newcommand{\ProductPrinciple}{\textit{Principio del producto} }
\newcommand{\AdditionPrinciple}{\textit{Principio de la suma} }
\renewcommand{\theenumi}{\alph{enumi}}

\newcommand{\sen}{\operatorname{sen}}
\newcommand{\arcsen}{\operatorname{arcsen}}
\newcommand{\li}{\displaystyle\lim}

\newcommand{\qedWhite}{\hfill\qed}
\newcommand{\qedBlack}{\hfill\ensuremath{\blacksquare}}

\newcommand{\sourceProblem}[1]
{
    \vspace{-4mm}
    \begin{flushright}
        \emph{(#1)}
    \end{flushright}
    \vspace{1mm}
}

\newcommand{\problemImage}[1]
{
    \begin{center}
        \includegraphics[width=5cm]{#1}
    \end{center}
}

\newcommand{\tempsolution}[1]
{
    \vspace{-3mm}
    \begin{proof}[\textbf{\textup{Solución}}]
        #1
    \end{proof}
    \vspace{1mm}
}

\newenvironment{solution}[1][\empty]
{
    \ifthenelse{ \equal{#1}{\empty} }
    {
        \textbf{Solución.}
    }
    {
        \textbf{Solución #1.}
    }
    }
    {
    \qedWhite
}

\newenvironment{math-proof}[1][\empty]
{
    \ifthenelse{ \equal{#1}{\empty} }
    {
        \textbf{Demostración.}
    }
    {
        \textbf{Demostración #1.}
    }
    }
    {
    \qedBlack
}

\newcommand{\enumSolution}[2]
{
    \vspace{-3mm}
    \begin{proof}[\textbf{\textup{Solución #1}}]
        #2
    \end{proof}
    \vspace{1mm}
}
