\theoremstyle{definition}

%With section
\newtheorem{lemma}{Lema}[section]
\newtheorem{example}{Ejemplo}[section]
\newtheorem{theorem}{Teorema}[section]
\newtheorem{problem}{Problema}[section]
\newtheorem{property}{Propiedad}[section]
\newtheorem{exercise}{Ejercicio}[section]
\newtheorem{corollary}{Corolario}[section]
\newtheorem{definition}{Definición}[section]

%Without numeration
\newtheorem*{note}{Nota}
\newtheorem*{hint}{Pista}

\newenvironment{solution}[1][]
{
    \begin{proof}[\textnormal{\textbf{Solución\ifthenelse{\equal{#1}{}}{}{ #1}}}]
    }{
    \end{proof}
}


\newtcbtheorem[use counter*= theorem, number within=section]{theorem.box}{Teorema}
{
    enhanced,
    coltitle = black,
    colbacktitle = gray!15!white,
    colback = white,
    fonttitle = \bfseries,
    breakable = true,
    separator sign none,
    description delimiters parenthesis,
    top=0mm,
    right=0mm,
    bottom=0mm,
    terminator sign={.},
    colframe=black,
    boxrule=0pt,
    leftrule=2pt,
    left=1mm,
    arc = 1.5pt,
    outer arc = 1.5pt,
}
{t}



\newtcbtheorem[number within=section]{definition.tcb}{Definición}
{
    enhanced,
    frame empty,
    interior empty,
    coltitle = black,
    colbacktitle = gray!15!white,
    fonttitle = \bfseries,
    extras broken = {frame empty, interior empty},
    borderline = {0.3mm}{0mm}{black},
    breakable = true,
    top = 4mm,
    before skip = 3.5mm,
    attach boxed title to top left = {yshift = -3mm, xshift = 3mm},
    boxed title style = {boxrule = 0mm, borderline = {0.1mm}{0mm}{black}},
    varwidth boxed title,
    separator sign none, description delimiters parenthesis,
    description font=\bfseries,
    terminator sign={.\hspace{1mm}}
}
{d}


\newtcolorbox[auto counter]{remark.tcb}[1][]
{
    breakable,
    title = Observación~\thetcbcounter.,
    colback = white,
    colbacktitle = gray!15!white,
    coltitle = black,
    fonttitle = \bfseries,
    bottomrule = 0pt,
    toprule = 0pt,
    leftrule = 2.5pt,
    rightrule = 0pt,
    titlerule = 0pt,
    arc = 2pt,
    outer arc = 2pt,
    colframe = black
}

\newtcbtheorem[use counter*= theorem, number within=section]{principle.tcb}{Principio}
{
    enhanced,
    frame empty,
    interior empty,
    coltitle = black,
    colbacktitle = gray!15!white,
    fonttitle = \bfseries,
    extras broken = {frame empty, interior empty},
    borderline = {0.3mm}{0mm}{black},
    breakable = true,
    top = 4mm,
    before skip = 3.5mm,
    attach boxed title to top left = {yshift = -3mm, xshift = 3mm},
    boxed title style = {boxrule = 0mm, borderline = {0.1mm}{0mm}{black}},
    varwidth boxed title,
    separator sign none, description delimiters parenthesis,
    description font=\bfseries,
    terminator sign={.\hspace{1mm}}
}
{t}

\newtcbtheorem[number within=section]{definition.box}{Definición}
{
    enhanced,
    coltitle = black,
    colback = white,
    fonttitle = \bfseries,
    breakable = true,
    attach title to upper,
    separator sign none,
    description delimiters parenthesis,
    top = 0mm,
    right = 0mm,
    bottom=0mm,
    description font=\bfseries,
    terminator sign={:\hspace{2mm}},
    colframe=black,
    boxrule=0pt,
    leftrule=2pt,
    left=1mm,
    arc = 0pt,
    outer arc = 0pt,
}
{d}
