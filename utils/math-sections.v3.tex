\theoremstyle{definition}

%With section
\newtheorem{lemma}{Lema}[section]
\newtheorem{example}{Ejemplo}[section]
\newtheorem{theorem}{Teorema}[section]
\newtheorem{problem}{Problema}[section]
\newtheorem{property}{Propiedad}[section]
\newtheorem{exercise}{Ejercicio}[section]
\newtheorem{corollary}{Corolario}[section]
\newtheorem{definition}{Definición}[section]

%Without numeration
\newtheorem*{note}{Nota}
\newtheorem*{hint}{Pista}

\newenvironment{solution}[1][]
{
    \begin{proof}[\textnormal{\textbf{Solución\ifthenelse{\equal{#1}{}}{}{ #1}}}]
    }{
    \end{proof}
}


\newtcbtheorem[use counter*= theorem, number within=section]{theorem.box}{Teorema}
{
    enhanced
    ,colback = gray!13!white
    ,frame hidden
    ,boxrule = 0sp
    ,borderline west = {2.5pt}{0pt}{black}
    ,sharp corners
    ,attach title to upper
    ,coltitle = black
    ,fonttitle = \bfseries
    ,description font = \mdseries
    ,separator sign none,
    ,terminator sign={.\hspace{2mm}}
    ,description delimiters parenthesis,
    right=1mm,
    top=0mm,
    left=1.5mm,
    bottom=0mm,
    breakable = true
}
{t}

\newtcbtheorem[number within=section]{definition.box}{Definición}
{
    colback = gray!13!white
    ,coltitle = black
    ,colframe = white
    ,boxrule = 0.3mm
    ,attach title to upper
    ,fonttitle = \bfseries
    ,description font = \mdseries
    ,separator sign none
    ,terminator sign={.\hspace{2mm}}
    ,description delimiters parenthesis,
    right=1mm,
    top=0mm,
    left=1mm,
    bottom=0mm,
    arc = 3pt,
    outer arc = 3pt,
}
{d}




\newtcolorbox[auto counter]{remark.box}[1][]
{
    breakable,
    title = Observación~\thetcbcounter.,
    colback = white,
    colbacktitle = gray!15!white,
    coltitle = black,
    fonttitle = \bfseries,
    bottomrule = 0pt,
    toprule = 0pt,
    leftrule = 2.5pt,
    rightrule = 0pt,
    titlerule = 0pt,
    arc = 2pt,
    outer arc = 2pt,
    colframe = black
}