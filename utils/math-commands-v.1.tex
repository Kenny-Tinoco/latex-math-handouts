\renewcommand{\qedsymbol}{$\blacksquare$}
\addto\captionsspanish{\renewcommand{\proofname}{\textnormal{\textbf{Demostración}}}}
\DeclareSymbolFont{yhlargesymbols}{OMX}{yhex}{m}{n}
\DeclareMathAccent{\wideparen}{\mathord}{yhlargesymbols}{"F3}

%\renewcommand{\theenumi}{\alph{enumi}}

%Number sets
\newcommand{\N}{\ensuremath{\mathbb{N}}}
\newcommand{\Z}{\ensuremath{\mathbb{Z}}}
\newcommand{\Q}{\ensuremath{\mathbb{Q}}}
\newcommand{\R}{\ensuremath{\mathbb{R}}}
\newcommand{\C}{\ensuremath{\mathbb{C}}}

\newcommand{\ZP}{\ensuremath{\mathbb{Z^+}}}
\newcommand{\ZN}{\ensuremath{\mathbb{Z^-}}}
\newcommand{\ZNN}{\ensuremath{\mathbb{Z}^{\geq 0}}}

\newcommand{\RP}{\ensuremath{\mathbb{R}^+}}
\newcommand{\RN}{\ensuremath{\mathbb{R}^-}}
\newcommand{\RNN}{\ensuremath{\mathbb{R}^{\geq 0}}}

%Useful commands
\newcommand{\refTheorem}[1]{\textbf{Teorema #1}}
\newcommand{\refDefinition}[1]{\textbf{Definición #1}}
\newcommand{\theTriangle}[1]{\ensuremath{\triangle #1}}
%ration Ceva and Menelao
\newcommand{\ratioCM}[6]{\ensuremath{\frac{#1 #4}{#4 #2} \cdot \frac{#2 #5}{#5 #3} \cdot \frac{#3 #6}{#6 #1}}}
\renewcommand{\emptyset}{\varnothing}
\newcommand{\inverseOf}[1]{\frac{1}{#1}}
\newcommand{\inverseOfD}[1]{\ensuremath{\dfrac{1}{#1}}}
\newcommand{\homothety}[4]{\ensuremath{H\hspace{-1mm}\left(#1, #2\right) : #3 \to #4}}
\newcommand{\fullMod}[2]{\equiv #1 \pmod{#2}}
\newcommand{\upla}[2]
{
    (#1_1, #1_2, \ldots, #1_{#2})
}
\newcommand{\kupla}[3][k]
{
    (#2_{#3}, \ldots, #2_{#1})
}
\newcommand{\polynom}[3][x]
{
    #3_{#2} #1^{#2} + #3_{#2 - 1} #1^{#2 - 1} + \cdots + #3_{1} #1 + #3_{0}
}
\newcommand{\asum}[3][x]%arithmetic sum
{
    \ifstrempty{#2}
    {
        #1_{1} +
        #1_{2}
    }
    {
        #1_{#2} +
        #1_{#2 + 1}
    }
    + \cdots + #1_{#3}
}

\newcommand{\gprod}[3][x]%product
{
    \ifstrempty{#2}
    {
        #1_{1} \cdot
        #1_{2}
    }
    {
        #1_{#2} \cdot
    #1_{#2 + 1}
    }
    \cdots #1_{#3}
}

\newcommand{\problemImage}[2]
{
    \begin{center}
        \includegraphics[width=#2]{#1}
    \end{center}
}




%Enviroments
\newenvironment{source-problem}
{
    \vspace{-5mm}
    \begin{flushright}
        \begin{itshape}
        (}{)
        \end{itshape}
    \end{flushright}
}

\newenvironment{solution}[1][]
{
    \ifstrempty{#1}
    {
        \begin{proof}[\textnormal{\textbf{Solución}}]
    }
    {
        \begin{proof}[\textnormal{\textbf{Solución #1}}]
    }
    }{
    \end{proof}
}