\section{OMCC y PAGMO}

\subsection{Encuentro 1}

\begin{section-problem}
    Sean reales $a$, $b$, $c > 0$, tales que $\dfrac{1}{a} + \dfrac{1}{b} + \dfrac{1}{c} = 3$.
    Demostrar que se cumple
    \[\sqrt {a} + \sqrt {b} + \sqrt {c} \geq 3.\]
\end{section-problem}

\begin{section-problem}
    Si $x + y + z = 1$, con $x, y, z \in \R^+$.
    Probar que
    \[xy + yz + 2zx \leq \frac{1}{2}.\]
\end{section-problem}

\begin{section-problem}
    Si $a^4 + b^4 + c^4 + d^4 = 16$, con $a, b, c, d \in \R$
    Probar que
    \[a^5 + b^5 + c^5 + d^5 \leq 32.\]
\end{section-problem}

\begin{section-problem}
    Para $a, b, c \in \R$ con $a^2 + b^2 + c^2 = 3$.
    Probar que
    \[\inverseOf{1 + ab} + \inverseOf{1 + bc} + \inverseOf{1 + ac} \geq \frac{3}{2}.\]
\end{section-problem}

\begin{section-problem}
    Para todo $x, y, z \in \R$, con $y \neq - z$, $z \neq - x$ y $x \neq - y$.
    Probar que
    \[\frac{x^2 - z^2}{y + z} + \frac{y^2 - x^2}{z + x} + \frac{z^2 - y^2}{x + y} \geq 0.\]
\end{section-problem}

\begin{section-problem}
    Definamos la secuencia $a_0 = 1$ y $a_n = \displaystyle\prod_{i = 0}^{n - 1} a_i + 1, n \geq 1$.
    Probar que
    \[\inverseOf{a_1} + \inverseOf{a_2} + \cdots + \inverseOf{a_n} + \inverseOf{a_{n + 1} - 1} = 1.\]
\end{section-problem}

\begin{section-problem}
    Definimos la secuencia $\{x_i\}_{i \geq 1}$ por $x_1 = \dfrac{1}{1012}$ y para $n \geq 1$
    \[x_{n + 1} = \frac{x_n + x_n^2}{1 + x_n + x_n^2}\]
    Hallar el valor de
    \[\inverseOf{x_1 + 1} + \inverseOf{x_2 + 1} + \cdots + \inverseOf{x_{1011} + 1} + \inverseOf{x_{1012}}.\]
\end{section-problem}

\begin{section-problem}
    Definamos la siguiente secuencia como
    \begin{gather*}
        B_1 = B_2 = 1\\
        B_n = 2 B_{n - 2} + B_{n - 1}, \quad n \geq 3.
    \end{gather*}
    Probar que para $n$ impar
    \[\sum_{i = 1}^{n - 1} B_i = B_n - 1.\]
\end{section-problem}

\begin{section-problem}
    Sea $x$, $y$, $z \in \R^+$, tal que $xyz = 1$, probar que la siguiente desigualdad se cumple
    \[\frac{x^3 + y^3}{x^2 + xy + y^2} + \frac{y^3 + z^3}{y^2 + yz + z^2} + \frac{z^3 + x^3}{z^2 + zx + x^2} \geq 2.\]
\end{section-problem}

\begin{section-problem}
    Sea $a_0 = a_1 = 1$ y
    \[a_{n + 1} = 1 + \frac{a^2_1}{a_0} + \frac{a^2_2}{a_1} + \cdots + \frac{a^2_n}{a_{n - 1}}, n \geq 1.\]
    Hallar $a_n$ en función de $n$.
\end{section-problem}

\begin{section-problem}
    Sea $P(x)$ un polinomio no nulo tal que $(x - 1)P(x + 1) = (x + 2)P(x)$ para todo real $x$, y $P(2)^2 = P(3)$.
    Hallar $m + n$, si $P\left(\dfrac{7}{2}\right) = \dfrac{m}{n}$ donde $m$ y $n$ son primos relativos.
\end{section-problem}





\subsection{Encuentro 2}

\begin{section-problem}
    Dado $a - b = 2$, $b - c = 4$.
    Hallar el valor de $a^2 + b^2 + c^2 - ab - bc - ca$.
\end{section-problem}

\begin{section-problem}
    Dado que $\dfrac{x}{x^2 + 3x + 1} = a$, $(a \neq 0)$.
    Encontrar el valor de
    \[\frac{x^2}{x^4 + 3x^2 + 1}.\]
\end{section-problem}

\begin{section-problem}
    Resolver
    \[\sqrt[3]{x^3 + 3x^2 - 4} - x = \sqrt[3]{x^3 - 3x + 2} - 1.\]
\end{section-problem}

\begin{section-problem}
    Pruebe que si $a$, $b$, $c$ son números reales positivos, entonces:
    \[\frac{a}{bc} + \frac{b}{ca} + \frac{c}{ab} \geq \frac{2}{a} + \frac{2}{b} - \frac{2}{c}.\]
\end{section-problem}

\begin{section-problem}
    Sean $a$, $b$, $c$ números reales positivos, muestre que
    \begin{gather*}
        \sum_{cyc} \frac{a}{(a + b)(a + c)} \leq \frac{9}{4(a + b + c)}.
    \end{gather*}
\end{section-problem}

\begin{section-problem}
    Sean $a$, $b$, $c$ números reales positivos, con $abc = 8$, muestre que
    \begin{gather*}
        \frac{a - 2}{a + 1} +
        \frac{b - 2}{b + 1} +
        \frac{c - 2}{c + 1} \leq 0
    \end{gather*}
\end{section-problem}

\begin{section-problem}
    Sea $P(x)$ un polinomio de coeficientes reales no negativos.
    Sean $x_1$, $x_2$, $\cdots$, $x_k$ reales positivos tales que $x_1 x_2 \cdots x_k = 1$.
    Demostrar que
    \[P(x_1) + P(x_2) + \cdots + P(x_k) \geq kP(1).\]
\end{section-problem}

\begin{section-problem}
    Consideremos la secuencia de números racionales definida por
    \begin{gather*}
        x_1 = \frac{4}{3} \\
        x_{n + 1} = \frac{x_n^2}{x_n^2 - x_n + 1}, n \geq 1
    \end{gather*}
    Demostrar que el numerador de la sumatoria
    \[\sum_{i = 1}^{n} x_i\]
    reducida en su mínima expresión es un cuadrado perfecto.
\end{section-problem}




\subsection{Encuentro 3}

\begin{section-problem}
    Hallar el valor de
    \[\frac{2^2}{2^2 - 1} \times \frac{3^2}{3^2 - 1} \times \frac{4^2}{4^2 - 1} \times \cdots \times \frac{2023^2}{2023^2 - 1}.\]
\end{section-problem}

\begin{section-problem}
    Determine el valor de la suma
    \[\frac{3}{1^2\times 2^2} + \frac{5}{2^2\times 3^2} + \frac{7}{3^2\times 4^2} + \cdots + \frac{4045}{2022^2\times 2023^2}.\]
\end{section-problem}

\begin{section-problem}
    Dado que $a + b = c + d$ y $a^3 + b^3 = c^3 + d^3$.
    Probar que
    \[a^{2023} + b^{2023} = c^{2023} + d^{2023}.\]
\end{section-problem}

\begin{section-problem}
    Sean $a, b$ y $c$ números naturales tales que
    \[ab(c + ab^2) + c^2(b^{2} c + a^3) = b^2 c(a^2 c + b) + a(a^2 b + c^3)\]
    Desmostrar que al menos uno de los números $a$, $b$ o $c$ es un cuadrado perfecto.
\end{section-problem}

\begin{section-problem}
    Sean $a_1, a_2, \cdots, a_{2023}$ números reales y para cada entero $1 \leq n \leq 2023$ sea
    \[S_n = a_1 + a_2 + \cdots + a_n\]
    Si $a_1 = 2023$ y se cumple que $S_n =  n^2 a_n$ para todo $n$, determina el valor de $a_{2023}$.
\end{section-problem}

\begin{section-problem}
    Existe un único polinomio con coeficiente reales de la forma
    \[P(x) =  7x^6 + a_5 x^5 + a_4 x^4 + a_3 x^3 + a_2 x^2 + a_1 x + a_0\]
    tal que $P(1) = 1$, $P(2) = 2$, $\cdots$,$P(6) = 6$.
    Hallar el valor de $P(0)$.
\end{section-problem}

\begin{section-problem}
    El entero positivo $n$ verifica
    \begin{gather*}
        \inverseOf{1\cdot\left(\sqrt {1} + \sqrt {2}\right) + \sqrt {1}} +
        \inverseOf{2\cdot\left(\sqrt {2} + \sqrt {3}\right) + \sqrt {2}} + \\
        \vdots\\
        + \inverseOf{n\cdot\left(\sqrt {n} + \sqrt {n + 1}\right) + \sqrt {n + 1}} = \frac{2022}{2023}
    \end{gather*}
    Hallar la suma de digitos de $n$.
\end{section-problem}

\begin{section-problem}
    Sean los reales positivos $a_1, a_2, \cdots, a_n$ tales que $a_1 \cdot a_2 \cdots a_n = 1$.
    Probar que
    \begin{gather*}
        \frac{a_1}{1 + a_1} + \frac{a_2}{(1 + a_1)(1 + a_2)} + \\
        \frac{a_3}{(1 + a_1)(1 + a_2)(1 + a_3)} + \\
        \vdots\\
        + \frac{a_n}{(1 + a_1)(1 + a_2)\cdots(1 + a_n)} \geq \frac{2^n - 1}{2^n}.
    \end{gather*}
\end{section-problem}

\begin{section-problem}
    Sea $n\geq 2$ un entero positivo y $a_1, a_2, \cdots, a_n$ números reales positivos tales que $a_1 + a_2 + \cdots + a_n = 1$.
    Probar que la siguiente desigualdad se cumple
    \begin{gather*}
        \frac{a_1}{1 + a_2 + a_3 + \cdots + a_n} + \frac{a_2}{1 + a_1 + a_3 + \cdots + a_n} +\\
        \vdots\\
        + \frac{a_n}{1 + a_2 + a_3 + \cdots + a_{n - 1}} \geq \frac{n}{2n - 1}.
    \end{gather*}
\end{section-problem}

\begin{section-problem}
    Definamos $a_k = (k^2 + 1)k!$ y $b_k = a_1 + a_2 + \cdots + a_k$.
    Sea
    \[\frac{a_{100}}{b_{100}} = \frac{m}{m}\]
    donde $m$ y $n$ naturales primos relativos.
    Hallar $n - m.$
\end{section-problem}

\begin{section-problem}
    Determine $a^2 + b^2 + c^2 + d^2$ si
    \begin{gather*}
        \frac{a^2}{2^2 - 1} + \frac{b^2}{2^2 - 3^2} + \frac{c^2}{2^2 - 5^2} + \frac{d^2}{2^2 - 7^2} = 1\\
        \frac{a^2}{4^2 - 1} + \frac{b^2}{4^2 - 3^2} + \frac{c^2}{4^2 - 5^2} + \frac{d^2}{4^2 - 7^2} = 1\\
        \frac{a^2}{6^2 - 1} + \frac{b^2}{6^2 - 3^2} + \frac{c^2}{6^2 - 5^2} + \frac{d^2}{6^2 - 7^2} = 1\\
        \frac{a^2}{8^2 - 1} + \frac{b^2}{8^2 - 3^2} + \frac{c^2}{8^2 - 5^2} + \frac{d^2}{8^2 - 7^2} = 1
    \end{gather*}
\end{section-problem}

\begin{section-problem}
    Probar que para todo entero positivo $n$, se puede encontrar una permutación del conjunto $\left\{ 1, 2, 3 , \cdots, n \right\}$ de manera que el promedio de dos enteros no aparece en medio de ellos.
    
    Por ejemplo, si se tiene $n = 4$, la permutación $\left\{ 1, 3, 2, 4 \right\}$ sirve, mientras que $\left\{ 1, 4, 2, 3 \right\}$ no, ya que 2 está entre 1 y 3, y $2 = \dfrac{1 + 3}{2}$.
\end{section-problem}

\begin{section-problem}
    Sea ${a_i}_{n}$ una sucesión de reales positivos que satisface
    \[a_{k + 1} \geq \frac{k a_k}{a_k^2 + (k - 1)}\]
    para todo entero positivo $k$.
    Muestra que $a_1 + a_2 + \cdots + a_n \geq n$, para todo $n \geq 2$.
\end{section-problem}

\begin{section-problem}
    Hallar todas las funciones $f: \R \rightarrow \R$ que satisfacen
    \[f(x^2 - y^2) = (x - y)\left(f(x) + f(y)\right) \quad \forall x, y \in \R.\]
\end{section-problem}

\begin{section-problem}
    Encontrar todas las funciones $f: \R \rightarrow \R$ que satisfacen
    \[f(x + y) + xy = f(x)f(y) \quad \forall x, y \in \R.\]
\end{section-problem}

\begin{section-problem}
    Encontrar todas las funciones $f: \R \rightarrow \R$ tales que, para cualesquiera $x, y, u, v$ reales, se cumple
    \[\left(f(x) + f(y)\right)\left(f(u) + f(v)\right) = f(xu - yv).\]
\end{section-problem}

\begin{section-problem}
    Hallar todas las funciones $f: \N \rightarrow \R$ que satisfacen que $f(1) = 3$, $f(2) =  2$ y
    \[f(n + 2) + \inverseOf{f(n)} = 2, \quad \forall n \in \N.\]
\end{section-problem}

\begin{section-problem}
    Un \textit{triminó} es una ficha rectangular de $1\times 3$.
    ¿Es posible cubrir un tablero de ajedrez de $8 \times 8$ usando 21 \textit{triminó}, de manera que hay exactamente una casilla $1 \times 1$ sin ser cubierta?.
    En caso de que la respuesta sea afirmativa, determine todas las posibles casillas que pueden quedar sin ser cubiertas.
\end{section-problem}

\begin{section-problem}
    En un tablero cuadriculado $n \times n$ se escriben números dentro de cada casilla mediante el siguiente proceso:
    \begin{itemize}
        \item Se seleccionan números reales $a_1, a_2, \cdots, a_n$, $b_1, b_2, \cdots, b_n$ todos distintos entre sí.
        \item En la casilla de la fila $i$ columna $j$ se escribe el número $a_i + b_j$.
    \end{itemize}
    Suponiendo que los $n$ productos de los números en cada fila del tablero son iguales entre sí, demostrar que los $n$ productos de los números en cada columna también son iguales entre sí.
\end{section-problem}

