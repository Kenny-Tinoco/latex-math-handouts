\section{Congruencias}
\vspace{-4mm}

\begin{definition}[Congruencias de Números Enteros]
    Sea $n$ un entero positivo.
    Si $a$ y $b$ son enteros cualesquiera, decimos que $a \fullModIn{b}{n}$ si $n \mid a - b$.
    Es decir, ambos números dejan el mismo resto en la división por $n$.
    Y se lee como
    \vspace{-2mm}
    \begin{center}
        $a$ es congruente con $b$ en módulo $n$.
    \end{center}
\end{definition}

\subsection{Propiedades básicas}
Sea $n$ un entero positivo.
Si $a, b, c, d$ son todos números enteros, entonces se cumplen las siguientes propiedades
\begin{itemize}
    \item $a \fullModIn{a}{n}$
    \item $a \fullModIn{b}{n} \Longrightarrow b \fullModIn{a}{n}$
    \item $a \fullModIn{b}{n} \Longrightarrow b \fullModIn{a}{n}$
    \item $a \fullModIn{b}{n} \mbox{ y } b \fullModIn{c}{n} \Longrightarrow a \fullModIn{c}{n}$
    \item $a \fullModIn{b}{n} \mbox{ y } c \fullModIn{d}{n} \Longrightarrow a + b \fullModIn{c + d}{n}$
    \item $a \fullModIn{b}{n} \mbox{ y } c \fullModIn{d}{n} \Longrightarrow a - b \fullModIn{c - d}{n}$
    \item $a \fullModIn{b}{n} \mbox{ y } c \fullModIn{d}{n} \Longrightarrow ab \fullModIn{cd}{n}$
    \item $ka \fullModIn{kb}{n}$ $\forall k \in \Z$
    \item $a^k \fullModIn{b^k}{n}$ $\forall k \in \Z$
    \item Si $mcd(k, n) =  d$, entonces $ka \fullModIn{kb}{n} \Leftrightarrow a \fullModIn{b}{\dfrac{n}{d}}$
\end{itemize}

\subsection{Algunas congruencias potenciales útiles}
Dado $x \in \Z$, entonces se cumple
\begin{table}[H]
    \centering
    \begin{tabular}{p{5cm} p{5cm}}
        $x^2 \fullModIn{0, 1}{3}$ & $x^3 \fullModIn{-1, 0, 1}{9}$
        \\
        $x^2 \fullModIn{0, 1}{4}$ & $x^4 \fullModIn{0, 1}{16}$
        \\
        $x^2 \fullModIn{0, 1, 4}{8}$ & $x^5 \fullModIn{-1, 0, 1}{11}$
        \\
        $x^2 \fullModIn{0, 1, 4, 9}{16}$
    \end{tabular}
\end{table}

\subsection{Ejercicios}

\begin{exercise}
    Sean $n, r \in \Z$, tal que $n \fullModIn{r}{7}$.
    Probar que $1000n \fullModIn{7 - r}{7}.$
\end{exercise}

\begin{exercise}
    Calcular el resto de $4^{100}$ por 3.
\end{exercise}

\begin{exercise}
    Calcular el resto de $4^{100}$ por 5.
\end{exercise}

\begin{exercise}
    Calcular el resto de $4^{100}$ por 7.
\end{exercise}

\begin{exercise}
    Demuestre que $n$ es divisible por 5 si y solo si su dígito de las unidades es divisible por 5.
\end{exercise}

\begin{exercise}
    ¿Cuál es el resto de $36^{36} + 41^{41}$ en la división por 77?
\end{exercise}

\begin{exercise}
    Probar que $p^2 - 1$ es divisible por $24$ si $p$ es un primo mayor que $3$.
\end{exercise}

\begin{exercise}
    Hallar el menor natural $n$ tal que 2001 es la suma de los cuadrados de $n$ enteros.
\end{exercise}

\begin{exercise}
    Sea $s(n)$ la suma de los dígitos de $n$.
    Si $N = 4444^{4444}$, $A = s(N)$ y $B = s(A)$.
    ¿Cuánto vale $s(B)$?
\end{exercise}

\begin{exercise}
    Pruebe que $11^{n + 2} + 12^{12n + 1}$ es divisible por 5 para todo entero $n$.
\end{exercise}

\begin{exercise}
    Sea $n > 6$ un entero positivo tal que $n - 1$ y $n + 1$ son primos.
    Muestre que $n^2 (n^2 + 16)$ es divisible por 720.
    Además, ¿el recíporoco de este ejercicio es verdadero?
\end{exercise}