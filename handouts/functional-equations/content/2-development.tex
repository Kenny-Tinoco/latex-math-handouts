\section{Métodos básicos}
{
    \begin{itemize}
        \item Sustitución de valores en las variables. El abordaje más común es darles valores pequeños o algo evidentes (p.e 0, 1, -1), después de eso (si es posible) algunas expresiones que harán que alguna parte de la ecuación se convierta en constante o invariante. Como es de esperase las sustituciones se hacen menos evidentes a medida que aumenta la dificultad de los problemas.
        \item Indución matemática. Este método se basa en usar el valor de $f(1)$ para encontrar todos los valores $f(n)$ con $n$ entero. Después encontramos $f(\frac{1}{n})$ y $f(r)$ para $r$ racional. Este método resulta de mucha utilidad en problemas donde la función está definida en los racionales ($f \in \Q$), especialmente en los problemas más sencillos.
        \item Invertigar sobre la injectividad or sobrejectividad de la fuciones implicadas en la ecuación. En muchos problemas estos hechos no muy dífiles de deducir pueden ser de gran importancia.
        \item Encontrar puntos fijos o ceros de las funciones. El número de problemas que se resuelven con este método para resolverse es considerablemente más pequeño que el número de problemas que se resuelven con los primeros tres métodos. Este método lo encontramos sobre todo en problemas de mayor dificultad.
        \item Usar las clásicas equaciones funcionales de Cauchy.
        \item Construir relaciones recurrentes.
        \item Sustituir la función. Este método a menudo es usado para simplificar la ecuación dada y rara vez es de importacia crucial
        \item NO OLVIDAR NUNCA verficar que la soluciones encontradas satisfagan las condiciones dadas.
    \end{itemize}
}

\section{Ecuaciones funcionales de Cauchy}
{
    \begin{section-definition}
        \emph{(Función aditiva)} Una función $f: A \rightarrow B$ es llamada aditiva si $$f(a_1+a_2) = f(a_1)+f(a_2)$$
        para toda $a_1, a_2$ tal que $a_1, a_2, a_1+a_2 \in A$
    \end{section-definition}
}