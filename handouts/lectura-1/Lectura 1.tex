\documentclass[12pt]{article}
\usepackage[charter]{mathdesign}
\usepackage[spanish]{babel}
\let\circledS\undefined % here - PS
\spanishdecimal{.}


\usepackage{cite}
\usepackage{fancyhdr}
\usepackage{multicol}
\usepackage[margin=2.4cm]{geometry}


%math packages
\usepackage{amsmath}
\usepackage{amssymb}
\usepackage{amsthm}
\usepackage{amsfonts}
\allowdisplaybreaks



%Figures dependences
\usepackage{pgf}
\usepackage{tikz}
\usepackage{float}
\usepackage{graphicx}
\usepackage{subcaption}
\usetikzlibrary{arrows}


%Text color box dependences
\usepackage{tcolorbox}
\usepackage{varwidth}
\tcbuselibrary{theorems}
\tcbuselibrary{breakable}
\tcbuselibrary{skins}


\setlength{\parskip}{2mm}
\setlength{\parindent}{0pt}

\renewcommand{\qedsymbol}{$\blacksquare$}
\addto\captionsspanish{\renewcommand{\proofname}{\textnormal{\textbf{Demostración}}}}
\DeclareSymbolFont{yhlargesymbols}{OMX}{yhex}{m}{n}
\DeclareMathAccent{\wideparen}{\mathord}{yhlargesymbols}{"F3}

%\renewcommand{\theenumi}{\alph{enumi}}

%Useful text commands
\newcommand{\ie}{\ensuremath{\text{i.e.}\ }}
\newcommand{\emphtext}[1]{\textbf{\emph{#1}}}


%Number sets
\newcommand{\N}{\ensuremath{\mathbb{N}}}
\newcommand{\Z}{\ensuremath{\mathbb{Z}}}
\newcommand{\Q}{\ensuremath{\mathbb{Q}}}
\newcommand{\R}{\ensuremath{\mathbb{R}}}
\newcommand{\C}{\ensuremath{\mathbb{C}}}

\newcommand{\positiveSet}[1]{\ensuremath{#1^+}}
\newcommand{\negativeSet}[1]{\ensuremath{#1^-}}
\newcommand{\nonnegativeSet}[1]{\ensuremath{#1^{\geq 0}}}

%Useful math commands
\newcommand{\ds}{\displaystyle}
\renewcommand{\max}[1]{\ensuremath{máx #1}}
\newcommand{\inverseOf}[1]{\frac{1}{#1}}
\newcommand{\fullMod}[2]{\equiv #1 \pmod{#2}}
\newcommand{\refTheorem}[1]{\textbf{Teorema #1}}
\newcommand{\refDefinition}[1]{\textbf{Definición #1}}
\newcommand{\inverseOfD}[1]{\ensuremath{\dfrac{1}{#1}}}

\newcommand{\showLine}{
    \setlength{\columnsep}{0.9cm}
    \setlength{\columnseprule}{0.2pt}
}
\newcommand{\hideLine}{
    \setlength{\columnsep}{0.4cm}
    \setlength{\columnseprule}{0pt}
}


% -- Geometry commads --
\renewcommand{\emptyset}{\varnothing}
\newcommand{\theTriangle}[1]{\ensuremath{\triangle #1}}
\newcommand{\homothety}[4]{\ensuremath{H\hspace{-1mm}\left(#1, #2\right) : #3 \to #4}}
\newcommand{\ratioCM}[6]{\ensuremath{\frac{#1 #4}{#4 #2} \cdot \frac{#2 #5}{#5 #3} \cdot \frac{#3 #6}{#6 #1}}}
\newcommand{\problemImage}[2]{
    \begin{center}
        \includegraphics[width=#2]{#1}
    \end{center}}

\newenvironment{solution}[1][]{
    \ifstrempty{#1}{
        \begin{proof}[\textnormal{\textbf{Solución}}]}
    {
        \begin{proof}[\textnormal{\textbf{Solución #1}}]}
    }{
    \end{proof}}
\theoremstyle{definition}

%With section
\newtheorem{section-lemma}{Lema}[section]
\newtheorem{section-example}{Ejemplo}[section]
\newtheorem{section-theorem}{Teorema}[section]
\newtheorem{section-problem}{Problema}[section]
\newtheorem{section-exercise}{Ejercicio}[section]
\newtheorem{section-corollary}{Corolario}[section]
\newtheorem{section-definition}{Definición}[section]

%Without section
\newtheorem{case}{Caso}
\newtheorem{example}{Ejemplo}
\newtheorem{problem}{Problema}
\newtheorem{remark}{Observación}
\newtheorem{corollary}{Corolario}

%Without numeration
\newtheorem*{note}{Nota}
\newtheorem*{definition}{Definición}


%Text Color Box styles
\newtcbtheorem[number within=section]{tcb-theorem-style}{Teorema}
{
    enhanced,
    frame empty,
    interior empty,
    coltitle = black,
    colbacktitle = gray!15!white,
    fonttitle = \bfseries,
    extras broken = {frame empty, interior empty},
    borderline = {0.3mm}{0mm}{black},
    breakable = true,
    top = 4mm,
    before skip = 3.5mm,
    attach boxed title to top left = {yshift = -3mm, xshift = 3mm},
    boxed title style = {boxrule = 0mm, borderline = {0.1mm}{0mm}{black}},
    varwidth boxed title,
    separator sign none, description delimiters parenthesis,
    description font=\bfseries,
    terminator sign={.\hspace{1mm}}
}
{theorem.tcb-label}

\newtcbtheorem[number within=section]{tcb-problem-style}{Problema}
{
    enhanced,
    frame empty,
    interior empty,
    coltitle = black,
    colbacktitle = white,
    fonttitle = \bfseries,
    extras broken = {frame empty, interior empty},
    borderline = {0.4mm}{0mm}{black},
    breakable = true,
    top = 4mm,
    before skip = 3.5mm,
    attach boxed title to top left = {yshift = -3mm, xshift = 3mm},
    boxed title style = {boxrule = 0.3mm, borderline = {0.3mm}{0mm}{black}},
    varwidth boxed title,
    terminator sign={.\hspace{1mm}}
}
{problem.tcb-label}

\newtcbtheorem[number within=section]{tcb-example-style}{Ejemplo}
{
    enhanced,
    frame empty,
    interior empty,
    coltitle = black,
    colback = white,
    colbacktitle = gray!15!white,
    fonttitle = \bfseries,
    borderline = {0.2mm}{0mm}{black},
    breakable = true,
    before skip = 3.5mm,
    after skip = 0mm,
    top = 1mm,
    bottom = 1mm,
    left = 1mm,
    right = 1mm,
    separator sign none, description delimiters parenthesis,
    description font=\bfseries,
    terminator sign={.\hspace{1mm}}
}
{example.tcb-label}

\newtcbtheorem[number within=section]{tcb-definition-style}{Definición}
{
    enhanced,
    frame empty,
    interior empty,
    coltitle = black,
    colbacktitle = gray!15!white,
    fonttitle = \bfseries,
    extras broken = {frame empty, interior empty},
    borderline = {0.3mm}{0mm}{black},
    breakable = true,
    top = 4mm,
    before skip = 3.5mm,
    attach boxed title to top left = {yshift = -3mm, xshift = 3mm},
    boxed title style = {boxrule = 0mm, borderline = {0.1mm}{0mm}{black}},
    varwidth boxed title,
    separator sign none, description delimiters parenthesis,
    description font=\bfseries,
    terminator sign={.\hspace{1mm}}
}
{definition.tcb-label}

\newtcbtheorem[number within=section]{tcb-lemma-style}{Lema}
{
    enhanced,
    frame empty,
    interior empty,
    coltitle = black,
    colbacktitle = white,
    fonttitle = \bfseries,
    extras broken = {frame empty, interior empty},
    borderline = {0.4mm}{0mm}{black},
    breakable = true,
    top = 4mm,
    before skip = 3.5mm,
    attach boxed title to top left = {yshift = -3mm, xshift = 3mm},
    boxed title style = {boxrule = 0.3mm, borderline = {0.3mm}{0mm}{black}},
    varwidth boxed title,
    separator sign none, description delimiters parenthesis,
    description font=\bfseries,
    terminator sign={.\hspace{1mm}}
}
{lemma.tcb-label}

\newtcolorbox[auto counter]{remark.tcb}[1][]
{
    breakable,
    title = Observación~\thetcbcounter.,
    colback = white,
    %colbacktitle = green!20!white,
    colbacktitle = gray!15!white,
    coltitle = black,
    fonttitle = \bfseries,
    bottomrule = 0pt,
    toprule = 0pt,
    leftrule = 2.5pt,
    rightrule = 0pt,
    titlerule = 0pt,
    arc = 2pt,
    outer arc = 2pt,
    colframe = black
}


%Enviroments section-x.tcb
\newenvironment{section-theorem.tcb}[1][]
{
    \ifstrempty{#1}
    {
        \begin{tcb-theorem-style}{}{}
    }
    {
        \begin{tcb-theorem-style}{#1}{}
    }
    }{
    \end{tcb-theorem-style}
}

\newenvironment{section-problem.tcb}
{
    \begin{tcb-problem-style}{}{}
    }{
    \end{tcb-problem-style}
}

\newenvironment{section-example.tcb}[1][]
{
    \ifstrempty{#1}
    {
        \begin{tcb-example-style}{}{}
    }
    {
        \begin{tcb-example-style}{#1}{}
    }
    }{
    \end{tcb-example-style}
}

\newenvironment{section-definition.tcb}[1][]
{
    \ifstrempty{#1}
    {
        \begin{tcb-definition-style}{}{}
    }
    {
        \begin{tcb-definition-style}{#1}{}
    }
    }{
    \end{tcb-definition-style}
}

\newenvironment{section-lemma.tcb}[1][]
{
    \ifstrempty{#1}
    {
        \begin{tcb-lemma-style}{}{}
    }
    {
        \begin{tcb-lemma-style}{#1}{}
    }
    }{
    \end{tcb-lemma-style}
}

\title{El Problema de Aprender a Enseñar: La enseñanza de la Solución de Problemas}
\author{Paul R. Halmos}
\date{Enero 1974}

\begin{document}
    \maketitle

    La mejor forma de aprender es hacer; la peor forma de enseñar es hablar.

    Sobre esto último: ¿ha notado que algunos de los mejores maestros del mundo son los peores conferencistas?
    (puedo probarlo, pero prefiero no perder tantos buenos amigos) Y, por otro lado, ¿ha notado que los buenos
    conferencistas no son necesariamente buenos maestros?
    Una buena conferencia es usualmente sistemática, completa, precisa—y aburrida: es un mal instrumento de enseñanza.
    Cuando es dada por legendarios y destacados expositores como Emil Artin y John von Neumann, aún una conferencia
    puede ser una herramienta útil—su carisma y entusiasmo sobresalen lo suficiente para inspirar al que escucha para
    seguir adelante y hacer algo—parece que es tan divertido.
    Sin embargo, para la mayoría de los mortales ordinarios, que no son tal malos conferencistas como lo fue Wiener
    —¡ni tan estimulantes!— y no tan buenos como Artin —¡ni tan dramáticos!— la conferencia es un instrumento de último
    recurso para la buena enseñanza.

    Mi prueba para lo que hace a un buen maestro es muy sencilla: es la pragmática de juzgar el desempeño por el
    producto.
    Si un maestro de estudiantes de posgrado consistentemente produce Doctores en Ciencias, que sean
    matemáticos y crean matemáticas de alta calidad, él es un buen maestro.
    Si un maestro de cálculo consistentemente produce graduados que se convierten en estudiantes sobresalientes de posgrado en
    matemáticas, o en ingenieros, biólogos o economistas líderes, él es un buen maestro.
    Si un maestro de tercer grado de la “nueva matemática” (o la vieja) produce consistentemente estudiantes sobresalientes de cálculo, o
    cajeros de mercados, o carpinteros, o mecánicos automotrices, él es un buen maestro.

    Para un estudiante de matemáticas, escuchar a alguien hablar sobre las matemáticas difícilmente le hace
    mayor bien que a un estudiante de natación el escuchar a alguien hablar sobre la natación.
    Tú no puedes aprender la técnica de natación teniendo a alguien que te dice cómo poner tus brazos y piernas; y no puedes
    aprender a resolver problemas teniendo a alguien que te dice cómo completar el cuadrado o sustituir $sen(u)$
    por $y$.

    ¿Puede uno aprender matemáticas leyendo?
    Estoy inclinado a decir que no.
    Leer tiene una ventaja sobre escuchar porque leer es más activo —pero no mucho.
    Leer con papel y lápiz al lado es mucho mejor —es un gran paso en la dirección correcta.
    La mejor forma de leer un libro, sin embargo, ciertamente con papel y
    lápiz al lado, es mantener el lápiz ocupado y abandonar el libro.
    Habiendo declarado esta posición extrema.
    La derogo de inmediato.
    Conozco que es extrema, y realmente no es lo que quiero decir—pero quiero hacer
    mucho énfasis en no dejarse llevar con la idea de que aprender significa asistir a las conferencias y leer libros.

    Si tuviéramos vidas más largas, y cerebros más grandes, y suficientes maestros expertos dedicados para tener
    una relación alumno/maestro de uno a uno, mantendría esta posición extrema —pero no los tenemos.
    Los libros y las conferencias no hacen buen trabajo en trasplantar los hechos y técnicas del pasado a la corriente
    sanguínea de los científicos del futuro —pero debemos tomar la segunda mejor opción para ahorrar tiempo y dinero.
    Pero, y este es el texto de mi sermón para hoy, si sólo dependemos de las conferencias y los libros, estamos
    haciendo a nuestros estudiantes, y a sus estudiantes, un grave perjuicio.

    De lo que realmente trata la matemática es sobre la solución de problemas concretos.
    Hilbert dijo una vez (pero no puedo recordar dónde) que la mejor forma de entender una teoría es encontrar, y luego
    estudiar, un ejemplo concreto prototipo de tal teoría, un ejemplo de raíz que ilustre todo lo que puede pasar.
    La falla más grande de muchos estudiantes, aún de los buenos, es que aunque pueden ser capaces de lanzar declaraciones
    correctas de teoremas, y recordar demostraciones correctas, no pueden dar ejemplos, construir
    contraejemplos, o resolver problemas especiales.
    He visto muchos estudiantes que pueden declarar algo que llaman el teorema espectral de operadores Hermiteanos
    en el espacio de Hilbert, pero que no tienen una idea de cómo diagonalizar una matriz simétrica real de $3\times 3$.
    Esto es malo —esto es mal aprendizaje, causado probablemente, al menos en parte, por mala enseñanza.
    El matemático profesional de tiempo-completo y el usuario ocasional de matemáticas, y todo el espectro de la
    comunidad científica en medio—todos necesitan resolver problemas, problemas matemáticos, y nuestro trabajo es
    enseñarles cómo hacerlo, o bien, enseñar a sus futuros maestros cómo enseñarles a hacerlo.
    Me gusta empezar cada curso que enseño con un problema.
    La última vez que enseñe el curso introductorio de teoría de conjuntos, mi primera oración fue la definición de
    números algebraicos, y la segunda fue la pregunta: ¿hay números que no sean algebraicos?
    La última vez que enseñe un curso introductorio de teoría
    de variable real, mi primera oración fue una pregunta: ¿existe una función continua no decreciente que mapea
    el intervalo unitario en el intervalo unitario de forma que la longitud de su gráfica es igual a 2?
    Para casi cualquier curso uno puede encontrar un conjunto pequeño de preguntas como éstas—preguntas que pueden
    hacerse con un mínimo de lenguaje técnico, que son lo suficientemente llamativas para captar el interés, que
    no tienen respuestas triviales, y que se las arreglan para contener, en sus respuestas, todas las ideas
    importantes de la materia.

    La existencia de tales preguntas es lo que uno quiere decir cuando expresa que las matemáticas tratan
    realmente sobre la resolución de problemas, y mi énfasis en la solución de problemas (en contraposición a
    asistir a conferencias o leer libros) está motivado por ellas.

    Un dicho famoso de Pólya sobre solución de problemas es que, si tú no puedes resolver un problema,
    entonces hay un problema más fácil que tampoco puedes resolver —¡encuéntralo!
    Si puedes enseñar este dicho a tus estudiantes, enseñarlo de manera que ellos puedan enseñarlo a los suyos, habrás
    resuelto el problema de crear maestros de solución de problemas.

    La parte más difícil de contestar preguntas es hacerlas; nuestro trabajo como maestros y maestros de maestros
    es enseñar cómo hacer preguntas.
    Es fácil enseñar a un ingeniero a utilizar un libro de recetas de ecuaciones diferenciales; lo que es difícil es
    enseñarle (y a su maestro) qué hacer cuando la respuesta no está en el recetario.
    En ese caso, de nuevo, el problema principal probablemente sea “¿cuál es el problema?”.
    Encuentre la pregunta correcta a contestar, y estará avanzado en el camino hacia la resolución del problema
    sobre el cual trabaja.

    ¿Cuál es entonces el secreto —cuál es la mejor forma de enseñar a resolver problemas?
    La respuesta se
    implica por la oración con la que inicié: resuelve problemas.
    El método que recomiendo se conoce algunas veces como el “método de Moore” ya que R. L. Moore lo desarrolló y
    utilizó en la Universidad de Texas.
    Es un método de enseñanza, un método para crear la actitud de resolución de problemas en un estudiante, el cual
    es una mezcla de lo que Sócrates enseñó y el espíritu competitivo y feroz de los juegos Olímpicos.
    La forma en que un mal conferencista puede ser un buen maestro, en el sentido de producir buenos estudiantes, es la
    forma en que un grano de sal puede producir ostras productoras de perlas.
    Una conferencia suave y un libro titulado “Álgebra introductoria para jovencitas” puede ser agradable; un buen
    maestro desafía, pregunta, incómoda, irrita, y mantiene altos estándares —todo lo cual es generalmente no placentero.
    Un buen maestro puede que no sea popular (excepto posiblemente con sus ex-alumnos), porque a algunos
    estudiantes no les gusta ser desafiados, cuestionados, incomodados e irritados —pero él produjo perlas (en
    lugar de lanzarlas de la manera proverbial).

    Permítanme contarles de la ocasión que enseñé un curso de álgebra lineal para estudiantes de segundo semestre.
    La primera hora repartí a cada estudiante algunas hojas de papel donde estaban escritas las declaraciones precisas
    de cincuenta problemas.
    Sólo eso—justamente las proposiciones de los teoremas.
    No había introducción, ni definiciones, no había explicaciones, y ciertamente, tampoco había demostraciones.
    El resto de la primera hora hablé al grupo algo sobre el método de Moore.
    Les dije que dejaran de leer álgebra lineal (¡tan sólo por ese semestre!), y dejar de consultarse el uno al otro
    (sólo para dicho semestre).
    Y les dije que el curso estaba en sus manos.
    El curso eran esos cincuenta teoremas; cuando los entendieran, cuando los pudieran explicar, cuando los pudieran
    apoyar con los ejemplos y contraejemplos necesarios, y claro, cuando los pudieran demostrar, entonces habrían terminado el curso.
    Se quedaron mirándome fijamente.
    No me creían.
    Creyeron que simplemente tenía pereza y quería ahorrarme trabajo.
    Estaban seguros que nunca aprenderían algo de esa manera.
    Todo esto no tomó más de media hora.
    Terminé la hora dándoles las definiciones básicas que ocupaban para entender aproximadamente la primera media
    docena de teoremas, y dándoles mis buenos deseos, los dejé a su suerte.
    La segunda hora, y cada hora sucesiva, le pedí a Smith
    demostrar el teorema 1, a Kovacs demostrar el teorema 2, y así sucesivamente.
    Animé a Kovacs y a Herrero y
    a todos, a observar a Smith como halcones, y a caerle si se equivocaba.
    Yo mismo escuchaba tan
    cuidadosamente como podía, y aunque trataba de no ser sádico, yo también caía cuando sentía la necesidad.
    Señalaba brechas, constantemente decía que no entendía, hacía preguntas sobre asuntos laterales, preguntaba
    y algunas veces ofrecía contraejemplos, les hablaba de la historia de la materia cuando tenía la oportunidad, y
    señalaba conexiones con otras partes de la matemática.
    Además, tomaba aproximadamente cinco minutos de
    la mayoría de las horas para introducir las nuevas definiciones requeridas.
    En conjunto probablemente hablé
    20 minutos de cada 50 minutos de las horas académicas que estuvimos juntos.
    Esto es mucho —pero es
    mucho menos que 50 (o 55) de las 50.
    Funcionó de maravilla.
    A la segunda semana estaban demostrando
    teoremas y encontrando errores en las demostraciones de otros, y obviamente encontrando placer en el
    proceso.
    Algunos de ellos tuvieron la gracia de buscarme y confesar que al principio estaban escépticos, pero luego se convirtieron.
    La mayoría de ellos dijeron que dedicaron más tiempo a la materia que a sus otros
    cursos dicho semestre, y que aprendieron más.

    Lo que acabo de describir es semejante al “método de Moore” como R. L. Moore lo usaba, pero es un método
    de Moore muy modificado.
    Estoy seguro que podrán hacerse cientos de modificaciones, para adaptarse a los
    temperamentos de diferentes maestros y a las necesidades de diferentes materias.
    Los detalles no importan.
    Lo que importa es hacer que los estudiantes formulen preguntas y las contesten.
    Muchas ocasiones cuando he
    utilizado el método de Moore, mis colegas me han comentado, posiblemente un semestre o dos después, que
    pueden reconocer aquellos estudiantes en sus clases, que han sido expuestos a una “clase de Moore” por las
    actitudes y comportamiento de dichos estudiantes.
    Las características distintivas son mayor madurez
    matemática que la de otros (la actitud de investigación), y una mayor inclinación y habilidad para formular
    preguntas penetrantes.

    La “actitud de investigación” es una tremenda ayuda para todos los maestros, estudiantes, creadores, y
    usuarios de las matemáticas.
    Para ilustrar, por ejemplo, cómo es una ayuda para mí cuando enseño cálculo
    elemental (a una clase que es demasiado grande para utilizar con ella el método de Moore).
    Debo antes hacer alarde ante ustedes de mi maravillosa memoria.
    Maravillosamente mala quiero decir.
    Si no enseño cálculo
    digamos por un semestre o dos, lo olvido.
    Olvido los teoremas, los problemas, las fórmulas, las técnicas.
    Como resultado, cuando preparo la clase de la siguiente semana, lo que hago dando un vistazo al programa
    pre escrito, o si no lo hay, a la tabla de contenido del texto, pero nunca en el texto mismo, empiezo casi de
    cero —hago investigación en cálculo.
    El resultado es que tengo más diversión que si lo tuviera todo
    memorizado, que una y otra vez me sorprendo genuinamente y me agrado de algunos estudiantes que
    redescubren lo que Leibniz probablemente sabía cuando era un adolescente, y que mi disfrute, sorpresa,
    placer y entusiasmo se siente por el grupo, y se toma como un reconocimiento a cada uno de los
    descubridores.

    Para enseñar la actitud de investigación, cada maestro debe hacer investigación y debe tener entrenamiento en
    hacer investigación.
    No estoy diciendo que cada uno que enseña trigonometría deba pasar la mitad de su tiempo demostrando teoremas
    abstrusos sobre teratología categórica y unirse a la carrera “publica -o- perece”.
    Lo que estoy diciendo es que cada uno que enseña, aún si enseña álgebra de secundaria, sería un mejor
    maestro si pensara sobre las implicaciones de la materia fuera de la materia, si lee sobre las conexiones de la
    materia con otras materias, si trata de resolver los problemas que tales implicaciones y conexiones sugieren—
    si, en otras palabras, si hace investigación en y alrededor del álgebra de secundaria.
    Esa es la única forma de mantener la actitud de investigación, la actitud de formulación de preguntas, viva en uno
    mismo, y así conservarla en una condición adecuada para ser transmitida a otros.

    Aquí está, resumido, en unas cuantas cáscaras de nuez:

    La mejor forma de aprender es hacer —preguntar y hacer.

    La mejor forma de enseñar es hacer que los estudiantes pregunten, y hagan.

    No predique hechos —estimule acciones.

    La mejor forma de enseñar a los maestros es hacer que ellos pregunten y hagan lo que ellos, a su vez, harán a
    sus estudiantes preguntar y hacer.

    Buena suerte, y feliz enseñanza, para todos ustedes.
\end{document}