\section{Desarrollo}

\begin{figure}[htb]
    \centering
    %\includegraphics[width=14cm]{images/colinealidad-1}
\end{figure}

Tres puntos son colineales si se encuentran sobre una misma línea.
Dicho esto, presentaremos algunos enfoques que nos ayudarán a probar que tres puntos son colineales al resolver problemas de geometría.

Hay tres formas más comunes de angulear que nos permiten probar que tres puntos $A$, $B$ y $C$ son colineales.

\begin{figure}[htb]
    \centering
    %\includegraphics[width=14cm]{images/colinealidad-6}
    \caption{Tres configuraciones de colinealidad.}
\end{figure}

En la primera configuración\footnote{Comenzando de izquierda a derecha.}, necesitaremos dos puntos adicionales que ya son colineales con nuestro punto \("\)medio\("\) B.
Sean esos puntos $X$ e $Y$.
Si $\angle XBA = \angle YBC$, entonces los puntos $A$, $B$ y $C$ son colineales.

En la segunda configuración, necesitaremos un punto extra $X$ que no esté en la supuesta línea $A - B - C$.
Si $\angle ABX + \angle XBC = 180^\circ$, entonces los puntos $A$, $B$ y $C$ son colineales.

En la tercera configuración, también necesitaremos un punto extra $X$ que no esté en la supuesta línea $A - B - C$.
Si $\angle XAB = \angle XAC$, entonces los puntos $A$, $B$ y $C$ son colineales.%
\begin{figure}[htb]
    \centering
    %\includegraphics[width=14cm]{images/colinealidad-1}
\end{figure}

Tres puntos son colineales si se encuentran sobre una misma línea.
Dicho esto, presentaremos algunos enfoques que nos ayudarán a probar que tres puntos son colineales al resolver problemas de geometría.

Hay tres formas más comunes de angulear que nos permiten probar que tres puntos $A$, $B$ y $C$ son colineales.

\begin{figure}[htb]
    \centering
    %\includegraphics[width=14cm]{images/colinealidad-6}
    \caption{Tres configuraciones de colinealidad.}
\end{figure}

En la primera configuración\footnote{Comenzando de izquierda a derecha.}, necesitaremos dos puntos adicionales que ya son colineales con nuestro punto \("\)medio\("\) B.
Sean esos puntos $X$ e $Y$.
Si $\angle XBA = \angle YBC$, entonces los puntos $A$, $B$ y $C$ son colineales.

En la segunda configuración, necesitaremos un punto extra $X$ que no esté en la supuesta línea $A - B - C$.
Si $\angle ABX + \angle XBC = 180^\circ$, entonces los puntos $A$, $B$ y $C$ son colineales.

En la tercera configuración, también necesitaremos un punto extra $X$ que no esté en la supuesta línea $A - B - C$.
Si $\angle XAB = \angle XAC$, entonces los puntos $A$, $B$ y $C$ son colineales.%

\subsection{Agregados culturales y preguntas}
{

}

\section{Ejercicios y Problemas}
{
    Sección de ejercicios y problemas para el autoestudio.
}