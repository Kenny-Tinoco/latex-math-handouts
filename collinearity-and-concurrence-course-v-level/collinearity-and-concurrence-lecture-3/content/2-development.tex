\section{Desarrollo}

\begin{section-theorem.tcb}[Steiner]
    Sean $D$ y $E$ dos puntos sobre el segmento $BC$ del triángulo $ABC$ tal que $\angle BAE = \angle DAC$.
    Así
    \[\frac{BD}{CD} \cdot \frac{BE}{CE} = \frac{AB^2}{AC^2}\]
\end{section-theorem.tcb}


\subsection{Simedianas}

La simediana es la reflexión de la mediana con respecto a la bisectriz interna del mismo vértice.
El siguiente lema proporciona una construcción alternativa a esta recta.

\begin{section-lemma.tcb}\label{simedian-lemma}
    Sea $D$ el punto de intersección de las tangentes por $B$ y $C$ al circuncírculo del triángulo $ABC$.
    Luego, $AD$ es una simediana de $\triangle ABC$.
\end{section-lemma.tcb}

Existen diversas pruebas del lema~\ref{simedian-lemma}.
Un camino considera polares y cuartetas armónicos.
Una vía adicional utiliza manipulación trigonométrica y el siguiente hecho.

\begin{section-lemma.tcb}
    Sea $N$ un punto sobre el segmento $BC$; entonces, $N$ pertenece a la $A$\nobreakdash-simediana si y solo si
    \[\frac{BN}{NC} = \left(\frac{AB}{AC}\right)^2\]
\end{section-lemma.tcb}






\section{Ejercicios y Problemas}
Sección de ejercicios y problemas para el autoestudio.
