\section{Problemas propuestos}

\begin{section-problem}
    Sean $BE$, $AD$ y $CF$ líneas tales que dividen a un triángulo en $AE = 12$, $EC = 6$, $CD = 7$, $DB = 10$, $BF = 5$, $FA = 7$.
    Demostrar que $BE$, $AD$ y $CF$ son concurrentes.
\end{section-problem}

\begin{section-problem}
    Sean $E$ y $F$ puntos sobre los lados $AB$ y $AC$ del triángulo \theTriangle{ABC}, respectivamente.
    Si $EF || BC$ y $M$ es el punto de corte de $BF$ y $CE$, demostrar que $AM$ biseca a $BC$.
\end{section-problem}

\begin{section-problem}
    Sean $AD$, $BE$ y $CF$ tres cevianas concurrentes en un triángulo \theTriangle{ABC}.
    Se toma un punto $D'$ en $BC$ tal que $BD = CD'$.
    Las paralelas a $BC$ por $E$ y $F$ cortan $AD$ y $AD'$ en $G$ y $H$ respectivamente.
    Prueba que $C$, $G$ y $H$ son colineales.
\end{section-problem}

\begin{section-problem}
    Los lados $BC$, $CA$ y $AB$ del triángulo \theTriangle{ABC} son cortados por dos rectas $l_1$ y $l_2$ en los puntos $D_1$, $E_1$, $F_1$ y $D_2$, $E_2$, $F_2$, respectivamente.
    Si $E_1 F_2 \cap BC = R$, $F_1 D_2 \cap CA = P$ y $D_1 E_2 \cap AB = Q$, demostrar que $R$, $P$ y $Q$ son colineales.
\end{section-problem}

\begin{section-problem}
    Sean $C$ y $F$ puntos sobre los respectivos lados $AE$ y $BD$ de un paralelogramo $AEBD$.
    Si $M$ y $N$ denotan los puntos de intersección de $CD$ con $FA$ y de $EF$ con $BC$, la línea $MN$ intersecará a $DA$ en $P$ y a $EB$ en $Q$.
    Demostrar que $AP = QB$.
\end{section-problem}

