\newpage
\section{Problemas}

\begin{section-problem}[\textbf{Capítulo 4~\cite{AKP16}, ejemplo 4.1}]
    Sea el triángulo \theTriangle{ABC} y $P$ un punto en su interior.
    Sea $A_1$, $B_1$ y $C_1$ las intersecciones de $AP$, $BP$ y $CP$ con los lados $BC$, $CA$ y $AB$, respectivamente.
    Considerando a $X$, $Y$ y $Z$ como la intersecciones de $BC$ con $B_1 C_1$, $CA$ con $C_1 A_1$ y $AB$ con $A_1 B_1$, respectivamente.
    Probar que $X$, $Y$ y $Z$ son colineales.
\end{section-problem}

\begin{section-problem}[\textbf{Capítulo 3~\cite{AKP16}, teorema 3.3}]
    Dado el cuadrilátero convexo $ABCD$ con $P$ la intersección de sus diagonales, entonces
    \[
        \frac{\sen(\angle PAD)}{\sen(\angle PAB)} \cdot \frac{\sen(\angle PBA)}{\sen(\angle PBC)} \cdot \frac{\sen(\angle PCB)}{\sen(\angle PCD)} \cdot \frac{\sen(\angle PDC)}{\sen(\angle PDA)} = 1.
    \]
\end{section-problem}

\begin{section-problem}[\textbf{Capítulo 3~\cite{AKP16}, ejemplo 3.10}]
    Sea \theTriangle{ABC} un triángulo cualquiera y $D$, $E$ y $F$ puntos cualesquiera sobre las rectas $BC$, $CA$ y $AB$ tal que las rectas $AD$, $BE$ y $CF$ concurren.
    La paralela a $AB$ por $E$ interseca a la recta $DF$ en el punto $Q$, la paralela a $AB$ por $D$ interseca a $EF$ en $T$.
    Probar que la rectas $CF$, $DE$ y $QT$ son concurrentes.
\end{section-problem}

\begin{section-problem}
    Sea $ABCD$ un cuadrado y sea $X$ un punto en lado $BC$.
    Sea $Y$ un punto en la recta $CD$ tal que $BX = YD$ y $D$ se encuentra entre $C$ y $Y$.
    Demuestra que el punto medio de $XY$ se encuetra sobre la diagonal $BD$.
\end{section-problem}

\begin{section-problem}[\textbf{Ceva's Theorem~\cite{TKT12}, problema 4}]
    Los puntos $P$, $Q$ y $R$ están sobre los lados $AB$, $BC$ y $CA$ del triángulo acutángulo \theTriangle{ABC}, respectivamente.
    Si $\angle BAQ = \angle CAQ$, $QP \perp AB$, $QR \perp AC$ y $CP$ y $BR$ se intersecan en $S$ probar que $AS \perp BC$.
\end{section-problem}

\begin{section-problem}
    Sea el triángulos \theTriangle{ABC} tal que una circunferencia que pasa por $A$ y $B$ interseca los segmentos $AC$ y $BC$ en $D$ y $E$, respectivamente.
    Las rectas $AB$ y $DE$ se intersecan en $F$, mientras que las rectas $BD$ y $CF$ se intersecan en $M$.
    Probar que $MF = MC$ si y solo si $MB \cdot MD = MC^2$.
\end{section-problem}

\begin{section-problem}[\textbf{Ceva's Theorem~\cite{TKT12}, problema 5}]
    Los lados opuestos de un hexágono son paralelos.
    Probar que las rectas que pasan por los puntos medio de los lados concurren.
\end{section-problem}

\begin{section-problem}[\textbf{Capítulo 3~\cite{AKP16}, ejemplo 3.8}]
    Sea $D$ el pie de altura desde $A$ en el triángulo \theTriangle{ABC} y $M$, $N$ puntos en los lados $CA$ y $AB$ talque las rectas $BM$ y $CN$ se intersecan en $AD$.
    Probar que $AD$ es bisectriz del ángulo $\angle MDN$.
\end{section-problem}