\section{Ejercicios}

\subsection{Ejercicios de construcción - Homotecia en polígonos}

\begin{section-exercise}
    Sabiendo que $D$ es el homotético de $A$ respecto a $P$.
    Dibujar el triángulo \theTriangle{DEF} tal que \homothety{P}{-k}{\theTriangle{ABC}}{\theTriangle{DEF}}.
    \vspace*{\fill}
    \begin{figure}[H]
        \centering
        
%dash pattern=on 5pt off 2pt
%[fill = white, rounded corners = 4pt, inner sep = 1pt]
\begin{tikzpicture}[scale = 1]
    \clip(-6.1,-7.42) rectangle (6.49,9.03);
    \draw [line width=1.2pt] (1.32,6.26)-- (5.04,6.17);
    \draw [line width=1.2pt] (5.04,6.17)-- (4.58,7.54);
    \draw [line width=1.2pt] (4.58,7.54)-- (1.32,6.26);
    \begin{scriptsize}
        \normalsize
        \fill [color=black] (4.58,7.54) circle (2pt);
        \draw[color=black] (4.52,8.06) node {$A$};
        \fill [color=black] (1.32,6.26) circle (2pt);
        \draw[color=black] (0.87,6.29) node {$B$};
        \fill [color=black] (5.04,6.17) circle (2pt);
        \draw[color=black] (5.27,5.92) node {$C$};
        \fill [color=black] (-4.44,-7.1) circle (2pt);
        \draw[color=black] (-3.85,-6.74) node {$D$};
        \fill [color=black] (2.38,3.97) circle (2pt);
        \draw[color=black] (1.98,4) node {$P$};
    \end{scriptsize}
\end{tikzpicture}
    \end{figure}
    \vspace*{\fill}
\end{section-exercise}

\newpage
\begin{section-exercise}
    Sabiendo que $D$ es el homotético de $A$ respecto a $P$.
    Dibujar el triángulo \theTriangle{DEF} tal que \homothety{P}{-k}{\theTriangle{ABC}}{\theTriangle{DEF}}.
    \vspace*{\fill}
    \begin{figure}[H]
        \centering
        
%dash pattern=on 5pt off 2pt
%[fill = white, rounded corners = 4pt, inner sep = 1pt]
\begin{tikzpicture}[scale = 1.15]
    \clip(-6.08,-7.42) rectangle (9.47,8.97);
    \draw [line width=1.2pt] (-3.5,-2.85)-- (7.15,-6.77);
    \draw [line width=1.2pt] (7.15,-6.77)-- (-1.42,7.66);
    \draw [line width=1.2pt] (-1.42,7.66)-- (-3.5,-2.85);
    \begin{scriptsize}
        \normalsize
        \fill [color=black] (-1.42,7.66) circle (2pt);
        \draw[color=black] (-1.48,8.17) node {$A$};
        \fill [color=black] (-3.5,-2.85) circle (2pt);
        \draw[color=black] (-3.96,-2.83) node {$B$};
        \fill [color=black] (7.15,-6.77) circle (2pt);
        \draw[color=black] (7.52,-6.85) node {$C$};
        \fill [color=black] (0.24,1.23) circle (2pt);
        \draw[color=black] (-0.16,1.26) node {$P$};
        \fill [color=black] (1.48,-3.6) circle (2pt);
        \draw[color=black] (1.72,-3.23) node {$D$};
    \end{scriptsize}
\end{tikzpicture}
    \end{figure}
    \vspace*{\fill}
\end{section-exercise}

\newpage
\begin{section-exercise}
    Sabiendo que $D$ es el homotético de $A$ respecto a $P$.
    Dibujar el triángulo \theTriangle{DEF} tal que \homothety{P}{-k}{\theTriangle{ABC}}{\theTriangle{DEF}}.
    \vspace*{\fill}
    \begin{figure}[H]
        \centering
        
%dash pattern=on 5pt off 2pt
%[fill = white, rounded corners = 4pt, inner sep = 1pt]
\begin{tikzpicture}[scale = 1.2]
    \clip(-6.1,-7.42) rectangle (8.27,9.03);
    \draw [line width=1.2pt] (0.78,3.77)-- (5.04,6.17);
    \draw [line width=1.2pt] (5.04,6.17)-- (1.01,7.52);
    \draw [line width=1.2pt] (1.01,7.52)-- (0.78,3.77);
    \begin{scriptsize}
        \normalsize
        \fill [color=black] (1.01,7.52) circle (2pt);
        \draw[color=black] (0.95,8.03) node {$A$};
        \fill [color=black] (0.78,3.77) circle (2pt);
        \draw[color=black] (0.32,3.52) node {$B$};
        \fill [color=black] (5.04,6.17) circle (2pt);
        \draw[color=black] (5.55,6.37) node {$C$};
        \fill [color=black] (2.38,3.97) circle (2pt);
        \draw[color=black] (2.32,3.32) node {$P$};
        \fill [color=black] (6.63,-7.01) circle (2pt);
        \draw[color=black] (6.87,-6.62) node {$D$};
    \end{scriptsize}
\end{tikzpicture}
    \end{figure}
    \vspace*{\fill}
\end{section-exercise}

\newpage
\begin{section-exercise}
    Sabiendo que $B$ es el homotético de $E$ respecto a $P$.
    Dibujar el triángulo \theTriangle{ABC} tal que \homothety{P}{k}{\theTriangle{DEF}}{\theTriangle{ABC}}.
    \vspace*{\fill}
    \begin{figure}[H]
        \centering
        
%dash pattern=on 5pt off 2pt
%[fill = white, rounded corners = 4pt, inner sep = 1pt]
\begin{tikzpicture}[scale = 1.25]
    \clip(0.78,-7.42) rectangle (9.44,9);
    \draw [line width=1.2pt] (7.65,-0.57)-- (3.65,7.14);
    \draw [line width=1.2pt] (3.65,7.14)-- (5.91,7.09);
    \draw [line width=1.2pt] (5.91,7.09)-- (7.65,-0.57);
    \begin{scriptsize}
        \normalsize
        \fill [color=black] (1.32,6.26) circle (2pt);
        \draw[color=black] (1.04,6.8) node {$B$};
        \fill [color=black] (7.27,8.52) circle (2pt);
        \draw[color=black] (6.87,8.54) node {$P$};
        \fill [color=black] (7.65,-0.57) circle (2pt);
        \draw[color=black] (7.87,-0.2) node {$D$};
        \fill [color=black] (5.91,7.09) circle (2pt);
        \draw[color=black] (6.15,7.46) node {$E$};
        \fill [color=black] (3.65,7.14) circle (2pt);
        \draw[color=black] (3.84,7.52) node {$F$};
    \end{scriptsize}
\end{tikzpicture}
    \end{figure}
    \vspace*{\fill}
\end{section-exercise}

\newpage
\begin{section-exercise}
    Sabiendo que $A'$ es el homotético de $A$ respecto a $O$.
    Dibujar el hexágono $A' B' C' D' E' F'$ tal que \homothety{O}{-k}{ABCDEF}{A' B' C' D' E' F'}.
    \vspace*{\fill}
    \begin{figure}[H]
        \centering
        
%dash pattern=on 5pt off 2pt
%[fill = white, rounded corners = 4pt, inner sep = 1pt]
\begin{tikzpicture}[scale = 0.65]
    \clip(-15.5,-12.02) rectangle (11.16,14.82);
    \begin{scriptsize}
        \normalsize
        \fill [color=black] (-5.22,13.47) circle (4pt);
        \draw[color=black] (-5.31,14.31) node {$A$};
        \fill [color=black] (-10.15,12.63) circle (4pt);
        \draw[color=black] (-10.38,13.42) node {$B$};
        \fill [color=black] (3.44,10.49) circle (4pt);
        \draw[color=black] (3.81,10.08) node {$C$};
        \fill [color=black] (-3.17,1.93) circle (4pt);
        \draw[color=black] (-3.82,1.98) node {$O$};
        \fill [color=black] (-14.1,0.03) circle (4pt);
        \draw[color=black] (-13.73,0.63) node {$D$};
        \fill [color=black] (-12.75,11.84) circle (4pt);
        \draw[color=black] (-13.03,12.59) node {$E$};
        \fill [color=black] (2.69,6.07) circle (4pt);
        \draw[color=black] (3.02,6.68) node {$F$};
        \fill [color=black] (-0.89,-10.92) circle (4pt);
        \draw[color=black] (-0.47,-10.3) node {$A'$};
    \end{scriptsize}
\end{tikzpicture}
    \end{figure}
    \vspace*{\fill}
\end{section-exercise}






\newpage
\subsection{Ejercicios de construcción - Homotecia en círcunferencias}

\begin{section-exercise}
    Trazar el insimilicentro y exsilimicentro de las siguientes circunferencias.
    \textbf{a)}
    \vspace*{\fill}
    \begin{figure}[H]
        \centering
        
%dash pattern=on 5pt off 2pt
%[fill = white, rounded corners = 4pt, inner sep = 1pt]
\begin{tikzpicture}[scale = 0.15]
    \clip(3.3,-69.54) rectangle (54.51,40.3);
    \draw(25.35,16.74) circle (20.26cm);
    \draw(21.96,-25.37) circle (9.67cm);
    \begin{scriptsize}
        \normalsize
        \fill [color=black] (25.35,16.74) circle (18pt);
        \draw[color=black] (28.24,19.17) node {$O_1$};
        \draw[color=black] (35.85,31.73) node {$\Omega_1$};
        \fill [color=black] (21.96,-25.37) circle (18pt);
        \draw[color=black] (24.81,-22.9) node {$O_2$};
        \draw[color=black] (37.76,-19.85) node {$\Omega_2$};
    \end{scriptsize}
\end{tikzpicture}
    \end{figure}
    \vspace*{\fill}
    \newpage
    \textbf{b)}
    \vspace*{\fill}
    \begin{figure}[H]
        \centering
        \begin{tikzpicture}[scale = 0.39]
    \clip(9.34,-21.01) rectangle (41.22,34.49);
    \draw(25.35,16.74) circle (13.82cm);
    \draw(29.98,0.35) circle (7.19cm);
    \draw(70.21,15.61) circle (19.85cm);
    \draw(65.05,25.42) circle (7.37cm);
    \begin{scriptsize}
        \normalsize
        \fill [color=black] (25.35,16.74) circle (6pt);
        \draw[color=black] (26.88,18.09) node {$O_1$};
        \draw[color=black] (28.33,31.71) node {$\Omega_1$};
        \fill [color=black] (29.98,0.35) circle (6pt);
        \draw[color=black] (31.52,1.69) node {$O_2$};
        \draw[color=black] (21.45,0.66) node {$\Omega_2$};
    \end{scriptsize}
\end{tikzpicture}
    \end{figure}
    \vspace*{\fill}
    \newpage
    \textbf{c)}
    \vspace*{\fill}
    \begin{figure}[H]
        \centering
        
%dash pattern=on 5pt off 2pt
%[fill = white, rounded corners = 4pt, inner sep = 1pt]
\begin{tikzpicture}[scale = 0.43]
    \clip(49.37,-4.91) rectangle (91.36,36.25);
    \draw(25.35,16.74) circle (13.82cm);
    \draw(30.29,6.64) circle (8cm);
    \draw(70.21,15.61) circle (19.85cm);
    \draw(65.05,25.42) circle (7.37cm);
    \begin{scriptsize}
        \normalsize
        \fill [color=black] (70.21,15.61) circle (6pt);
        \draw[color=black] (71.76,16.96) node {$O_1$};
        \draw[color=black] (70.83,34.18) node {$\Omega_1$};
        \fill [color=black] (65.05,25.42) circle (6pt);
        \draw[color=black] (66.6,26.76) node {$O_2$};
        \draw[color=black] (61.65,22.53) node {$\Omega_2$};
    \end{scriptsize}
\end{tikzpicture}
    \end{figure}
    \vspace*{\fill}
\end{section-exercise}

\newpage
\begin{section-exercise}
    Dadas las siguientes circunferencias, aplicar el teorema de Monge.
    \vspace*{\fill}
    \begin{figure}[H]
        \centering
        
%dash pattern=on 5pt off 2pt
%[fill = white, rounded corners = 4pt, inner sep = 1pt]
\begin{tikzpicture}[scale = 0.85]
    \clip(8.64,-13.34) rectangle (22.64,11.2);
    \draw (11.71,-2.71) circle (1.81cm);
    \draw(17.62,-10.22) circle (2.73cm);
    \draw(15.22,5.62) circle (4.97cm);
    \begin{scriptsize}
        \normalsize
        \fill [color=black] (11.71,-2.71) circle (3pt);
        \draw[color=black] (11.28,-3) node {$O_1$};
        \fill [color=black] (15.22,5.62) circle (3pt);
        \draw[color=black] (16.22,5.63) node {$O_2$};
        \fill [color=black] (17.62,-10.22) circle (3pt);
        \draw[color=black] (18.43,-10.28) node {$O_3$};
    \end{scriptsize}
\end{tikzpicture}
    \end{figure}
    \vspace*{\fill}
\end{section-exercise}

\newpage
\begin{section-exercise}
    Dadas las siguientes circunferencias, aplicar el teorema de Monge D'Alembert.
    \vspace*{\fill}
    \begin{figure}[H]
        \centering
        
%dash pattern=on 5pt off 2pt
%[fill = white, rounded corners = 4pt, inner sep = 1pt]
\begin{tikzpicture}[scale = 1.1]
    \clip(8.42,-12.18) rectangle (23.85,4.58);
    \draw (19.99,-5.71) circle (1.52cm);
    \draw (11.39,-9.47) circle (2.57cm);
    \draw (17.98,0.41) circle (4.08cm);
    \begin{scriptsize}
        \normalsize
        \fill [color=black] (19.99,-5.71) circle (2.5pt);
        \draw[color=black] (20.37,-5.18) node {$O_1$};
        \fill [color=black] (17.98,0.41) circle (2.5pt);
        \draw[color=black] (18.51,0.95) node {$O_2$};
        \fill [color=black] (11.39,-9.47) circle (2.5pt);
        \draw[color=black] (11.88,-8.99) node {$O_3$};
    \end{scriptsize}
\end{tikzpicture}
    \end{figure}
    \vspace*{\fill}
\end{section-exercise}

\newpage
\begin{section-exercise}
    Dadas las siguientes circunferencias, aplicar los teoremas de Monge y Monge D'Alembert, con los insimilicentro de $\Omega_1$ con $\Omega_2$ y $\Omega_1$ con $\Omega_3$.
    \vspace*{\fill}
    \begin{figure}[H]
        \centering
        \begin{tikzpicture}[scale = 0.43]
    \clip(3.3,-17.04) rectangle (46.94,11.67);
    \draw(16.54,5.25) circle (4.26cm);
    \draw(14.8,-3.82) circle (2.34cm);
    \draw(34.62,-0.25) circle (10.92cm);
    \begin{scriptsize}
        \normalsize
        \fill [color=black] (16.54,5.25) circle (6pt);
        \draw[color=black] (16.34,6.5) node {$O_1$};
        \fill [color=black] (34.62,-0.25) circle (6pt);
        \draw[color=black] (35.35,0.38) node {$O_2$};
        \fill [color=black] (14.8,-3.82) circle (6pt);
        \draw[color=black] (15.19,-2.96) node {$O_3$};
        \draw[color=black] (13.9,9.78) node {$\Omega_1$};
        \draw[color=black] (28.63,10.28) node {$\Omega_2$};
        \draw[color=black] (13.25,-1.02) node {$\Omega_3$};
    \end{scriptsize}
\end{tikzpicture}
    \end{figure}
    \vspace*{\fill}
\end{section-exercise}

\newpage
\begin{section-exercise}
    Dibujar los circuncírculos de \theTriangle{ABC}, \theTriangle{DEF} y \theTriangle{XYZ} y aplicar los teoremas de Monge y Monge D'Alembert, con los insimilicentro de \theTriangle{ABC} con \theTriangle{DEF} y \theTriangle{XYZ}.
    \vspace*{\fill}
    \begin{figure}[H]
        \centering
        
%dash pattern=on 5pt off 2pt
%[fill = white, rounded corners = 4pt, inner sep = 1pt]
\begin{tikzpicture}[scale = 0.301]
    \clip(-8.38,-35.12) rectangle (48.28,34.14);
    \draw (17.04,11.82)-- (24.86,5.22);
    \draw (24.86,5.22)-- (13.53,3.47);
    \draw (13.53,3.47)-- (17.04,11.82);
    \draw (34.97,-5.09)-- (38.28,-21.18);
    \draw (38.28,-21.18)-- (32.49,-24.79);
    \draw (32.49,-24.79)-- (34.97,-5.09);
    \draw (10.58,-12.3)-- (8,-5.57);
    \draw (8,-5.57)-- (7.05,-12.17);
    \draw (7.05,-12.17)-- (10.58,-12.3);
    \begin{scriptsize}
        \normalsize
        \fill [color=black] (17.04,11.82) circle (8pt);
        \draw[color=black] (15.87,13.86) node {$A$};
        \fill [color=black] (13.53,3.47) circle (8pt);
        \draw[color=black] (10.95,3.17) node {$B$};
        \fill [color=black] (24.86,5.22) circle (8pt);
        \draw[color=black] (26.43,5.45) node {$C$};

        \fill [color=black] (34.97,-5.09) circle (8pt);
        \draw[color=black] (35.91,-3.55) node {$X$};
        \fill [color=black] (38.28,-21.18) circle (8pt);
        \draw[color=black] (40,-22.39) node {$Y$};
        \fill [color=black] (32.49,-24.79) circle (8pt);
        \draw[color=black] (31.95,-26.83) node {$Z$};

        \fill [color=black] (8,-5.57) circle (8pt);
        \draw[color=black] (8.43,-3.67) node {$D$};
        \fill [color=black] (7.05,-12.17) circle (8pt);
        \draw[color=black] (5.79,-13.39) node {$E$};
        \fill [color=black] (10.58,-12.3) circle (8pt);
        \draw[color=black] (11.43,-13.15) node {$F$};
    \end{scriptsize}
\end{tikzpicture}
    \end{figure}
\end{section-exercise}