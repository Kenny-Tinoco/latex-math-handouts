\section{Problemas propuestos}

\begin{section-problem}
    Sea \theTriangle{ABC} un triángulo cualquiera y $D$, $E$ y $F$ puntos cualesquiera sobre las rectas $BC$, $CA$ y $AB$ tal que las rectas $AD$, $BE$ y $CF$ concurren.
    La paralela a $AB$ por $E$ interseca a la recta $DF$ en el punto $Q$, la paralela a $AB$ por $D$ interseca a $EF$ en $T$.
    Probar que la rectas $CF$, $DE$ y $QT$ son concurrentes.
\end{section-problem}

\begin{section-problem}
    Sea $BCXY$ un rectángulo construido fuera del triángulo \theTriangle{ABC}.
    Sea $D$ pie de altura desde $A$ hacía $BC$ y sean $U$ y $V$ los puntos de intersección de $DY$ con $AB$ y $DX$ con $AC$, respectivamente.
    Probar que $UV || BC$.
\end{section-problem}

\begin{section-problem}
    Probar que la recta que pasa por el baricentro, $G$, de \theTriangle{ABC}, corta los lados $AB$ y $AC$ en los puntos $M$ y $N$, respectivamente, tal que $AM \cdot NC + AN \cdot MB = AM \cdot AN$.
\end{section-problem}

\begin{section-problem}
    Sea $ABCD$ un cuadrilátero cíclico.
    Las diagonales $AC$ y $BD$ se cortan en $E$, y los lados $AB$ y $CD$ se cortan en $F$.
    Sean $J$ y $K$ los ortocentros de \theTriangle{ADF} y \theTriangle{BCF}, respectivamente.
    Demostrar que $J$, $E$ y $K$ están alineados.
\end{section-problem}

\begin{section-problem}
    El punto $D$ está sobre el lado $AB$ del triángulo \theTriangle{ABC}.
    Sea $\omega_1$ y $\Omega_1$, $\omega_2$ y $\Omega_2$ los incírculos y los excírculos (tangentes al segmento $AB$) de los triángulos \theTriangle{ACD} y \theTriangle{BCD}, respectivamente.
    Probar que las tangentes externas comunes a $\omega_1$ y $\omega_2$, $\Omega_1$ y $\Omega_2$ se intersecan en $AB$.
\end{section-problem}