\chapter{\textbf{Propiedades básicas - Parte I}}

Un \textit{polinomio} es una expresión de la forma
\[
    P(x) = a_n x^n + a_{n - 1} x^{n - 1} + \ldots + a_1 x + a_0 = \sum_{i = 0}^{n} a_i x^i.
\]
Los números $a_i$ diremos se dicen \textit{coefientes} del polinomio $P(x)$.
Usualmente, consideramos $a_i$, en $\Z$, $\Q$, $\R$, $\C$ y decimos que el polinomio tiene coeficientes enteros, racionales, reales o complejos, respectivamente.
Denotamos por $\Z[x]$, $\Q[x]$, $\R[x]$, $\C[x]$ el conjunto de polinomios con coeficientes enteros, racionales reales o complejos, respectivamente.

El coeficiente $a_0$ se dice el \textit{término constante}.
Por la definición, se sigue que dos polinomios
\[
    P(x) = \sum_{i = 0}^{n} a_i x^i \ \text{y} \ Q(x) = \sum_{i = 0}^{m} b_i x^i
\]
son iguales si y solo si $a_i = b_i$ para todo $i$ (si $m > n$, entonces $b_{n + 1} = \ldots = b_m = 0$).

Definimos el \textit{grado} del polinomio $P(x) = \ds \sum_{i = 0}^{n} a_i x^i $ como el mayor entero $i$ tal que $a_i \neq 0$ y denotamos el grado por $\deg P(x)$.
Si $i$ es el mayor entero $i$ tal que $a_i \neq 0$, decimos que $a_i$ es el \textit{coeficiente principal} de $P(x)$.
Si el coeficiente principal es igual a 1, decimos que el polinomio es \textit{mónico}.
Observe que el grado de un polinomio constante $P(x) = a_0 \neq 0$ es cero.
No damos ningún grado al \textit{polinomio cero} $P(x) \equiv 0$ (i.e. el polinomio cuyos coeficientes son todos ceros)\footnote{Por conveción, también podemos asignar al polinomio cero el grado $- \infty$.}.
Normalmente, omitimos los términos que tiene coeficiente cero.
Por ejemplo, escribimos el polinomio $0x^3 + 1x^2 + 2x + 0$ como $x^2 + 2x$.
Este es claramente un ejemplo de un polinomio mónico de grado 2.

Podemos realizar algunas operaciones sobre polinomios.

Por ejemplo, si $P(x) = \ds \sum_{i = 0}^{n} a_i x^i$ y $Q(x) = \ds \sum_{i = 0}^{m} b_i x^i $ son dos polinomios y $n \leq m$, entonces la suma y el producto de $P(x)$ y $Q(x)$ es definida por
\[
    P(x) + Q(x) = \sum_{h = 0}^{m} (a_h + b_h) x^h \ \ \text{y} \ \ P(x)Q(x) = \sum_{h = 0}^{n + m} \left(\sum_{i + j = h} a_i b_j \right)x^h,
\]
respectivamente.
Sabiendo lo anterior obtenemos lo siguiente.
\begin{theorem.tcb}{}{}
    Sea $P(x)$ y $Q(x)$ dos polinomios y sea $k$ un entero positivo.
    Entonces,
    \begin{enumerate}
        \item $\deg [P(x) Q(x)] = \deg P(x) + \deg Q(x)$
        \item $\deg [P(x) + Q(x)] \leq \max (\deg P(x), \deg Q(x))$
        \item $\deg [P(x)^k] = k \cdot \deg P(x)$.
    \end{enumerate}
\end{theorem.tcb}




\section{Identidades}

El término \textit{identidad} designa una igualdad que se cumple para todo valor permitido de las incognitas que contenga (normalmente todo los números reales o todos los números complejos).
Cuando está claro por el contexto a menudo se omite específicar explicidamente el rango permitido de las incognitas.
Por ejemplo, las siguientes ecuaciones son identidades:
\begin{align*}
    a^2 - b^2 &= (a - b)(a + b),\\
    a^3 - b^3 &= (a - b)(a^2 + ab + b^2),\\
    (a + b + c)^2 &= a^2 + b^2 + c^2 + 2(a + b + c),\\
    \frac{a^2}{a + b} + \frac{b^2}{b + c} + \frac{c^2}{c + a} &= \frac{b^2}{a + b} + \frac{c^2}{b + c} + \frac{a^2}{a + c}.
\end{align*}
Las primeras tres se cumplen para todos los valores reales (o complejos) $a$, $b$ y $c$, mientras que el último solo se cumple si ninguno de los valores $a + b$, $b + c$ y $c + a$ es cero.
Sin embargo, la ecuaciones a continuación no cumplen con este criterio ya que no se cumplen universalmente:
\begin{align*}
    2x + 1 &= 5,\\
    \frac{1}{x - 2} + \frac{1}{x} &= 3,\\
    a^3 + b^3 + c^3 &= 3abc.
\end{align*}
Las identidades son los pilares de los cálculos matemáticos.
Se encuentran comunmente en las competiciones matemáticas, donde muchos problemas requieren de su conocimiento.

Aquí recopilamos algunas de las identidades más útiles.
Es importante que, para mejorar tu fortaleza en el trabajo con expresiones algebraicas, te esforcés en aprender estas identidades.

\begin{box.tcb}[Identidades Útiles]
    \vspace{-0.5cm}
    \begin{align*}
        &a^2 - b^2 = (a - b)(a + b) && \text{(Diferencia de cuadrados)}\\[2mm]
        &(a + b)^2 = a^2 + 2ab + b^2 && \text{(Binomio al cuadrado)}\\[2mm]
        &(a + b + c)^2 = a^2 + b^2 + c^2 + 2(ab + bc + ca) && \text{(Trinomio al cuadrado)}\\[2mm]
        &a^n - b^n = (a - b)(a^{n - 1} + a^{n - 2}b + \cdots + b^{n - 1}) && \text{(Diferencia de potencias)}\\[2mm]
        &a^n + b^n = (a + b)(a^{n - 1} - a^{n - 2}b + \cdots + b^{n - 1}) \ \text{con $n$ impar} && \text{(Suma de potencias)}\\[2mm]
        &a^3 + b^3 + c^3 - 3abc = (a + b + c)(a^2 + b^2 + c^2 - ab - bc - ca) && \text{(Identidad de Gauss)}\\[2mm]
        &(a + b)^n = a^n + \binom{n}{1} a^{n - 1}b + \cdots + \binom{n}{k} a^{n -k}b^k + \cdots + b^n && \text{(Binomio de Newton)}
    \end{align*}
\end{box.tcb}

\begin{remark.tcb}
    La \textit{Identidad de Gauss} implica que si $a + b + c = 0$, entonces $a^3 + b^3 + c^3 = 3abc$.
\end{remark.tcb}

\begin{remark.tcb}
    El \textit{Binomio al cuadrado} tiene un caso general, esto es, sean $x_1, x_2 , \ldots, x_n$ $n$ números, entonces
    \[
        (x_1 + \cdots + x_n)^2 = x_1^2 + \cdots + x_n^2 + 2 \sum_{1 \leq i < j \leq n} x_i x_j.
    \]
\end{remark.tcb}

Por ejemplo, cuando $n = 4$ tenemos
\[
    (a + b + c + d)^2 = a^2 + b^2 + c^2 + d^2 + 2 (ab + ac + ad + bc + bd + cd).
\]

\begin{remark.tcb}
    El caso general del \textit{Binomio de Newton} es de especial interés.
    Sean $x_1, x_2,\ldots, x_n$ $n$ números y sea $m$ un entero positivo, entonces
    \[
        (x_1 + \cdots + x_n)^m = \sum_{0 \leq i_1, i_2, \ldots, i_n \leq m}^{ i_1 + i_2 + \cdots + i_n = m } \binom{m}{i_1, i_2, \ldots, i_n} x_1^{i_1} x_2^{i_2} \cdot \ldots \cdot x_n^{i_n}.
    \]
    Esta identidad es llamada \textit{Identidad Multinomial}.
\end{remark.tcb}
Por ejemplo, cuando $n = m = 3$ tenemos
\[
    (a + b + c)^3 = a^3 + b^3 + c^3 + 3(a^2 b + b^2 c + c^2 a + b^2 a + c^2 b + a^2 c) + 6abc.
\]

Muchos problemas elementales de álgebra se simplifican enormemente con las identidades.
No obstante, veremos algunas de las aplicaciones creativas en soluciones más complejas.




\section{El coeficiente de $x^d$}

\section{Factorización y sus implicaciones}

\section{Valores de polinomios}

\section{División, MCD de polinomios}

\section{La composición de polinomios}

\section{Polinomios pares e impares}

