\section{OMCC y PAGMO}

\begin{section-problem}
    El entero positivo $n$ verifica
    \[\inverseOf{1\cdot\left(\sqrt {1} + \sqrt {2}\right) + \sqrt {1}} + \inverseOf{2\cdot\left(\sqrt {2} + \sqrt {3}\right) + \sqrt {2}} + \cdots + \inverseOf{n\cdot\left(\sqrt {n} + \sqrt {n + 1}\right) + \sqrt {n + 1}} = \frac{2022}{2023}\]
    Hallar la suma de digitos de $n$.
\end{section-problem}

\begin{section-problem}
    Sean reales $a$, $b$, $c > 0$, tales que $\dfrac{1}{a} + \dfrac{1}{b} + \dfrac{1}{c} = 3$.
    Demostrar que se cumple
    \[\sqrt {a} + \sqrt {b} + \sqrt {c} \geq 3.\]
\end{section-problem}

\begin{section-problem}
    Si $x + y + z = 1$, con $x, y, z \in \R^+$
    Probar que
    \[xy + yz + 2zx \leq \frac{1}{2}.\]
\end{section-problem}

\begin{section-problem}
    Si $a^4 + b^4 + c^4 + d^4 = 16$, con $a, b, c, d \in \R$
    Probar que
    \[a^5 + b^5 + c^5 + d^5 \leq 32.\]
\end{section-problem}

\begin{section-problem}
    Para $a, b, c \in \R$ con $a^2 + b^2 + c^2 = 3$.
    Probar que
    \[\inverseOf{1 + ab} + \inverseOf{1 + bc} + \inverseOf{1 + ac} \geq \frac{3}{2}.\]
\end{section-problem}

\begin{section-problem}
    Para todo $x, y, z \in \R$, con $y \neq - z$, $z \neq - x$ y $x \neq - y$.
    Probar que
    \[\frac{x^2 - z^2}{y + z} + \frac{y^2 - x^2}{z + x} + \frac{z^2 - y^2}{x + y} \geq 0.\]
\end{section-problem}

\begin{section-problem}
    Sea $a_0 = 1$ y $a_n = \displaystyle\prod_{i = 0}^{n - 1} a_i + 1, n \geq 1$.
    Probar que
    \[\inverseOf{a_1} + \inverseOf{a_2} + \cdots + \inverseOf{a_n} + \inverseOf{a_{n + 1} - 1} = 1.\]
\end{section-problem}

\begin{section-problem}
    Definimos la secuencia $\{x_i\}_{i \geq 1}$ por $x_1 = \dfrac{1}{1012}$ y para $n \geq 1$
    \[x_{n + 1} = \frac{x_n + x_n^2}{1 + x_n + x_n^2}\]
    Hallar el valor de
    \[\inverseOf{x_1 + 1} + \inverseOf{x_2 + 1} + \cdots + \inverseOf{x_{1011} + 1} + \inverseOf{x_{1012}}.\]
\end{section-problem}

\begin{section-problem}
    Sean los reales positivos $a_1, a_2, \cdots, a_n$ tales que $a_1 \cdot a_2 \cdots a_n = 1$.
    Probar que
    \[\frac{a_1}{1 + a_1} + \frac{a_2}{(1 + a_1)(1 + a_2)} + \frac{a_3}{(1 + a_1)(1 + a_2)(1 + a_3)} + \cdots + \frac{a_n}{(1 + a_1)(1 + a_2)\cdots(1 + a_n)} \geq \frac{2^n - 1}{2^n}.\]
\end{section-problem}

\begin{section-problem}
    Sea $n\geq 2$ un entero positivo y $a_1, a_2, \cdots, a_n$ números reales positivos tales que $a_1 + a_2 + \cdots + a_n = 1$.
    Probar que la siguiente desigualdad se cumple
    \[\frac{a_1}{1 + a_2 + a_3 + \cdots + a_n} + \frac{a_2}{1 + a_1 + a_3 + \cdots + a_n} + \cdots + \frac{a_n}{1 + a_2 + a_3 + \cdots + a_{n - 1}} \geq \frac{n}{2n - 1}.\]
\end{section-problem}

\begin{section-problem}
    Definamos la siguiente secuencia como
    \begin{gather*}
        B_1 = B_2 = 1\\
        B_n = 2 B_{n - 2} + B_{n - 1}, \quad n \geq 3.
    \end{gather*}
    Probar que para $n$ impar
    \[\sum_{i = 1}^{n - 1} B_i = B_n - 1.\]
\end{section-problem}

\begin{section-problem}
    Sea $x$, $y$, $z \in \R^+$, tal que $xyz = 1$, probar que la siguiente desigualdad se cumple
    \[\frac{x^3 + y^3}{x^2 + xy + y^2} + \frac{y^3 + z^3}{y^2 + yz + z^2} + \frac{z^3 + x^3}{z^2 + zx + x^2} \geq 2.\]
\end{section-problem}

\begin{section-problem}
    Sea $a_0 = a_1 = 1$ y
    \[a_{n + 1} = 1 + \frac{a^2_1}{a_0} + \frac{a^2_2}{a_1} + \cdots + \frac{a^2_n}{a_{n - 1}}, n \geq 1.\]
    Hallar $a_n$ en función de $n$.
\end{section-problem}

\begin{section-problem}
    Sea $P(x)$ un polinomio no nulo tal que $(x - 1)P(x + 1) = (x + 2)P(x)$ para todo real $x$, y $P(2)^2 = P(3)$.
    Hallar $m + n$, si $P\left(\dfrac{7}{2}\right) = \dfrac{m}{n}$ donde $m$ y $n$ son primos relativos.
\end{section-problem}

\begin{section-problem}
    En un tablero cuadriculado $n \times n$ se escriben números dentro de cada casilla mediante el siguiente proceso:
    \begin{itemize}
        \item Se seleccionan números reales $a_1, a_2, \cdots, a_n$, $b_1, b_2, \cdots, b_n$ todos distintos entre sí.
        \item En la casilla de la fila $i$ columna $j$ se escribe el número $a_i + b_j$.
    \end{itemize}
    Suponiendo que los $n$ productos de los números en cada fila del tablero son iguales entre sí, demostrar que los $n$ productos de los números en cada columna también son iguales entre sí.
\end{section-problem}