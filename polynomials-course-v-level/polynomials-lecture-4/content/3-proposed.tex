\section{Problemas propuestos}
{
    Recordar que los problemas de esta sección son los asignados como \textbf{tarea}.
    Es el deber del estudiante resolverlos y entregarlos de manera clara y ordenada el próximo encuentro
     (de ser necesario, también se pueden entregar borradores).

    \begin{section-problem}
        Si $p$ y $q$ son las raíces del polinomio $P(x) = 4x^2 + 5x + 3$.
        Determina $(p + 7)(q + 7)$, sin calcular los valores de $p$ y $q$.
    \end{section-problem}

    \begin{section-problem}
        Supóngase que el polinomio $5x^3 + 4x^2 - 8x + 6$ tiene tres raíces reales $a, b \mbox{ y } c$.
        Demostra que \[(5a)^2\left(\frac{b}{2}\right)\left(\frac{1}{a} + \frac{1}{b}\right) + (5b)^2\left(\frac{c}{2}\right)\left(\frac{1}{b} + \frac{1}{c}\right)+ (5c)^2\left(\frac{a}{2}\right)\left(\frac{1}{c} + \frac{1}{a}\right) = 3^3 + 1.\]
    \end{section-problem}

    \begin{section-problem}
        Sean $r_1, r_2, r_3$ las raíces del polinomio $P(x) = x^3 - x^2 + x - 2$.
        Determina el valor de $r^3_1 + r^3_2 + r^3_3$, sin calcular los valores de $r_1, r_2, r_3$.
    \end{section-problem}
}
\label{sec:problemas-propuestos}