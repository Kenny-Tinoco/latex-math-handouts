\section{Desarrollo}
{
    \begin{section-definition}[\textbf{Fórmulas de Vieta}]
        Sea $P(x) = a_n x^n + a_{n - 1} x^{n - 1} + \cdots  + a_1 x + a_0$ un polinomio con $n$ raíces $r_1, r_2, \cdots, r_n$,
        el cual, por el \textit{Teorema del Factor}\footnote{Ver \cite{TD23-clase2} página 1.}, podemos escribir como
        \[a_n x^n + a_{n - 1} x^{n - 1} + \cdots  + a_1 x + a_0 = a_n (x - r_1)(x - r_2) \cdots (x - r_n)\]
        Cuando expandimos los $n$ factores lineales del lado derecho de la ecuación, obtenemos
        \[a_n x^n - a_n(r_1 + r_2 + \cdots + r_n)x^{n - 1} + a_n(r_1 r_2 + r_1 r_3 + \cdots + r_{n - 1} r_n)x^{n - 2} + \cdots + (-1)^n a_n r_1 r_2 \cdots r_n\]
        donde el signo del coeficiente $x^k$ está dado por $(-1)^{n - k}$. Al comparar los coeficientes podemos ver que
        \begin{gather*}
            r_1 + r_2 + \cdots + r_n = - \frac{a_{n - 1}}{a_n}\\
            r_1 r_2 + r_1 r_3 + \cdots + r_{n - 1} r_n = \frac{a_{n - 2}}{a_n}\\
            \cdots\\
            r_1 r_2 \cdots r_n = (-1)^n \frac{a_0}{a_n}
        \end{gather*}
        A estas ecuaciones entre las raíces de un polinomio y sus coeficientes son llamadas las fórmulas de Vieta y puede
        ayudar a calcular varias expresiones que involucran las raíces de un polinomio sin tener que calcular las propias raíces.
    \end{section-definition}
    \newpage
    \begin{example}
        Si $p$ y $q$ son raíces de la ecuación $x^2 + 2bx + 2c = 0$, determina el valor de $\frac{1}{p^2} + \frac{1}{q^2}.$

        \solution
        {
            Aplicando las fórmulas de Vieta
            \begin{gather*}
                p + q = - \frac{2b}{1} = - 2b \\
                pq = \frac{2c}{1} = 2c
            \end{gather*}

            La expresión que nos piden hallar puede ser transformada de la siguente manera
            \[\frac{1}{p^2} + \frac{1}{q^2} = \frac{p^2 + q^2}{p^2 q^2} = \frac{p^2 + 2pq + q^2 - 2pq}{p^2 q^2} = \frac{ (p + q)^2 - 2pq}{(pq)^2}\]
            Finalmente, al sustituir llegamos a
            \[\frac{ (p + q)^2 - 2pq}{(pq)^2} = \frac{ (-2b)^2 - 2(2c)}{(2c)^2} = \boxed{\frac{b^2 - c}{c^2}}\]
        }
    \end{example}

    \begin{example}
        Encuentre el valor de $\frac{1}{r_1} + \frac{1}{r_2} + \frac{1}{r_3}$ si $r_1, r_2, r_3$ son raíces del polinomio $P(x) = x^3 - 7^2 + 3x + 1$.
        \solution
        {
            Por la fórmulas de Vieta sabemos lo siguiente
            \begin{gather*}
                r_1 r_2 + r_2 r_3 + r_3 r_1 = \frac{3}{1} = 3 \\
                r_1 r_2 r_3 = - \frac{1}{1} = -1
            \end{gather*}
            Ahora, notemos que
            \[\frac{r_1 r_2 + r_2 r_3 + r_3 r_2}{r_1 r_2 r_3} = \frac{r_1 r_2}{r_1 r_2 r_3} + \frac{r_2 r_3}{r_1 r_2 r_3} + \frac{r_3 r_1}{r_1 r_2 r_3} = \frac{1}{r_3} + \frac{1}{r_1} + \frac{1}{r_2}\]
            Luego $\frac{1}{r_3} + \frac{1}{r_1} + \frac{1}{r_2} = \frac{3}{-1} = \boxed{-3}$
        }
    \end{example}

    \begin{example}[1986 URSS]
        Las raíces del polinomio $x^2 + ax + b + 1$ son números naturales. Mostrar que $a^2 + b^2$ no es un primo.
        \solution
        {
            Para demostrar que $a^2 + b^2$ no es primo, entonces tendremos que demostrar que es igual a producto de dos enteros mayores a 1.
            Digamos entonces que $r_1$ y $r_2$ son las raíces, por las Fórmulas de Vieta
            \begin{gather*}
                r_1 + r_2 = -a\\
                r_1 r_2 = b + 1
            \end{gather*}
            Que al elevar al cuadrado y sumar tenemos $a^2 + b^2 = (r_1 + r_2)^2 + (r_1 r_2 - 1)^2$. Así, desarrollando
            \begin{gather*}
                a^2 + b^2 = r^2_1 + 2r_1 r_2 + r^2_2 + r^2_1 r^2_2 - 2r_1 r_2 + 1 \\
                a^2 + b^2 = r^2_1 + r^2_2 + r^2_1 r^2_2+ 1 = \boxed{(r^2_1 + 1)(r^2_2 + 1)}
            \end{gather*}
            Por dato del problemas sabemos que $r_1$ y $r_2$ son naturales así $(r^2_1 + 1)$ y $(r^2_2 + 1)$ son mayores a 1.
            Luego la prueba está hecha.
        }
    \end{example}

    \begin{example}[2008 AIME II #7]
        Sean $r, s$ y $t$ las tres raíces de la ecuación
        \[8x^3 + 1001x + 2008 = 0.\]
        Encontrar $(r + s)^3 + (s + t)^3 + (t + r)^3.$

        \solution
        {
            Por las fórmulas de Vieta, sabemos lo siguiente
            \begin{gather*}
                r +  s + t = - \frac{0}{8} = 0 \\
                rs +  st + tr = \frac{1001}{8} \\
                r s t = - \frac{2008}{8} = -251
            \end{gather*}
            Por la primera ecuación, podemos ver inmediantamente que $r + s = -t$ y por lo tanto $(r + s)^3 = - t^3$.
            Análogamente, llegaremos a
            \[(r + s)^3 + (s + t)^3 + (t + r)^3 =  -(r^3 + s^3 + t^3)\]
            El mejor método para resolver este problema es expresar $r^3 + s^3 + t^3$ en termino de $r +  s + t$, $rs +  st + tr$ y $r s t$.
            Para lo cual podemos análizar las relaciones entre el cúbo de un trinomio y la suma de tres cubos.
            De ellas\footnote{Se deja como ejercicio al lector la búsqueda de otras identidades similares.} tenemos la siguiente
            \[(x + y + z)^3 = x^3 + y^3 + z^3 + 3(x + y)(y + z)(z + x)\]
            De donde rápidamente vemos que
            \[-(r^3 + s^3 + t^3) = - (r + s + t)^3 + 3(r + s)(s + t)(t + r)\]
            \[-(r^3 + s^3 + t^3) = -(0)^3 + 3(-t)(-r)(-s)\]
            \[-(r^3 + s^3 + t^3) = - 3rst = -3(-251) = \boxed{753}\]
        }
    \end{example}

}\label{sec:desarrollo}

\vspace{-10mm}

\section{Ejercicios y Problemas}
{
    Sección de ejercicios y problemas para el autoestudio.

    \begin{exercise}
        Sean $r_1$ y $r_2$ las raíces del polinomio $P(x) = ax^2 + bx + c$. Encontrar los siguientes valores en función del los coeficientes de $P.$
        \begin{multicols}{3}
            \begin{enumerate}
                \item $r_1 - r_2$
                \item $\frac{1}{r_1} - \frac{1}{r_2}$
                \item $r^2_1 + r^2_2$
                \item $r^3_1 + r^3_2$
                \item $(r_1 + 1)^2 + (r_2 + 1)^2$
            \end{enumerate}
        \end{multicols}
    \end{exercise}

    \begin{section-problem}
        Sean $a$ y $b$ las raíces de la ecuación $x^2 - 6x + 5 = 0$, encuentra $(a + 1)(b + 1).$
    \end{section-problem}

    \begin{section-problem}
        Dado que $m$ y $n$ son raíces del polinomio $6x^2 - 5x - 3$, encuentra un polinomio cuyas raíces sean
        $m - n^2$ y $n - m^2$, sin calcular los valores de $m$ y $n$.
    \end{section-problem}

    \begin{section-problem}
        ¿Para qué valores reales positivos de $m$, las raíces $x_1$ y $x_2$ de la ecuación
        \[x^2 - \left( \frac{2m - 1}{2} \right)x  + \frac{m^2 - 3}{2} = 0\]
        cumplen que $x_1 = x_2 - \frac{1}{2}$?
    \end{section-problem}

}\label{sec:ejercicios-y-problemas}