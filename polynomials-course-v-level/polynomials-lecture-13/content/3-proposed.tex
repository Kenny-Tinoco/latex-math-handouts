\section{Problemas propuestos}

\begin{section-problem}
    Sea $Q(x) = 2x - 4096$ y $P(x) = Q^{12}(x)$, hallar la raíz de $P$.
\end{section-problem}

\begin{section-problem}
    Si $a$ y $b$ son raíces distintas del polinomio $x^2 + 1012x + 1011$, entonces
    \[\frac{1}{a^2 + 1011a + 1011} + \frac{1}{b^2 + 1011b + 1011} = \frac{m}{n},\]
    donde $m$ y $n$ son primos relativos.
    Calcular $m + n$.
\end{section-problem}

\begin{section-problem}
    Dado el polinomio $P(x)$ para el cual se cumple que
    \[x^{23} + 23x^{17} - 18x^{16} - 24x^{15} + 108x^{14} = (x^4 - 3x^2 - 2x + 9)P(x)\]
    Calcular la suma de coeficientes de $P$.
\end{section-problem}

\begin{section-problem}
    Sea $r_1$, $r_2$ y $r_3$ las tres raíces de polinomio cúbico $P$.
    También, que
    \[\frac{P(2) + P(-2)}{P(0)} = 52\]
    La expresión $\dfrac{1}{r_1 r_2} + \dfrac{1}{r_2 r_3} + \dfrac{1}{r_3 r_1}$ puede ser escrita como $\dfrac{m}{n}$ para $m$ y $n$ coprimos.
    Encontrar $m\times n$.
\end{section-problem}

\begin{section-problem}
    Dado que
    \begin{gather*}
        Q(x) = 2x + 3 \\
        Q( F(x) + G(x) ) = 4x + 3 \\
        Q( F(x) \times G(x) ) = 5
    \end{gather*}
    Calcular $F(G(F(G(\dots F(G(1))\dots))))$.
\end{section-problem}

\begin{section-problem}
    Dado que $m$ y $n$ son raíces del polinomio $6x^2 - 5x - 3$, encuentra un polinomio cuyas raíces sean
    $m - n^2$ y $n - m^2$, sin calcular los valores de $m$ y $n$.
\end{section-problem}

\begin{section-problem}
    Si $P(x - 2) = x^3 - 10x^2 + 28x - 24$, hallar el resto de dividir $P(x)$ por $x - 3$
\end{section-problem}

\begin{section-problem}
    Encontrar el resto cuando $(x + 3)^5 + (x + 2)^8 + (5x + 9)^{1997}$ es dividido por $x + 2$.
\end{section-problem}

\begin{section-problem}
    Encontrar el resto cuando $x^{2006} + x^{2005} + \cdots + x + 1$ es dividido por $x + 1$.
\end{section-problem}

\begin{section-problem}
    Sea el polinomio $P_0(x) = x^3 + 313x^2 - 77x - 8$.
    Para enteros $n \geq 0$, definimos $P_n(x) = P_{n - 1}(x - n)$.
    ¿Cuál es el coeficiente de $x$ en $P_{20}(x)$?
\end{section-problem}

\begin{section-problem}
    Sean $a$, $b$ y $c$ números reales distintos de cero, con $a + b + c \neq 0$.
    Probar que si
    \[\frac{1}{a} + \frac{1}{b} + \frac{1}{c} = \frac{1}{a + b + c},\]
    entonces para $n$ impar se cumple
    \[\frac{1}{a^n} + \frac{1}{b^n} + \frac{1}{c^n} = \frac{1}{a^n + b^n + c^n}.\]
\end{section-problem}

\begin{section-problem}
    Sean $x$, $y$ y $z$ números reales, encontrar todas las soluciones del siguiente sistema de ecuaciones
    \[
        \left\{
        \begin{array}{rcl}
            x + y + z & =& 6\\
            x^2 + y^2 + z^2 & =& 30\\
            x^3 + y^3 + z^3 & =& 132
        \end{array}
        \right.
    \]
\end{section-problem}

\begin{section-problem}
    Demostrar $\forall n \in \N$, que
    \[8 \mid 3^{2n + 2} + 8n - 9\]
\end{section-problem}


\begin{section-problem}
    Con la ayuda del teorema de la raíz racional, encontrar todas las raíces de los siguientes polinomios
    \[2 x^3 - 21 x^2 + 52 x - 21\]
\end{section-problem}

\begin{section-problem}
    Si la división
    \[\frac{x^{80} - 7 x^{30} + 9x^5 - mx + 1}{x^3 + x - 2}\]
    Deja como resto a $R(x) = x^2 + x - 1$, hallar el valor de $m$.
\end{section-problem}

\begin{section-problem}
    Hallar $Q(x)$, si $P\left[Q(x) - 3\right] = 6x + 2$ y $P(x + 3) = 2x + 10$.
\end{section-problem}

\begin{section-problem}
    Indique el valor de la expresión
    \[E(3)  \cdot E(5) \cdot E(7) \cdots E(2021) \cdot E(2023)\]
    Si $E(x) = 1 + \dfrac{2}{x - 1}$
\end{section-problem}

\begin{section-problem}
    Hallar el resto de la división de $\left[(x - 1)(x)(x + 2)(x + 3)\right]^2 + (x^2 + 2x)^3 x - 50$ entre $x^2 + 2x - 5$.
\end{section-problem}

\begin{section-problem}
    Si $P\left(x + \dfrac{1}{x}\right) = x^2 + \dfrac{1}{x^2} + 2023$, ¿cuál es el valor de $P(2024)$?
\end{section-problem}

\begin{section-problem}
    Dado el polinomio $S(x) = (11 - 15x^3)(17x^6 - 43) + 2^8 x^6 (1 - x + x^2)(1 + x)$, responda lo siguiente:
    \begin{multicols}{2}
        \begin{enumerate}
            \item ¿$S(x)$ es mónico? \\R: \rule{1cm}{0.1mm}
            \item ¿$S(x)$ es completo? \\R: \rule{1cm}{0.1mm}
            \item ¿$S(x)$ es simétrico? \\R: \rule{1cm}{0.1mm}
            \item Escriba el coeficiente de $x^6$. \\R: \rule{1cm}{0.1mm}
            \item Escriba el término independiente.\\ R: \rule{1cm}{0.1mm}
            \item ¿Es $S(\sqrt[3]{x})$ un polimonio? \\ R: \rule{2cm}{0.1mm}
        \end{enumerate}
    \end{multicols}
\end{section-problem}

\begin{section-problem}
    Si tenemos que
    \begin{align*}
        P(x) = \frac{x - 1}{3} \\
        Q(x) = 3x^2 - 2x \\
        R(x) = (Q \circ P)(x) - 673x
    \end{align*}
    ¿Cuál es el valor de $R(2023)$?
    \begin{multicols}{5}
        \begin{enumerate}
            \item -4
            \item 2023
            \item 2729701
            \item 0
            \item 1
        \end{enumerate}
    \end{multicols}
\end{section-problem}

\begin{section-problem}
    Sea $R(y) = 3x^2 - 12x - 36 + 3y^2 + 6xy - 12y$, factorize $R$ y responda.
    ¿Cuáles son las raíces de $R$?
    \begin{multicols}{3}
        \begin{enumerate}
            \item $6$ y $-2$
            \item $6 - y$ y $-2 - y$
            \item $-6 + y$ y $-2 + y$
            \item $6 - x$ y $-2 - x$
            \item $-6 - y$ y $2 + x$
        \end{enumerate}
    \end{multicols}
\end{section-problem}