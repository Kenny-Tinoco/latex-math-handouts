\section{Problemas propuestos}

Los siguientes problemas propuestos deben ser resueltos y entregados por los estudiantes al final del encuentro en forma de trabajo.
Si el tiempo de la sesión se vuelve insuficiente, los estudiantes deberán mostrar un avanze de $x$ problemas\footnote{Lo decidirá el profesor en la sesión.} para poder entregar su trabajo el próximo encuentro.

\begin{section-problem}
    Para que la división de $6x^4 - 11x^2 + ax + b$ entre $3x^2 - 3x - 1$ sea exacta, encuentre los valores de $a$ y $b$ apropiados.
\end{section-problem}

\begin{section-problem}
    Calcular la suma de coeficientes del resto que deja $x^{3333} - 9$ entre $x^2 - 729$.
\end{section-problem}

\begin{section-problem}
    Sea $r$ una raíz de $x^2 - x + 7$.
    Hallar el valor de $r^3 + 6r + \pi$.
\end{section-problem}

\begin{section-problem}
    Sean $a$, $b$ y $c$ las raíces reales de la ecuación $x^3 + 3x^2 - 24x + 1 = 0$.
    Probar que $\sqrt[3]{a} + \sqrt[3]{b} + \sqrt[3]{c} = 0$.
\end{section-problem}

\begin{section-problem}
    Sean $r_1$, $r_2$ y $r_3$ raíces distintas del polinomio $y^3 - 22 y^2 + 80 y - 67$.
    De tal manera que existen números reales $\alpha$, $\beta$ y $\theta$ tal que
    \[\frac{1}{y^3 - 22 y^2 + 80 y - 67} = \frac{\alpha}{y - r_1} + \frac{\beta}{y - r_2} + \frac{\theta}{y - r_3}\]
    $\forall y \notin \left\{ r_1, r_2, r_3 \right\}$.
    ¿Cuál es valor de $\dfrac{1}{\alpha} + \dfrac{1}{\beta} + \dfrac{1}{\theta}$?
\end{section-problem}

\begin{section-problem}
    La ecuación $2^{333x - 2} + 2^{111x + 2} = 2^{222x + 1} + 1$ tiene tres raíces reales.
    Dado que su suma es $\dfrac{m}{n}$ con $m, n \in \ZP$ y $mcd(m, n) = 1$.
    Calcular $m + n$.
\end{section-problem}

\begin{section-problem}
    Si $P(x) = x^4 + ax^3 + bx^2 + cx + d$ es un polinomio tal que $P(1) = 10$, $P(2) = 20$ y $P(3) = 30$, determine el valor de
    \[\frac{P(12) + P(-8)}{10}.\]
\end{section-problem}

\begin{section-problem}
    Sea $F(x)$ un polinomio mónico con coeficientes enteros.
    Probar que si existen cuatro enteros diferentes $a$, $b$, $c$ y $d$ tal que $F(a) = F(b) = F(c)  = F(d) = 5$,
    entonces no existe un entero $k$ tal que $F(k) = 8$.
\end{section-problem}

\begin{section-problem}
    Sea el polinomio $P_0(x) = x^3 + 313x^2 - 77x - 8$.
    Para enteros $n \geq 0$, definimos $P_n(x) = P_{n - 1}(x - n)$.
    ¿Cuál es el coeficiente de $x$ en $P_{20}(x)$?
\end{section-problem}