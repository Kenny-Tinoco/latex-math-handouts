\section{Desarrollo}

\begin{theorem}
    Si $P(x)$ es un polinomio con coeficientes enteros, entonces $P(a) - P(b)$ es divisible entre $(a - b)$, para cualesquiera enteros distintos $a$ y $b$.
\end{theorem}

En particular, todas las raíces enteras de $P(x)$ dividen a $P(0)$.
Esto nos conduce a la siguente propiedad aritmética.

\begin{section-definition}
    Sea $P(x) = a_n x^n + \cdots + a_0$ en los enteros y sea $z \in \Z$.
    Entonces \[P(z) = 0 \Leftrightarrow z \mid a_0.\]
\end{section-definition}

En efecto, $a_n z^n + \cdots + a_1 z + a_0 = 0 \Leftrightarrow a_0 = -z(a_n z^{n - 1} + \cdots + a_1)$.
Además, si $a_n = 1$, entonces cada raíz racional de $P$ es un entero. En efecto, sea $\frac{p}{q}$ una raíz con $p, q \in \Z$ y $mcd(p, q) = 1$.
Entonces
\begin{gather*}
    \dfrac{p^n}{q^n} + a_{n - 1} \dfrac{p^{n - 1}}{q^{n - 1}} + \cdots + a_1 \dfrac{p}{q} + a_0  = 0 \\
    \hookrightarrow \dfrac{p^n}{q} = - a_{n - 1} p^{n - 1} - a_{n - 2} p^{n - 2} q - \cdots - a_1 p q^{n - 2} - a_0 q^{n - 1}
\end{gather*}

El lado derecho de la ecuación es entera, por lo tanto $q = 1$.

\begin{theorem}[\textbf{Teorema de la raíz racional}]
    Sea $P(x) = a_n x^n + \cdots + a_0$ en los enteros y sea $\frac{p}{q}$ con $mcd(p, q) = 1$ una raíz cualquiera de $P$.
    Entonces se cumple que $p \mid a_0$ y $q \mid a_n$.
\end{theorem}

\textbf{Demostración.} En efecto, tenemos que
\begin{gather*}
    a_n \dfrac{p^n}{q^n} + a_{n - 1} \dfrac{p^{n - 1}}{q^{n - 1}} + \cdots + a_1 \dfrac{p}{q} + a_0  = 0 \\
    \hookrightarrow a_n p^n + a_{n - 1}p^{n - 1}q + \cdots + a_1 p q^{n - 1} + a_0 q^n = 0
\end{gather*}

Todos los sumandos excepto, posiblemente, el primero, son múltiplos de $q$ y todos los sumandos excepto, posiblemente, el último son múltiplos de $p$.
Como $p$ y $q$ dividen a 0, se deberá tener que $q \mid a_n p^n$ y $p \mid a_0 q^n$, y de aquí se sigue la afirmación, ya que\footnote{$(p, q) =  1$ es una manera corta de escribir $mcd(p, q) = 1$.} $(p, q) = 1$.

\subsection{Agregados culturales y preguntas}
{

}

\section{Ejercicios y Problemas}
{
    Sección de ejercicios y problemas para el autoestudio.

    \begin{section-problem}
        Encontrar todas la raíces racionales del polinomio $x^4 - 4x^3 + 6x^2 - 4x + 1$.
    \end{section-problem}

    \begin{section-problem}
        Encontrar todas las raíces racionales del polinomio $6x^4 + x^3 - 3x^2 - 9x - 4$.
    \end{section-problem}

    \begin{section-problem}
        Si la división entre los polinomios $12x^5 - 9x^4 + 14x^3 - mx^2 + nx - p$
    \end{section-problem}

    \begin{section-problem}
        Si el polinomio $P(x) = 16x^5 + ax^2 + bx + c$ es divisible por $2x^3 - x^2 + 1$, hallar $a + b + c$.
    \end{section-problem}

    \begin{section-problem}
        Si el polinomio $P(x) = x^{n + 2} + Ax^{n + 1} + ABx^n$ es divisible por $C(x) = x^2 - (A + B)x + AB$ con $AB \neq 0$.
        Hallar el valor de $E = \frac{A}{B}$.
    \end{section-problem}
}