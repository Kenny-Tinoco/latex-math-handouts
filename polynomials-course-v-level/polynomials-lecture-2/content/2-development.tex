\section{Desarrollo}\label{sec:desarrollo}

\subsection{Definiciones}
{
    \begin{section-definition}[\textbf{Raíz de un Polinomio}]
        La raíz de un polinomio $P(x)$ es un número $r$, tal que $P(r) = 0$. También, diremos que $r$ es una solución de la ecuación $P(x) = 0$.
    \end{section-definition}

    \begin{example}
        Demuestre que $u$ es raíz del polinomio $R(x) = x^2 - (u + 17) x + 17u$.
        \solution
        {
            Para demostrar que $u$ es raíz\footnote{¿Podés encontrar otra raíz de $R(x)$?} de $R(x)$, basta probar que $R(u) = 0$. Lo cual es fácil ver cuando evaluamos $R(u) = u^2 - (u+17)u + 17u = u^2 - u^2 - 17u + 17u = 0.$
        }
    \end{example}


    \begin{section-definition}[\textbf{Factor de un Polinomio}]
        Sea $P$ un polinomio con $\deg{(P)} = n$ y $a \in \R$. Entonces, $(x - a)$ es un \emph{factor} de $P(x)$ si existe un polinomio\footnote{¿Por qué $\deg{(Q)} = (n-1)$?} $Q(x)$ tal que \[P(x) = (x-a)Q(x).\]
    \end{section-definition}

    \begin{theorem}[\textbf{Teorema del factor}]\label{factor-theorem}
        Dado un polinomio $P$, de grado $n$ y $a \in \R$, diremos que $a$ es una raíz de $P$ si y sólo si $(x-a)$ es un factor de $P(x)$. Es decir \[P(a) = 0 \Leftrightarrow P(x) = (x-a)Q(x)\] para algún polinomio $Q(x).$
    \end{theorem}

    Si $a_1, a_2 \mbox{ y } a_3$ son tres raíces distintas del polinomio cúbico $P(x)$, por el ~\textbf{Teorema \ref{factor-theorem}},
    \[P(x) = (x - a_1)Q(x)\]
    Para algún $Q(x)$, pero como $P(a_2) = (a_2 - a_1)Q(a_2) = 0$ y $a_2 \neq a_1$, entonces $Q(a_2) = 0$, es decir $a_2$ es raíz de $Q$, por lo tanto por el \textbf{Teorema \ref{factor-theorem}}
    \[Q(x) = (x - a_2)R(x)\]
    Para algún $R(x)$. Análogamente, tendremos que $R(x) = (x - a_3)S(x)$, para algún $S$ constante. Así,
    \[P(x) = c(x - a_1)(x - a_2)(x - a_3) \mbox{, con } c \in \R.\]
    Vemos que saber las raíces de $P$ nos condujo a su factorización\footnote{Tema que se introduce en la sección~\ref{factorization}.}. Y en general, para un polinomio $P(x)$ de grado $n$ y raíces $r_i \mbox{ con } 1 \leq i \leq n$, este puede ser expresado como:
    \[P(x) = c(x - r_1)(x - r_2)\cdots(x - r_{n-1})(x - r_n), \mbox{ con } c \in \R.\]

    \textbf{Cantidad de raíces de un polinomio:} Un polinomio de grado $n$ tiene como máximo $n$ raíces (o ceros). Así, por ejemplo, un polinomio $P$ con $\deg{(P)} = 7$, tiene a lo más 7 raíces.

    \textbf{Multiplicidad de raíces:} Si existe $m \in \N$ y un polinomio $Q(x)$ tal que \[P(x) = (x - a)^m Q(x)\] diremos que la raíz $a$ tiene multiplicidad $m$. Si $m = 1$ diremos que la raíz $a$ es simple.

    \begin{example}
        Sea $P(x)$ un polinomio con coeficientes enteros y suponga que $P(1)$ y $P(2)$ son ambos impares. Demuestre que no existe ningún entero $n$ para el cual $P(n) = 0$.
        \solution
        {
            Nos piden mostrar que $P(x)$ no tiene raíces enteras, entonces supongamos por el contrario, que existe un entero $n$ tal que $P(n) = 0$. Entonces, por el \textbf{Teorema \ref{factor-theorem}} $P(x) = (x - n)Q(x)$, con $Q(x)$ un polinomio con coeficientes enteros. Así podemos ver que, $P(1) = (1 - n)Q(1)$ y $P(2) = (2 - n)Q(2)$ son impares, pero $(1 - n)$ y $(2 - n)$ son enteros consecutivos, así que uno de ellos debe ser par. Por lo tanto, $P(1)$ o bien $P(2)$ tiene que ser par, lo cual contradice las condiciones del problema. Luego, $n$ no existe.
        }
    \end{example}

    \begin{example}
        Sea $M(x)$ un polinomio cúbico con coeficientes enteros y sean $a, b, c \in \Z$, con $a \neq b \neq c$ tal que $M(a) = M(b) = M(c) = 2$. Demostrar que no existe un $d \in \Z$ para el que $M(d) = 3.$
        \solution
        {
            Sea $N(x) = M(x) - 2$, como $a, b \mbox{ y } c$ son raíces de $N(x)$, es claro que $N(x) = \alpha (x - a)(x - b)(x - c)$, para algún entero $\alpha$. Si para algún entero $d$ se tiene que $M(d) = 3$, entonces $N(d) = \alpha (d - a)(d - b)(d - c) = 1$. Para que esto suceda los factores deben ser 1 o $-1$ y por lo tanto dos de ellos tendrían que ser iguales. Pero por la condición $a \neq b \neq c$ esto no puede ser, luego $d$ no existe.
        }
    \end{example}
}
        Sean $a, b, c \in R$ tal que los polinomios $ax^2 + bx + c$ y $cx^2 + bx + a$
        tienen dos raíces reales distintas cada uno ($r_1$, $r_2$) y ($r_3$, $r_4$) respectivamente.
        Se sabe que los números $r_1, r_2, r_3, r_4$ forman, en ese orden, una progresión aritmética.
        Demostar que $a + c = 0.$
\label{subsec:definiciones}

\subsection{Métodos para determinar raíces de polinomios}
{
    En este apartado nos centraremos en los métodos para la determinación de raíces de polinomios, particularmente para polinomios cuadráticos y cúbicos.
    Para determinar el valor de las raíces de polinomios se pueden utilizar diversos métodos, como por ejemplo; la factorización, las fórmulas de Cardano, la completación de cuadrados, fórmulas cuadráticas y cúbicas general, soluciones trigonométricas e hiperbólicas con valores auxiliares, métodos númericos, entre otros.
    El presente escrito, solo abordará algunos de estos métodos y se invita al lector complementar su aprendizaje con la búsqueda e investigación de otros métodos.

    \subsubsection{Factorización}
    {
        Si un polinomio $P(x)$ es equivalente al producto de otros polinomios con grado menor, entonces diremos que $P(x)$ está factorizado. Por ejemplo, el polinomio $M(x) = 5x^3 + 4x^2 + 5x + 4$, es equivalente a $(5x+4)(x^2 + 1)$, así diremos que $M(x)$ está factorizado y sus factores son $(5x+4) \mbox{ y } (x^2 + 1)$.

        \begin{section-definition}
            Dado un polinomio cuadrático $P(x) = ax^2 + bx + c$ con $a, b, c \in \R$, este puede ser factorizado como
            \begin{gather*}
                P(x) = \frac{(ax+m)(ax+n)}{a}
                \mbox{, donde  } \left\{{m + n = b \atop mn = ac} \right
            \end{gather*}
        \end{section-definition}
    }\label{factorization}

    \subsubsection{Completación de cuadrados}
    {
        No todos los polinomios cuadráticos pueden ser factorizados fácilmente.
        Por ejemplo, al tratar de factorizar $x^2 + 6x - 1$ llegar directamente a $(x + 3 - \sqrt {10})(x-3-\sqrt {10})$ no resulta tan evidente, por lo cual podemos auxiliarnos en técnicas como la \textbf{completación de cuadrados}.
        Si se tiene el polinomio\footnote{¿Cómo sería la fórmula si $P(x) = ax^2 + bx$?} $P(x) = x^2 + bx$ entonces podemos expresarlo de la forma:
        \[P(x) = \left( x + \frac{b}{2} \right)^2 - \left( \frac{b}{2} \right)^2\]
        Lo cual facilita aplicar la \textbf{diferencia de cuadrados}.

        \begin{example}
            Hallar las raíces del polinomio $R(r) = r^2 - 10r + 7.$
            \solution
            {
                Utilizando la completación de cuadrados, tenemos que
                \begin{gather*}
                    R(r) = r^2 - 10r + 7\\
                    = \left( r - \frac{10}{2} \right)^2 - \left( \frac{10}{2} \right)^2 + 7\\
                    = \left( r - 5 \right)^2 - 18
                    = \left( r - 5 + \sqrt {18} \right)\left( r - 5 - \sqrt {18} \right)\\
                    = \left[ r - \left( 5 - 3\sqrt {2} \right)\right]\left[ r - \left( 5 + 3\sqrt {2} \right)\right]
                \end{gather*}
                De esta manera sabemos que $R(r)$ tiene como raíces a $\left( 5 - 3\sqrt {2} \right) \mbox{ y } \left( 5 + 3\sqrt {2} \right).$
            }
        \end{example}
    }
    \subsubsection{Fórmula general}
    {
        Cuando tenemos un polinomio cuadrático $P(x) = ax^2 + bx + c$ con $a \neq 0$, podemos encontrar los valores para sus dos raíces en función de los coeficientes, a esta fórmula le conoceremos como fórmula general \[x_1 = \frac{-b - \sqrt {b^2 - 4ac}}{2a} \land x_2 = \frac{-b + \sqrt {b^2 - 4ac}}{2a}.\]
        \textbf{Demostración:} Al completar cuadrado en $P(x)$ tenemos que
        \begin{gather*}
            P(x) = ax^2 + bx + c = a \left( x^2 + \frac{b}{a}x + \frac{c}{a} \right) \\
            = a \left[ \left( x + \frac{b}{2a} \right)^2 - \left( \frac{b}{2a} \right)^2 + \frac{c}{a} \right]
            = a \left[ \left( x + \frac{b}{2a} \right)^2 - \left( \frac{b^2 - 4ac}{4a^2}\right) \right] \\
            = a \left[ x + \frac{b}{2a} - \frac{\sqrt {b^2 - 4ac}}{2a}\right]\left[ x + \frac{b}{2a} + \frac{\sqrt {b^2 - 4ac}}{2a}\right] \\
            = a \left[ x - \left( \frac{-b + \sqrt {b^2 - 4ac}}{2a} \right) \right]\left[ x - \left( \frac{-b - \sqrt {b^2 - 4ac}}{2a} \right) \right]
        \end{gather*}
        Así, $\frac{-b - \sqrt {b^2 - 4ac}}{2a}$ y $\frac{-b + \sqrt {b^2 - 4ac}}{2a}$ son la raíces del polinomio.
    }

    \subsubsection{Análisis del discriminante}
    {
        Sea el polinomio cuadrático $P(x) = ax^2 + bx + c$ y sea $\Delta = b^2 - 4ac$.
        Diremos que $\Delta$ es el \textbf{discriminante} de $P$ y que dependiendo de su signo se cumplirán los siguientes hechos:
        \begin{itemize}
            \item Si $\Delta > 0$, entonces $P$ tiene dos raíces reales distintas, las cuales son:
            \[x_1 = \frac{-b - \sqrt{\Delta}}{2a} \land x_2 = \frac{-b + \sqrt {\Delta}}{2a}.\]
            \item Si $\Delta = 0$, entonces $P$ tiene una raíz real de multiplicidad 2, la cual es $x = -\frac{b}{2a}.$
            \item Si $\Delta < 0$, entonces $P$ no tiene raíces reales, sino raíces complejas conjugadas\footnote{Vale aclarar que el presente curso no entrará de lleno con las raíces complejas.\\Aunque sí veremos algunos ejercicios y problemas bonitos.}.
        \end{itemize}

        Nos piden hallar las raíces del polinomio $P(x) = ax^3 + bx^2 + cx + d$. Para ello podemos plantear la ecuación
        \[ ax^3 + bx^2 + cx + d = 0, \mbox{ con } a \neq 0. \]
        Al hacer la sustitución $y = x + \frac{b}{3a}$ nos da como resultado la ecuación de la forma $y^3 + py + q$, que desde ahora llamaremos la ecuación \textbf{Cúbica reducida}.
        La cual nos ayuda a obtener la expresión $\Delta = \frac{q^2}{4} + \frac{p^3}{27}$ que llamaremos \textbf{discriminante} y que dependiendo de su signo se cumplirán los siguientes hechos:
        \begin{itemize}
            \item Si $\Delta > 0$, entonces $P$ tiene una raíz real y dos raíces complejas.
            \item Si $\Delta = 0$, entonces $P$ tiene una raíz real de multiplicidad tres en el caso de que $p = q = 0$ o bien dos raíces reales (de multiplicidad uno y dos, respectivamente) en el caso de que $p^3 = -q^3 \neq 0.$
            \item Si $\Delta < 0$, entonces $P$ tiene tres raíces reales diferentes.
        \end{itemize}
    }

    \subsubsection{Método de Cardano}
    {
        Si se hace $y = A + B$, elevando al cubo y reacomodano se obtiene: \[y^3 -3ABy - (A^3 + B^3) = 0\]
        Así, al comparar coeficientes homólogos con la ecuación \textbf{cúbica reducida}, se obtiene que $3AB = - p$ y $A^3 + B^3 = -q$, y en base a estas dos ecuaciones podemos formar:
        \[(A^3)^2 + q(A^3) - \frac{p^3}{27} = 0\] la cual es una ecuación cuadrática en $A^3$, que por la \textbf{fórmula general} podemos encontrar sus soluciones. Procediendo análogamente para $B^3$, llegamos a
        \begin{gather*}
            A = \sqrt[3]{\frac{-q - \sqrt{q^2 + \frac{4p^3}{27}}}{2}} = \sqrt[3]{\frac{-q}{4} - \sqrt{\frac{q^2}{4} + \frac{p^3}{27}}}\\
            B = \sqrt[3]{\frac{-q + \sqrt{q^2 + \frac{4p^3}{27}}}{2}} = \sqrt[3]{\frac{-q}{4} + \sqrt{\frac{q^2}{4} + \frac{p^3}{27}}}
        \end{gather*}
        Así obtenemos que $y = \sqrt[3]{\frac{-q}{4} - \sqrt{\Delta}} + \sqrt[3]{\frac{-q}{4} + \sqrt{\Delta}}$. Finalmente, obtuvimos las soluciones de la ecuación \textbf{cúbica reducida} y por lo tanto las soluciones de la ecuación cúbica general.
        A este resultado le conoceremos como la fórmula o Método de Cardano para un polinomio cúbico.
    }

    \begin{example}
        ¿Para qué valores de $\delta$ el polinomio $\delta x^2 + 2x + 1 - \frac{1}{\delta}$ tiene sus dos raíces iguales?
        \solution
        {
            El polinomio tiene raíces iguales si el discriminante es nulo. Es decir , $\Delta = 4 - 4\delta (1 - \frac{1}{\delta})$ de donde es fácil ver que $\delta (1 - \frac{1}{\delta}) = 1.$ Luego, $\delta = 2$ es la única posibilidad.
        }
    \end{example}
}
\label{subsec:determinar-raices}

\subsection{Agregados culturales y preguntas}
{
    \begin{enumerate}
        \item Los números reales son un subconjunto de los números complejos. $(\R \subset \mathbb{C}).$
        \item El Teorema Fundamental del Álgebra dice que cualquier polinomio de grado mayor a cero con coeficientes complejos tiene al menos una raíz compleja\footnote{Este teorema lo veremos más adelante.}.
        \item \textbf{Pregunta:} ¿Cuántas raíces reales tiene el polinomio $P(x) = x^2+1$?
    \end{enumerate}
}\label{subsec:agregados-culturales}

\section{Ejercicios y Problemas}
{
    \begin{exercise}
        Determina las raíces los siguientes polinomios con el método que más te guste
        \begin{multicols}{3}
            \begin{enumerate}
                \item $x^2 + x - 20$
                \item $9t^2 + 88t - 20$
                \item $x^3 - 6x + 9$
                \item $x^3 - 1331$
                \item $-21x^2 - 11x + 2$
                \item $(c + d)^2 - 18(c + d) + 65$
                \item $r^4 - 13r^2 + 36$
                \item $x^3 - 9x^2 - 9x - 15$
                \item $12p^2 - 7p - 12$
            \end{enumerate}
        \end{multicols}
    \end{exercise}

    \begin{section-problem}
        Determine todos los posibles valores que puede tomar $\frac{x}{y}$ si $x, y \neq 0$ y $6x^2 + xy = 15y^2.$
    \end{section-problem}

    \begin{section-problem}
        Hallar $K \in \R$ tal que $x = K^2(x - 1)(x - 2)$ tiene raíces reales.
    \end{section-problem}

    \begin{section-problem}
        Encontrar todas las soluciones de la ecuación $m^2 - 3m + 1 = n^2 + n - 1$, con $m, n \in \ZP$.
    \end{section-problem}

    \begin{section-problem}
        Sean $a, b$ y $c$ números reales positivos. ¿Es posible que cada uno de los polinomios $P(x) = ax^2 + bx + c \mbox{, } Q(x) = bx^2 + cx + a \mbox{ y } R(x) = cx^2 + ax + b$ tenga sus dos raíces reales?
    \end{section-problem}

}\label{sec:ejercicios-y-problemas}