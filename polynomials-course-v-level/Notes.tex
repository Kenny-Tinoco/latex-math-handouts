\documentclass[12pt]{article}

\usepackage{amsmath}
\usepackage{amsthm}
\usepackage{amsfonts}
\usepackage{amssymb}
\usepackage[spanish]{babel}
\usepackage{float}
\usepackage[table,xcdraw]{xcolor}
\usepackage{graphicx}
\usepackage{grffile}
\usepackage{multicol}
\usepackage{booktabs}
\usepackage{subfigure}
\usepackage[left=2.54cm,right=2.54cm,top=2.54cm,bottom=2.54cm]{geometry}
\usepackage{polynom}
\usepackage{tikz,lipsum,lmodern}
\usepackage[most]{tcolorbox}
\usepackage{titlesec}
\usepackage{pst-node,wrapfig}
\usepackage{fancyhdr}

\label{key}

\spanishdecimal{.}
\newcommand{\cvec}[1]{{\mathbf #1}}
\newcommand{\rvec}[1]{\vec{\mathbf #1}}
\newcommand{\ihat}{\hat{\textbf{\i}}}
\newcommand{\jhat}{\hat{\textbf{\j}}}
\newcommand{\khat}{\hat{\textbf{k}}}
\newcommand{\minor}{{\rm minor}}
\newcommand{\trace}{{\rm trace}}
\newcommand{\spn}{{\rm Span}}
\newcommand{\rem}{{\rm rem}}
\newcommand{\ran}{{\rm range}}
\newcommand{\range}{{\rm range}}
\newcommand{\mdiv}{{\rm div}}
\newcommand{\proj}{{\rm proj}}
\newcommand{\R}{\mathbb{R}}
\newcommand{\N}{\mathbb{N}}
\newcommand{\Q}{\mathbb{Q}}
\newcommand{\Z}{\mathbb{Z}}
\newcommand{\C}{\mathbb{C}}
\newcommand{\ZP}{\mathbb{Z^+}}
\newcommand{\ZN}{\mathbb{Z^-}}

\newcommand{\<}{\langle}
\renewcommand{\>}{\rangle}

\newcommand{\attn}[1]{\textbf{#1}}
\renewcommand{\emptyset}{\varnothing}

\newcommand{\modIn}[1]{\mbox{ (mód }#1\mbox{)}}
\newcommand{\fullModIn}[2]{\equiv #1 \mbox{ (mód }#2\mbox{)}}

\newcommand{\MDIF}{\textit{(MDIF)}}
\newcommand{\WellOrderingPrinciple}{\textit{Principio del Buen Orden}}

\newcommand{\ProductPrinciple}{\textit{Principio del producto} }
\newcommand{\AdditionPrinciple}{\textit{Principio de la suma} }
\renewcommand{\theenumi}{\alph{enumi}}

\newcommand{\sen}{\operatorname{sen}}
\newcommand{\arcsen}{\operatorname{arcsen}}
\newcommand{\li}{\displaystyle\lim}
\newcommand*{\QEDA}{\hfill\ensuremath{\blacksquare}}
\newcommand{\QED}{\hfill {\qed}}

\newcommand{\bproof}{\bigskip {\bf Proof. }}
\newcommand{\eproof}{\hfill\qedsymbol}
\newcommand{\Disp}{\displaystyle}
\newcommand{\qe}{\hfill\(\bigtriangledown\)}

\newcommand{\sourceProblem}[1]
{
    \vspace{-4mm}
    \begin{flushright}
        \emph{(#1)}
    \end{flushright}
    \vspace{1mm}
}

\newcommand{\problemImage}[1]
{
    \begin{center}
        \includegraphics[width=5cm]{#1}
    \end{center}
}

\newcommand{\generalFormPolynomail}
{
    P(x) = a_nx^n+a_{n-1}x^{n-1}+\dots+a_{1}x+a_0
}

\newcommand{\solution}[1]
{
    \vspace{-3mm}
    \begin{proof}[\textbf{\textup{Solución}}]
        #1
    \end{proof}
    \vspace{1mm}
}

\newcommand{\enumSolution}[2]
{
    \vspace{-3mm}
    \begin{proof}[\textbf{\textup{Solución #1}}]
        #2
    \end{proof}
    \vspace{1mm}
}
\theoremstyle{definition}

\newtheorem*{definition}{Definición}
\newtheorem{section-definition}{Definición}[section]
\newtheorem*{note}{Nota}
\newtheorem{lemma}{Lema}[section]
\newtheorem{cor}{Corolario}[section]
\newtheorem{case}{Caso}
\newtheorem{exercise}{Ejercicio}
\newtheorem{example}{Ejemplo}
\newtheorem{problem}{Problema}
\newtheorem{section-problem}{Problema}[section]
\newtheorem{corollary}{Corolario}
\newtheorem{theorem}{Teorema}[section]

\begin{document}

    \begin{center}
        \textbf{\Large Notas para la clase}
    \end{center}

    \section{Clase 02}
    {
        Sea el polinomio $M(x) = x^5 - 3x^4 - 29x^3 - 13x^2 + 120x + 140.$ Encontrar sus raíces.

        Para empezar, tratemos que describir el polinomio a manera de ejercicio con ayuda de las siguietes preguntas; ¿qué característcas tiene $M$? ¿es mónico? ¿es completo? ¿es simétrico? ¿está ordenado?.
        Luego que estudiante intentó encontrar soluciones por su cuenta, anunciar que 7 es una raíz. A continuación, comprobar que $x = 7$ es una raíz.

        \begin{align*}
              1\times 16807= +16807\\
             -3\times 2401 = -\mbox{..}7203\\
            -29\times 343  = -\mbox{..}9947\\
            -13\times 49   = -\mbox{....}637\\
            120\times 7    = +\mbox{....}840\\
              1\times 140  = +\mbox{....}140
            \end{align*}

        ¿Es fácil deducir que $x = 7$ es una raíz?. Mostrar la factorización
        \begin{align*}
            x^5 + 4x^4 - x^3 - 20x^2 - 20x + 0 \\
            0x^5 - 7x^4 - 28x^3 + 7x^2 + 140x + 140\\
            x^5 - 3x^4 - 29x^3 - 13x^2 + 120x + 140
        \end{align*}

        Es decir $M(x) = (x - 7)(x^4 + 4x^3 - x^2 - 20x - 20)$
        \begin{align*}
            x^4 + 4x^3 + 4x^2 \\
            -5x^2 - 20x - 20\\
            x^4 + 4x^3 - x^2 - 20x - 20
        \end{align*}

        Es decir $M(x) = (x - 7)(x^2 - 5)(x + 2)^2$\\
        Indicar que $x = -2$ tiene multiplicidad dos.
    }\label{sec:clase-02}

\end{document}