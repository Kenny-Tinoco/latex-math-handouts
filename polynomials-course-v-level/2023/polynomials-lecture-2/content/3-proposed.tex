\section{Problemas propuestos}

\begin{section-problem}
    Si $P(x) = x^4 + ax^3 + bx^2 + cx + d$ es un polinomio tal que $P(1) = 10$, $P(2) = 20$ y $P(3) = 30$, determine el valor de $\frac{P(12) + P(-8)}{10}.$
\end{section-problem}

\begin{section-problem}
    Sea $P(x)$ un polinomio cuadrático.
    Demostrar que existen polinomios cuadráticos $G(x)$ y $H(x)$ tales que $P(x)P(x+1) = (G \circ H)(x).$
\end{section-problem}

\begin{section-problem}
    Sea $P(x) = mx^3 + mx^2 + nx + n$ un polinomio cuyas raíces son $a, b \mbox{ y } c$. Demostrar que
    \[\frac{1}{a} + \frac{1}{b} + \frac{1}{c} = \frac{1}{a + b + c}.\]
\end{section-problem}

\begin{section-problem}
    Sea $P(x)$ un polinomio cúbico mónico tal que $P(1) = 1$, $P(2) = 2$ y $P(3) = 3$. Encontrar $P(4).$
\end{section-problem}