\section{Problemas propuestos}

Recordar que los problemas de esta sección son los asignados como \textbf{tarea}.
Es el deber del estudiante resolverlos y entregarlos de manera clara y ordenada el próximo encuentro
(de ser necesario, también se pueden entregar borradores).

\begin{section-problem}
    El polinomio $P(x)$ deja residuo $-2$ en la división entre $x - 1$ y residuo $-4$ en la división entre $x + 2$.
    Encontrar el residuo cuando el polinomio es dividido por $x^2 + x - 2$.
\end{section-problem}

\begin{section-problem}
    Encontrar el resto cuando el polinomio $x^{81} + x^{49} + x^{25} + x^9 + x$ es dividido entre $x^3 - x$.
\end{section-problem}

\begin{section-problem}
    Probar que si un polinomio $F(x)$ deja un residuo de la forma $px + q$ cuando es dividido entre $(x - a)(x - b)(x - c)$
    donde $a$, $b$ y $c$ son todos distintos, entonces
    \[(b - c)F(a) + (c - a)F(b) + (a - b)F(c) = 0.\]
\end{section-problem}