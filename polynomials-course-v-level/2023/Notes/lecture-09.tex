\section{Clase 09}

\begin{section-problem}
    Sean $x$, $y$ y $z$ números reales tales que
    \[
        \left\{
        \begin{array}{rcl}
            x + y + z & =& 3\\
            x^2 + y^2 + z^2 & =& 5\\
            x^3 + y^3 + z^3 & =& 7
        \end{array}
        \right.
    \]
    Hallar el valor de $x^4 + y^4 + z^4$.
\end{section-problem}

\begin{solution}
    Utilizando la notación de polinomios simétricos y suma simétrica de potencias, tenemos lo siguiente
    \[
        \left\{
        \begin{array}{rcl}
            \sigma_1 & =& 3\\
            s_2 & =& 5\\
            s_3 & =& 7
        \end{array}
        \right.
    \]

    Por propiedad $s_2 = \sigma_1^2 - 2\sigma_2$ y $s_3 = \sigma_1^3 - 3\sigma_1 \sigma_2 + 3\sigma_3$.
    De aquí que
    \vspace{-10mm}
    \begin{table}[H]
        \centering
        \begin{tabular}{p{5cm} p{5cm}}
            \begin{gather*}
                s_2 = \sigma_1^2 - 2\sigma_2\\
                5 = 3^2 - 2\sigma_2\\
                5 - 9 = - 2\sigma_2\\
                - 4 = - 2\sigma_2\\
                \boxed{\sigma_2 = 2}
            \end{gather*}
            &
            \begin{gather*}
                s_3 = \sigma_1^3 - 3\sigma_1 \sigma_2 + 3\sigma_3\\
                7 = 3^3 - 3\cdot 3\cdot 2 + 3\sigma_3\\
                7 - 27 + 18 = 3\sigma_3\\
                -2 = 3\sigma_3\\
                \boxed{\sigma_3 = - \frac{2}{3}}
            \end{gather*}
        \end{tabular}
    \end{table}
    \vspace{-10mm}

    De estos resultados podemos considerar al polinomio $w^3 - 3w^2 + 2w + \dfrac{2}{3}$, que tiene como raíces a $x$, $y$ y $z$.
    Por propiedad
    \vspace{-10mm}
    \begin{table}[H]
        \centering
        \begin{tabular}{p{5cm} p{5cm}}
            \begin{gather*}
                x^3 - 3x^2 + 2x + \frac{2}{3} = 0\\
                x^4 - 3x^3 + 2x^2 + \frac{2}{3}x = 0\\
                x^4 = 3x^3 - 2x^2 - \frac{2}{3}x
            \end{gather*}
            &
            \begin{gather*}
                \mbox{Análogamente}\\
                y^4 = 3y^3 - 2y^2 - \frac{2}{3}y\\
                z^4 = 3z^3 - 2z^2 - \frac{2}{3}z
            \end{gather*}
        \end{tabular}
    \end{table}
    \vspace{-10mm}

    Por lo tanto
    \begin{gather*}
        x^4 + y^4 + z^4 = 3s_3 - 2s_2 - \frac{2}{3}\sigma_1\\
        x^4 + y^4 + z^4 = 3\cdot7 - 2\cdot5 - \frac{2}{3} \cdot 3\\
        x^4 + y^4 + z^4 = 21 - 10 - 2\\
        \boxed{x^4 + y^4 + z^4 = 9}
    \end{gather*}
\end{solution}


\begin{section-problem}
    Sean $a$, $b$ y $c$ las raíces del polinomio $3x^3 + x + 2023$.
    Calcular \[(a + b)^3 + (b + c)^3 + (c + a)^3.\]
\end{section-problem}

\begin{solution}
    Por Vieta
    \begin{gather*}
        \sigma_1 = a + b + c = 0\\
        \sigma_2 = ab + bc + ca = \frac{1}{3}\\
        \sigma_3 = abc = - \frac{2023}{3}
    \end{gather*}
    De aquí que $a + b = -c$, $b + c = -a$ y $c + a = -b$.
    Por lo que la expresión en cuestión toma la forma
    \[(a + b)^3 + (b + c)^3 + (c + a)^3 = - (a^3 + b^3 + c^3) = - s_ 3\]
    Por otro lado, sabemos que
    \begin{gather*}
        s_3 = \sigma_1^3 - 3\sigma_1 \sigma_2 + 3\sigma_3\\
        s_3 = 0^3 - 3\cdot 0 \cdot \frac{1}{3} + 3\cdot (- \frac{2023}{3})\\
        s_3 = -2023
    \end{gather*}
    De esta forma llegamos a que
    \[(a + b)^3 + (b + c)^3 + (c + a)^3 = \boxed{2023}\]
\end{solution}

\begin{section-problem}
    Considera el polinomio $P(x) = x^6 - x^5 - x^3 - x^2 - x$ y $Q(x) = x^4 - x^3 - x^2 - 1$.
    Sean $z_1$, $z_2$, $z_3$ y $z_4$ las raíces de $Q$, encontrar $P(z_1) + P(z_2) + P(z_3) + P(z_4)$.
\end{section-problem}

\begin{solution}
    Por propiedad, $Q(z_1) = 0 \rightarrow z_1^4 - z_1^3 - z_1^2 - 1 = 0$, así que
    \vspace{-10mm}
    \begin{table}[H]
        \centering
        \begin{tabular}{p{5cm} p{5cm}}
            \begin{gather*}
                z_1^4 - z_1^3 - z_1^2 - 1 = 0\\
                z_1^6 - z_1^5 - z_1^4 - z_1^2 = 0\\
                \boxed{z_1^6 - z_1^5 - z_1^2 = z_1^4}
            \end{gather*}
            &
            \begin{gather*}
                \mbox{También}\\
                z_1^4 - z_1^3 - z_1^2 - 1 = 0\\
                \boxed{z_1^4 - z_1^3 = z_1^2 + 1}
            \end{gather*}
        \end{tabular}
    \end{table}
    \vspace{-10mm}

    Claramente $P(z_1) = z_1^6 - z_1^5 - z_1^3 - z_1^2 - z_1 = z_1^4 - z_1^3 - z_1 = z_1^2 + 1 - z_1 = \boxed{z_1^2 - z_1 + 1}$.
    Análogamente con $P(z_2)$, $P(z_3)$ y $P(z_4)$.
    Sea $R = P(z_1) + P(z_2) + P(z_3) + P(z_4)$, entonces $R = s_2 - \sigma_1 + 4$.
    Utilizando la relación $s_2 = \sigma_1^2 - 2\sigma_2$ y las fórmulas de Vieta, llegamos a
    \begin{gather*}
        R = s_2 - \sigma_1 + 4\\
        R = \sigma_1^2 - 2\sigma_2 - \sigma_1 + 4\\
        R = 1^2 - 2\cdot(-1) - 1 + 4\\
        R = 2 + 4\\
        \boxed{R = 6}
    \end{gather*}

\end{solution}