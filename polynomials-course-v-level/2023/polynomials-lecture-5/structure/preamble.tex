\usepackage{amsmath}
\usepackage{amsthm}
\usepackage{amsfonts}
\usepackage{amssymb}
\usepackage[spanish]{babel}
\usepackage{float}
\usepackage[table,xcdraw]{xcolor}
\usepackage{graphicx}
\usepackage{grffile}
\usepackage{multicol}
\usepackage{booktabs}
\usepackage{subfigure}
\usepackage[left=2.54cm,right=2.54cm,top=2.54cm,bottom=2.54cm]{geometry}
\usepackage{polynom}
\usepackage{tikz,lipsum,lmodern}
\usepackage[most]{tcolorbox}
\usepackage{titlesec}
\usepackage{pst-node,wrapfig}
\usepackage{fancyhdr}
\usepackage{xparse}

\label{key}

\spanishdecimal{.}

\setlength{\columnsep}{0.8cm}
\setlength{\parindent}{0pt}
\setlength{\parskip}{2.5mm}

\newcommand{\cvec}[1]{{\mathbf #1}}
\newcommand{\rvec}[1]{\vec{\mathbf #1}}
\newcommand{\ihat}{\hat{\textbf{\i}}}
\newcommand{\jhat}{\hat{\textbf{\j}}}
\newcommand{\khat}{\hat{\textbf{k}}}
\newcommand{\minor}{{\rm minor}}
\newcommand{\trace}{{\rm trace}}
\newcommand{\spn}{{\rm Span}}
\newcommand{\rem}{{\rm rem}}
\newcommand{\ran}{{\rm range}}
\newcommand{\range}{{\rm range}}
\newcommand{\mdiv}{{\rm div}}
\newcommand{\proj}{{\rm proj}}
\newcommand{\R}{\mathbb{R}}
\newcommand{\N}{\mathbb{N}}
\newcommand{\Q}{\mathbb{Q}}
\newcommand{\Z}{\mathbb{Z}}
\newcommand{\C}{\mathbb{C}}
\newcommand{\ZP}{\mathbb{Z^+}}
\newcommand{\ZN}{\mathbb{Z^-}}

\newcommand{\<}{\langle}
\renewcommand{\>}{\rangle}

\newcommand{\attn}[1]{\textbf{#1}}
\renewcommand{\emptyset}{\varnothing}

\newcommand{\modIn}[1]{\mbox{ (mód }#1\mbox{)}}
\newcommand{\fullModIn}[2]{\equiv #1 \mbox{ (mód }#2\mbox{)}}

\newcommand{\MDIF}{\textit{(MDIF)}}
\newcommand{\WellOrderingPrinciple}{\textit{Principio del Buen Orden}}

\newcommand{\ProductPrinciple}{\textit{Principio del producto} }
\newcommand{\AdditionPrinciple}{\textit{Principio de la suma} }
\renewcommand{\theenumi}{\alph{enumi}}

\newcommand{\sen}{\operatorname{sen}}
\newcommand{\arcsen}{\operatorname{arcsen}}
\newcommand{\li}{\displaystyle\lim}
\newcommand*{\QEDA}{\hfill\ensuremath{\blacksquare}}
\newcommand{\QED}{\hfill {\qed}}

\newcommand{\bproof}{\bigskip {\bf Proof. }}
\newcommand{\eproof}{\hfill\qedsymbol}
\newcommand{\Disp}{\displaystyle}
\newcommand{\qe}{\hfill\(\bigtriangledown\)}

\newcommand{\sourceProblem}[1]
{
    \vspace{-4mm}
    \begin{flushright}
        \emph{(#1)}
    \end{flushright}
    \vspace{1mm}
}

\newcommand{\problemImage}[1]
{
    \begin{center}
        \includegraphics[width=5cm]{#1}
    \end{center}
}

\newcommand{\generalFormPolynomail}
{
    P(x) = a_nx^n+a_{n-1}x^{n-1}+\dots+a_{1}x+a_0
}

\newcommand{\solution}[1]
{
    \vspace{-3mm}
    \begin{proof}[\textbf{\textup{Solución}}]
        #1
    \end{proof}
    \vspace{1mm}
}

\newcommand{\enumSolution}[2]
{
    \vspace{-3mm}
    \begin{proof}[\textbf{\textup{Solución #1}}]
        #2
    \end{proof}
    \vspace{1mm}
}
\theoremstyle{definition}

\newtheorem*{definition}{Definición}
\newtheorem{section-definition}{Definición}[section]
\newtheorem*{note}{Nota}
\newtheorem{lemma}{Lema}[section]
\newtheorem{cor}{Corolario}[section]
\newtheorem{case}{Caso}
\newtheorem{exercise}{Ejercicio}
\newtheorem{example}{Ejemplo}
\newtheorem{problem}{Problema}
\newtheorem{section-problem}{Problema}[section]
\newtheorem{corollary}{Corolario}
\newtheorem{theorem}{Teorema}[section]


\ExplSyntaxOn

\NewDocumentCommand{\ruffini}{mmmm}
{% #1 = polynomial, #2 = divisor, #3 = middle row, #4 = result
    \franklin_ruffini:nnnn { #1 } { #2 } { #3 } { #4 }
}

\seq_new:N \l_franklin_temp_seq
\tl_new:N \l_franklin_scheme_tl
\int_new:N \l_franklin_degree_int

\cs_new_protected:Npn \franklin_ruffini:nnnn #1 #2 #3 #4
    {
    % Start the first row
    \tl_set:Nn \l_franklin_scheme_tl { & }
    % Split the list of coefficients
    \seq_set_split:Nnn \l_franklin_temp_seq { , } { #1 }
    % Remember the number of columns
    \int_set:Nn \l_franklin_degree_int { \seq_count:N \l_franklin_temp_seq }
    % Fill the first row
    \tl_put_right:Nx \l_franklin_scheme_tl
    { \seq_use:Nn \l_franklin_temp_seq { & } }
    % End the first row and leave two empty places in the next
    \tl_put_right:Nn \l_franklin_scheme_tl { \\ #2 & & }
    % Split the list of coefficients and fill the second row
    \seq_set_split:Nnn \l_franklin_temp_seq { , } { #3 }
    \tl_put_right:Nx \l_franklin_scheme_tl
    { \seq_use:Nn \l_franklin_temp_seq { & } }
    % End the second row
    \tl_put_right:Nn \l_franklin_scheme_tl { \\ \hline }
    % Split and fill the third row
    \seq_set_split:Nnn \l_franklin_temp_seq { , } { #4 }
    \tl_put_right:Nx \l_franklin_scheme_tl
    { & \seq_use:Nn \l_franklin_temp_seq { & } }
    % Start the array (with \use:x because the array package
    % doesn't expand the argument)
    \use:x
    {
        \exp_not:n { \begin{array} } { r | *{\int_eval:n { \l_franklin_degree_int - 1 }} { r } | r }
    }
    % Body of the array and finish
    \tl_use:N \l_franklin_scheme_tl
    \end{array}
}
\ExplSyntaxOff