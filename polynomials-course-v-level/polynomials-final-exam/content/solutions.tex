\newpage
\section*{\large Soluciones}

    \textbf{Ejercicio 1.}

    \textbf{Ejercicio 2.}
    Podemos expresar el problema de la siguiente manera
    \[x^{2023} + 2^3 = (x - 2)(x + 2)Q(x) + ax + b\]
    Por el teorema del resto $P(2) = 2a + b$ y $P(-2) = -2a + b$, con lo cual podemos hacer el siguiente sistema de ecuaciones
    \[
        \left\{
        \begin{array}{rcl}
            2^{2023} + 2^3 & =& 2a + b\\
            -2^{2023} + 2^3 & =& -2a + b
        \end{array}
        \right
    \]
    De donde obtenemos las soluciones $(a, b) = (2^{2022}, 2^3)$.
    Luego, el producto que nos piden es
    \[a\cdot b = 2^{2022}\cdot 2^3 = \boxed{2^{2025}}\]

    \textbf{Ejercicio 3.}
    Restando los dos polinomios tenemos que
    \begin{gather*}
        (3202 - 2023)x^2 - (3202 - 2023) = 1179x^2 - 1179
    \end{gather*}
    De donde claramente, las soluciones para $x$ son $\pm 1$.
    Luego, sustituyendo $x = 1$ en cualquiera de las ecuaciones (ya que tienen una raíz en común) tenemos que
    \[2023(1)^2 + a(1) + 3202 = 0 \longrightarrow \boxed{a = -5225}\]

    \textbf{Ejercicio 4.}
    Al aplicar Vieta al primer polinomio
    \begin{gather*}
        a + b + c = - 3\\
        ab + bc + ca = 4\\
        abc = 11
    \end{gather*}

    Por Vieta aplicado al segundo polinomio sabemos que
    \[(a + b)(b + c)(c + a) = - t\]
    Al desarrollar este producto tenemos que
    \[(a + b + c)(ab + bc + ca) - abc = - t\]
    De donde es claro que
    \[t = abc - (a + b + c)(ab + bc + ca) = 11 - (-3)(4) = 11 + 12 = \boxed{23}\]

    \textbf{Ejercicio 5.}
    Sabemos que $a^3 + 3a - 1 = 0 \rightarrow a^3 = 1 - 3a$.
    También, sabemos por Vieta que $a + b + c = 0$, $ab + bc + ca = 3$ y $abc = 1$.
    Por lo tanto, la expresión es equivalente a
    \begin{gather*}
        \frac{1}{1 - 3a + 1 - 3b} + \frac{1}{1 - 3b + 1 - 3c} + \frac{1}{1 - 3c + 1 - 3a}\\
        \hookrightarrow \frac{1}{2 - 3(a + b)} + \frac{1}{2 - 3(b + c)} + \frac{1}{2 - 3(c + a)}\\
        \hookrightarrow \frac{1}{2 + 3a} + \frac{1}{2 + 3b} + \frac{1}{2 + 3c}
    \end{gather*}
    Luego, solo desarrollamos y evaluamos
    \begin{gather*}
        \hookrightarrow \frac{(3a + 2)(3b + 2) + (3b + 2)(3c + 2) + (3c + 2)(3a + 2)}{(3a + 2)(3b + 2)(3c + 2)}\\
        \hookrightarrow \frac{(9ab + 6a + 6b + 4) + (9bc + 6b + 6c + 4) + (9ca + 6c + 6a + 4)}{27abc + 12(a + b + c) + 18(ab + bc + ca) + 8}\\
        \hookrightarrow \frac{ 12(a + b + c) + 9(ab + bc + ca) + 12}{27abc + 12(a + b + c) + 18(ab + bc + ca) + 8}\\
        \hookrightarrow \frac{ 12(0) + 9(3) + 12}{27(1) + 12(0) + 18(3) + 8} = \frac{27 + 12}{27 + 54 + 8}  = \boxed{\frac{39}{89}}
    \end{gather*}


