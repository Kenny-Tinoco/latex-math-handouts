\newpage
\section{Problemas}

\subsection{Enunciados}

\begin{section-problem}
    Si $P(x^2 - 2x + 1) = x^2 - 3$, determine $P((x - 2)^2).$
\end{section-problem}

\begin{section-problem}
    Sea $P(x) = x^2$, encontrar $Q(x)$ si $(P \circ Q)(x) = 4x^2 - 12x + 9$.
\end{section-problem}

\begin{section-problem}
    Sea $Q(x) = \frac{1}{5} x - 2$ y $P(x) = Q^4(x)$, probar que $P(1560) = 0$.
\end{section-problem}


\begin{section-problem}
    Sea $P(x)$ un polinomio cuadrático.
    Demostrar que existen polinomios cuadráticos $G(x)$ y $H(x)$ tales que $P(x)P(x+1) = (G \circ H)(x).$
\end{section-problem}

\begin{section-problem}
    Sea $P(x) = mx^3 + mx^2 + nx + n$ un polinomio cuyas raíces son $a, b \mbox{ y } c$.
    Demostrar que \[\frac{1}{a} + \frac{1}{b} + \frac{1}{c} = \frac{1}{a + b + c}.\]
\end{section-problem}

\begin{section-problem}
    Sea $P(x)$ un polinomio cúbico mónico tal que $P(1) = 1$, $P(2) = 2$ y $P(3) = 3$.
    Encontrar $P(4).$
\end{section-problem}


\begin{section-problem}
    Sea el polinomio $P$ para el cual $P(x^2 + 1) = x^4 + 4x^2$.
    Encontrar $P(x^2 - 1).$
\end{section-problem}

\begin{section-problem}
    Sea $S(x)$ un polinomio cúbico con $S(1) = 1$, $S(2) = 2$, $S(3) = 3$ y $S(4) = 5$.
    Encontrar $S(6)$.
\end{section-problem}

\begin{section-problem}
    Determine un polinomio cúbico $P$ en los reales, con una raíz igual a cero y que satisface $P(x - 1) = P(x) + 25x^2$.
\end{section-problem}

\begin{section-problem}
    Si $p$ y $q$ son las raíces del polinomio $P(x) = 4x^2 + 5x + 3$.
    Determina $(p + 7)(q + 7)$, sin calcular los valores de $p$ y $q$.
\end{section-problem}

\begin{section-problem}
    Supóngase que el polinomio $5x^3 + 4x^2 - 8x + 6$ tiene tres raíces reales $a, b \mbox{ y } c$.
    Demostra que \[(5a)^2\left(\frac{b}{2}\right)\left(\frac{1}{a} + \frac{1}{b}\right) + (5b)^2\left(\frac{c}{2}\right)\left(\frac{1}{b} + \frac{1}{c}\right)+ (5c)^2\left(\frac{a}{2}\right)\left(\frac{1}{c} + \frac{1}{a}\right) = 3^3 + 1.\]
\end{section-problem}

\begin{section-problem}
    Sean $r_1, r_2, r_3$ las raíces del polinomio $P(x) = x^3 - x^2 + x - 2$.
    Determina el valor de $r^3_1 + r^3_2 + r^3_3$, sin calcular los valores de $r_1, r_2, r_3$.
\end{section-problem}


\begin{section-problem}
    Dado los polinomios $P(x) = 2x^5 + x^4 + ax^2 + bx + c$ y $Q(x) = x^4 - 1$, se sabe que $Q \mid P$.
    Hallar $\dfrac{a + b}{a - b}$.
\end{section-problem}

\begin{section-problem}
    Dado los polinomios $P(x) = 16x^5 + ax^2 + bx + c$ y $Q(x) = 2 x^3 - x^2 + 1$, se sabe que $Q \mid P$.
    Hallar $a + b + c$.
\end{section-problem}

\begin{section-problem}
    Dado los polinomios $P(x) = 6x^4 + 4x^3 - 5x^2 - 10x + a$ y $Q(x) = 3 x^2 + 2x + b$, se sabe que $Q \mid P$.
    Hallar $a^2 + b^2$.
\end{section-problem}


\begin{section-problem}
    El polinomio $P(x)$ deja residuo $-2$ en la división entre $x - 1$ y residuo $-4$ en la división entre $x + 2$.
    Encontrar el residuo cuando el polinomio es dividido por $x^2 + x - 2$.
\end{section-problem}

\begin{section-problem}
    Encontrar el resto cuando el polinomio $x^{81} + x^{49} + x^{25} + x^9 + x$ es dividido entre $x^3 - x$.
\end{section-problem}

\begin{section-problem}
    Probar que si un polinomio $F(x)$ deja un residuo de la forma $px + q$ cuando es dividido entre $(x - a)(x - b)(x - c)$
    donde $a$, $b$ y $c$ son todos distintos, entonces
    \[(b - c)F(a) + (c - a)F(b) + (a - b)F(c) = 0.\]
\end{section-problem}

\begin{section-problem}
    Para que la división de $6x^4 - 11x^2 + ax + b$ entre $3x^2 - 3x - 1$ sea exacta, encuentre los valores de $a$ y $b$ apropiados.
\end{section-problem}

\begin{section-problem}
    Calcular la suma de coeficientes del resto que deja $x^{3333} - 9$ entre $x^2 - 729$.
\end{section-problem}

\begin{section-problem}
    Sea $r$ una raíz de $x^2 - x + 7$.
    Hallar el valor de $r^3 + 6r + \pi$.
\end{section-problem}

\begin{section-problem}
    Sean $a$, $b$ y $c$ las raíces reales de la ecuación $x^3 + 3x^2 - 24x + 1 = 0$.
    Probar que $\sqrt[3]{a} + \sqrt[3]{b} + \sqrt[3]{c} = 0$.
\end{section-problem}

\begin{section-problem}
    Sean $r_1$, $r_2$ y $r_3$ raíces distintas del polinomio $y^3 - 22 y^2 + 80 y - 67$.
    De tal manera que existen números reales $\alpha$, $\beta$ y $\theta$ tal que
    \[
        \frac{1}{y^3 - 22 y^2 + 80 y - 67} = \frac{\alpha}{y - r_1} + \frac{\beta}{y - r_2} + \frac{\theta}{y - r_3}
    \]
    $\forall y \notin \left\{ r_1, r_2, r_3 \right\}$.
    ¿Cuál es valor de $\dfrac{1}{\alpha} + \dfrac{1}{\beta} + \dfrac{1}{\theta}$?
\end{section-problem}

\begin{section-problem}
    La ecuación $2^{333x - 2} + 2^{111x + 2} = 2^{222x + 1} + 1$ tiene tres raíces reales.
    Dado que su suma es $\dfrac{m}{n}$ con $m, n \in \ZP$ y $mcd(m, n) = 1$.
    Calcular $m + n$.
\end{section-problem}

\begin{section-problem}
    Si $P(x) = x^4 + ax^3 + bx^2 + cx + d$ es un polinomio tal que $P(1) = 10$, $P(2) = 20$ y $P(3) = 30$, determine el valor de
    \[\frac{P(12) + P(-8)}{10}.\]
\end{section-problem}

\begin{section-problem}
    Sea $F(x)$ un polinomio mónico con coeficientes enteros.
    Probar que si existen cuatro enteros diferentes $a$, $b$, $c$ y $d$ tal que $F(a) = F(b) = F(c)  = F(d) = 5$,
    entonces no existe un entero $k$ tal que $F(k) = 8$.
\end{section-problem}


\begin{section-problem}
    Sea el polinomio $F(x) = x^3 + \frac{3}{4}x^2 - 4x - 3$.
    Notemos que $F(2) = 0$, así que 2 es raíz de $F$.
    Pero 2 no divide al término independiente de $F$, es decir que esta raíz viola el Teorema de la raíz racional.
    Argumente por qué pasa esto.
\end{section-problem}

\begin{section-problem}
    Encontrar las raíces del polinomio $R(x) = 6x^3 + x^2 - 19x + 6$.
\end{section-problem}

\begin{section-problem}
    Encontrar las raíces del polinomio $G(x) = 12x^3 - 107x^2 - 15x + 54$.
\end{section-problem}



\begin{section-problem}
    Sean $x$, $y$ y $z$ números reales tales que
    \[
        \left\{
        \begin{array}{rcl}
            x + y + z & =& 3\\
            x^2 + y^2 + z^2 & =& 5\\
            x^3 + y^3 + z^3 & =& 7
        \end{array}
        \right.
    \]
    Hallar el valor de $x^4 + y^4 + z^4$.
\end{section-problem}

\begin{section-problem}
    Sean $a$, $b$ y $c$ las raíces del polinomio $3x^3 + x + 2023$.
    Calcular
    \[
        (a + b)^3 + (b + c)^3 + (c + a)^3.
    \]
\end{section-problem}

\begin{section-problem}
    Considera el polinomio $P(x) = x^6 - x^5 - x^3 - x^2 - x$ y $Q(x) = x^4 - x^3 - x^2 - 1$.
    Sean $z_1$, $z_2$, $z_3$ y $z_4$ las raíces de $Q$, encontrar $P(z_1) + P(z_2) + P(z_3) + P(z_4)$.
\end{section-problem}



\begin{section-problem}
    Si $a + b + c = \sqrt{2023}$ y $a^2 + b^2 + c^2 = 2021$, hallar el valor de
    \[E = \frac{(a + b)^2 (b + c)^2 (c + a)^2}{(a^2 + 1) (b^2 + 1) (c^2 + 1)}.\]
\end{section-problem}

\begin{section-problem}
    Porbar que para todo $n$ entero positivo se cumple que
    \begin{align*}
        1 + 3 + 5 + \cdots + \left(2n - 1\right) &= n^2\\
        1^3 + 2^3 + 3^3  + \cdots + n^3 &= \left(\frac{n (n + 1)}{2}\right)^2
    \end{align*}
\end{section-problem}

\begin{section-problem}
    Con la ayuda del teorema de la raíz racional, encontrar todas las raíces de los siguientes polinomios
    \begin{align*}
        2 x^3 - 21 x^2 + 52 x - 21& \\
        x^4 - 7 x^3 - 19 x^2 + 103 x + 210&
    \end{align*}
\end{section-problem}

\begin{section-problem}
    El cociente de la división $\dfrac{x^{n + 1} + 2x + 5}{x - 3}$ es $Q(x)$, la suma de coeficientes de $Q$ es $\dfrac{9^{10} + 3}{2}$.
    Hallar el valor de $n$.
\end{section-problem}

\begin{section-problem}
    Si la división
    \[\frac{x^{80} - 7 x^{30} + 9x^5 - mx + 1}{x^3 + x - 2}\]
    Deja como resto a $R(x) = x^2 + x - 1$, hallar el valor de $m$.
\end{section-problem}

\begin{section-problem}
    Suponga que $x$, $y$ y $z$ son números distintos de cero tal que $(x + y + z)(x^2 + y^2 + z^2) = x^3 + y^3 + z^3$.
    Hallar el valor de
    \[(x + y + z)\left(\frac{1}{x} + \frac{1}{y} + \frac{1}{z}\right).\]
\end{section-problem}

\begin{section-problem}
    Sean $a$, $b$ y $c$ números reales no nulos tales que $a + b + c = 0$.
    Determine el valor de la expresión
    \[\frac{(a^2 + b^2)(b^2 + c^2) + (b^2 + c^2)(c^2 + a^2) + (c^2 + a^2)(a^2 + b^2)}{a^4 + b^4 + c^4}.\]
    \begin{source-problem}
        Lista corta, OMCC XXII. Álgebra, P1
    \end{source-problem}
\end{section-problem}

\begin{section-problem}
    Si $a$, $b$, $c$ y $d$ son las raíces de la ecuación $x^4 - 3x^3 + 1 = 0$,
    calcular el valor de
    \[\inverseOf{a^6} + \inverseOf{b^6} + \inverseOf{c^6} + \inverseOf{d^6}.\]
\end{section-problem}



\begin{section-problem}
    Dado el polinomio $P(x)$ para el cual se cumple que
    \[x^{23} + 23x^{17} - 18x^{16} - 24x^{15} + 108x^{14} = (x^4 - 3x^2 - 2x + 9)P(x)\]
    Calcular la suma de coeficientes de $P$.
\end{section-problem}

\begin{section-problem}
    Sea $r_1$, $r_2$ y $r_3$ las tres raíces de polinomio cúbico $P$.
    También, que
    \[
        \frac{P(2) + P(-2)}{P(0)} = 52
    \]
    La expresión $\dfrac{1}{r_1 r_2} + \dfrac{1}{r_2 r_3} + \dfrac{1}{r_3 r_1}$ puede ser escrita como $\dfrac{m}{n}$ para $m$ y $n$ coprimos.
    Encontrar $m\times n$.
\end{section-problem}

\begin{section-problem}
    Dado que
    \[
        \left\{
        \begin{array}{rcl}
            Q(x + 1) & =& \dfrac{3}{2} x + 3\\
            Q( F(x) + G(x) ) & =& 3x + \dfrac{3}{2}\\
            Q( F(x) \times G(x) ) & =& \dfrac{15}{2}
        \end{array}
        \right.
    \]
    Calcular $F(G(F(G(\dots F(G(-2))\dots))))$.
\end{section-problem}

\begin{section-problem}
    Sea $Q(x) = 2x - 4096$ y $P(x) = Q^{12}(x)$, hallar la raíz de $P$.
\end{section-problem}

\begin{section-problem}
    Hallar el resto de la división de $\left[(x - 1)(x)(x + 2)(x + 3)\right]^2 + (x^2 + 2x)^3 x - 50$ entre $x^2 + 2x - 5$.
\end{section-problem}

\begin{section-problem}
    Si $P\left(x + \dfrac{1}{x}\right) = x^2 + \dfrac{1}{x^2} + 227$, ¿cuál es el valor de $\sqrt {P(20)}$?
\end{section-problem}

\begin{section-problem}
    Si $a$ y $b$ son raíces distintas del polinomio $x^2 + 1012x + 1011$, entonces
    \[\frac{1}{a^2 + 1011a + 1011} + \frac{1}{b^2 + 1011b + 1011} = \frac{m}{n},\]
    donde $m$ y $n$ son primos relativos.
    Calcular $m + n$.
\end{section-problem}

\begin{section-problem}
    Encontrar el resto cuando $(5x + 16)^{2023} + (x + 6)^{98} + (7x + 30)^{49}$ es dividido por $x + 3$.
\end{section-problem}

\begin{section-problem}
    Si la división
    \[\frac{x^{80} - 7 x^{30} + 9x^5 - mx + 1}{x^3 + x - 2}\]
    Deja como resto a $R(x) = x^2 + x - 1$, hallar el valor de $m$.
\end{section-problem}

\begin{section-problem}
    Demostrar por inducción matemática, que $\forall n \in \Z^{\geq 0}$, se cumple
    \[17 \mid 2^{5n + 3} + 5^n \cdot 3^{n + 2}.\]
\end{section-problem}

\begin{section-problem}
    Sea $R(c) = a^2 + b^2 + 65c^2 + 2ab - 18bc - 18ca$, factorize $R$ y responda.
    ¿Cuáles son las raíces de $R$?
\end{section-problem}

\begin{section-problem}
    Encontrar el resto cuando $x^{2022} + x^{2021} + \cdots + x + 1$ es dividido por $x - 3$.
\end{section-problem}

\begin{section-problem}
    Sea el polinomio $P_0(x) = x^3 + 313x^2 - 77x - 8$.
    Para enteros $n \geq 0$, definimos $P_n(x) = P_{n - 1}(x - n)$.
    ¿Cuál es el coeficiente del término cuadrático en $P_{23}(x)$?
\end{section-problem}

\begin{section-problem}
    Indique el valor de la expresión
    \[M(3)  + M(5) + M(7) + \cdots + M(2021) + M(2023)\]
    Si $M(x) = \dfrac{2\cdot 2023}{x(x - 2)}$.
\end{section-problem}

\begin{section-problem}
    Dado el polinomio $S(x) = (11 - 15x^3)(17x^6 - 37) + 2^8 x^6 (16 - x + x^2)(16 + x)$, responda lo siguiente:
    \begin{multicols}{2}
        \begin{enumerate}
            \item ¿$S(x)$ es mónico? \\R: \rule{1cm}{0.1mm}
            \item ¿$S(x)$ es completo? \\R: \rule{1cm}{0.1mm}
            \item ¿$S(x)$ es simétrico? \\R: \rule{1cm}{0.1mm}
            \item Escriba el coeficiente de $x^6$. \\R: \rule{1cm}{0.1mm}
            \item Escriba el término independiente.\\ R: \rule{1cm}{0.1mm}
            \item ¿Es $S\left(\sqrt[3]{x}\right)$ un polimonio? \\ R: \rule{2cm}{0.1mm}
        \end{enumerate}
    \end{multicols}
\end{section-problem}

\begin{section-problem}
    Si $P(x - 2) = x^3 - 10x^2 + 28x - 24$, hallar el resto de dividir $P(x)$ por $x - 3$.
\end{section-problem}

\begin{section-problem}
    Encontrar todas las tripletas $(x, y, z)$ de números reales, tal que cumplen el siguiente sistema de ecuaciones
    \[
        \left\{
        \begin{array}{rcl}
            x + y + z & =& 17\\
            xy + yz + zx & =& 94\\
            x y z & =& 168
        \end{array}
        \right.
    \]
\end{section-problem}

\begin{section-problem}
    Sean $a$, $b$ y $c$ números reales distintos de cero, con $a + b + c \neq 0$.
    Probar que si
    \[\frac{1}{a} + \frac{1}{b} + \frac{1}{c} = \frac{1}{a + b + c},\]
    entonces para $n$ impar se cumple
    \[\frac{1}{a^n} + \frac{1}{b^n} + \frac{1}{c^n} = \frac{1}{a^n + b^n + c^n}.\]
\end{section-problem}

\begin{section-problem}
    Hallar $Q(x)$, si $P\left(Q(x) - 3\right) = 6x + 2$ y $P(x + 3) = 2x + 10$.
\end{section-problem}

\begin{section-problem}
    Con la ayuda del teorema de la raíz racional, encontrar todas las raíces de los siguiente polinomio
    \[2 x^3 - 21 x^2 + 52 x - 21.\]
\end{section-problem}

\begin{section-problem}
    Dado que $m$ y $n$ son raíces del polinomio $6x^2 - 5x - 3$, encuentra un polinomio cuyas raíces sean
    $m - n^2$ y $n - m^2$, sin calcular los valores de $m$ y $n$.
\end{section-problem}

\begin{section-problem}
    Si tenemos que
    \[
        \left\{
        \begin{array}{rcl}
            P(x) & =& 3x^2 - 2x \\
            Q(x) & =& \dfrac{x - 1}{3} \\
            R(x) & =& (P \circ Q)(x) - 673x
        \end{array}
        \right.
    \]
    Calcular el valor de $R(2023)$.
\end{section-problem}

\subsection{Pistas y Soluciones}