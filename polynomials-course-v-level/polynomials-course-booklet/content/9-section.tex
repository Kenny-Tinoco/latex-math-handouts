\newpage
\section{Polinomios simétricos}

\subsection{Polinomios simétricos elementales}

Un polinomio de dos variables $P(x, y)$ es simétrico si $P(x, y) = P(y, x)$ para toda $x$, $y$.
Por ejemplo, $7x^2 - 5xy + 7y^2$ es simétrico, ya que $P(x, y) = 7x^2 - 5xy + 7y^2$ es igual a $P(y, x) = 7y^2 - 5yx + 7x^2$.

Así mismo, un polinmio de tres variables $P(x, y, z)$ es simétrico si $P(x, y, z) = P(x, z, y) = P(y, x, z) = P(y, z, x) = P(z, x, y) = P(z, y, x)$.
Por ejemplo, $13x^5 + 13y^5 + 13z^5 - 8xyz + 1$ es simétrico, ya que
\begin{gather*}
    P(x, y, z) = 13x^5 + 13y^5 + 13z^5 - 8xyz + 1\\
    P(x, z, y) = 13x^5 + 13z^5 + 13y^5 - 8xzy + 1\\
    P(y, x, z) = 13y^5 + 13x^5 + 13z^5 - 8yxz + 1\\
    P(y, z, x) = 13y^5 + 13z^5 + 13x^5 - 8yzx + 1\\
    P(z, x, y) = 13z^5 + 13x^5 + 13y^5 - 8zxy + 1\\
    P(z, y, x) = 13z^5 + 13y^5 + 13x^5 - 8zyx + 1
\end{gather*}
son todos iguales.

Análogamente, un polinomio de $n$ variables $P(x_1, x_2, \dots, x_n)$ es simétrico si $P(x_1, x_2, \cdots, x_n) = P(x_1, x_3, \cdots, x_n) = \cdots$, es decir si $P$ evaluado en todas las permutaciones de las $n$ variables da el mismo resultado.

\begin{remark.tcb}
    En el polinomio de ejemplo $7x^2 - 5xy + 7y^2$, veamos que $P(x, 1) =  P(1, x)$.
    Es decir, $7x^2 - 5x + 7$ es un polinomio simétrico o \textbf{recíproco}, esta es la manera en la que se presentaron los polinomios recíprocos en la primera sección\footnote{Ver página~\pageref{reci-polynomial}.}.
\end{remark.tcb}

Resulta que los polinomios simétricos más simples son los que tienen sus variables con grado 1, como por ejemplo $x + y$, $xy$, $x + y + z$, etc.
Esto se puede llevar a la formalización, por lo cual veamos la siguiente definición.

\begin{section-definition.tcb}[Polinomio simétrico elemental]
    Sea $k \in \left\{ 1, 2, \cdots, n \right\}$.
    El $k$-ésimo polinomio simétrico elemental en las variables $x_1, \cdots, x_n$ es el polinomio\footnote{$\sigma_k$: Se lee sigma sub k.} $\sigma_k$ definido por
    \[
        \sigma_k = \sigma_k(x_1, x_2, \cdots, x_n) = \Sigma x_{i_1} x_{i_2} \cdots x_{i_k}
    \]
    donde la suma se realiza sobre los subconjuntos $\left\{ i_1, i_2, \cdots, i_k \right\}$ de tamaño $k$ del conjunto $\left\{ 1, 2, \cdots, n\right\}$.
\end{section-definition.tcb}

Para más claridad veamos algunos casos:

\begin{table}[H]
    \centering
    \begin{tabular}{|p{2.5cm} | p{12cm}|}
        \hline
        Variables & Polinomios simétricos elementales \\
        \hline \hline
        $a$, $b$ &
        $
        \begin{array}{lcl}
            \sigma_1 &=& a + b\\
            \sigma_2 &=& ab
        \end{array}
        $
        \\\hline
        $a$, $b$, $c$
        &
        $
        \begin{array}{lcl}
            \sigma_1 &=& a + b + c\\
            \sigma_2 &=& ab + bc + ca\\
            \sigma_3 &=& abc
        \end{array}
        $
        \\\hline
        $a$, $b$, $c$, $d$
        &
        $
        \begin{array}{lcl}
            \sigma_1 &=& a + b + c + d\\
            \sigma_2 &=& ab + ac + ad + bc + bd + da\\
            \sigma_3 &=& abc + bcd + cda\\
            \sigma_4 &=& abcd
        \end{array}
        $
        \\\hline
        $a$, $b$, $c$, $d$, $e$
        &
        $
        \begin{array}{lcl}
            \sigma_1 &=& a + b + c + d + e\\
            \sigma_2 &=& ab + ac + ad + ae + bc + bd + be + cb + ce + de\\
            \sigma_3 &=& abc + abd + abe + acd + ace + ade + bcd + bce + bde + ced\\
            \sigma_4 &=& abcd + abce + abde + acde + bced\\
            \sigma_5 &=& abcde
        \end{array}
        $
        \\\hline
    \end{tabular}
\end{table}

\begin{section-example.tcb}
    Hallar el valor $x^2 + y^2$ sabiendo que
    \[
        \left\{
        \begin{array}{rcl}
            x + y & =& 1\\
            x^4 + y^4 & =& 7
        \end{array}
        \right.
    \]
\end{section-example.tcb}
\begin{solution}
    De la primera ecuación $x + y = 1$ al cuadrado obtenemos que $x^2 + 2xy + y^2 = 1$, es decir $x^2 + y^2 = 1 - 2xy$.
    Por lo tanto sólo basta encontrar el valor de $xy$ para solucionar el problema.
    Veamos que
    \begin{gather*}
        (x^2 + y^2)^2 = (1 - 2xy)^2 \\
        x^4 + 2x^2 y^2 + y^4 = 1 - 4xy + 4x^2 y^2\\
        x^4 + y^4 = 2x^2 y^2 - 4xy + 1\\
        7 = 2x^2 y^2 - 4xy + 1\\
        2x^2 y^2 - 4xy - 6 = 0\\
        2(x^2 y^2 - 2xy - 3) = 0\\
        2(xy - 3)(xy + 1) = 0
    \end{gather*}
    Es decir, $xy$ puede tomar los valores de $3$ y $-1$.
    Luego, $x^2 + y^2$ puede tomar los valores de $\boxed{-5}$ y $\boxed{3}$.
\end{solution}

Otra notación útil para polinomios simétricos, es la siguiente.

\begin{section-definition.tcb}[Suma simétrica de potencias]
    La suma simétrica de la $k$-ésimas potencias en las variables $x_1, \cdots, x_n$ es el polinomio $s_k$ definido por
    \[s_k = s_k(x_1, x_2, \cdots, x_n) = x_1^k + x_2^k + \cdots + x_n^k, \quad \forall k \in \Z^{\geq 0}.\]
\end{section-definition.tcb}

Algunas relaciones clásicas entre las sumas simétricas de potencias y los polinomios simétricos elementales, como por ejemplo $\boxed{s_2 = \sigma_1^2 - 2\sigma_2}$ para cualquier cantidad de variables.
La relación $\boxed{s_k = \sigma_1 s_{k - 1} - \sigma_2 s_{k - 2}}$ para dos variables y $\boxed{s_k = \sigma_1 s_{k - 1} - \sigma_2 s_{k - 2} + \sigma_3 s_{k - 3}}$ para tres variables.

Estas relaciones casos particulares de las llamadas \textbf{Fórmulas de Newton}.
Se invita al estudiante investigar por su propia cuenta sobre estas fórmulas.



\subsection{Fórmulas de Vieta}

Ya conociendo los polinomios simétricos elementales, podemos volver a ver la defición de las fórmulas de Vieta que ya conocemos\footnote{Ver~\refDefinition{\ref{vietas-formula}}, página \pageref{vietas-formula}.}.

\begin{section-definition.tcb}[Fórmulas de Vieta]
    Sea $P(x) = a_n x^n + a_{n - 1} x^{n - 1} + \cdots  + a_1 x + a_0$ un polinomio con raíces $r_1, r_2, \cdots, r_n$,
    entonces
    \begin{gather*}
        \sigma_k = (-1)^k\times \frac{a_{n - k}}{a_n}.
    \end{gather*}
\end{section-definition.tcb}

\begin{section-example.tcb}
    Determine el producto de las raíces de $50x^{50} + 49x^{49} + \cdots + x + 1.$
\end{section-example.tcb}
\begin{solution}
    Por las fórmulas de Vieta, tenemos que
    \[
        \sigma_{50} = (-1)^{50} \times \frac{a_0}{a_{50}} = \boxed{\frac{1}{50}}. \qedhere
    \]
\end{solution}



\subsection{Ejercicios y Problemas}

Sección de ejercicios y problemas para el autoestudio.

\begin{section-problem}
    Consideremos el polinomio $P(x) = x^n - (x - 1)^n$, donde $n$ es un entero positivo impar.
    Encontrar el valor de la suma y el valor del producto de sus raíces.
\end{section-problem}

\begin{section-problem}
    Encontrar los números reales $x$ y $y$, que satisfacen
    \[
        \left\{
        \begin{array}{rcl}
            x^3 + y^3 & =& 7\\
            x^2 + y^2 + x + y + xy & =& 4
        \end{array}
        \right.
    \]
\end{section-problem}

\begin{section-problem}
    Encontrar todas las soluciones reales del sistema
    \[
        \left\{
        \begin{array}{rcl}
            x + y + z & =& 1\\
            x^3 + y^3 + z^3 + xyz & =& x^4 + y^4 + z^4 + 1
        \end{array}
        \right.
    \]
\end{section-problem}

\begin{section-problem}
    Sean $x$, $y$ y $z$ números reales, encontrar todas las soluciones del siguiente sistema de ecuaciones
    \[
        \left\{
        \begin{array}{rcl}
            x + y + z & =& 6\\
            x^2 + y^2 + z^2 & =& 30\\
            x^3 + y^3 + z^3 & =& 132
        \end{array}
        \right.
    \]
\end{section-problem}

\begin{section-problem}
    Si $\alpha + \beta + \gamma = 0$, demostrar que
    \[3 (\alpha^2 + \beta^2 + \gamma^2) (\alpha^5 + \beta^5 + \gamma^5) = 5 (\alpha^3 + \beta^3 + \gamma^3) (\alpha^4 + \beta^4 + \gamma^4).\]
\end{section-problem}

\begin{section-problem}
    Sean $a$, $b$ y $c$ números reales distintos de cero, con $a + b + c \neq 0$.
    Probar que si
    \[\frac{1}{a} + \frac{1}{b} + \frac{1}{c} = \frac{1}{a + b + c},\]
    entonces para $n$ impar se cumple
    \[\frac{1}{a^n} + \frac{1}{b^n} + \frac{1}{c^n} = \frac{1}{a^n + b^n + c^n}.\]
\end{section-problem}