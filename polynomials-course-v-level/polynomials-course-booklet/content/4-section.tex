\newpage
\section{Fórmulas de Vieta}

\subsection{Definiciones}

\begin{section-definition.tcb}[Fórmulas de Vieta]\label{vietas-formula}
    Para un polinomio de la forma $P(x) = a_n x^n + a_{n - 1} x^{n - 1} + \cdots  + a_1 x + a_0$ con raíces $r_1, r_2, \cdots, r_n$ y $a_n \neq 0$, se denomina como Fórmulas de Vieta a las ecuaciones
    \begin{align*}
        r_1 + r_2 + \cdots + r_n &= - \frac{a_{n - 1}}{a_n}\\
        r_1 r_2 + r_1 r_3 + \cdots + r_{n - 1} r_n &= \frac{a_{n - 2}}{a_n}\\
        r_1 r_2 r_3 + r_1 r_2 r_4 + \cdots + r_{n - 2} r_{n - 1} r_n &= -\frac{a_{n - 3}}{a_n}\\
        \vdots\\
        r_1 r_2 \cdots r_n &= (-1)^n \frac{a_0}{a_n}
    \end{align*}
\end{section-definition.tcb}

Veamos que por el~\refTheorem{\ref{factor-theorem}}, podemos escribir a $P$ como
\[
    P(x) = a_n x^n + a_{n - 1} x^{n - 1} + \cdots  + a_1 x + a_0 = a_n (x - r_1)(x - r_2) \cdots (x - r_n)
\]
Cuando expandimos los $n$ factores lineales del lado derecho de la ecuación, obtenemos
\[
    a_n x^n - a_n(r_1 + r_2 + \cdots + r_n)x^{n - 1} + a_n(r_1 r_2 + r_1 r_3 + \cdots + r_{n - 1} r_n)x^{n - 2} + \cdots + (-1)^n a_n r_1 r_2 \cdots r_n
\]
donde el signo del coeficiente $x^k$ está dado por $(-1)^{n - k}$.
Al comparar los coeficientes podemos ver que
\begin{align*}
    a_{n - 1} = - a_n(r_1 + r_2 + \cdots + r_n) &\to& r_1 + r_2 + \cdots + r_n = - \frac{a_{n - 1}}{a_n}\\
    a_{n - 2} =   a_n (r_1 r_2 + r_1 r_3 + \cdots + r_{n - 1} r_n) &\to& r_1 r_2 + r_1 r_3 + \cdots + r_{n - 1} r_n = \frac{a_{n - 2}}{a_n}\\
    \vdots\\
    a_{0} = - a_n (r_1 r_2 \cdots r_n)&\to& r_1 r_2 \cdots r_n = (-1)^n \frac{a_0}{a_n}
\end{align*}

Las fórmulas de Vieta son de mucha utilidad para calcular expresiones que involucran las raíces de un polinomio sin tener que calcular las propias raíces.
Ya que dichas raíces se puede dejar en función de los coeficientes, los cuales muchas veces ya se conocen de antemano.
En particular es útil conocer la fórmulas de Vieta para polinomios cuadráticos y cúbicos, gran variedad de los problemas que veremos las involucran.

\begin{multicols}{2}
    \begin{enumerate}
        \item Sea el polinimio cuadrático \[P(x) = a_2 x^2 + a_1 x + a_0\] con raíces $r_1$ y $r_2$, entonces se cumple que
        \[
            \begin{cases}
                r_1 + r_2 = - \frac{a_1}{a_2} \\
                r_1 r_2 = \frac{a_0}{a_2}
            \end{cases}
        \]
        \\
        \item Sea el polinimio cúbico \[P(x) = a_3 x^3 + a_2 x^2 + a_1 x + a_0\] con raíces $r_1, r_2$ y $r_3$, entonces se cumple que
        \[
            \begin{cases}
                r_1 + r_2 + r_3 = - \frac{a_2}{a_3}\\
                r_1 r_2 + r_2 r_3 + r_3 r_1 = \frac{a_1}{a_3}\\
                r_1 r_2 r_3 = - \frac{a_0}{a_3}
            \end{cases}
        \]
    \end{enumerate}
\end{multicols}

Ahora veamos algunos ejemplos de cómo se utilizan las formulas de Vieta en problemas de polinomios.

\begin{section-example.tcb}
    Si $p$ y $q$ son raíces de la ecuación $x^2 + 2bx + 2c = 0$, determina el valor de $\frac{1}{p^2} + \frac{1}{q^2}.$
\end{section-example.tcb}
\begin{solution}
    Aplicando las fórmulas de Vieta
    \begin{gather*}
        p + q = - \frac{2b}{1} = - 2b \\
        pq = \frac{2c}{1} = 2c
    \end{gather*}
    La expresión que nos piden hallar puede ser transformada de la siguente manera
    \[
        \frac{1}{p^2} + \frac{1}{q^2} = \frac{p^2 + q^2}{p^2 q^2} = \frac{p^2 + 2pq + q^2 - 2pq}{p^2 q^2} = \frac{ (p + q)^2 - 2pq}{(pq)^2}
    \]
    Finalmente, al sustituir llegamos a
    \[
        \frac{ (p + q)^2 - 2pq}{(pq)^2} = \frac{ (-2b)^2 - 2(2c)}{(2c)^2} = \boxed{\frac{b^2 - c}{c^2}} \qedhere
    \]
\end{solution}

\begin{section-example.tcb}
    Encuentre el valor de $\frac{1}{r_1} + \frac{1}{r_2} + \frac{1}{r_3}$ si $r_1, r_2, r_3$ son raíces de $P(x) = x^3 - 7 x^2 + 3x + 1$.
\end{section-example.tcb}
\begin{solution}
    Por la fórmulas de Vieta sabemos lo siguiente
    \begin{gather*}
        r_1 r_2 + r_2 r_3 + r_3 r_1 = \frac{3}{1} = 3 \\
        r_1 r_2 r_3 = - \frac{1}{1} = -1
    \end{gather*}
    Ahora, notemos que
    \[\frac{r_1 r_2 + r_2 r_3 + r_3 r_2}{r_1 r_2 r_3} = \frac{r_1 r_2}{r_1 r_2 r_3} + \frac{r_2 r_3}{r_1 r_2 r_3} + \frac{r_3 r_1}{r_1 r_2 r_3} = \frac{1}{r_3} + \frac{1}{r_1} + \frac{1}{r_2}\]
    Luego $\frac{1}{r_3} + \frac{1}{r_1} + \frac{1}{r_2} = \frac{3}{-1} = \boxed{-3}.$
\end{solution}

\begin{section-example.tcb}[1986 URSS]
    Las raíces del polinomio $x^2 + ax + b + 1$ son números naturales. Mostrar que $a^2 + b^2$ no es un primo.
\end{section-example.tcb}
\begin{solution}
    Para demostrar que $a^2 + b^2$ no es primo, entonces tendremos que demostrar que es igual a producto de dos enteros mayores a 1.
    Digamos entonces que $r_1$ y $r_2$ son las raíces, por las Fórmulas de Vieta
    \begin{gather*}
        r_1 + r_2 = -a\\
        r_1 r_2 = b + 1
    \end{gather*}
    Que al elevar al cuadrado y sumar tenemos $a^2 + b^2 = (r_1 + r_2)^2 + (r_1 r_2 - 1)^2$.
    Desarrollando
    \begin{gather*}
        a^2 + b^2 = r^2_1 + 2r_1 r_2 + r^2_2 + r^2_1 r^2_2 - 2r_1 r_2 + 1 \\
        a^2 + b^2 = r^2_1 + r^2_2 + r^2_1 r^2_2+ 1 = \boxed{(r^2_1 + 1)(r^2_2 + 1)}
    \end{gather*}
    Por dato del problemas sabemos que $r_1$ y $r_2$ son naturales así $(r^2_1 + 1)$ y $(r^2_2 + 1)$ son mayores a 1.
    Luego la prueba está hecha.
\end{solution}

\begin{section-example.tcb}[2008 AIME II \#7]
    Sean $r, s$ y $t$ las tres raíces de la ecuación
    \[8x^3 + 1001x + 2008 = 0.\]
    Encontrar $(r + s)^3 + (s + t)^3 + (t + r)^3.$
\end{section-example.tcb}
\begin{solution}
    Por las fórmulas de Vieta, sabemos lo siguiente
    \begin{gather*}
        r +  s + t = - \frac{0}{8} = 0 \\
        rs +  st + tr = \frac{1001}{8} \\
        r s t = - \frac{2008}{8} = -251
    \end{gather*}
    Por la primera ecuación, podemos ver inmediantamente que $r + s = -t$ y por lo tanto $(r + s)^3 = - t^3$.
    Análogamente, llegaremos a
    \[(r + s)^3 + (s + t)^3 + (t + r)^3 =  -(r^3 + s^3 + t^3)\]
    Para evitar calcular los valores individuales de las raíces, podemos auxiliarnos de indentidades algebraicas
    que nos relacionan $r^3 + s^3 + t^3$ con la expresiones $r +  s + t$, $rs +  st + tr$ y $rst$ cuyos valores ya conocemos.
    Una de estas identidades es\footnote{Se deja como ejercicio al lector la demostración de esta indentidad. Así como la búsqueda de otras identidades similares.}
    \[(x + y + z)^3 = x^3 + y^3 + z^3 + 3(x + y)(y + z)(z + x)\]
    De donde rápidamente vemos que
    \begin{align*}
        -(r^3 + s^3 + t^3) &= - (r + s + t)^3 + 3(r + s)(s + t)(t + r)\\
        -(r^3 + s^3 + t^3) &= -(0)^3 + 3(-t)(-r)(-s)\\
        -(r^3 + s^3 + t^3) &= - 3rst = -3(-251) = \boxed{753} \qedhere
    \end{align*}
\end{solution}



\subsection{Agregados culturales y preguntas}

\begin{enumerate}
    \item IMO son las siglas de \textit{International Mathematical Olympiad}. La IMO, básicamente, es la olimpiada mundial más dificil e importante en la que Nicaragua participa.
    \item El método de Descenso Infinito de Fermat es una propiedad de los enteros no negativos. El cual dice que no puede existir una secuencia $n_1 > n_2 > \cdots$ con $n_i \in \Z^{\geq 0}$.
    \item Existe una técnica llamada \textbf{Vieta Jumping} la cual es una combinación de las fórmulas de Vieta y el método de Descenso Infinito de Fermat.
    Esta técnica es muy útil en problemas nivel IMO.
\end{enumerate}

\subsection{Ejercicios y Problemas}

Sección de ejercicios y problemas para el autoestudio.

\begin{section-exercise}
    Sean $r_1$ y $r_2$ las raíces del polinomio $P(x) = ax^2 + bx + c$.
    Encontrar los siguientes valores en función del los coeficientes de $P.$
    \begin{multicols}{3}
        \begin{enumerate}
            \item $r_1 - r_2$
            \item $\frac{1}{r_1} - \frac{1}{r_2}$
            \item $r^2_1 + r^2_2$
            \item $r^3_1 + r^3_2$
            \item $(r_1 + 1)^2 + (r_2 + 1)^2$
        \end{enumerate}
    \end{multicols}
\end{section-exercise}

\begin{section-problem}
    Sean $a$ y $b$ las raíces de la ecuación $x^2 - 6x + 5 = 0$, encuentra $(a + 1)(b + 1).$
\end{section-problem}

\begin{section-problem}
    Dado que $m$ y $n$ son raíces del polinomio $6x^2 - 5x - 3$, encuentra un polinomio cuyas raíces sean
    $m - n^2$ y $n - m^2$, sin calcular los valores de $m$ y $n$.
\end{section-problem}

\begin{section-problem}
    ¿Para qué valores reales positivos de $m$, las raíces $x_1$ y $x_2$ de la ecuación
    \[
        x^2 - \left( \frac{2m - 1}{2} \right)x  + \frac{m^2 - 3}{2} = 0
    \]
    cumplen que $x_1 = x_2 - \frac{1}{2}$?
\end{section-problem}
