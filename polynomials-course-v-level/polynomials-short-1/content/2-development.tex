\section*{\large Problemas}

\vspace{-3mm}

Estimado estudiante, resolver los siguientes ejercicios de manera clara y ordenada.

\begin{exercise}
    Dado los polinomios
    \begin{multicols}{2}
        \begin{enumerate}
            \item $4x^3 - 1$
            \item $x^{-2} - 4x^{-1} + 4$
            \item $3y^2 + y^3 + 1$
            \item $8t^3 - 9t^2 + 9t - 8$
            \item $x^2 + 2xy + y^2$
            \item $3s^4 - 6t^4 + 8 s^{2} t^{2} + s^{3} t + 5 st^{3} $
            \item $2x^2 - 2x + 2$
            \item $-x^3 - 1 + x^4 - x^2 + x - x^5$
            \item $x - 2\sqrt{x} + 1$
            \item $x^3 + 3x^2 + 3x + 1$
        \end{enumerate}
    \end{multicols}
    \vspace{-5mm}
    Escriba como repuesta la letra de los polinomios que cumplen lo siguiente
    \begin{multicols}{3}
        \begin{enumerate}
            \item Homogéneo: \rule{1.5cm}{0.1mm}
            \item Multivariable: \rule{1.3cm}{0.1mm}
            \item Mónico: \rule{1.5cm}{0.1mm}
            \item Completo: \rule{1.5cm}{0.1mm}
            \item Ordenado: \rule{1.5cm}{0.1mm}
            \item Simétrico: \rule{1.5cm}{0.1mm}
        \end{enumerate}
    \end{multicols}
\end{exercise}

\begin{exercise}
    Simplifique $S(x) = x^2(x^2 - 7)^3 + (13 - 2x)(3x + x^7)$ y responda lo siguiente

    \begin{multicols}{2}
        \begin{enumerate}
            \item ¿Es mónico? R: \rule{1cm}{0.1mm}
            \item ¿Es completo? R: \rule{1cm}{0.1mm}
            \item ¿Es simétrico? R: \rule{1cm}{0.1mm}
            \item Escriba el coeficiente de $x^4$. \\R: \rule{1cm}{0.1mm}
            \item Escriba el término independiente. \\R: \rule{1cm}{0.1mm}
        \end{enumerate}
    \end{multicols}
\end{exercise}

\begin{exercise}
    Si tenemos que
    \vspace{-5mm}
    \begin{align*}
        P(x) = 3x^2 - 2x \\
        Q(x) = \frac{x - 1}{3} \\
        R(x) = (P \circ Q)(x) - 673x
    \end{align*}
    ¿Cuál es el valor de $R(2023)$?\footnote{Justificar la respuesta.}
    \begin{multicols}{5}
        \begin{enumerate}
            \item -4
            \item 2023
            \item 12
            \item 0
            \item 1
        \end{enumerate}
    \end{multicols}
\end{exercise}