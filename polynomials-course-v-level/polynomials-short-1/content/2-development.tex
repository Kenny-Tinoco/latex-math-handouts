\section*{\large Problemas}

Estimado estudiante, resolver los siguientes ejercicios de manera clara y ordenada.

\begin{exercise}
    Dado el polinomio $S(x) = x^2(x^2 - 7)^3 + (13 - 2x)(3x + x^7)$, responda lo siguiente:
    \begin{multicols}{2}
        \begin{enumerate}
            \item ¿$S(x)$ es mónico? R: \rule{1cm}{0.1mm}
            \item ¿$S(x)$ es completo? R: \rule{1cm}{0.1mm}
            \item ¿$S(x)$ es simétrico? R: \rule{1cm}{0.1mm}
            \item Escriba el coeficiente de $x^4$. R: \rule{1cm}{0.1mm}
            \item Escriba el término independiente.\\ R: \rule{1cm}{0.1mm}
            \item ¿Qué pasa con $S(\sqrt{x})$?. R: \rule{1cm}{0.1mm}
        \end{enumerate}
    \end{multicols}
\end{exercise}

\begin{exercise}
    Si tenemos que
    \begin{align*}
        P(x) = 3x^2 - 2x \\
        Q(x) = \frac{x - 1}{3} \\
        R(x) = (P \circ Q)(x) - 673x
    \end{align*}
    ¿Cuál es el valor de $R(2023)$?\footnote{Justificar la respuesta.}
    \begin{multicols}{5}
        \begin{enumerate}
            \item -4
            \item 2023
            \item 12
            \item 0
            \item 1
        \end{enumerate}
    \end{multicols}
\end{exercise}

\newpage

\section*{\large Soluciones}
{
    \textbf{Ejercicio 1.}
    \[S(x) = x^2(x^2 - 7)^3 + (13 - 2x)(3x + x^7)\]
    \[S(x) = x^2 \left[ (x^2)^3 + 3 (x^2)^2(-7) + 3 (x^2)(-7)^2 + (-7)^3 \right] + (39x + 13x^7 - 6x^2 - 2x^8)\]
    \[S(x) = x^2 \left[ x^6 - 21x^4 + 147 x^2 - 343 \right] + (39x + 13x^7 - 6x^2 - 2x^8)\]
    \[S(x) = (x^8 - 21x^6 + 147 x^4 - 343x^2 + 39x) + (13x^7 - 6x^2 - 2x^8)\]
    \[S(x) = (x^8 - 2x^8) + 13x^7 - 21x^6 + 147 x^4 + (- 343x^2 - 6x^2) + 39x\]
    \[S(x) = -x^8 + 13x^7 - 21x^6 + 147 x^4 - 349x^2 + 39x\]

        \begin{enumerate}
            \item No, ya que el su término principal es $-1$.
            \item No, ya que faltan los términos de $x^5$, $x^3$.
            \item No, simplemente no.
            \item El coeficiente es 147.
            \item El coeficiente independiente es cero (0).
            \item Sabemos por definición que un polinomio tiene sus variables con exponentes \textbf{enteros} no negativos. Al analizar el término lineal de $S(\sqrt {x})$ vemos que $39\sqrt {x} = 39 x^{\frac{1}{2}}$, lo cual no cumple.
        \end{enumerate}

    \textbf{Ejercicio 2.} Primero encontremos $(P \circ Q)(x)$
    \[ (P \circ Q)(x) = P(Q(x)) = 3\left( \frac{x - 1}{3} \right)^2 - 2\left( \frac{x - 1}{3} \right)\]
    \[ (P \circ Q)(x) = 3\left( \frac{x^2 - 2x + 1}{9} \right) - \frac{2x - 2}{3}\]
    \[ (P \circ Q)(x) = \frac{x^2 - 2x + 1}{3} - \frac{2x - 2}{3} = \frac{x^2 - 2x + 1 - (2x - 2)}{3}\]
    \[ (P \circ Q)(x) = \frac{x^2 - 4x + 3}{3}\]
    Luego sustituimos $(P \circ Q)(x)$ en $R(x)$
    \[R(x) = (P \circ Q)(x) - 673x = \frac{x^2 - 4x + 3}{3} - 673x\]
    \[R(x) = \frac{x^2 - 4x + 3 - 2019x}{3} = \frac{x^2 - 2023x + 3}{3}\]
    Finalmente, evaluamos $R(2023)$
    \[R(2023) = \frac{(2023)^2 - 2023\times 2023 + 3}{3} = \frac{0 + 3}{3} = 1\]
    Respuesta correcta es la opción 'e'.
}\label{sec:soluciones}