\section{Clase 06}
    
    \begin{section-problem}[\cite{BLW22}. Example 4.4. Page 7]
        Sea $P$ un polinomio con coeficientes enteros tal que $P(1) = 2$, $P(2) = 3$ y $P(3) = 2016$.
        Si $n$ es el menor valor positivo posible de $P(2016)$, encontrar el resto cuando $n$ es dividido por 2016.
    \end{section-problem}

\section{Clase 07}

    \begin{section-problem}[\cite{Eng97}. Problem 15. Page 255]
        Sea $N(x) = (1 - x + x^2 - \cdots + x^{100})(1 + x + x^2 + \cdots + x^{100})$.
        Probar que después de multiplicar y reducir términos solo quedan potencias pares de $x$.
    \end{section-problem}

    \begin{section-problem}[\cite{BLW22}. Problem 6.2. Page 12]
        Sea $R(x) = 15x - 2016$.
        Si $R^5(x) = R(x)$, encontrar la suma de todos los posibles valores de $x$.
    \end{section-problem}

    \begin{section-problem}[\cite{Ihatemath123}. Exercise 14. Page 12]
        Sean $r$ y $s$ raíces reales distintas de $P(x) = x^3 + ax + b$.
        También, sean $r + 4$ y $s - 3$ raíces de $Q(x) = x^3 + ax + b + 240$.
        Encontrar la suma de todos los posible valores de $|b|$.
    \end{section-problem}

\section{Clase práctica #2}

    \begin{section-problem}
        Si $P(x) = x^4 + ax^3 + bx^2 + cx + d$ es un polinomio tal que $P(1) = 10$, $P(2) = 20$ y $P(3) = 30$, determine el valor de
        \[\frac{P(12) + P(-8)}{10}.\]
    \end{section-problem}

    \begin{section-problem}[\cite{RC08}. Problem 8.25. Page 253\footnote{El el archivo pdf es la página 273.}.]
        Sea $r$ una raíz de $x^2 - x + 7$. Hallar el valor de $r^3 + 6r + \pi$.
    \end{section-problem}

    \begin{section-problem}[\cite{RC08}. Problem 8.27. Page 254.]
        Sean $a$, $b$ y $c$ las raíces reales de la ecuación $x^3 + 3x^2 - 24x + 1 = 0$.
        Probar que $\sqrt[3]{a} + \sqrt[3]{b} + \sqrt[3]{c} = 0$.
    \end{section-problem}

    \begin{section-problem}[\cite{NF21}. Exercise 3.14. Page 12]
        La ecuación $2^{333x - 2} + 2^{111x + 2} = 2^{222x + 1} + 1$ tiene tres raíces reales.
        Dado que su suma es $\dfrac{m}{n}$ con $m, n \in \ZP$ y $mcd(m, n) = 1$.
        Calcular $m + n$.
    \end{section-problem}

    \begin{section-problem}[\cite{NF21}. Exercise 3.15. Page 12]
        Sea el polinomio $P_0(x) = x^3 + 313x^2 - 77x - 8$.
        Para enteros $n \geq 0$, definimos $P_n(x) = P_{n - 1}(x - n)$.
        ¿Cuál es el coeficiente de $x$ en $P_{20}(x)$?
    \end{section-problem}

    \begin{section-problem}[\cite{Ihatemath123}. Exercise 13. Page 12]
        Sean $r_1$, $r_2$ y $r_3$ raíces distintas del polinomio $y^3 - 22 y^2 + 80 y - 67$.
        De tal manera que existen números reales $\alpha$, $\beta$ y $\theta$ tal que
        \[\frac{1}{y^3 - 22 y^2 + 80 y - 67} = \frac{\alpha}{y - r_1} + \frac{\beta}{y - r_2} + \frac{\theta}{y - r_3}\]
        $\forall y \notin \left\{ r_1, r_2, r_3 \right\}$.
        ¿Cuál es valor de $\frac{1}{\alpha} + \frac{1}{\beta} + \frac{1}{\theta}$?
    \end{section-problem}

    \begin{section-problem}[\cite{Eng97}. Problem 34. Page 256]
        Sea $F(x)$ un polinomio mónico con coeficientes enteros.
        Probar que si existen cuatro enteros diferentes $a$, $b$, $c$ y $d$ tal que $F(a) = F(b) = F(c)  = F(d) = 5$,
        entonces no existe un entero $k$ tal que $F(k) = 8$.
    \end{section-problem}

    \begin{section-problem}
        Para que la división de $6x^4 - 11x^2 + ax + b$ entre $3x^2 - 3x - 1$ sea exacta, encuentre los valores de $a$ y $b$ apropiados.
    \end{section-problem}

\section{Corto #3}

    \begin{section-problem}
        Para que la división de $x^4 + ax^2 + b$ entre $x^2 + x + 1$ sea exacta, encuentre los valores de $a$ y $b$ apropiados.
    \end{section-problem}

    \begin{section-problem}
        Calcular el producto de los coeficientes del resto que deja $x^{2023} - 1$ entre $x^2 - 4$. (La respuesta se puede dejar indicada)
    \end{section-problem}

    \begin{section-problem}[\cite{NF21}. Example 3.13. Page 12]
        Supongase que las raíces de $x^3 + 3x^2 + 4x - 11 = 0$ son $a$, $b$ y $c$.
        También, que las raíces de $x^3 + rx^2 + sx + t = 0$ son $a + b$, $b + c$ y $c + a$.
        Hallar $t$.
    \end{section-problem}

    \begin{section-problem}[\cite{Ihatemath123}. Page 9]
        Sea el polinomo $f(x) = x^3 + 3x - 1$ con raíces $a$, $b$ y $c$.
        Calcular
        \[\frac{1}{a^3 + b^3} + \frac{1}{b^3 + c^3} + \frac{1}{c^3 + a^3}.\]
    \end{section-problem}

\section{Examen final}
    
    \begin{section-problem}
        Sea $r_1$, $r_2$ y $r_3$ las tres raíces de polinomio cúbico $P$.
        También, que
        \[\frac{P(2) + P(-2)}{P(0)} = 200\]
        La expresión $\dfrac{1}{r_1 r_2} + \dfrac{1}{r_2 r_3} + \dfrac{1}{r_3 r_1}$ puede ser escrito como $\dfrac{m}{n}$ para $m$ y $n$ coprimos.
        Encontrar $m + n$.
    \end{section-problem}

    \begin{section-problem}
        Si $a$ y $b$ son raíces distintas del polinomio $x^2 + 2023x + 2020$, entonces
        \[\frac{1}{a^2 + 2020a + 2020} + \frac{1}{b^2 + 2020b + 2020} = \frac{m}{n},\]
        donde $m$ y $n$ son primos relativos.
        Calcular $m + n$.
    \end{section-problem}