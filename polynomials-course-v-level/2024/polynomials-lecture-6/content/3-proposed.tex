\section{Problemas propuestos}

Los problemas de esta sección es la \textbf{tarea}.
El estudiante tiene el deber de entregar sus soluciones en la siguiente sesión de clase (también se pueden entregar borradores).
Recordar realizar un trabajo claro, ordenado y limpio.

    \begin{exercise}
        Nos dan el polinomio $F(x) = x^3 + \frac{3}{4}x^2 - 4x - 3$, donde vemos que $F(2) = 0$, por tanto 2 es raíz.
        Sin embargo, 2 no divide al término independiente de $F$, es decir, esta raíz viola el teorema de la Raíz Racional.
        Argumente por qué pasa esto.
    \end{exercise}

    \begin{exercise}
        Encontrar las raíces del polinomio $R(x) = 6x^3 + x^2 - 19x + 6$.
    \end{exercise}

    \begin{exercise}
        Encontrar las raíces del polinomio $G(x) = 12x^3 - 107x^2 - 15x + 54$.
    \end{exercise}