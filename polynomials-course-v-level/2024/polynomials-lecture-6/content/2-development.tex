\section{Desarrollo}

\begin{theorem.tcb}{}{}
    Sean $a$ y $b$ dos enteros distintos y $P(x)$ un polinomio con coeficientes enteros, entonces
    \[
        (a - b) \mid P(a) - P(b).
    \]
\end{theorem.tcb}
\begin{proof}
    Sea $P(x) = c_n x^n + \cdots + c_1 x + c_0$, al evaluarlo en $a$, $b$ y restar, encontramos que
    \begin{align*}
        P(a) - P(b) &= c_n a^n + \cdots + c_1 a + c_0 - (c_n b^n + \cdots + c_1 b + c_0)\\[1.2mm]
        &= c_n (a^n - b^n) + c_{n - 1} (a^{n - 1} - b^{n - 1}) + \cdots + c_1(a - b)\\[1.2mm]
        &= (a - b)\left[c_n (a^{n - 1} + a^{n - 2}b +\cdots + b^{n - 1}) + \cdots + c_1\right]
    \end{align*}
    Como el lado derecho de la ecuación es un entero, necesariamente $(a - b) \mid P(a) - P(b)$.
\end{proof}
En particular, vemos que cuando $a$ es una raíz y $b$ es cero, resulta que la raíz divide al término constante del polinomio.
Al generalizar, esto nos conduce al siguiente colorario.

\begin{corollary}
    Sea $z$ un número entero y $P(x)$ un polinomio con coeficientes enteros, entonces
    \[
        P(z) = 0 \iff z \mid P(0).
    \]
\end{corollary}
\begin{proof}
    En efecto, sea $P(x) = a_n x^n + \cdots + a_1 x + a_0$, entonces $a_n z^n + \cdots + a_1 z + a_0 = 0 \iff a_0 = -z(a_n z^{n - 1} + \cdots + a_1)$.
\end{proof}

Cuando el polinomio es mónico, entonces cada raíz racional de $P(x)$ es un entero.
En efecto, sea $\frac{p}{q}$ una raíz, donde $p$ y $q$ son enteros coprimos
\footnote{Es decir $\mcd{p}{q} = 1$. Muchas veces, para escribir $\mcd{a}{b} = d$ solo pondremos $(a,b) = d$.}, entonces
\begin{align*}
    &\sum_{i = 0}^{n} a_i \left(\dfrac{p}{q}\right)^{i} = 0 \iff \sum_{i = 0}^{n - 1} a_i \left(\dfrac{p}{q}\right)^{i} = -\left(\dfrac{p}{q}\right)^{n}\\
    &\iff \dfrac{p^n}{q}= - q^{n - 1} \left(\sum_{i = 0}^{n - 1} a_i \left(\dfrac{p}{q}\right)^{i}\right)\\
    &\iff \dfrac{p^n}{q} = -\left(a_{n - 1} p^{n - 1} + a_{n - 2} p^{n - 2} q + \cdots + a_1 p q^{n - 2} + a_0 q^{n - 1}\right).
\end{align*}
Donde el lado derecho de la ecuación es entera y por tanto $q = 1$.

\begin{theorem.tcb}{Teorema de la raíz racional}{}
    Sea $P(x) = a_n x^n + \cdots + a_0$ un polinomio con coeficientes enteros, y una raíz $\frac{p}{q}$ con $(p,q) = 1$.
    Entonces, se cumple que $p \mid a_0$ y $q \mid a_n$.
\end{theorem.tcb}
\begin{proof}
    En efecto, tenemos que
    \begin{align*}
        a_n \left(\frac{p}{q}\right)^n + a_{n - 1} \left(\frac{p}{q}\right)^{n - 1} + \cdots + a_1 \left(\frac{p}{q}\right) + a_0 = 0\\[1.2mm]
        \iff a_n p^n + a_{n - 1} p^{n - 1}q + \cdots + a_1 p q^{n - 1} + a_0 q^n = 0
    \end{align*}
    Todos los sumandos excepto (posiblemente) $a_n p^n$ son múltiplos de $q$ y todos los sumandos excepto (posiblemente) $a_0 q^n$ son múltiplos de $p$.
    Como $p$ y $q$ dividen a 0, necesariamente $q \mid a_n p^n$ y $p \mid a_0 q^n$.
    Luego, al ser $p$ y $q$ coprimos, se obtiene el resultado.
\end{proof}


\begin{example}
    Encontrar todas las raíces de $x^3 - 2x^2 - 5x + 6$.
\end{example}
\begin{solution}
    Como el polinomio tiene coeficientes enteros, primero acotemos la búsqueda a las raíces racionales.
    Por el teorema de la Raíz racional, las raíces con la forma $\frac{p}{q}$ donde $(p,q) = 1$, deben cumplir que $p \mid 6$ y $q \mid 1$.

    Analizando los divisores de 1 y 6, vemos que $p \in \{\pm 1, \pm 2, \pm 3, \pm6\}$ y $q \in \{-1, 1\}$.
    Por lo tanto, el conjunto de posibles valores para las raíces racionales es
    \[
        \frac{p}{q} \in \left\{ \pm 1, \pm 2, \pm 3, \pm 6 \right\}.
    \]
    Como vemos, obtuvimos un conjunto de ocho posibilidades, pero, ¿cómo sabemos cuáles de estos números es una raíz?
    Para responder esto tenemos dos opciones\footnote{Como hacer este análisis es elección tuya.}, la primera es evaluar valor por valor y ver cuál de ellos cumple.
    La otra es hacer uso de un teorema ya conocido, el teorema del factor.

    Sabemos que $(x - r)$ es factor de $(x^3 - 2x^2 - 5x + 6)$ si y solo si $r$ es una raíz.
    Es decir que, si $r$ es una raíz, entonces la división por $(x - r)$ es exacta, \eg\ cuando $r = 2$, vemos que la división
    \[
        \polylongdiv[style=D]{x^3 - 2x^2 - 5x + 6}{x - 2}
    \]
    es inexacta, por lo que 2 no es una raíz.
    Aplicando este proceso a todos los valores del conjunto, vemos que $r = 1$ es una raíz, ya que
    \[
        \polylongdiv[style=D]{x^3 - 2x^2 - 5x + 6}{x - 1}
    \]
    Ahora, las dos raíces restantes están incluidas en $(x^2 - x - 6) = (x - 3)(x + 2)$.
    Finalmente, las raíces del polinomio son $\left\{ -2, 1, 3 \right\}$.
\end{solution}

\begin{example}
    Sea $P(x)$ un polinomio con coeficientes enteros, que cumple $P(1) = 2$, $P(2) = 3$ y $P(3) = 2016$.
    Si $n$ es un número positivo, tal que es el menor valor posible de $P(2016)$, encontrar el resto cuando $n$ es dividido por 2016.
\end{example}
\begin{solution}
    Como $P(1) = 2$ y $P(2) = 3$, por el teorema del resto obtenemos
    \[
        P(x) = (x - 1)(x - 2)Q(x) + (x + 1).
    \]
    Para algún polinomio $Q(x)$ con coeficientes enteros.
    Evaluando en $x = 3$, encontramos que $2016 = 2Q(3) + 4 \iff Q(3) = 1006$.
    Ahora bien, al evaluar en $x = 2016$, obtenemos que $P(2016) = (2015)(2014)Q(2016) + 2017$.
    Usando el teorema 1.1 con el polinomio $Q(x)$ llegamos a
    \begin{align*}
        2016 - 3 &\mid Q(2016) - Q(3)\\
        2013 &\mid Q(2016) - Q(3)
    \end{align*}
    Para que $P(2016)$ sea positivo debe pasar que $Q(2016) - Q(3) \geq 0$, por lo cual, el mínimo valor de $Q(2016) = Q(3) = 1006$.
    Así, el valor de $n$ es $P(2016) = (2015)(2014)(1006) + 2017$.
    Luego,
    \begin{align*}
        P(2016) &\fullMod{(2015)(2014)(1006) + 2017}{2016}\\
         &\fullMod{(-1)(-2)(1006) + 1}{2016}\\
         &\fullMod{2013}{2016}.
    \end{align*}
    Finalmente, el resto de $n$ dividido por 2016 es 2013.
\end{solution}



\subsection{Ejercicios y problemas}

Ejercicios y problemas para el autoestudio.

    \begin{problem}
        Encontrar todas las raíces racionales del polinomio $x^4 - 4x^3 + 6x^2 - 4x + 1$.
    \end{problem}

    \begin{problem}
        Encontrar todas las raíces racionales del polinomio $6x^4 + x^3 - 3x^2 - 9x - 4$.
    \end{problem}

    \begin{problem}
        Encontrar todas las raíces racionales del polinomio $30x^4 - 133 x^3 - 121x^2 + 189x - 45$.
    \end{problem}

    \begin{problem}
        Encontrar todos los $r$ tal que $12r^4 - 16r^3 > 41r^2 - 69r + 18$.
    \end{problem}

    \begin{problem}
        Si el polinomio $P(x) = x^{n + 2} + Ax^{n + 1} + ABx^n$ es divisible por $C(x) = x^2 - (A + B)x + AB$ con $AB \neq 0$.
        Hallar el valor de $E = \frac{A}{B}$.
    \end{problem}