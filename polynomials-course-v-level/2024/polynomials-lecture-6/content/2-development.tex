\section{Desarrollo}

\begin{theorem.tcb}{}{}
    Si $P(x)$ es un polinomio con coeficientes enteros, entonces $P(a) - P(b)$ es divisible entre $(a - b)$, para cualesquiera enteros distintos $a$ y $b$.
    \[
        (a - b) \mid P(a) - P(b).
    \]
\end{theorem.tcb}

En particular, todas las raíces enteras de $P(x)$ dividen a $P(0)$.
Esto nos conduce a la siguiente propiedad aritmética.

\begin{definition}
    Sea $P(x) = a_n x^n + \cdots + a_1 x + a_0$ en los enteros y sea $z \in \Z$.
    Entonces \[P(z) = 0 \Leftrightarrow z \mid a_0.\]
\end{definition}

En efecto, $a_n z^n + \cdots + a_1 z + a_0 = 0 \Leftrightarrow a_0 = -z(a_n z^{n - 1} + \cdots + a_1)$.
Además, si $a_n = 1$, entonces cada raíz racional de $P$ es un entero.
En efecto, sea $\frac{p}{q}$ una raíz con $p, q \in \Z$ y $mcd(p, q) = 1$.
Entonces
\begin{gather*}
    \dfrac{p^n}{q^n} + a_{n - 1} \dfrac{p^{n - 1}}{q^{n - 1}} + \cdots + a_1 \dfrac{p}{q} + a_0  = 0 \\
    \hookrightarrow \dfrac{p^n}{q} = - a_{n - 1} p^{n - 1} - a_{n - 2} p^{n - 2} q - \cdots - a_1 p q^{n - 2} - a_0 q^{n - 1}
\end{gather*}

El lado derecho de la ecuación es entera, por lo tanto $q = 1$.

\begin{theorem.tcb}{Teorema de la raíz racional}{}
    Sea $P(x) = a_n x^n + \cdots + a_0$ en los enteros y sea $\frac{p}{q}$, con $mcd(p, q) = 1$, una raíz cualquiera de $P$.
    Entonces se cumple que $p \mid a_0$ y $q \mid a_n$.
\end{theorem.tcb}
\begin{proof}
    En efecto, tenemos que
    \begin{gather*}
        a_n \dfrac{p^n}{q^n} + a_{n - 1} \dfrac{p^{n - 1}}{q^{n - 1}} + \cdots + a_1 \dfrac{p}{q} + a_0  = 0 \\
        \hookrightarrow a_n p^n + a_{n - 1}p^{n - 1}q + \cdots + a_1 p q^{n - 1} + a_0 q^n = 0
    \end{gather*}
    Todos los sumandos excepto, posiblemente, el primero, son múltiplos de $q$ y todos los sumandos excepto, posiblemente, el último son múltiplos de $p$.
    Como $p$ y $q$ dividen a 0, se deberá tener que $q \mid a_n p^n$ y $p \mid a_0 q^n$, y de aquí se sigue la afirmación, ya que\footnote{$(p, q) =  1$ es una manera corta de escribir $mcd(p, q) = 1$.} $(p, q) = 1$.
\end{proof}


\begin{example}
    Encontrar todas las raíces de $x^3 - 2x^2 - 5x + 6$.
\end{example}
\begin{solution}
    Primero tratemos de ver las raíces racionales de la ecuación.
    Los divisores del término principal y los divisores del término independiente son
    \begin{gather*}
        \left\{ -1, 1 \right\}\\
        \left\{ \pm 1, \pm 2, \pm 3, \pm6 \right\}
    \end{gather*}
    respectivamente.
    Por el Teorema de la raíz racional, las raíces de la ecuación tiene la forma $\dfrac{p}{q}$, es decir, el conjunto de valores candidatos a ser raíces es (ya simplificado)
    \begin{gather*}
        \left\{ \dfrac{\pm 1}{\pm 1}, \dfrac{\pm 2}{\pm 1}, \dfrac{\pm 3}{\pm 1}, \dfrac{\pm 6}{\pm 1} \right\} =
        \left\{ \pm 1, \pm 2, \pm 3, \pm 6 \right\}
    \end{gather*}
    Pero, ¿cómo sabemos cuáles de estos valores es una raíz?.
    Por el Teorema del factor, sabemos que $x^3 - 2x^2 - 5x + 6$ es divisible por $x - r$, donde $r$ es una raíz.

    Entonces, por ejemplo si $x = -6$ la división $\frac{x^3 - 2x^2 - 5x + 6}{x + 6} = x^2 - 8x + 3 - \frac{12}{x + 6}$ no es exacta, por lo tanto $-6$ no es raíz\footnote
    {También, se puede ir evaluado los valores en el polinomio y ver si da cero. Como hacer este análisis es elección del estudiante.}.
    Este proceso lo aplicamos a todos los valores del conjunto.

    En particular, si $x = 1$ la división $\frac{x^3 - 2x^2 - 5x + 6}{x - 1} = x^2 - x - 6$ en este caso es exacta, por lo tanto 1 es raíz.
    Para términar solo falta encontar las raíces de $x^2 - x - 6 = (x - 3)(x + 2)$, las cuales son $-2$ y 3.
    Luego, las raíces del polinomio son $\left\{ -2, 1, 3 \right\}$.
\end{solution}

\begin{example}
    Sea $P$ un polinomio con coeficientes enteros tal que $P(1) = 2$, $P(2) = 3$ y $P(3) = 2016$.
    Si $n$ es el menor valor positivo posible de $P(2016)$, encontrar el resto cuando $n$ es dividido por 2016.
\end{example}
\begin{solution}
    Rápidamente, por el Teorema del resto
    \[
        P(x) = (x - 1)(x - 2)Q(x) + x + 1.
    \]
    Cuando hacemos $x = 3$, entonces obtenemos $2016 = 2Q(3) + 4$, lo cual es $Q(3) = 1006$.
    Ahora bien, cuando hacemos $x = 2016$, obtenemos $P(2016) = 2015\cdot2014Q(2016) + 2017$.
    Por el Teorema 1.1, sabemos que
    \begin{gather*}
        2016 - 3 \mid Q(2016) - Q(3)\\
        2013 \mid Q(2016) - Q(3)
    \end{gather*}
    El mínimo valor de $Q(2016)$ tal que $P(2016) \geq 0$ es $Q(2016) = Q(3) = 1006$, por lo tanto
    \begin{gather*}
        P(2016) = 2015\cdot2014\cdot1006 + 2017 \equiv (-1)(-2)(1006) + 2017 \fullMod{\boxed{2013}}{2016}
    \end{gather*}
\end{solution}
Otra solución al problema anterior sería por medio del Teorema Chino del Resto.



\subsection{Ejercicios y problemas}

Ejercicios y problemas para el autoestudio.

\showLine
\begin{multicols}{2}
    \begin{problem}
        Encontrar todas las raíces racionales del polinomio $x^4 - 4x^3 + 6x^2 - 4x + 1$.
    \end{problem}

    \begin{problem}
        Encontrar todas las raíces racionales del polinomio $6x^4 + x^3 - 3x^2 - 9x - 4$.
    \end{problem}

    \begin{problem}
        Encontrar todas las raíces racionales del polinomio $30x^4 - 133 x^3 - 121x^2 + 189x - 45$.
    \end{problem}

    \begin{problem}
        Encontrar todos los $r$ tal que $12r^4 - 16r^3 > 41r^2 - 69r + 18$.
    \end{problem}

    \begin{problem}
        Si el polinomio $P(x) = x^{n + 2} + Ax^{n + 1} + ABx^n$ es divisible por $C(x) = x^2 - (A + B)x + AB$ con $AB \neq 0$.
        Hallar el valor de $E = \frac{A}{B}$.
    \end{problem}
\end{multicols}