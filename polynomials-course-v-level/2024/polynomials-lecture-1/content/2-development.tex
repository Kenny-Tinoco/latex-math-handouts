\section{Desarrollo}

\subsection{Definiciones}

Un \textbf{\emph{polinomio}} en $x$ es una expresión de la forma
\[
    P(x) = a_d x^d + a_{d - 1} x^{d - 1} + \ldots + a_1 x + a_0
\]
donde $d$ es un entero mayor o igual que cero.
Los números $a_1, a_2, a_3, \dots, a_d$ son llamados los \textbf{\emph{coeficientes}} de $P(x)$, y estos pueden ser enteros, racionales, reales o complejos.
Por la definición, se sigue que dos polinomios
\[
    P(x) = a_d x^d + a_{d - 1} x^{d - 1} + \ldots + a_1 x + a_0 \quad \text{y} \quad Q(x) = b_e x^e + a_{e - 1} x^{e - 1} + \ldots + b_1 x + b_0
\]
son iguales si y solo si $a_i = b_i$ para todo $i$ (si $d > e$, entonces $b_{e + 1} = \ldots = b_d = 0$).

Definimos el \textbf{\emph{grado}} del polinomio $P(x)$ como el mayor entero $i$ tal que $a_i \neq 0$ y denotamos el grado por $\deg P(x)$.
Si $i$ es el mayor entero $i$ tal que $a_i \neq 0$, decimos que $a_i$ es el \textbf{\emph{coeficiente principal}} de $P(x)$.
Si el coeficiente principal es igual a 1, decimos que el polinomio es \textbf{\emph{mónico}}.

Observemos que el grado de un polinomio constante $P(x) = a_0 \neq 0$ es cero.
No damos ningún grado al \textbf{\emph{polinomio cero}} $P(x) \equiv 0$ (\ie el polinomio cuyos coeficientes son todos ceros).

Podemos realizar algunas operaciones sobre polinomios.
Por ejemplo, si $P(x) = a_n x^n + \ldots + a_1 x + a_0$ y $Q(x) = b_m x^m + \ldots + b_1 x + b_0$ son dos polinomios y $m \geq n$, entonces la suma y el producto de $P(x)$ y $Q(x)$ es definida por
\begin{align*}
    P(x) + Q(x) =& (a_0  + b_0) + (a_1 + b_1)x + (a_2 + b_2)x^2 + \cdots\\
    &+ (a_n + b_n)x^n + b_{n + 1} x^{n + 1} + \cdots + b_m x^m\\[2mm]
    P(x)Q(x) =& a_0 b_0 + (a_0 b_1 + a_1 b_0)x + (a_0 b_2 + a_1 b_1 + a_2 b_0)x^2 + \cdots \\
    &+ (a_0 b_r + a_1 b_{r - 1} + \cdots + a_{r - 1}b_1 + a_r b_0)x^r + \cdots + (a_n b_m)x^{m + n}
\end{align*}
respectivamente.

Se dan nombres especiales a polinomios con grados bajos:
\begin{table}[H]
    \centering
    \begin{tabular}{p{1.5cm} p{2.5cm} p{6.5cm}}
        \hline
        Grado & Nombre & Forma \\
        \hline \hline
        1 & Líneal & $P(x) = a_1 x + a_0$\\
        2 & Cuadrático & $P(x) = a_2 x^2 + a_1 x + a_0$\\
        3 & Cúbico & $P(x) = a_3 x^3 + a_2 x^2 + a_1 x + a_0$\\
        4 & Cuártico & $P(x) = a_4 x^4 + a_3 x^3 + a_2 x^2 + a_1 x + a_0$\\
        \hline
    \end{tabular}
    \caption{Nombre de polinomios comunes.}
\end{table}
Los polinomios de grado dos y tres son de especial interés debido a la gran cantidad de problemas en la que estos aparecen, a lo largo del curso veremos las propiedades más importantes de estos polinomios.

\begin{example}
    Sean $P(x) = x^3 - 5x^2 + 7$ y $Q(x) = -2x^2 + x - 10$ dos polinomios.
    Desarrolle las siguientes operaciones,
    \begin{tasks}[label=\alph*.](4)
        \task $P(x) + Q(x)$
        \task $P(x) Q(x)$
        \task $Q(x) - P(x)$
        \task $11 P(x) - x Q(x)$
    \end{tasks}
\end{example}
\begin{solution}
    a. Sustituyendo los polinomios $P(x) + Q(x) = (x^3 - 5x^2 + 7) + (-2x^2 + x - 10)$, tenemos
    \begin{align*}
        P(x) + Q(x) &= (x^3 + 0x^3) + (-5 x^2 - 2x^2) + (0x + x) + (7 - 10)\\
        &= (x^3) + (-7x^2) + (x) + (-3) = \boxed{x^3 - 7x^2 + x - 3}.
    \end{align*}
    b. Sustituyendo los polinomios $P(x) Q(x) = (x^3 - 5x^2 + 7)(-2x^2 + x - 10)$, tenemos
    \begin{align*}
        P(x) Q(x) &= (x^3)(-2x^2 + x - 10) + (-5x^2)(-2x^2 + x - 10) + (7)(-2x^2 + x - 10)\\
        &= (-2x^5 + x^4 - 10x^3) + (10x^4 - 5x^3 + 50x^2) + (-14x^2 + 7x - 70)\\
        &= -2x^5 + (x^4 + 10x^4) + ( - 10x^3 - 5x^3) + (50x^2 -14x^2) + 7x - 70\\
        &= \boxed{-2x^5 + 11x^4 - 15x^3 + 36x^2 + 7x - 70}.
    \end{align*}
    c. Sustituyendo los polinomios $Q(x) - P(x) = (-2x^2 + x - 10) - (x^3 - 5x^2 + 7)$, tenemos
    \begin{align*}
        Q(x) - P(x) &= (0x^3 - x^3) + (- 2x^2 + 5 x^2) + (x - 0x) + (- 10 - 7)\\
        &= (-x^3) + (3x^2) + (x) + (-17) = \boxed{-x^3 + 3x^2 + x - 17}.
    \end{align*}
    d. Sustituyendo los polinomios $11P(x) - xQ(x) = 11(x^3 - 5x^2 + 7) - x(-2x^2 + x - 10)$, tenemos
    \begin{align*}
        P(x) + Q(x) &= (11x^3 - 55x^2 + 77) - (-2x^3 + x^2 - 10x)\\
        &= (11x^3 + 2x^3) + (-55 x^2 - x^2) + (0x + 10x) + (77 - 0)\\
        &= (13x^3) + (-56x^2) + (10x) + (77) = \boxed{13x^3 - 56x^2 + 10x + 77}.\qquad\qquad\qquad\qedhere
    \end{align*}
\end{solution}


Visto lo anterior, damos paso al siguiente teorema.
\begin{theorem.tcb}{}{}
    Sea $P(x)$ y $Q(x)$ dos polinomios y sea $k$ un entero positivo.
    Entonces,
    \begin{enumerate}
        \item $\deg [P(x) Q(x)] = \deg P(x) + \deg Q(x)$
        \item $\deg [P(x) + Q(x)] \leq \max (\deg P(x), \deg Q(x))$
        \item $\deg [P(x)^k] = k \cdot \deg P(x)$.
    \end{enumerate}
\end{theorem.tcb}

Si tenemos dos polinomios $P(x) = a_n x^n + \ldots + a_1 x + a_0$ y $Q(x) = b_m x^m + \ldots + b_1 x + b_0$ y sustituimos cada $x$ de $P(x)$ por el polinomio $Q(x)$ obtenemos la expresión $P\left(Q(x)\right) = (P \circ Q)(x)$ y decimos que es una \textbf{\emph{composición}} de $P(x)$ con $Q(x)$.
Por ejemplo, si tenemos $P(x) = x^2 - x + 5$ y $Q(x) = 7 - 3x$ vemos que $(P \circ Q)(x) = (7 - 3x)^2 - (7 - 3x) + 5 = (49 - 42x + 9x^2) - 7 + 3x + 5 = 9x^2 - 39x + 47$.

La composición de polinomios es asociativa, es decir, $(P \circ Q) \circ R = P \circ (Q \circ R)$, mas no conmutativa, salvo casos especiales, así que generalmente supondremos que $P \circ Q \neq  Q \circ P$.

\begin{remark.tcb}
    Notemos que $P(x)^n \neq P^n(x)$.
    Entendemos a $P(x)^n$ como $P(x)$ elevado a la $n$-ésima potencia, en cambio $P^n(x)$ es la composición $n$-ésima de $P(x)$ consigo mismo, es decir
    \[
        P^n(x) = \underbrace{P(P(P\dots P(P}_{n\ \text{veces}\ P} (x))\dots)).
    \]
\end{remark.tcb}


Como hemos visto, cualquier polinomio $P(x)$ de grado no negativo tiene una representación \textit{genérica} como la siguiente
\[
    P(x) = a_d x^d + a_{d - 1} x^{d - 1} + \ldots + a_1 x + a_0 = \sum_{i = 0}^{d} a_i x^i,
\]
donde $a_d, a_{d - 1}, \ldots, a_1, a_0$ son números complejos.
El término $a_d x^d$ es llamado el \textbf{\emph{término principal}} y $a_0$ es llamado el \textbf{\emph{término constante}}.
El valor del polinomio en $x = c$, el cual denotamos por $P(c)$, es
\[
    P(c) = a_d c^d + a_{d - 1}c^{d - 1} + \ldots + a_1 c + a_0.
\]
De especial interés son los casos de $P(0)$, $P(1)$ y $P(-1)$.
    $P(0) = a_0$, el cual es el término constante,
\[
    P(1) = a_d + a_{d - 1} + \ldots + a_0
\]
que es llamado la \textbf{\emph{suma de coeficientes}} y $P(-1) = a_0 - a_1 + \ldots + (-1)^d a_d.$

Otras cuestiones sobre polinomios son:
\begin{enumerate}
    \item \textbf{Completo}: Un polinomio completo cumple que $a_i \neq 0$ para todo entero $i$.
    \item \textbf{Ordenado}: Un polinomio está ordenado si sus términos están escritos de manera decreciente (o creciente) respecto al exponente de $x$.
    \item \textbf{Multi variable}: Un polinomio puede estar en términos de más de una variable, por ejemplo el polinomio $P(x, y) = x^3 + 3x^2 y + 3xy^2 + y^3$.
\end{enumerate}

\begin{example}
    Sabiendo que $P(x + 1) = x + 3$, hallar $P(x)$.
\end{example}
\begin{solution}
    \textbf{1ra forma.} Escribiendo $(x + 3)$ en función de $x$, esto es $P(x + 1) = x + 1 + 2$.
    Luego, donde aparezca $(x + 1)$ se colocará $x$, es decir $P(x) = x + 2$.

    \textbf{2da forma.} Haciendo un cambio de variable $y = x + 1 \implies x = y - 1$.
    Escribiendo la expresión original en términos de $y$: $P(y) = y - 1 + 3 \implies P(y) = y + 2$.
    Una vez reducida, se hace $y = x$ y por lo tanto $P(x) = x + 2$.
\end{solution}

\begin{example}
    Sabiendo que los polinomios $P(x)$ y $F(x)$ cumplen $P(x - 3) = 5x - 7$ y $P[F(x) + 2] = 10x - 17$, hallar $F(x - 2)$
\end{example}
\begin{solution}
    En el primer polinomio en lugar de $x$ ponemos $F(x) + 5$:
    \begin{align*}
        P(F(x) + 5 - 3) =&  5 (F(x) + 5) - 7\\
        P(F(x) + 2) =& 5F(x) + 18\\
        10x - 17 =& 5F(x) + 18
    \end{align*}
    Despejando tenemos $F(x) = 2x - 7$.
    Así $F(x - 2)$ será $F(x - 2) = 2(x - 2) - 7 = \boxed{2x - 11}$.
\end{solution}

\begin{example}
    Si $P(x) = ax^2 + b$ y $P^2(x) = 8x^4 + 24x^2 + c$, hallar el valor de $a + b + c$.
\end{example}
\begin{solution}
    Evaluamos $P^2(x) = P(P(x))$,
    \begin{align*}
        P(P(x)) =& a (ax^2 + b)^2 + b\\
        8x^4 + 24x^2 + c =& a (a^2 x^4 + 2abx^2 + b^2) + b\\
        8x^4 + 24x^2 + c =& a^3 x^4 + 2a^2 bx^2 + ab^2 + b
    \end{align*}
    Ya que los polinomios son iguales, por simple inspección vemos que $a^3 = 8$, $2a^2 b = 24$ y $ab^2 + b = c$.
    Una vez resuelto este sistema de ecuaciones, tenemos que $a = 2$, $b = 3$ y $c = 21$, por lo tanto $a + b + c = 26$.
\end{solution}

\begin{example}
    El polinomio $P(x) = ax^4 + bx^3 + cx^2 + dx + e$ es tal que, $P(0) = 0$, $P(-1) = 6$ y $P(x) = P(1 - x)$.
    Hallar el valor de $(2c - b)$
    \begin{tasks}[label=\Alph*)](5)
        \task $2a$
        \task $(3 - a)$
        \task 4
        \task 5
        \task 6
    \end{tasks}
\end{example}

\begin{solution}
    Como $P(0) = 0 \implies e = 0$.
    Además $P(x) = P(1 - x)$, implica que $P(0) = P(1) = 0$
    \begin{equation}
        \implies a + b + c + d = 0.
    \end{equation}
    También $P(-1) = 6$
    \begin{equation}
        \implies a - b + c - d = 6
    \end{equation}
    Y por último $P(-1) = P(2) = 6 \implies 16a + 8b + 4c + 2d = 6$
    \begin{equation}
        \implies 8a + 4b + 2c + d = 3
    \end{equation}
    Por lo hemos obtenido un sistema de ecuaciones.
    Sumando $(1) + (2)$ obtenemos $a + c = 3 \implies a = 3 - c$.
    Restando $(1) - (2)$ obtenemos $b + d = 3 \implies d = 3 - b$.
    Sustituyendo esto dos resultados en $(3)$
    \[
        8(3 - c) + 4b + 2c - 3 - b = \implies -6c + 3b = -18 \implies \boxed{2c - b = 6}. \qedhere
    \]
\end{solution}



\subsection{Ejercicios y problemas}
Ejercicios y problemas para el autoestudio.

\begin{multicols}{2}
    \begin{exercise}
        Calcular la suma de coeficientes de $P(x) = (x - 1)^{20} + (x - 2)^7 + x^3 + 5$
        \begin{tasks}[label=\Alph*)](5)
            \task 3
            \task 5
            \task 7
            \task 9
            \task 11
        \end{tasks}
    \end{exercise}

    \begin{exercise}
        Si $P(x) = (x + 2)^5 + (x - 3)^3 - (x + 2)(x - 3)$, el término independiente de $P$ es
        \begin{tasks}[label=\Alph*)](5)
        \task 15
        \task 13
        \task 11
        \task 10
        \task 9
        \end{tasks}
    \end{exercise}

    \begin{exercise}
        Hallar el término independiente del polinomio de grado siete
        \[
            P(x) = x^{n + 2} + x^{m - 1} + \ldots + mx + (m + n)
        \]
        que es completo y ordenado.
        \begin{tasks}[label=\Alph*)](5)
        \task 16
        \task 12
        \task 8
        \task 10
        \task 6
        \end{tasks}
    \end{exercise}

    \begin{exercise}
        Si el polinomio ordenado decreciente y completo
        \[
            P(x) = x^{2a + 1} + 2x^{b + 3} + 3x^{c + 2} + \ldots
        \]
        posee $2c$ términos, hallar $a + b + c$.
        \begin{tasks}[label=\Alph*)](5)
        \task 12
        \task 13
        \task 14
        \task 15
        \task 16
        \end{tasks}
    \end{exercise}

    \begin{exercise}
        Si $P(x, y)$ es un polinomio cero, hallar $m^n$
        \[
            P(x, y) = (10 - m)x^2 y + nxy^2 + 5x^2 y - 2xy^2.
        \]
        \begin{tasks}[label=\Alph*)](3)
        \task 15
        \task 30
        \task 125
        \task 225
        \task N.A
        \end{tasks}
    \end{exercise}


    \begin{exercise}
        Multiplica y simplifica
        \[
            (2x^2 - x - 3)(1 + 2x - x^2).
        \]
        Escribir la respuesta en potencias ascendentes de $x$.
    \end{exercise}

    \begin{exercise}
        Encontrar el coeficiente de $x^3$ en la expansión de
        \[
            (2x^3 - 5x^2 + 2x - 1)(3x^3 + 2x^2 - 9x + 7).
        \]
    \end{exercise}

    \begin{exercise}
        Multiplica y simplifica
        \[
            (1 + x)(1 + x^2)(1 - x + x^2).
        \]
        Escribir la respuesta en potencias ascendentes de $x$.
    \end{exercise}

    \begin{exercise}
        Nos dan el polinomio
        \[
            f(x) = 2x^3 + 3x^2 - 8x + c,
        \]
        donde $c$ es un número distinto de cero.
        Se da además que $f(-3) = 0$, hallar $c$.
    \end{exercise}

    \begin{exercise}
        Hallar el valor de cada una de las constantes $a, b, c$ tal que
        \[
            6x^3 - 7x^2 - x + 2 = (x - 1)(ax^2 + bx + c).
        \]
    \end{exercise}

    \begin{exercise}
        Un polinomio $f(x)$ es definido en término de las constantes $a, b, c$ como
        \[
            f(x) = 2x^3 + ax^2 + bx + c.
        \]
        Además $f(2) = f(-1) = 0$ y $f(1) = -14$.
        Hallar el valor de $a + bc$
    \end{exercise}

    \begin{exercise}
        Sean los polinomios $P(t) = 3t - 4$ y $Q(t) = 2t^2 - 5t + 8$.
        Encontrar las siguientes operaciones
        \begin{tasks}(2)
            \task $P(t) + Q(t)$
            \task $7P(t) - 6Q(t)$
            \task $P(t)Q(t)$
            \task $(P \circ Q)(t)$
            \task $(Q \circ P)(t)$
        \end{tasks}
    \end{exercise}

    \begin{exercise}
        Sea $Q(x) = \frac{1}{5} x - 2$ y $P(x) = Q^4(x)$, probar que $P(1560) = 0$.
    \end{exercise}

    \begin{exercise}
        Dado que $P(x^3 + x^2) = x^5 + x$, hallar el valor de $P(1)$.
        \begin{tasks}[label=\Alph*)](5)
        \task 5
        \task 4
        \task 3
        \task 2
        \task 1
        \end{tasks}
    \end{exercise}

    \begin{exercise}
        Sabiendo que $P(x^x + 2n + 1) = 6x^x + 12n$ y $P\left(F(x)\right) = 24x + 12$, hallar la expresión de $F(n - 1)$.
        \begin{tasks}[label=\Alph*)](3)
        \task $2n$
        \task $2n - 2$
        \task $4n - 1$
        \task $4(n - \inverseOf{4})$
        \task $4n$
        \end{tasks}
    \end{exercise}

    \begin{exercise}
        Señalar las proposiciones falsas:
        \begin{enumerate}[label=\Roman*.]
            \item Si sumamos un polinomios de grado cuatro con uno de grado seis, entonces el grado del polinomio resultante es seis.
            \item Si restamos dos polinomios del mismo grado, el resultado siempre será de un grado menor.
            \item Si el grado de $P(x)$ es mayor que el grado de $Q(x)$, el que tiene más términos es $P(x)$.
        \end{enumerate}
        \begin{tasks}(2)
            \task I y II
            \task II solamente
            \task I solamente
            \task II y III
            \task Ninguna
        \end{tasks}
    \end{exercise}

    \begin{problem}
        Sea $P(x)$ un polinomio mónico de grado tres, donde su término constante es 5 y se cumple que $P(x + 1) = P(x) + nx + 2$.
        Hallar la suma de los coeficientes del término cuadrático y el término líneal.
    \end{problem}

    \begin{problem}
        Hallar un polinomio cuadrático cuyo coeficiente de $x$ y término independiente son iguales, además cumple que $P(1) = 7$ y $P(2) = 18$.
    \end{problem}

    \begin{problem}
        Hallar el valor de $\alpha_1 + \alpha_2 - \alpha_3$ si se cumple que
        \[
            50x^3 + 5x^2 - 8x + 1 = \alpha_1 (\alpha_2 x + 1)^{\alpha_1} (\alpha_2 x - \alpha_1)^{\alpha_3}.
        \]
    \end{problem}

    \begin{problem}
        Calcular la suma de los coeficientes de siguiente polinomio completo
        \[
            P(x) = c(x^a + x^b) + a(x^b + x^c) + b(x^a + x^c) + abc.
        \]
    \end{problem}

    \begin{problem}
        Si $P(x) = x^2 + xy + xz + yz$, hallar el valor de $E = \sqrt {P(y)P(z)P(0)}$.
    \end{problem}

    \begin{problem}
        Sabiendo que $P(x) = ax + b$ y $P^3(x) = 8x + 154$, determinar el valor de $P^2(3)$.
    \end{problem}

    \begin{problem}
        Dado que
        \begin{align*}
            Q(x) &= 2x + 3 \\
            Q\left[ F(x) + G(x) \right] &= 4x + 3 \\
            Q\left[ F(x) - G(x) \right] &= 7
        \end{align*}
        Calcular $F(G(F(G(\dots F(G(1))\dots))))$.
    \end{problem}
\end{multicols}