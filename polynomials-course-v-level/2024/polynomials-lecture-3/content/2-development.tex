\section{Desarrollo}

\begin{definition}[Fórmulas de Vieta]
    Sea el polinomio $P(x) = a_n x^n + a_{n - 1} x^{n - 1} + \cdots  + a_1 x + a_0$ de grado $n$ con raíces $r_1, r_2, \ldots, r_n$,
    entonces
    \begin{align*}
        r_1 + r_2 + \cdots + r_n &= - \frac{a_{n - 1}}{a_n}\\
        r_1 r_2 + r_1 r_3 + \cdots + r_{n - 1} r_n &= \frac{a_{n - 2}}{a_n}\\
        r_1 r_2 r_3 + r_1 r_2 r_4 + \cdots + r_{n - 2} r_{n - 1} r_n &= -\frac{a_{n - 3}}{a_n}\\
        \vdots\\
        r_1 r_2 \cdots r_n &= (-1)^n \cdot \frac{a_0}{a_n}
    \end{align*}
\end{definition}

Por el teorema del factor, podemos escribir el polinomio como
\[
    a_n x^n + a_{n - 1} x^{n - 1} + \cdots  + a_1 x + a_0 = a_n (x - r_1)(x - r_2) \cdots (x - r_n).
\]
Al expandir los $n$ factores lineales del lado derecho, obtenemos el siguiente resultado
\[
    a_n x^n - a_n(r_1 + r_2 + \cdots + r_n)x^{n - 1} + a_n(r_1 r_2 + r_1 r_3 + \cdots + r_{n - 1} r_n)x^{n - 2} + \cdots + (-1)^n a_n r_1 r_2 \cdots r_n,
\]
donde el signo del coeficiente de $x^k$ está dado por $(-1)^{n - k}$.
Comparando los coeficientes correspondientes obtenemos el resultado deseado.

Las fórmulas de Vieta pueden ser útiles al calcular expresiones que involucran las raíces de un polinomio sin tener que calcular los valores de las propias raíces.
Para ser más prácticos, en esta sesión de clase solo nos centraremos en las fórmulas de Vieta para polinomios cuadráticos y cúbicos.

\begin{remark.tcb}
        Para el polinomio cuadrático $P(x) = a_2 x^2 + a_1 x + a_0$ con raíces $r_1$ y $r_2$, tenemos
        \begin{align*}
            r_1 + r_2 = - \frac{a_1}{a_2} \quad \land \quad
            r_1 r_2 = +\frac{a_0}{a_2}
        \end{align*}
\end{remark.tcb}
\begin{remark.tcb}
    Para el polinomio cúbico $P(x) = a_3 x^3 + a_2 x^2 + a_1 x + a_0$ con raíces $r_1$, $r_2$ y $r_3$, tenemos
    \begin{align*}
        r_1 + r_2 + r_3 = - \frac{a_2}{a_3} \quad \land \quad
        r_1 r_2 + r_2 r_3 + r_3 r_1 = +\frac{a_1}{a_3} \quad \land \quad
        r_1 r_2 r_3 = - \frac{a_0}{a_3}
    \end{align*}
\end{remark.tcb}

\begin{example}
    Si $p$ y $q$ son raíces de la ecuación $x^2 + 2bx + 2c = 0$, determina el valor de $\frac{1}{p^2} + \frac{1}{q^2}.$
\end{example}
\begin{solution}
    Aplicando las fórmulas de Vieta, tenemos $p + q = - 2b$ y $pq = 2c$.
    Podemos transformar la expresión deseada de la siguiente manera,
    \[
        \inverseOf{p^2} + \inverseOf{q^2} = \frac{p^2 + q^2}{p^2 q^2} = \frac{p^2 + 2pq + q^2 - 2pq}{p^2 q^2} = \frac{ (p + q)^2 - 2pq}{(pq)^2}
    \]
    Sustituyendo valores, obtenemos $\dfrac{(p + q)^2 - 2pq}{(pq)^2} = \dfrac{ (-2b)^2 - 2(2c)}{(2c)^2} = \boxed{\dfrac{b^2 - c}{c^2}}$.
\end{solution}

\begin{example}
    El polinomio $x^3 - 7 x^2 + 3x + 1$ tiene raíces $r_1, r_2$ y  $r_3$, hallar el valor de $\dfrac{1}{r_1} + \dfrac{1}{r_2} + \dfrac{1}{r_3}$.
\end{example}
\begin{solution}
    Por Vieta sabemos que $r_1 r_2 + r_2 r_3 + r_3 r_1 = 3$ y $r_1 r_2 r_3 = -1$.
    Además, notemos que
    \[
        \frac{r_1 r_2 + r_2 r_3 + r_3 r_2}{r_1 r_2 r_3} = \frac{r_1 r_2}{r_1 r_2 r_3} + \frac{r_2 r_3}{r_1 r_2 r_3} + \frac{r_3 r_1}{r_1 r_2 r_3} = \frac{1}{r_3} + \frac{1}{r_1} + \frac{1}{r_2}
    \]
    Por lo tanto $\frac{1}{r_3} + \frac{1}{r_1} + \frac{1}{r_2} = \frac{3}{-1} = \boxed{-3}$.
\end{solution}

\begin{example}[1986 URSS]
    Las raíces del polinomio $x^2 + ax + b + 1$ son números naturales.
    Mostrar que $a^2 + b^2$ no es un primo.
\end{example}
\begin{solution}
    Para demostrar que $a^2 + b^2$ no es primo, basta demostrar que es el producto de dos enteros mayores a 1.
    Digamos entonces que $r_1$ y $r_2$ son las raíces, por Vieta $r_1 + r_2 = -a$ y $r_1 r_2 = b + 1$.
    Elevando al cuadrado y sumando tenemos $a^2 + b^2 = (r_1 + r_2)^2 + (r_1 r_2 - 1)^2$, al desarrollar llegamos a
    \begin{align*}
        a^2 + b^2 &= r^2_1 + 2r_1 r_2 + r^2_2 + r^2_1 r^2_2 - 2r_1 r_2 + 1 \\
        &= r^2_1 + r^2_2 + r^2_1 r^2_2+ 1\\
        &= (r^2_1 + 1)(r^2_2 + 1).
    \end{align*}
    Por dato $r_1$ y $r_2$ son naturales por lo que $(r^2_1 + 1)$ y $(r^2_2 + 1)$ son naturales mayores a 1 y hemos terminado.
\end{solution}

\begin{example}[AIME II, 2008]
    Sean $r, s$ y $t$ las tres raíces de la ecuación $8x^3 + 1001x + 2008 = 0$.
    Hallar el valor de $(r + s)^3 + (s + t)^3 + (t + r)^3$.
\end{example}
\begin{solution}
    Por Vieta, sabemos que $r +  s + t = 0$ y $r s t = -251$.
    Por la primera ecuación, obtenemos $r + s = -t$ y por tanto $(r + s)^3 = - t^3$.
    Análogamente, llegamos a
    \[
        (r + s)^3 + (s + t)^3 + (t + r)^3 =  -(r^3 + s^3 + t^3)
    \]
    Por la identidad de Gauss, tenemos que $r^3 + s^3 + t^3 - 3rst = 0$ y por tanto $-(r^3 + s^3 + t^3) = -3rst \implies -(r^3 + s^3 + t^3) = -3(-251) = \boxed{753}$.
\end{solution}




\subsection{Ejercicios y problemas}

Ejercicios y problemas para el autoestudio.

\begin{multicols}{2}
    \begin{exercise}
        Sean $r_1$ y $r_2$ las raíces del polinomio $P(x) = ax^2 + bx + c$.
        Encontrar los siguientes valores en función del los coeficientes de $P.$
        \begin{multicols}{3}
            \begin{enumerate}
                \item $r_1 - r_2$
                \item $\frac{1}{r_1} - \frac{1}{r_2}$
                \item $r^2_1 + r^2_2$
                \item $r^3_1 + r^3_2$
                \item $(r_1 + 1)^2 + (r_2 + 1)^2$
            \end{enumerate}
        \end{multicols}
    \end{exercise}

    \begin{problem}
        Sean $a$ y $b$ las raíces de la ecuación $x^2 - 6x + 5 = 0$, encuentra $(a + 1)(b + 1).$
    \end{problem}

    \begin{problem}
        Dado que $m$ y $n$ son raíces del polinomio $6x^2 - 5x - 3$, encuentra un polinomio cuyas raíces sean
        $m - n^2$ y $n - m^2$, sin calcular los valores de $m$ y $n$.
    \end{problem}

    \begin{problem}
        ¿Para qué valores reales positivos de $m$, las raíces $x_1$ y $x_2$ de la ecuación
        \[x^2 - \left( \frac{2m - 1}{2} \right)x  + \frac{m^2 - 3}{2} = 0\]
        cumplen que $x_1 = x_2 - \frac{1}{2}$?
    \end{problem}
\end{multicols}