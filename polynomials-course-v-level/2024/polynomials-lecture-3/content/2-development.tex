\section{Desarrollo}

\subsection{Factorización}

Factorizar un polinomio significa descomponer la expresión original en un producto de polinomios con grados menores.
Tenemos dos herramientas principales para esto: agrupar o usar identidades.
La aplicación de la primera herramienta incluye dividir la expresión en grupos con factores comunes.
Por ejemplo,
\[
    a^2 + ab + bc + ca = (a^2 + ab) + (ac + bc) = a(a + b) + c(a + b) = (a + c)(a + b).
\]

\begin{example}
    Factorizar la siguiente expresión $xyz + 3xy + 2xz - yz + 6x - 3y - 2z - 6$.
\end{example}
\begin{solution}
    Agrupamos la expresión de arriba como $(xyz + 3xy) + (2xz + 6x) - (yz + 3y) - (2z + 6)$.
    Ahora podemos sacar el factor $z + 3$ para cada grupo, $(z + 3)(xy + 2x - y - 2)$.
    Nuevamente, podemos agrupar la expresión $xy + 2x - y - 2$ como $x(y + 2) - (y + 2) = (x - 1)(y + 2)$.
    Por tanto, nuestra expresión queda factorizada como $(x - 1)(y + 2)(z + 3)$.
\end{solution}

El segundo caso se refiere a identidades para factorizar.
Las identidades son los pilares de los cálculos matemáticos.
Se encuentran comúnmente en las competiciones matemáticas, donde muchos problemas requieren de su conocimiento.

Aquí recopilamos algunas de las identidades más útiles.
Es importante que, para mejorar tu fortaleza en el trabajo con expresiones algebraicas, te esforcés en aprender estas identidades.

\textbf{Identidades útiles}
\begin{align*}
    &\text{(Diferencia de cuadrados)}     && a^2 - b^2 = (a - b)(a + b)\\[2mm]
    &\text{(Binomio al cuadrado)}         && (a + b)^2 = a^2 + 2ab + b^2\\[2mm]
    &\text{(Trinomio al cuadrado)}        && (a + b + c)^2 = a^2 + b^2 + c^2 + 2(ab + bc + ca)\\[2mm]
    &\text{(Identidad de Gauss)}          && a^3 + b^3 + c^3 - 3abc = (a + b + c)(a^2 + b^2 + c^2 - ab - bc - ca)\\[2mm]
    &\text{(Identidad de Argand)}         && a^4 + a^2 + 1 = (a^2 - a + 1)(a^2 + a + 1)\\[2mm]
    &\text{(Identidad de Sophie Germain)} && a^4 + 4 b^4 = (a^2 - 2ab + 2b^2)\left(a^2 + 2ab + 2b^2\right)\\[2mm]
    &\text{(Identidad de Brahmagupta)}    && (a^2 + b^2) (x^2 + y^2) = (ax + by)^2 + (ay - bx)^2\\[2mm]
    &\text{(Diferencia de potencias)}     && a^n - b^n = (a - b)(a^{n - 1} + a^{n - 2}b + \cdots + b^{n - 1})\\[2mm]
    &\text{(Suma de potencias)}           && a^n + b^n = (a + b)(a^{n - 1} - a^{n - 2}b + \cdots + b^{n - 1}) \ \text{con $n$ impar}\\[1.5mm]
    &\text{(Binomio de Newton)}           && (a + b)^n = a^n + \binom{n}{1} a^{n - 1}b + \cdots + \binom{n}{k} a^{n -k}b^k + \cdots + b^n
\end{align*}

Por ejemplo, podemos factorizar la expresión $(a^2 - b^2)^3 + (b^2 - c^2)^3 + (c^2 - a^2)^3$.
Primero, notemos que $a^2 - b^2 + b^2 - c^2 + c^2 - a^2 = 0$.
Por tanto, utilizando la Identidad de Gauss, obtenemos que
\[
    (a^2 - b^2)^3 + (b^2 - c^2)^3 + (c^2 - a^2)^3 = 3 (a^2 - b^2)(b^2 - c^2)(c^2 - a^2).
\]
Que por diferencia de cuadrados, $3 (a^2 - b^2)(b^2 - c^2)(c^2 - a^2) = 3 (a - b)(b - c)(c - a) (a + b)(b + c)(c + a)$.



\subsubsection{Factorización clásica}

Dado un polinomio cuadrático $P(x) = ax^2 + bx + c$ donde $a, b, c \in \R$, este puede ser factorizado como
\begin{gather*}
    P(x) = \frac{(ax + \triangle)(ax + \square)}{a},\ \text{donde}\
    \begin{cases}
        \triangle + \square = b\\
        \triangle \times \square = ac
    \end{cases}
\end{gather*}

\begin{example}
    Factorizar el polinomio $10x^2 - 13x - 3$.
\end{example}
\begin{solution}
    Primero calculamos el valor $(10)(-3) = -30$, después escribimos los siguientes paréntesis
    \[
        \inverseOf{10} (10x + \ \ )(10x + \ \ ).
    \]
    Luego, nos preguntamos lo siguiente ¿qué números cumplen que al sumarse den $-13$ y al multiplicarse den $-30$?
    Tras un breve análisis nos damos cuenta de que dichos números son $-15$ y 2.
    El mayor (sin considerar el signo) se coloca en el primer paréntesis, el menor en el segundo y finalmente se reduce la expresión.
    Es decir,
    \[
        \inverseOf{10} (10x - 15)(10x + 2) = \inverseOf{10} (5)(2x - 3)(2)(5x + 1) = \boxed{(2x - 3)(5x + 1)}. \qedhere
    \]
\end{solution}



\subsubsection{Completación de cuadrados}

No todos los polinomios cuadráticos pueden ser factorizados fácilmente.
Por ejemplo, al tratar de factorizar $x^2 + 6x - 1$ llegar directamente a $\left(x + 3 - \sqrt {10}\right)\left(x - 3-\sqrt {10}\right)$ no resulta tan evidente, por lo cual podemos auxiliarnos en técnicas como la \textbf{completación de cuadrados}.
Si se tiene el polinomio $P(x) = x^2 \pm bx$ entonces podemos expresarlo de la forma:
\[
    P(x) = \left( x \pm \frac{b}{2} \right)^2 - \left( \frac{b}{2} \right)^2.
\]
Lo cuál facilita aplicar la diferencia de cuadrados.

\begin{example}
    Factorizar el polinomio $R(r) = r^2 - 10r + 7$.
\end{example}
\begin{solution}
    Utilizando la completación de cuadrados en la expresión $r^2 - 10r$, tenemos que
    \begin{align*}
        R(r) =& r^2 - 10r + 7\\
        =& \left( r - \frac{10}{2} \right)^2 - \left( \frac{10}{2} \right)^2 + 7
        = \left( r - 5 \right)^2 - 18\\
        =& \left( r - 5 + \sqrt {18} \right)\left( r - 5 - \sqrt {18} \right)
    \end{align*}
    De esta manera sabemos que $R(r) = \left[ r - \left( 5 - 3\sqrt {2} \right)\right]\left[ r - \left( 5 + 3\sqrt {2} \right)\right]$.
\end{solution}


\subsection{Definiciones}

\begin{definition}[Raíz de un Polinomio]
    La \textbf{\emph{raíz}} o \textbf{\emph{cero}} de un polinomio $P(x) = a_d x^d + \cdots + a_1 x + a_0$ es un número $r$, tal que $P(r) = 0$.
    También, diremos que $r$ es una solución de la ecuación $ a_d x^d + \cdots + a_1 x + a_0 = 0$.
\end{definition}

\begin{definition}[Factor de un Polinomio]
    Dado $P(x)$ un polinomio con grado $n$ y un número $a \in \R$, diremos que $(x - a)$ es un \emphtext{factor} de $P(x)$ si se cumple que
    \[
        P(x) = (x - a)Q(x),
    \]
    para algún polinomio $Q(x)$ con $\deg Q(x) = n - 1$.
\end{definition}

\begin{example}
    Demuestre que $u$ es raíz del polinomio $R(x) = x^2 - (u + 17) x + 17u$.
\end{example}
\begin{solution}
    Para demostrar que $u$ es raíz de $R(x)$, basta probar que $R(u) = 0$.
    Lo cual es fácil ver cuando evaluamos $R(u) = u^2 - (u+17)u + 17u = u^2 - u^2 - 17u + 17u = 0.$
\end{solution}

El polinomio anterior resulta que puede ser factorizado como $R(x) = (x - u)(x - 17)$, que al igualarlo a cero y analizarlo, cada factor nos da una raíz.
Luego, vemos que $x = u, 17$ son sus raíces.
Es decir, que si factorizamos un polinomio hasta sus factores lineales automáticamente obtenemos sus raíces.

\begin{example}
    Dado $f(x) = x^3 - 3x^2 + 6x - 40$, probar que $(x - 5)$ no es factor un de $f(x)$.
\end{example}
\begin{solution}
    Supongamos que $(x - 5)$ es un factor de $f(x)$, entonces debe pasar que $f(x) = (x - 5)g(x)$ para un polinomio cuadrático $g(x)$.
    Si $(x - 5)$ es factor de $f(x)$, al evaluarlo en $x = 5$ vemos que dicho factor se hace cero, es decir $f(5) = 0$.
    Pero vemos que $f(5) = 5^3 - 3(5)^2 + 6(5) - 40 = 40 \neq 0$, una contradicción.
\end{solution}

Sabiendo las definiciones anteriores, veamos el siguiente teorema.

\begin{theorem.tcb}{Teorema del factor}{}\label{factor-theorem}
Dado un polinomio $P(x)$ con grado $n$ y un número $a \in \R$, diremos que $a$ es una raíz de $P(x)$ si y solo si $(x - a)$ es un factor de $P(x)$, \ie
\[
    P(a) = 0 \iff P(x) = (x - a)Q(x)
\]
para algún polinomio $Q(x)$.
\end{theorem.tcb}
Demostrar este teorema se reduce a observar la correlación de las raíces y los factores lineales, dicha demostración se deja como ejercicio al lector.
El teorema del factor es un teorema muy útil y recurrente en los problemas de polinomios cuando se conocen algunas raíces.
Además de implicar ciertos resultados como el siguiente.

Sean $a_1, a_2$ y $a_3$ tres raíces distintas de un polinomio cúbico $P(x)$, por el teorema anterior sabemos que
\[
    P(x) = (x - a_1)Q(x),
\]
para algún polinomio cuadrático $Q(x)$.
Como $P(a_2) = (a_2 - a_1)Q(a_2) = 0$ y tenemos que $a_2 \neq a_1$, necesariamente $Q(a_2) = 0$, es decir $a_2$ es raíz de $Q$, nuevamente por el teorema anterior sabemos que
\[
    Q(x) = (x - a_2)R(x)
\]
para algún polinomio lineal $R(x)$.
Análogamente, tendremos que $R(x) = (x - a_3)S(x)$, para algún polinomio constante $S(x) = C$.
Luego
\[
    P(x) = C(x - a_1)(x - a_2)(x - a_3),\ \text{con}\  C \in \R.
\]
Donde cabe mencionar que el número $C$ es igual al coeficiente principal.
En resumen, vemos que saber las raíces de $P(x)$ nos condujo a su factorización.
Esto, además, podemos generalizarlo y obtener el siguiente resultado.
\begin{remark.tcb}
    Sea el polinomio $P(x)$ de grado $n$ y raíces $r_i$ con $1 \leq i \leq n$, este puede ser expresado como
    \begin{equation}
        P(x) = C(x - r_1)(x - r_2)\cdots(x - r_{n-1})(x - r_n), \ \text{donde}\ C \in \R.
    \end{equation}
\end{remark.tcb}
De este resultado obtenemos las siguientes propiedades.
\begin{enumerate}
    \item Un polinomio de grado $n$ tiene como máximo $n$ raíces.
    \item Si una cantidad $m \leq n$ de los $r_i$ son iguales, digamos a $k$, se cumple que $P(x) = (x - k)^m Q(x)$.
    Entonces diremos que la raíz $k$ tiene multiplicidad $m$.
\end{enumerate}

\begin{example}
    Sea $P(x)$ un polinomio con coeficientes enteros y suponga que $P(1)$ y $P(2)$ son ambos impares.
    Demuestre que no existe ningún entero $n$ para el cual $P(n) = 0$.
\end{example}
\begin{solution}
    Nos piden demostrar que $P(x)$ no tiene raíces enteras.
    Supongamos lo contrario, que existe un entero $n$ tal que $P(n) = 0$.
    Entonces, por el teorema del factor tenemos $P(x) = (x - n)Q(x)$, con $Q(x)$ un polinomio con coeficientes enteros.
    Así podemos ver que, $P(1) = (1 - n)Q(1)$ y $P(2) = (2 - n)Q(2)$ son impares, pero $(1 - n)$ y $(2 - n)$ son enteros consecutivos, así que uno de ellos debe ser par.
    Por lo tanto, $P(1)$ o bien $P(2)$ tiene que ser par, lo cual contradice las condiciones del problema.
    Luego, $n$ no existe.
\end{solution}

\begin{example}
    Sea $M(x)$ un polinomio cúbico con coeficientes enteros y sean $a, b, c \in \Z$ distintos tal que $M(a) = M(b) = M(c) = 2$.
    Demostrar que no existe un $d \in \Z$ para el que $M(d) = 3.$
\end{example}
\begin{solution}
    Sea $N(x) = M(x) - 2$, como $a, b$ y $c$ son raíces de $N(x)$, por el teorema del factor $N(x) = \alpha (x - a)(x - b)(x - c)$, para algún entero $\alpha$.
    Si para algún entero $d$ se cumple $M(d) = 3$, entonces $N(d) = \alpha (d - a)(d - b)(d - c) = 1$.
    Para que esto suceda los factores deben ser 1 o $-1$, por lo tanto, dos de ellos tendrían que ser iguales.
    Pero por la condición $a \neq b \neq c$ esto no puede ser, luego $d$ no existe.
\end{solution}



\subsection{Análisis de raíces en polinomios cuadráticos y cúbicos}

\subsubsection{Fórmula general}

En un polinomio cuadrático $P(x) = ax^2 + bx + c$ con $a \neq 0$, podemos encontrar sus dos raíces en función de los coeficientes.
A esta fórmula le conoceremos como la \textbf{\emph{fórmula general}} cuadrática.
\[
    x_1 = \frac{-b - \sqrt {b^2 - 4ac}}{2a}\quad\land\quad x_2 = \frac{-b + \sqrt {b^2 - 4ac}}{2a}.
\]
\begin{proof}
    Completemos cuadrados en el polinomio
    \begin{align*}
        P(x) =& a \left( x^2 + \frac{b}{a}x + \frac{c}{a} \right) = a \left[ \left( x + \frac{b}{2a} \right)^2 - \left( \frac{b}{2a} \right)^2 + \frac{c}{a} \right] && (\text{Sumando y restando} \left( b/2a \right)^2)\\
        =& a \left[ \left( x + \frac{b}{2a} \right)^2 - \left( \frac{b^2 - 4ac}{4a^2}\right) \right]  && (\text{Simplificando})\\
        =& a \left[ x + \frac{b}{2a} - \frac{\sqrt {b^2 - 4ac}}{2a}\right]\left[ x + \frac{b}{2a} + \frac{\sqrt {b^2 - 4ac}}{2a}\right]  && (\text{Por diferencia de cuadrados})\\
        =& a \left[ x - \left( \frac{-b + \sqrt {b^2 - 4ac}}{2a} \right) \right]\left[ x - \left( \frac{-b - \sqrt {b^2 - 4ac}}{2a} \right) \right] && (\text{Agrupando coeficientes})
    \end{align*}
    Así, por el teorema del factor, se obtiene el resultado deseado.
\end{proof}

\begin{example}
    Sabiendo que $a_1 = m + 2$, $a_2 = m - 1$ y $a_3 = 1 - m$ son los coeficientes de la ecuación $a_1 x^4 + a_2 x^2 + a_3 = 0$, tal que $a_2 - a_1 = a_3 - a_2$.
    Determinar el valor de mayor raíz.
\end{example}
\begin{solution}
    De la condición $a_2 - a_1 = a_3 - a_2 \implies (m - 1) - (m + 2) = (2 - m) - (m - 1) \implies -3 = -2m + 3 \implies m = 3$.
    Luego, la ecuación es $5x^4 + 2x^2 - 1 = 0$, haciendo $y = x^2$ tenemos la cuadrática $5y^2 + 2y - 1$.
    Que aplicando la fórmula general, obtenemos
    \[
        y = \frac{-2 \pm \sqrt{2^2 - 4(5)(-1)}}{2(5)} = \frac{-2 \pm \sqrt{24}}{10} = \frac{-2 \pm 2\sqrt{6}}{10} = \frac{-1 \pm \sqrt{6}}{5}.
    \]
    Por consiguiente las raíces de la ecuación original, son las raíces cuadradas de estos dos valores, es decir
    \[
        x = \pm \sqrt {\frac{-1 \pm \sqrt{6}}{5}}.
    \]
    Después de un breve análisis, vemos que la mayor de las raíces es $\sqrt {\frac{-1 + \sqrt{6}}{5}}$.
\end{solution}

En el ejemplo anterior, las raíces
\[
    x = \pm \sqrt {\frac{-1 - \sqrt{6}}{5}} = \pm \sqrt {\frac{1 + \sqrt{6}}{5}} i
\]
son valores complejos.
Donde $i = \sqrt {-1}$ se conoce como la \textbf{\emph{unidad imaginaria}}.

En general, dependerá del contexto del problema si estos valores son tomados en cuenta o no.
En nuestro curso no veremos el tema de \emphtext{Números Complejos} tal cuál, pero al menos tendremos en cuenta elementos básicos como su definición y algunas propiedades.



\subsubsection{Análisis de discriminate}

Para el polinomio $P(x) = ax^2 + bx + c$, la cantidad $\Delta = b^2 - 4ac$, es llamada \textbf{\emph{discriminante}} y su signo conduce a tres casos.
\begin{table}[H]
    \centering
    \begin{tabular}{| p{1.4cm} | p{6.5cm} | p{6.5cm} |}
        \hline
        Caso & Descripción & Ejemplo \\ \hline
        $\Delta > 0$ & El polinomio $P(x)$ tiene dos raíces reales distintas.&
            $P(x) = x^2 - 5x + 6$ con raíces $x = 2, 5$ y $\Delta = 1$. \\\hline
        $\Delta = 0$ & El polinomio $P(x)$ tiene una raíz real, que es una raíz \textbf{\emph{doble}}.&
            $P(x) = x^2 - 6x + 9$ con raíz doble $x = 3$ y $\Delta = 0$. \\\hline
        $\Delta < 0$ & El polinomio $P(x)$ no tiene raíces reales.&
            $P(x) = x^2 - 5x + 7$ sin raíces reales y $\Delta = -3$. \\\hline
    \end{tabular}
    \caption{Análisis de discriminate para polinomios cuadráticos.}
\end{table}

\begin{example}
    ¿Para qué valores de $\delta$ el polinomio $\delta x^2 + 2x + 1 - \frac{1}{\delta}$ tiene sus dos raíces iguales?
\end{example}
\begin{solution}
    Si el polinomio tiene raíces iguales, quiere decir que su discriminante es nulo, es decir $\Delta = 4 - 4\delta \left(1 - \frac{1}{\delta}\right) = 0$.
    Multiplicando por $\delta$ y dividiendo por 4, $ \delta - \delta (\delta - 1) = 0 \implies 2\delta - \delta^2 = 0$.
    Luego, $\delta = 2$ es la única posibilidad.
\end{solution}


Para el polinomio $P(x) = ax^3 + bx^2 + cx + d$, podemos plantear la ecuación polinómica $ax^3 + bx^2 + cx + d = 0$, con $a \neq 0$.
Al hacer la sustitución $y = x + \frac{b}{3a}$ se obtiene como resultado la ecuación de la forma $y^3 + py + q$, que se nombra como la ecuación \textbf{\emph{cúbica reducida}}, donde
\[
    p = \frac{3ac - b^2}{3a^2}\quad\text{y}\quad q = \frac{2b^3 - 9abc + 27a^2 d}{27a^3}.
\]
Esta ecuación nos ayuda a obtener la valor $\Delta = \dfrac{q^2}{4} + \dfrac{p^3}{27}$ que llamaremos \textbf{\emph{discriminante}} y su signo conduce a tres casos.
\begin{table}[H]
    \centering
    \begin{tabular}{| p{1.4cm} | p{6.5cm} | p{6.5cm} |}
        \hline
        Caso & Descripción & Ejemplo \\ \hline
        $\Delta > 0$ & El polinomio $P(x)$ tiene una raíz real y dos raíces complejas.&
            $P(x) = x^3 - 3x^2 + x - 3$ con raíces $x = 3, \pm i$ y $\Delta = 0.074$.
            Cuya forma reducida es $y^3 - 2y - 4$. \\\hline
        $\Delta = 0$ & El polinomio $P(x)$ en el caso de que $p = q = 0$ tiene una raíz de multiplicidad tres.
        Si $p = -q \neq 0$ tiene dos raíces reales (una simple y una doble, respectivamente).&
            $P(x) = x^3 - 3x + 2$ con raíz simple $x = -2$, raíz doble $x = 1$ y $\Delta = 0$.
            Cuando $p = q = 0$, es claro que $P(x) = x^3$ y la raíz triple es 0.\\\hline
        $\Delta < 0$ & El polinomio $P(x)$ tiene tres raíces reales diferentes.&
            $P(x) = x^3 - 2x^2 - x + 2$ con raíces $x = -1, 1, 2$ y $\Delta = -0.33$. \\\hline
    \end{tabular}
    \caption{Análisis de discriminate para polinomios cúbicos.}
\end{table}

En las próximas clases veremos una manera directa de obtener las raíces de una ecuación cúbica reducida.
Por medio de un método conocido como el \emphtext{Método de Cardano}.



\subsection{Ejercicios y problemas}

Ejercicios y problemas para el autoestudio.

\showLine

\begin{multicols}{2}

    \begin{exercise}
        Factorizar y determinar las raíces usando factorización clásica.
        \begin{tasks}(1)
            \task $x^2 + x - 20$
            \task $9t^2 + 88t - 20$
            \task $12p^2 - 7p - 12$
            \task $(c + d)(c + d - 18) + 65$
            \task $-21x^2 - 11x + 2$
            \task $r^4 - 13r^2 + 36$
        \end{tasks}
    \end{exercise}

    \begin{exercise}
        Resolver las siguientes ecuaciones utilizando la fórmula general.
        \begin{tasks}(2)
            \task $x^2 = x + 3$
            \task $2x^2 + 39 = -18x$
            \task $5x^2 + 3x + 1 = 0$
            \task $2x^2 + 23 = 14x$
            \task $x^2 = 2x + 48$
        \end{tasks}
    \end{exercise}

    \begin{exercise}
        Determinar la cantidad de raíces de las siguientes ecuaciones cúbicas.
        \begin{tasks}(2)
            \task $x^3 - 6x^2 - 6x - 14$
            \task $x^3 - 7x + 6$
            \task $x^3 - 9x^2 + 24x - 16$
            \task $x^3 - 6x + 9$
        \end{tasks}
    \end{exercise}

    \begin{exercise}
        Indicar cuál no es un factor de $Q(x, y) = x^4 y^3 - 2x^3 y^4 + x^2 y^5$.
        \begin{tasks}(3)
            \task $x$
            \task $y$
            \task $x - y$
            \task $x^2 + xy$
            \task $x + y$
        \end{tasks}
    \end{exercise}

    \begin{exercise}
        Señalar un término de un factor de $F(x) = 8x^2 - 22x + 15$.
        \begin{tasks}(5)
            \task $8x$
            \task $x$
            \task 5
            \task 3
            \task $2x$
        \end{tasks}
    \end{exercise}

    \begin{exercise}
        Indicar un factor del polinomio $P(a) = a^2 + 2ab + b^2 - c^2 - d^2 - 2cd$.
        \begin{tasks}(2)
            \task $a + b - c + d$
            \task $a + b + c + d$
            \task $a - b + c - d$
            \task $a + b + c - d$
            \task $a + b - c - d$
        \end{tasks}
    \end{exercise}

    \begin{exercise}
        Factorizar el polinomio $P(x, y) = (x + 1)^2 - (y - 2)^2$ e indicar un factor.
        \begin{tasks}(3)
            \task $x + y - 1$
            \task $x - y - 2$
            \task $x - y - 3$
            \task $x - y - 4$
            \task $x - y - 5$
        \end{tasks}
    \end{exercise}

    \begin{exercise}
        Dar la suma de coeficientes de un factor del polinomio $R(y) = y^{2m + 2} - y^{m + 1} - 30$.
        \begin{tasks}(5)
            \task 1
            \task 2
            \task 7
            \task 3
            \task 6
        \end{tasks}
    \end{exercise}
    
    \begin{exercise}
        Indicar la cantidad de factores de $(mn + pq)^2 + (mq - pn)^2$.
        \begin{tasks}(5)
            \task 1
            \task 2
            \task 3
            \task 4
            \task 5
        \end{tasks}
    \end{exercise}

    \begin{exercise}
        Factorizar $M(x,y) = x^4 - 5x^2 y^2 + 4y^4$ y dar como respuesta la suma de los factores lineales.
        \begin{tasks}(3)
            \task $2x + y$
            \task $2x - y$
            \task $4y$
            \task $4x$
            \task $2x + 2y$
        \end{tasks}
    \end{exercise}

    \begin{exercise}
        Indicar la suma de los términos independientes de los factores de $R(x) = (x + 2y - 3)(x + 2y + 5) + 15$.
        \begin{tasks}(3)
            \task 2
            \task $2(2y + 1)$
            \task $y + 2$
            \task 3
            \task $2y + 2$
        \end{tasks}
    \end{exercise}

    \begin{exercise}
        Luego de factorizar $F(x,y) = (x + y + 2)^2 - 4x - 4y - 5$ el polinomio tiene la forma
        \[
            (ax + by + c)(dx + ey + f)
        \]
        sabiendo que $f < 0$, calcular $\frac{a + b + c}{d + e + f}$.
        \begin{tasks}(5)
            \task 3
            \task 1
            \task 0
            \task 4
            \task $-2$
        \end{tasks}
    \end{exercise}

    \begin{exercise}
        Reducir la expresión
        \[
            (x^2 - x + 1)(x^2 + x + 1)(x^4 - x^2 + 1)(x^4 - 1) + 1.
        \]
        \begin{tasks}(5)
            \task $x^{24}$
            \task $-x^{12}$
            \task $x^{12}$
            \task $x^2$
            \task $x^{18}$
        \end{tasks}
    \end{exercise}

    \begin{exercise}
        Teniendo $a + b + c = \inverseOf{(a - b)(b - c)} + \inverseOf{(b - c)(c - a)} + \inverseOf{(c - a)(a - b)}$, calcular
        \[
            \frac{(a + b - c)^2}{ab} + \frac{(b + c - a)^2}{bc} + \frac{(c + a - b)^2}{ca}.
        \]
        \begin{tasks}(5)
            \task 10
            \task 13
            \task 15
            \task 12
            \task 14
        \end{tasks}
    \end{exercise}

    \begin{exercise}
        Si $a^2 + b^2 + c^2 = 2$, calcular
        \begin{align*}
            &- 4(ab + bc + ca)[(a + b + c)^2 - (ab + bc + ca)]\\
            &+ (a + b + c)^4
        \end{align*}
        \begin{tasks}(5)
            \task 1
            \task 2
            \task 4
            \task 3
            \task 5
        \end{tasks}
    \end{exercise}

    \begin{exercise}
        Sabiendo que $x + y + z = 0$, calcular el valor de
        \[
            \frac{(x^3 + y^3 + z^3)(x^2 + y^2 + z^2)^2}{xyz^5 + xy^5 z + x^5 yz}.
        \]
        \begin{tasks}(5)
            \task 3
            \task 1
            \task $-4$
            \task $-6$
            \task 6
        \end{tasks}
    \end{exercise}

    \begin{exercise}
        Si $a + b + c = 1$ y $a^3 + b^3 + c^3 = 4$.
        Hallar el valor de $\inverseOfD{a + bc} + \inverseOfD{b + ac} + \inverseOfD{c + ab}$.
        \begin{tasks}(5)
            \task $-2$
            \task $-1$
            \task 0
            \task 1
            \task 2
        \end{tasks}
    \end{exercise}

    \begin{exercise}
        Sea $f(x) = 3x^3 - 2x^2 - 12x + 8$.
        \begin{tasks}[label=\alph*.](1)
            \task Usar el teorema del factor para probar que $(x + 2)$ es factor de $f(x)$.
            \task Factorizar $f(x)$ completamente.
        \end{tasks}
    \end{exercise}

    \begin{exercise}
        El polinomio $x^3 + 4x^2 + 7x + k$, donde $k$ es una constante, es denotado por $f(x)$.
        \begin{tasks}[label=\alph*.](1)
            \task Dado que $(x + 2)$ es un factor de $f(x)$, probar que $k = 6$.
            \task Expresar $f(x)$ como el producto de un factor lineal y uno cuadrático.
        \end{tasks}
    \end{exercise}

    \begin{exercise}
        Usar el teorema del factor para demostrar que $(x + 3)$ es un factor de $x^3 + 5x^2 - 2x - 24$.
        Luego, factorize completamente el polinomio.
    \end{exercise}

    \begin{exercise}
        Usar el teorema del factor para probar que $(x - 5)$ es un factor de $x^3 - 19x - 30$.
        Después, factorize el polinomio en tres factores lineales.
    \end{exercise}

    \begin{exercise}
        Un polinomio cúbico es definido en términos de la constante $k$ como
        \[
            P(x) = x^3 + x^2 - x + k,\quad x \in \R.
        \]
        Dado que $(x - k)$ es un factor de $P(x)$ determinar los posibles valores de $k$.
    \end{exercise}

    \begin{exercise}
        Use el teorema del factor para demostrar $(x + 2)$ es factor de $2x^3 + 3x^2 - 5x - 6$.
        Luego, factorize el polinomio en tres factores lineales.
    \end{exercise}

    \begin{exercise}
        Hallar las raíces del polinomio
        \[
            x^3 + x^2 - (x - 1)(x - 2)(x - 3) - 12.
        \]
    \end{exercise}

    \begin{exercise}
        Si la ecuación cumple $6x^2 + xy = 15y^2$ con $x, y \neq 0$, hallar todos los valores de $\frac{x}{y}$.
    \end{exercise}

    \begin{exercise}
        Sabiendo que la ecuación $x = K^2(x - 1)(x - 2)$ con $K \in \R$, tiene raíces reales, hallar $K$.
    \end{exercise}

    \begin{exercise}
        Encontrar las soluciones de la ecuación $m^2 - 3m + 1 = n^2 + n - 1$, con $m, n \in \positiveSet{\Z}$.
    \end{exercise}

    \begin{problem}
        Sean $a, b$ y $c$ números reales positivos.
        ¿Es posible que cada uno de los polinomios $P(x) = ax^2 + bx + c$, $Q(x) = bx^2 + cx + a$ y $R(x) = cx^2 + ax + b$ tenga sus dos raíces reales?
    \end{problem}

    \begin{problem}
        Sean $a$ y $b$ enteros.
        Determinar todas las soluciones de la ecuación
        \[
            (ax - b)^2 + (bx - a)^2 =  x,
        \] si se sabe que tiene una solución entera.
    \end{problem}

    \begin{problem}
        Sea $P(x)$ un polinomio cúbico mónico tal que $P(1) = 1$, $P(2) = 2$ y $P(3) = 3$.
        Encontrar $P(4).$
    \end{problem}
\end{multicols}