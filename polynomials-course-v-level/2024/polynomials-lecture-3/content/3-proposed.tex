\section{Problemas propuestos}

Los problemas de esta sección es la \textbf{tarea}.
El estudiante tiene el deber de entregar sus soluciones en la siguiente sesión de clase (también se pueden entregar borradores).
Recordar realizar un trabajo claro, ordenado y limpio.

\begin{multicols}{2}
    \begin{problem}
        Si $p$ y $q$ son las raíces del polinomio $P(x) = 4x^2 + 5x + 3$.
        Determina $(p + 7)(q + 7)$, sin calcular los valores de $p$ y $q$.
    \end{problem}

    \begin{problem}
        Supóngase que el polinomio $5x^3 + 4x^2 - 8x + 6$ tiene tres raíces reales $a, b \mbox{ y } c$.
        Demostra que \[(5a)^2\left(\frac{b}{2}\right)\left(\frac{1}{a} + \frac{1}{b}\right) + (5b)^2\left(\frac{c}{2}\right)\left(\frac{1}{b} + \frac{1}{c}\right)+ (5c)^2\left(\frac{a}{2}\right)\left(\frac{1}{c} + \frac{1}{a}\right) = 3^3 + 1.\]
    \end{problem}

    \begin{problem}
        Sean $r_1, r_2, r_3$ las raíces del polinomio $P(x) = x^3 - x^2 + x - 2$.
        Determina el valor de $r^3_1 + r^3_2 + r^3_3$, sin calcular los valores de $r_1, r_2, r_3$.
    \end{problem}
\end{multicols}