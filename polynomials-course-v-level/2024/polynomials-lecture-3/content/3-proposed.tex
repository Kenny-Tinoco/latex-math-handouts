\section{Problemas propuestos}

Los problemas de esta sección es la \textbf{tarea}.
El estudiante tiene el deber de entregar sus soluciones en la siguiente sesión de clase (también se pueden entregar borradores).
Recordar realizar un trabajo claro, ordenado y limpio.

\showLine
\begin{multicols}{2}
    \begin{problem}
        Dado el polinomio
        \[
            P(x) = x^4 + ax^3 + bx^2 + cx + d
        \]
        que cumple $P(1) = 10$, $P(2) = 20$ y $P(3) = 30$.
        Determinar el valor de
        \[
            \frac{P(12) + P(-8)}{10}.
        \]
    \end{problem}

    \begin{problem}
        Sea $P(x)$ un polinomio cuadrático.
        Demostrar que existen polinomios cuadráticos $G(x)$ y $H(x)$ tales que
        \[
            P(x)P(x+1) = (G \circ H)(x).
        \]
    \end{problem}
\end{multicols}