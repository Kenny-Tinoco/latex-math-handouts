\section{Clase 02}

Se empieza la clase dando un pequeño resumen de lo visto en la sesión anterior, se hace énfasis en la evaluación de polinomios
\begin{center}
    P(expresión) $\rightarrow$ expresión. (Expresión algebraica $\equiv$ polinomio)\\
    P(constante) $\rightarrow$ constante. (Constante $\equiv$ número)
\end{center}
Se hace mención a evaluaciones notables como $P(0)$ y $P(1)$.
Luego, se introduce la definición de raíz de un polinomio (alución a la \textbf{definición 1.1}) y la razón por la que se estudia.

Se muestra el polinomio $M(x) = x^5 - 3x^4 - 29x^3 - 13x^2 + 120x + 140$ y se pide a los estudiante hallar una raíz por tanteo.
También, describir el polinomio a manera de ejercicio.
El docente podrá hacer las siguientes preguntas ¿qué características tiene $M$? ¿es mónico? ¿es completo? ¿es simétrico? ¿está ordenado?.
Después que el estudiante intentó encontrar soluciones por su cuenta, anunciar que 7 es una raíz.
Y a continuación, comprobar que $x = 7$ es una raíz.

\begin{align*}
    1\times 16807= +16807\\
    -3\times 2401 = -\mbox{..}7203\\
    -29\times 343  = -\mbox{..}9947\\
    -13\times 49   = -\mbox{....}637\\
    120\times 7    = +\mbox{....}840\\
    1\times 140  = +\mbox{....}140
\end{align*}

Hacer la siguiente pregunta: ¿es fácil o obvio deducir que $x = 7$ es una raíz?.
Introducir la definición de factor (alución a la \textbf{definición 1.2}) y mencionar la factorización.
Luego, expresar el polinomio $M$ como el producto del factor $(x - 7)$ con otro polinomio\footnote{Preguntar nuevamente si los polinomios son completos y mostrar la completación de polinomios}:

\begin{align*}
    x^5 + 4x^4 - x^3 - 20x^2 - \:20x \hspace{3em}\\
    - 7x^4 - 28x^3 + 7x^2 + 140x + 140\\
    x^5 - 3x^4 - 29x^3 - 13x^2 + 120x + 140\\
    \Rightarrow M(x) = (x - 7)(x^4 + 4x^3 - x^2 - 20x - 20)
\end{align*}

Dar énfasis en cómo los factores dan información de las raíces de un polinomio y hacer referencia al \textbf{Teorema del factor}.
\newpage
Terminar la factorización de $M$
\begin{align*}
    x^4 + 4x^3 + 4x^2\hspace{5em}\\
    -5x^2 - 20x - 20\\
    x^4 + 4x^3 - x^2 - 20x - 20\\
    \Rightarrow M(x) = (x - 7)(x^2 - 5)(x + 2)^2\\
    \Rightarrow M(x) = (x - 7)(x - \sqrt{5})(x + 2)(x + 2)(x + \sqrt{5})
\end{align*}

Hacer referencia a que la cantidad de raíces de un polinomio está determinado por su grado e indicar las multiplicidades de las raíces del polinomio $M$.