\section{Desarrollo}

\subsection{Definiciones}

\textbf{Definición:} Un \textbf{\emph{polinomio}} en $x$ es una expresión de la forma \[\generalFormPolynomail,\]
donde $n$ es un entero mayor o igual que cero y $a_1, a_2, \dots, a_n$ son números que pueden ser enteros,
racionales, reales o complejos y son llamados los \textbf{\emph{coeficientes}} de $P(x)$. Si $a_n \neq 0$,
se dice que $P(x)$ es de \emph{grado n} y se denota por $deg(P) = n$; en este caso $a_n$ es llamado \textbf{\emph{coeficiente principal}}.

En particular, los polinomios de grado 1, 2 y 3 son llamados \emph{líneal, cuadrático y cúbico},
respectivamente, y son estos el caso de estudio de esta primera sesión.
\vspace{-2mm}
\begin{center}
    Líneal: $P(x) = a_{1}x+a_0$\\
    Cuadrático: $P(x) = a_2x^2+a_{1}x+a_0$\\
    Cúbico: $P(x) = a_3x^3+a_2x^2+a_{1}x+a_0$
\end{center}

Veremos a continuación algunas caracteristicas resaltables sobre los polinomios.

\textbf{Término principal:} El término (monomio) que contiene la mayor potencia es conocido \emph{término principal} y su coeficiente como \emph{coeficiente principal}.

\textbf{Valor numérico:} Resultado que se obtiene al evaluar el polinomio en una constante $c$, es decir $P(c)$. Además, cuando el valor de $c$ es 0 o 1, el valor numérico
es igual al término independiente y a la suma de todos los coeficientes, respectivamente. Siendo el término independiente aquel monomio que no tiene variable.

\textbf{Polinomio mónico:} Polinomio cuyo término principal tiene como coeficiente a 1.

\textbf{Polinomio completo:} Polinomio que tiene todos sus términos. Desde el termino con exponente $n$ hasta el término con exponente $1$.

\textbf{Polinomio iguales:} Diremos que dos polinomios $P(x)$ y $Q(x)$ son iguales si y solo si $deg(P) = deg(Q) = n$ y $a_i = b_i,$ con $0\leq i\leq n$. Donde $a_i, b_i$ son los coeficientes de $P(x)$ y $Q(x)$, respectivamente.

\textbf{Polinomio recíproco:} Un polinomio $P(x)$ es recíproco si cumple que $a_i = a_{n-i}$ con $0\leq i\leq n$, donde $deg(P) = n$.

\textbf{Polinomio de varias variables:} Un polinomio que dependen de más de una variable es llamado multivariante o multivariable, y es denotado por $P(x_1, x_2, \dots, x_k)$.

\textbf{Polinomio ordenado:} Un polinomio es ordenado cuando los exponente de la variable de referencia, guardan cierto orden, ya sea ascendente o descendente.

\textbf{Polinomio homogéneo:} Un polinomio multivariable es homogéneo si todos usus términos tienen el mismo grado absoluto.


\subsection{Operaciones con polinomios}

Sean dos polinomios $P(x)$ y $Q(x)$ de grado 3.
\begin{gather*}
    P(x) = a_3x^3+a_2x^2+a_1x+a_0\\
    Q(x) = b_3x^3+b_2x^2+b_1x+b_0
\end{gather*}
se definen:

\textbf{Suma:} $P(x) + Q(x) = (a_3+b_3)x^3+(a_2+b_2)x^2+(a_1+b_1)x+(a_0+b_0)$

\textbf{Resta:} $P(x) - Q(x) = (a_3-b_3)x^3+(a_2-b_2)x^2+(a_1-b_1)x+(a_0-b_0)$

\textbf{Multiplicación:} $P(x)\times Q(x) = a_3b_3x^6 + $
$(a_2b_2+a_3b_2)x^5 + $
$(a_1b_3+a_2b_2+a_3b_1)x^4 + $\\
$(a_0b_3+a_1b_2+a_2b_1+a_3b_0)x^3 + $
$(a_0b_2+a_1b_1+a_2b_0)x^2 + $
$(a_0b_1+a_1b_0)x + $
$a_0b_0$\\

\textbf{Composición:} $P(Q(x)) = a_3Q(x)^3+a_2Q(x)^2+a_1Q(x)+a_0$, es decir remplazar la variable $x$ por el polinomio $Q(x)$.

En general si $P(x)$ y $Q(x)$ son polinomios no nulos entonces, se verifica para la suma y la resta que
\[deg(P\pm Q) \leq máx\{deg(P), deg(Q)\}\]
y para el producto \[deg(P\times Q) = deg(P) + deg(Q)\]