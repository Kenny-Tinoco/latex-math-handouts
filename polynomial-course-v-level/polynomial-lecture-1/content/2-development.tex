\section{Desarrollo}

\textbf{Definición:} Un \textbf{\emph{polinomio}} en $x$ es una expresión de la forma \[\generalFormPolynomail,\]
donde $n$ es un entero mayor o igual que cero y $a_1, a_2, \dots, a_n$ son números que pueden ser enteros,
racionales, reales o complejos y son llamados los \textbf{\emph{coeficientes}} de $P(x)$. Si $a_n \neq 0$,
se dice que $P(x)$ es de \emph{grado n} y se denota por $grad(P) = n$; en este caso $a_n$ es llamado \textbf{\emph{coeficiente principal}}.

En particular, los polinomios de grado 1, 2 y 3 son llamados \emph{líneal, cuadrático y cúbico},
respectivamente, y son estos el caso de estudio de esta primera sesión.
\vspace{-2mm}
\begin{center}
    Líneal: $P(x) = a_{1}x+a_0$\\
    Cuadrático: $P(x) = a_2x^2+a_{1}x+a_0$\\
    Cúbico: $P(x) = a_3x^3+a_2x^2+a_{1}x+a_0$
\end{center}

