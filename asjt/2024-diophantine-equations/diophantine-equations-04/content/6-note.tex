\newpage
\section{Plan de clase}

\subsection{¿Qué?}

Mostrar el uso de congruencias en enteros para la solución de ecuaciones diofánticas, dando énfasis en cómo las
congruencias nos permiten probar que algunas ecuaciones son insolubles
También, su utilidad para obtener información de las soluciones, por medio de condiciones que deben cumplirse.

\subsection{¿Cómo?}

Problema para motivar el tema.\par
\begin{activity}[Ejercicio 2][10 min]
    Hallar los enteros positivos $a,b$ tales que $a^2 - 3b^2 = 8$.
    \textit{Primero dar tiempo para intentar resolverlo por medio de los métodos factorización, desigualdades y
        parametrización.
    Dar énfasis en la dificultad que esto conlleva.
    Para resolverlo, solo basta analizar la ecuación en módulo 3.}
\end{activity}

Dar una pequeña introducción a congruencias.
Explicar qué son las congruencias potenciales.\par
\begin{activity}[Prop. 6][10 min]
    Probar que para todo entero $x$ se tiene $x^3 \modulo{0,\pm 1}{9}$.
    \textit{Probarlo utilizando un análisis de cada cubo en módulo 9.
    Señalar que aquí no se puede usar los teoremas de Euler o Fermat}
\end{activity}

\begin{activity}[Ejercicio 1][15 min]
    Ejercicio para los estudiantes.
    \textit{Pasar a la pizarra.}
\end{activity}

\begin{activity}[Ejercicio 7][8 min]
    Ejercicio para los estudiantes.
    \textit{Este también se resuelve con un análisis en módulo 3.}
\end{activity}

\begin{activity}[Ejercicio 3][15 min]
    Ejemplo.
    \textit{Se señala como usar los retos cúbicos, dar una comparación del uso en módulo 9 y en módulo 7.
    Para resolverlo, solo basta análizar en módulo 7 y notar que las potencias de 2 tienen un patron de 1, 2 y 4 en módulo
    7, lo cual hace imposible que $x^3$ sea resto cúbico en módulo 7. Luego, no hay soluciones.}
\end{activity}

\begin{activity}[Ejercicio 8][10 min]
    Ejercicio para los estudiantes.
    \textit{Se analiza la ecuación en módulo 9, obteniendo $a^3 + 2b^3 + 4c^3 \modulo{0}{9}$ lo cual no es posible porque no se
    puede formar una combinación con 0, -1 y 1 (que son los restos cúbico módulo 9) de tal manera que se obtenga cero.}
\end{activity}

\begin{activity}[Problema 1][20 min]
    Ejemplo.
    \textit{Se hace la sustitución $x = w - 49$, con lo cual la ecuación se reduce a una cuadrática con ciertos coeficientes.
    Notar que la ecuación debe ser múltiplo de 3 y por tanto múltiplo de una potencia de 3, luego señalar que una
    parte de la ecuación no cumple esto, por tanto, no hay soluciones.}
\end{activity}

\begin{activity}[Ejercicio 9][8 min]
    Ejercicio para los estudiantes (para finalizar la clase).
    \textit{Se resuelve con un análisis en módulo 9.}
\end{activity}


\subsection{Comentarios}

Preguntas claves: ¿me entendieron?
¿me salté algún tema?
¿se dio tiempo suficiente para pensar los problemas?
¿participaron?
¿problemas muy fáciles o muy difíciles, demasiados o muy pocos?
¿mis explicaciones/ejemplos fueron suficientes o buenos?

\foreach \x in {1,...,20}{
    \myhrule{8.5}
}