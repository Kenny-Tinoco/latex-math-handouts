\section{Ecuaciones diofánticas lineales}

Antes de empezar recordemos los siguientes resultados.

\begin{definition.box}{Máximo divisor común}{}
    Dados $a,b \in \Z^{\neq 0}$, el máximo $d \in \positiveSet{\Z}$ tal que $d \mid a$ y $d \mid b$ es el máximo divisor común,
    y lo denotamos por $d = \mcd a b.$
\end{definition.box}

\begin{definition.box}{}{}
    Si $a,b \in \Z^{\neq 0}$ con $\mcd a b = 1$, entonces diremos que $a$ y $b$ son coprimos, primos relativos o primos entre sí.
\end{definition.box}

\begin{definition.box}{Combinación lineal}{}
    Dados los enteros $a_1, a_2, \ldots, a_k$, definimos una combinación lineal de los números $\{a_i\}$ como un número de la forma
    \[
        a_1 x_1 + a_2 x_2 + \ldots + a_k x_k,
    \]
    donde los $\{x_i\}$ son enteros cualesquiera.
\end{definition.box}

\begin{theorem.box}{Algoritmo de la división}{}
    Si $a, b \in \Z$ con $b \neq 0$, entonces existen enteros únicos $(q,r)$ tales que $a = bq + r$ con $0 \leq r < |b|$.
\end{theorem.box}

El siguiente resultado es un teorema muy conocido e importante en la teoría de números, con él damos inicio al
estudio de las ecuaciones diofánticas lineales, este resultado es utilizado tanto para la demostración de teoremas como
la resolución de problemas.

\begin{theorem.box}{Bezout}{}
    Si $d = \mcd a b$, entonces $d$ es el menor entero tal que
    \[
        ax + by = d,\ \text{con}\ x,y \in \Z.
    \]
\end{theorem.box}

\begin{theorem.box}{}{theorema3}
    La ecuación diofántica $ax + by = c$ tendrá soluciones si y solo si $\mcd a b \mid c$.
    Además, si $(x_0,y_0)$ es una solución particular, entonces la solución general es
    \[
        (x,y) = \left(x_0+\frac{b}{d}t,\ y_0 -\frac{a}{d}t\right) \quad \text{con}\ t \in \Z.
    \]
\end{theorem.box}

Ejemplos.
\begin{example}
    Resolver la ecuación $5x - 3y = 52$ en enteros positivos.
\end{example}
\begin{solution}
    Primero, como $\mcd 5 3 = 1$ y $1 \mid 52$, entonces la ecuación tiene soluciones enteras.
    Ahora, analizando en módulo 5 tenemos que
    \[
        - 3y \modulo{52}{5} \implies 2y \modulo{2}{5} \implies y \modulo{1}{5}.
    \]
    Claramente, $y = 1$ es solución a esta congruencia, sustituyendo en la ecuación original $5x - 3\cdot 1 = 52 \iff 5x = 55$
    por lo cual, la ecuación tiene una solución $(11, 1)$, luego con la solución $(x_0,y_0)=(11,1)$ llegamos a
    \[
        (x,y) = (11+3t,1-5t), \mbox{ donde } t\in \Z.\qedhere
    \]
\end{solution}

\begin{example}
    Resolver la siguiente ecuación $8c+7p=100$.
\end{example}
\begin{solution}
    Claramente, la ecuación tiene soluciones enteras, analizando en módulo 8,
    \[
        7p \modulo{100}{8} \implies -p \modulo{4}{8} \implies p \modulo{-4}{8} \implies p \modulo{4}{8}.
    \]
    Rápidamente, podemos decir que $p = 4$ es una solución para dicha congruencia, sustituyendo en la ecuación obtenemos $c = 9$.
    Luego, con la solución $(c_0, p_0) = (9, 4)$ tenemos
    \[
        (c,p) = (9 + 7t, 4 - 8t),\quad \text{con}\ t\in \Z.\qedhere
    \]
\end{solution}



\subsection{Aplicando el algoritmo de Euclides}

\begin{definition.box}{Algoritmo de Euclides}{}
    Si $a, b \in \Z^{\neq 0}$ y $a = bq + r$ para algunos enteros $q,r$ con $0 < r < b$, entonces
    \[
        \mcd a b = \mcd b r.
    \]
\end{definition.box}

Podemos usar este algoritmo para resolver ecuaciones lineales de una manera iterativa, similar, a la manera en cómo
se calcula el MCD de dos números grandes.

Sean $a,b, c \in \Z$ con $a,b \neq 0$ y $d = \mcd a b$, considerando la ecuación
\[
    ax + by = c,
\]
se tienen los siguientes pasos:
\begin{enumerate}
    \item Si $d\nmid c$, entonces no hay solución.
    \item Si $d \mid c$, entonces se divide la ecuación por $d$ obteniendo $a_1 x + b_1 y = c_1$.
    \item Por el paso anterior la ecuación tiene coeficiente coprimos;
    \begin{enumerate}
        \item[3.1.] Si $a_1 \mid c_1$, entonces $a_1 c_0 = c_1$, luego $(x,y) = (c_0, 0)$ es solución.
        \item[3.2.] Si $a_1 \nmid c_1$, entonces tomamos el menor de $|a_1|, |b_1|$ y obtenemos\footnote{Aquí vamos a suponer que $|a_1|$ es el menor.}
        \begin{align*}
            b_1 = a_1 q_1 + r_1,\ \text{con}\ 0<r_1<|a|, && c_1 = a_1 q_2 + r_2,\ \text{con}\ 0 < r_2 < |a|.
        \end{align*}
    \end{enumerate}
    \item Sustituimos en la ecuación
    \[
        a_1 x + (a_1 q_1 + r_1) y = a_1 q_2 + r_2 \iff a_1 (x + q_1 y - q_2) + r_1 y = r_2.
    \]
    Haciendo $z = x + q_1 y - q_2$, la ecuación anterior se transforma en $a_1 z + r_1 y = r_2$.
    \begin{enumerate}
        \item[4.1.] Si $r_1 \mid r_2$, entonces terminamos con el paso 3.1.
        \item[4.2.] Si $r_1 \nmid r_2$, entonces vamos al paso 3.2 y repetimos el proceso.
    \end{enumerate}
\end{enumerate}

Con estos pasos (o algoritmo) es posible resolver ecuaciones diofánticas lineales de coeficientes grandes, de una manera
cíclica o iterativa, lo cual se reduce la complejidad.
Veamos un ejemplo.

\begin{example}
    Resuelva la siguiente ecuación $350x + 425y = 1200$.
\end{example}
\begin{solution}
    Como $\mcd{350}{425} = 25$ y $25 \mid 1200$, dividimos ambos lados de la ecuación por $25$ y tenemos
    \[
        14x+17y=48
    \]
    Por el algoritmo de euclides se tiene que $17 = 1\cdot14+3$ y $48=3\cdot14+6$.
    Sustituyendo y agrupando, tenemos $14(x+y-3)+3y=6.$
    Haciendo $z=x+y-3$ y sustituyendo en esta última ecuación se tiene $14z+3y=6.$
    Como $3\mid6$, para esta ecuación tenemos una solución de la forma $z=0$ y $y=2$.
    Escribiendo $z$ en términos de $x$ y $y=2$, obtenemos el valor de $x=1$.
    Luego, la solución general de la ecuación inicial es:
    \[
        x=1+17k, \quad y=2-14k, \quad\text{con}\ k \in \Z. \qedhere
    \]
\end{solution}



\subsection{Caso general de ecuaciones lineales}

Hasta el momento solamente hemos trabajado en ecuaciones lineales de dos variables, pero en realidad la ecuación
$ax + by = c$ es un caso concreto de la ecuación
\[
    a_1 x_1 + a_2 x_2 + \ldots + a_n x_n = c,
\]
donde $a_1, a_2, \dots, a_n,$ y $c$ son coeficientes.
También, cabe mencionar que el teorema de bezout se cumple para una cantidad $n$ de números, con lo cual
se podrá hacer un tratamiento similar al caso de $n = 2$.

\begin{theorem.box}{}{}
    La ecuación diofántica $a_1 x_1 + a_2 x_2 + \ldots + a_n x_n = c$ tiene solución si y solo si $\mcd {a_1}{a_2,\cdots,a_n} \mid c.$
\end{theorem.box}

Veamos un ejemplo.

\begin{example}
    Resuelva la ecuación $3x + 4y + 5z = 6$.
\end{example}
\begin{solution}
    Primero, notamos que $\mcd{3}{4,5} = 1$ efectivamente divide a 6.
    Trabajando en módulo 5 tenemos que $3x + 4y\modulo{1}{5}$ por lo cual $3x + 4y = 1 + 5s$, con $s \in \Z$.
    Si tomamos a $s$ como una constante, una solución para esta ecuación es $(x_0, y_0) = (-1 + 3s, 1 - s)$.
    Usando el teorema~\ref{t:theorema3}, obtenemos las soluciones generales de esta ecuación de dos variables
    \[
        (x,y) = (-1+3s+4t, 1-s-3t)\quad\text{con}\ t \in \Z.
    \]
    Sustituyendo esto en la ecuación original obtenemos $z = 1 - s$, por lo que todas las soluciones son
    \[
        (x,y,z)=(-1+3s+4t,1-s-3t,1-s), \quad \text{con}\ s,t \in \Z.\qedhere
    \]
\end{solution}

Como vemos en este ejemplo, una ecuación de tres variables la podemos reducir a una ecuación de dos variables,
ecuación que sabemos cómo solucionarla, luego solo debemos revertir el proceso y dejar las soluciones en función
de los parámetros que aparezcan.
Análogamente, podemos resolver una ecuación de grado $n$ reduciéndola sucesivamente a una de grado dos y realizar un proceso similar.


\subsection{Existencia de soluciones}

Para concluir, veremos algunos resultados que estudian la existencia de soluciones para una ecuación diofántica, estos
resultados ya son parte de una teoría de mayor complejidad, si se tiene curiosidad se invita al lector a investigar más a fondo.

\begin{definition.box}{}{}
    Sean  $a_1,a_2,\ldots,a_n$ enteros positivos con $\mcd{a_1}{a_2,\ldots,a_n} = 1$ se define a $g(a_1,a_2,\ldots,a_n)$
    como el mayor entero positivo $N$ para el cual
    \[
        a_1 x_1 + a_2 x_2 + \ldots + a_n x_n = N,
    \]
    no tiene soluciones enteras.
\end{definition.box}

El problema de determinar $g(a_1,\ldots,a_n)$ es conocido como el problema de las monedas de Frobenius.
Este problema fue planteado por Ferdinand Frobenius, quien se interesó en encontrar la mayor cantidad de dinero que no
se puede representar como una combinación lineal de $n$ denominaciones de monedas,
quizás el ejemplo más sencillo es con monedas de 3 y 5 centavos con las cuales nunca se podrá pagar una deuda de 7 centavos.
El siguiente teorema brinda un valor de $N$ para el caso de $n=2$.

\begin{theorem.box}{Sylvester}{}
    Sean $a,b \in \positiveSet{\Z}$, con $\mcd a b = 1$, entonces
    \[
        g(a,b) = ab - a - b.
    \]
\end{theorem.box}
Con esto se puede analizar la ecuación $3x + 5y = 7$, como $g(3,5) = 3\cdot 5 - 3 - 5 = 7$
entonces se tiene que 7 es el mayor entero para el cual no hay soluciones.
Para los casos de $n\geq2$ hasta la fecha no se conoce ninguna fórmula explícita de $g$, cabe aclarar que estos temas
ya son de una complejidad mayor a este curso.
Finalmente, con el teorema sylvester podemos entender mejor el siguiente resultado.

\begin{theorem.box}{Chicken McNugget}{}
    Sean\footnote{Su historia es curiosa, debido a que fue enunciado en un McDonald's.} $a, b\in \positiveSet{\Z}$ con $\mcd a b = 1$, se tiene:
    \begin{enumerate}
        \item[i.] Si $n = ab - a - b$, entonces $ax + by = n$ es insoluble $\forall x, y \in \positiveSet{\Z}$.
        \item[ii.] Si $n > ab - a - b$, entonces la ecuación es soluble.
    \end{enumerate}
\end{theorem.box}
Como recomendación general, se aconseja siempre verificar que una ecuación cumpla con el segundo punto del teorema anterior.




\subsection{Ejercicios y problemas}

Ejercicios y problemas para el autoestudio.

\begin{exercise}
    Resolver las siguientes ecuaciones:
    \begin{multicols}{3}
        \begin{enumerate}
            \item $30x - 25y = 15$
            \item $2x - 3y = 5$
            \item $91a - 195b = 1079$
            \item $35m - 25n = 3$
            \item $24x + 18y = 12$
            \item $125x - 25y = 28$
            \item $69x + 123y = 3000$
            \item $12p + 501q=1$
            \item $97s + 98t=1000$
        \end{enumerate}
    \end{multicols}
\end{exercise}

\begin{exercise}
    Resuelva las siguientes ecuaciones:
    \begin{multicols}{3}
        \begin{enumerate}
            \item $3x+10y+8z=34$
            \item $5a+20b+5c=13$
            \item $6x+10y+15z=37$
            \item $m+p-q=6$
            \item $4r-32s+12t=8$
            \item $10x+2y-3z=7$
        \end{enumerate}
    \end{multicols}
\end{exercise}

\begin{problem}
    Suponga que Fabiana gasta 40 pesos en una pulpería.
    Si esta paga con un billete de 100, ¿cómo puede recibir el vuelto en billetes de 10 y 25 pesos?
\end{problem}

\begin{problem}
    En el correo solo se tienen sellos de 14 y 21 céntimos.
    ¿De qué manera puede formar con un valor de 7.77 euros?
\end{problem}

\begin{problem}
    Queremos echar 21 litros de gasolina a un depósito.
    Para ello, tenemos dos bidones, de 2 y 5 litros respectivamente.
    Responder a las siguientes cuestiones:
    \begin{enumerate}
        \item ¿Es posible medir 21 litros con nuestros bidones? ¿Por qué?
        \item En caso afirmativo, dar todas las combinaciones posibles.
        \item Si suponemos que en nuestro depósito caben exactamente 22 litros, ¿cómo podemos echar 21 litros sin desbordar el depósito y sin retirar gasolina de él?
    \end{enumerate}
\end{problem}

\begin{problem}
    Brisa Marina le paga 143 pesos por la compra de mangos y naranjas a Gerald.
    Si paga 15 pesos por cada mango y 17 pesos por cada naranja, ¿cuántos mangos y naranjas le compró a Gerald?
\end{problem}

\begin{problem}
    Determinar los enteros positivos $n$ tales que la ecuación $x + 2y + z = n$ tiene exactamente $100$ soluciones
    $(x,y,z)$ en enteros no negativos.
\end{problem}