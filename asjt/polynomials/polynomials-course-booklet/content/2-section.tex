\newpage
\section{Introducción a los Polinomios}



\subsection{Definiciones}

\begin{section-definition.tcb}
    Un \textbf{polinomio} en $x$ es una expresión de la forma
    \[
        P(x) = \polynom{n}{a},\quad \text{con} \ n \in \ZNN.
    \]
    Donde $a_1, a_2, \dots, a_n$ son números que pueden ser tanto reales como complejos y son llamados los \textbf{coeficientes} de $P(x)$.
\end{section-definition.tcb}

Cada uno de los sumandos $a_i x^i$ son llamados \textbf{términos} de $P(x)$.
Si $a_n \neq 0$, se dice que $P(x)$ es de \textbf{grado n} y se denota por $\deg{(P)} = n$.
En este caso $a_n x^n$ es llamado \textbf{término principal} y $a_n$ \textbf{coeficiente principal} del polinomio.

En particular, los polinomios de grado 1, 2, 3 y 4 son llamados \emph{líneal, cuadrático, cúbico y cuártico}, respectivamente.
He aquí la forma de dichos polinomios:

\begin{table}[H]
    \centering
    \begin{tabular}{p{2.5cm} p{6.5cm}}
        \hline
        Nombre & Forma \\
        \hline \hline
        Líneal & $P(x) = a_1 x + a_0$\\
        Cuadrático & $P(x) = a_2 x^2 + a_1 x + a_0$\\
        Cúbico & $P(x) = a_3 x^3 + a_2 x^2 + a_1 x + a_0$\\
        Cuártico & $P(x) = a_4 x^4 + a_3 x^3 + a_2 x^2 + a_1 x + a_0$\\
        \hline
    \end{tabular}
\end{table}

Veremos a continuación algunas caracteristicas y definiciones resaltables sobre los polinomios:

\begin{itemize}
    \item \textbf{Polinomio mónico:} Polinomio cuyo coeficiente principal es 1.
    \item \textbf{Polinomio completo:} Un polinomio de grado $n$ es completo si tiene todos sus $n + 1$ términos.
    \item \textbf{Polinomio ordenado:} Un polinomio es ordenado cuando los exponente de la variable de referencia, guardan cierto orden, ya sea ascendente o descendente.
    \item \textbf{Polinomios iguales:} Dos polinomios $P(x)$ y $Q(x)$ con coeficientes $a_i$ y $b_i$, respectivamente, son iguales si y solo si tienen el mismo grado y $a_i = b_i$, para todo $i$ en $0\leq i\leq n$.
    \item \textbf{Polinomio recíproco:}\label{reci-polynomial} Un polinomio $P(x)$ de grado $n$ es recíproco si cumple que $a_i = a_{n-i}$, para todo $i$ en $0\leq i\leq n$.
    \item \textbf{Polinomio de varias variables:} Un polinomio que dependen de más de una variable es llamado multivariante o multivariable, y es denotado por $P(x_1, x_2, \dots, x_k)$.
    \item \textbf{Grado absoluto:} El grado absoluto de un término de la forma $x_1^{\alpha_1} x_2^{\alpha_2} \cdots x_k^{\alpha_k}$, es la suma de las potencias de cada una de las variables.
    \item \textbf{Polinomio homogéneo:} Un polinomio multivariable es homogéneo si todos sus términos tienen el mismo grado absoluto.
    \item \textbf{Evaluación en un polinomio:} Valor númerico que se obtiene al reemplazar la $x$ por una constante $a$ en un polinomio $P(x)$, lo cual denotamos por $P(a)$\footnote{Se lee ``$P$ de $a$'' o ``$P$ evaluado en $a$''.}.
\end{itemize}



\subsection{Operaciones con polinomios}

Sean dos polinomios $P(x)$ y $Q(x)$ de grado 3,
\begin{align*}
    P(x) &= a_3 x^3 + a_2 x^2 + a_1 x + a_0\\
    Q(x) &= b_3 x^3 + b_2 x^2 + b_1 x + b_0
\end{align*}
se definen:
\[
    \begin{array}{lcll}
        \text{\textbf{Suma:}} && P(x) + Q(x) =& (a_3+b_3)x^3+(a_2+b_2)x^2+(a_1+b_1)x+(a_0+b_0)\\[3mm]
        \text{\textbf{Resta:}} && P(x) - Q(x) =& (a_3-b_3)x^3+(a_2-b_2)x^2+(a_1-b_1)x+(a_0-b_0)\\[3mm]
        \text{\textbf{Producto:}} && P(x)\times Q(x) =& a_3 b_3 x^6 + (a_2 b_3 + a_3 b_2)x^5 + (a_1 b_3 + a_2 b_2 + a_3 b_1)x^4 +\\
                                  &&&(a_0 b_3 + a_1 b_2 + a_2 b_1 + a_3 b_0)x^3 + (a_0 b_2 + a_1 b_1 + a_2 b_0)x^2 +\\
                                  &&&(a_0 b_1 + a_1 b_0)x + a_0 b_0\\[3mm]
        \text{\textbf{Composición:}} && (P \circ Q)(x) =& P(Q(x)) = a_3 Q(x)^3 + a_2 Q(x)^2 + a_1 Q(x) + a_0
    \end{array}
\]
Evidentemente estas operaciones son análogas para polinomios de grado mayor a 3.

\begin{remark.tcb}
    En general si $P(x)$ y $Q(x)$ son polinomios no nulos, entonces se verifica que
    \begin{align*}
        \deg{(P\pm Q)} &\leq m\acute ax\{\deg{(P)}, \deg{(Q)}\}\\
        \deg{(P\times Q)} &= \deg{(P)} + \deg{(Q)}
    \end{align*}
\end{remark.tcb}

\begin{remark.tcb}
    La composición $(P \circ Q)(x)$ se lee ``$P$ compuesto de $Q$'' o ``$Q$ evaluado en $P$'' y consiste en igualar la variable $x$ a $Q(x)$ ($x : = Q(x)$) y luego sustituir en el polinomio $P(x)$.
\end{remark.tcb}

La composición de polinomios es asociativa, es decir, $(P \circ Q) \circ R = P \circ (Q \circ R)$, mas no conmutativa, salvo casos especiales, así que generalmente supondremos que $P \circ Q \neq  Q \circ P$.

\begin{remark.tcb}
    $P(x)^n \neq P^n(x)$.
    Denotaremos como $P(x)^n$ al polinomio elevado a la $n$-ésima potencia y a $P^n(x)$ como la composición $n$-ésima de $P(x)$ con si mismo, es decir:
    \[
        P(x)^n = \underbrace{P(x) \cdot P(x) \cdots P(x)}_{n\ \text{veces}} \quad \land \quad
        P^n (x) = \underbrace{P(P(P\cdots(P}_{n\ \text{veces}}(x))\cdots)).
    \]
\end{remark.tcb}



\subsection{Ejercicios y problemas}

\begin{section-exercise}
    Sean $P(x) = 5x^2 - 33x + 59$ y $Q(x) = 3 - 2x$.
    Determine
    \begin{multicols}{2}
        \begin{enumerate}
            \item $P(x) + Q(x)$
            \item $Q(x) - P(x)$
            \item $P(x) + Q((x+1)^2)$
            \item $P(1-x) + Q(x-1)$
            \item $(P\circ Q)(x)$
            \item $Q(P(x))$
            \item $P(Q(3x - 4))$
            \item $Q^3(x) + P(x)^2$
        \end{enumerate}
    \end{multicols}
\end{section-exercise}

\begin{section-problem}
    Sea $P(x)$ un polinomio mónico de grado 3.
    Halle la suma de coeficientes del término cuadrático y lineal, siendo su término independiente igual a 5.
    Además, $P(x + 1) = P(x) + nx + 2$.
\end{section-problem}

\begin{section-problem}
    Hallar un polinomio cuadrático cuyo coeficiente de $x$ y término independiente son iguales y se cumple que $P(1) = 7$ y $P(2) = 18$.
\end{section-problem}

\begin{section-problem}
    Dado que

    \[
        \begin{cases}
            Q(x) = 2x + 3 \\
            Q( F(x) + G(x) ) = 4x + 3 \\
            Q( F(x) \times G(x) ) = 5
        \end{cases}
    \]

    Calcular $F(G(F(G(\dots F(G(1))\dots))))$.
\end{section-problem}