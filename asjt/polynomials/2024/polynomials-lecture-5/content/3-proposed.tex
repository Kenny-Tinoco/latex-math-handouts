\section{Problemas propuestos}

Los problemas de esta sección es la \textbf{tarea}.
El estudiante tiene el deber de entregar sus soluciones en la siguiente sesión de clase (también se pueden entregar borradores).
Recordar realizar un trabajo claro, ordenado y limpio.

\showLine
\begin{multicols}{2}

    \begin{exercise}
        Dado los polinomios
        \begin{align*}
            P(x) &= 6x^4 + 4x^3 - 5x^2 - 10x + a \\
            Q(x) &= 3 x^2 + 2x + b,
        \end{align*}
        sabemos que $Q(x)$ divide a $P(x)$.
        Hallar el valor de $a^2 + b^2$.
    \end{exercise}

    \begin{exercise}
        Un polinomio $P(x)$ deja resto $-2$ en la división entre $(x - 1)$ y resto $-4$ en la división entre $(x + 2)$.
        Hallar el resto en la división de $P(x)$ por $x^2 + x - 2$.
    \end{exercise}

    \begin{exercise}
        Probar que si un polinomio $F(x)$ deja un residuo de la forma $px + q$ cuando es dividido entre $(x - a)(x - b)(x - c)$
        donde $a$, $b$ y $c$ son todos distintos, entonces
        \[
            (b - c)F(a) + (c - a)F(b) + (a - b)F(c) = 0.
        \]
    \end{exercise}

    \begin{problem}
        Encontrar el resto cuando el polinomio $x^{81} + x^{49} + x^{25} + x^9 + x$ es dividido entre $x^3 - x$.
    \end{problem}
\end{multicols}