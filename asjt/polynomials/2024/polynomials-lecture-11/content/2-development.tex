\section{Ejercicios y problemas}

Ejercicios y problemas para el autoestudio.

\begin{exercise}
    Sea $P(x)$ un polinomio mónico de grado 3 tal que
    \[
        P(x + 1) = P(x) + nx + 2.
    \]
    Hallar la suma de coeficientes del término cuadrático y lineal, sabiendo que su término independiente igual a 5.
\end{exercise}
\textbf{* Pista.} Considerar un polinomio generico que cumpla las condiciones y utilzar las evaluaciones $P(0)$ y $P(1)$.


\begin{exercise}
    Determine todos los posibles valores que puede tomar $\dfrac{x}{y}$ si se cumple la ecuación $6x^2 + xy = 15y^2$ con $x,y \neq 0$.
\end{exercise}
\textbf{* Pista.} Formar una ecuación cuadrática y resolverla.


\begin{exercise}[Con correciones]
    Hallar $K \in \R$ tal que $P(x) = K^2(x - 1)(x - 2)$ tiene raíces reales iguales.
\end{exercise}
\textbf{* Pista.} Considerar el discriminante de la ecuación.


\begin{exercise}
    Encontrar todas las soluciones de la ecuación $m^2 - 3m + 1 = n^2 + n - 1$, con $m, n \in \Z^{+}$.
\end{exercise}
\textbf{* Pista.} Formar una ecuación cuadrática tomando una de las letras como variable.
Factorizar o usar la fórmula general cuadrática.


\begin{exercise}
    Sea el polinomio $P(x)$ tal que
    \[
        P(x^2 + 1) = x^4 + 4x^2,
    \]
    encontrar $P(x^2 - 1).$
\end{exercise}
\textbf{* Pista.} \begin{enumerate}
    \item Hallar una transformación para $x$ de tal manera que se obtenga el resultado deseado. Apoyarse en una ecuación.
    \item Encontrar $P(x)$ por medio de un cambio de variable.
\end{enumerate}


\begin{exercise}
    Sea $S(x)$ un polinomio cúbico tal que $S(1) = 1$, $S(2) = 2$, $S(3) = 3$ y $S(4) = 5$, encontrar $S(6)$.
\end{exercise}
\textbf{* Pista.} Considerar un polinomio auxiliar que tenga a cierto valores cómo raíces.
Expresar $S(x)$ en función de ese polinomio auxiliar.


\begin{exercise}
    Para que la división de $6x^4 - 11x^2 + ax + b$ entre $3x^2 - 3x - 1$ sea exacta, encuentre los valores de $a$ y $b$ apropiados.
\end{exercise}
\textbf{* Pista.} Realizar una división larga.


\begin{exercise}
    Calcular la suma de coeficientes del resto que deja $x^{3333} - 9$ entre $x^2 - 729$.
\end{exercise}
\textbf{* Pista.} Utilizar el teorema de resto.


\begin{exercise}
    Probar que para todo $n$ entero positivo se cumple que
    \begin{enumerate}
        \item $1 + 3 + 5 + \cdots + \left(2n - 1\right) = n^2$
        \item $1^3 + 2^3 + 3^3  + \cdots + n^3 = \left(\frac{n (n + 1)}{2}\right)^2$
    \end{enumerate}
\end{exercise}
\textbf{* Pista.} Utilizar inducción matemática.


\begin{exercise}
    Con la ayuda del teorema de la raíz racional, encontrar todas las raíces de los siguientes polinomios
    \begin{enumerate}
        \item $2 x^3 - 21 x^2 + 52 x - 21$
        \item $x^4 - 7 x^3 - 19 x^2 + 103 x + 210$
    \end{enumerate}
\end{exercise}
\textbf{* Pista.} Utilizar el teorema de la raíz racional.


\begin{exercise}[Con correciones]
    Dado el polinomio $P(a,b, c) = a^2 b + b^2 c + c^2 a$ expresarlo en términos de los polinomios simétricos elementales y en función de sí mismo.
\end{exercise}
\textbf{* Pista.} Probar con el producto de $\sigma_1$ y $\sigma_2$.
Considerar permutaciones de variables en la definción de $P$.

\begin{problem}
    Sea $r$ una raíz de $x^2 - x + 7$.
    Hallar el valor de $r^3 + 6r + \pi$.
\end{problem}
\textbf{* Pista.} Apoyarse en las propiedades de $r$ evaluado en el polinomio.


\begin{problem}
    Si $a + b + c = \sqrt{2023}$ y $a^2 + b^2 + c^2 = 2021$, hallar el valor de
    \[
        E = \frac{(a + b)^2 (b + c)^2 (c + a)^2}{(a^2 + 1) (b^2 + 1) (c^2 + 1)}.
    \]
\end{problem}
\textbf{* Pista.} Utilizar Vieta e identidades algébraicas.

\begin{problem}
    El cociente de la división $\dfrac{x^{n + 1} + 2x + 5}{x - 3}$ es $Q(x)$, la suma de coeficientes de $Q$ es $\dfrac{9^{10} + 3}{2}$.
    Hallar el valor de $n$.
\end{problem}
\textbf{* Pista.} Utilizar el teorema del resto.

\begin{problem}
    Sean $a$, $b$ y $c$ las raíces reales de la ecuación $x^3 + 3x^2 - 24x + 1 = 0$.
    Probar que
    \[
        \sqrt[3]{a} + \sqrt[3]{b} + \sqrt[3]{c} = 0.
    \]
\end{problem}
\textbf{* Pista.} Utilizar Vieta e identidades algébraicas.

\begin{problem}
    Si la división
    \[\frac{x^{80} - 7 x^{30} + 9x^5 - mx + 1}{x^3 + x - 2}\]
    Deja como resto a $R(x) = x^2 + x - 1$, hallar el valor de $m$.
\end{problem}
\textbf{* Pista.} Utilizar el teorema del resto (apoyarse en el teorema de la raíz racional para encontrar las raíces del divisor.)


\begin{problem}
    Sean $r_1$, $r_2$ y $r_3$ raíces distintas del polinomio $y^3 - 22 y^2 + 80 y - 67$.
    De tal manera que existen números reales $\alpha$, $\beta$ y $\theta$ tal que
    \[
        \frac{1}{y^3 - 22 y^2 + 80 y - 67} = \frac{\alpha}{y - r_1} + \frac{\beta}{y - r_2} + \frac{\theta}{y - r_3}
    \]
    para toda $y \notin \left\{ r_1, r_2, r_3 \right\}$.
    ¿Cuál es valor de $\dfrac{1}{\alpha} + \dfrac{1}{\beta} + \dfrac{1}{\theta}$?
\end{problem}
\textbf{* Pista.} Factorizar el polinomio y deshacer todas las fracciones tratando de encontrar $\frac{1}{\alpha}$, $\frac{1}{\beta}$ y $\frac{1}{\theta}$ de manera separada.
Utilizar una idea parecida al teorema de la raíz racional.

\begin{problem}
    La ecuación
    \[
        2^{333x - 2} + 2^{111x + 2} = 2^{222x + 1} + 1
    \]
    tiene tres raíces reales.
    Dado que su suma es $\dfrac{m}{n}$ con $m, n \in \Z^{+}$ y $mcd(m, n) = 1$.
    Calcular $m + n$.
\end{problem}
\textbf{* Pista.} Cambio de variable con la ayuda de las propiedades de potencia.
Reducirlo a una ecuación cúbica y encontrar sus raíces.

\begin{problem}
    Si $P(x) = x^4 + ax^3 + bx^2 + cx + d$ es un polinomio tal que $P(1) = 10$, $P(2) = 20$ y $P(3) = 30$, determine el valor de
    \[\frac{P(12) + P(-8)}{10}.\]
\end{problem}
\textbf{* Pista.} Considerar un polinomio auxiliar y factorizarlo.
Operar sobre ese polinomio auxiliar para obtener el valor de la expresión.

\begin{problem}
    Sea $F(x)$ un polinomio mónico con coeficientes enteros.
    Probar que si existen cuatro enteros diferentes $a$, $b$, $c$ y $d$ tal que $F(a) = F(b) = F(c)  = F(d) = 5$,
    entonces no existe un entero $k$ tal que $F(k) = 8$.
\end{problem}
\textbf{* Pista.} Suponer lo contrario y llegar a una contradicción.
Utilizar las propiedades de raíces.


\begin{problem}
    Sea el polinomio $P_0(x) = x^3 + 313x^2 - 77x - 8$.
    Para enteros $n \geq 0$, definimos $P_n(x) = P_{n - 1}(x - n)$.
    ¿Cuál es el coeficiente de $x$ en $P_{20}(x)$?
\end{problem}
\textbf{* Pista.} Ver que pasa con casos pequeños de $n$ y generalizar.



\begin{problem}
    Determine un polinomio cúbico $P(x)$ en los reales, con una raíz igual a cero y que satisface $P(x - 1) = P(x) + 25x^2$.
\end{problem}
\textbf{* Pista.} Considerar una suma telescópica e utilizar ciertas propiedades de sumas de cuadrados.


\begin{problem}
    Suponga que $x$, $y$ y $z$ son números distintos de cero tal que $(x + y + z)(x^2 + y^2 + z^2) = x^3 + y^3 + z^3$.
    Hallar el valor de
    \[(x + y + z)\left(\frac{1}{x} + \frac{1}{y} + \frac{1}{z}\right).\]
\end{problem}
\textbf{* Pista.} Utilizar polinomios simétricos elementales e identidades algebraicas.


\begin{problem}[OMCC 2020, shortlist]
    Sean $a$, $b$ y $c$ números reales no nulos tales que $a + b + c = 0$.
    Determine el valor de la expresión
    \[\frac{(a^2 + b^2)(b^2 + c^2) + (b^2 + c^2)(c^2 + a^2) + (c^2 + a^2)(a^2 + b^2)}{a^4 + b^4 + c^4}.\]
\end{problem}
\textbf{* Pista.} Considerar $(a + b + c)^2$ y desarrollar el numerador para lograr simplificar.
Utilizar los polinomios simétricos elementales.

\begin{problem}
    Si $a$, $b$, $c$ y $d$ son las raíces de la ecuación $x^4 - 3x^3 + 1 = 0$,
    calcular el valor de
    \[
        \inverseOf{a^6} + \inverseOf{b^6} + \inverseOf{c^6} + \inverseOf{d^6}.
    \]
\end{problem}
\textbf{* Pista.} Utilizar la ecuación para encontrar $\frac{1}{x^3}$ elevar al cuadrado para reducir la expresión en función de los polinomios simétricos elementales.
Utilizar las fórmulas de vieta.