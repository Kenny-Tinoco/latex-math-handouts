\section{Desarrollo}

\subsection{Definiciones}

\textbf{Definición:} Un \textbf{\emph{polinomio}} en $x$ es una expresión de la forma \[\generalFormPolynomail,\]
donde $n$ es un entero mayor o igual que cero y $a_1, a_2, \dots, a_n$ son números que pueden ser enteros,
racionales, reales o complejos y son llamados los \textbf{\emph{coeficientes}} de $P(x)$. Si $a_n \neq 0$,
se dice que $P(x)$ es de \emph{grado n} y se denota por $\deg{(P)} = n$; en este caso $a_n$ es llamado \textbf{\emph{coeficiente principal}}.

En particular, los polinomios de grado 1, 2 y 3 son llamados \emph{líneal, cuadrático y cúbico},
respectivamente, y son estos el caso de estudio de esta primera sesión.
\vspace{-2mm}
\begin{center}
    Líneal: $P(x) = a_1 x + a_0$\\
    Cuadrático: $P(x) = a_2 x^2 + a_1 x + a_0$\\
    Cúbico: $P(x) = a_3 x^3 + a_2 x^2 + a_1 x + a_0$
\end{center}

Veremos a continuación algunas caracteristicas resaltables sobre los polinomios.

\textbf{Término principal:} El término (monomio) que contiene la mayor potencia es conocido \emph{término principal} y su coeficiente como \emph{coeficiente principal}.

\textbf{Valor numérico:} Resultado que se obtiene al evaluar el polinomio en una constante $c$, es decir $P(c)$. Además, cuando el valor de $c$ es 0 o 1, el valor numérico
es igual al \textbf{\textit{término independiente}} y a la suma de todos los coeficientes, respectivamente. Siendo el término independiente aquel monomio que no tiene variable.

\textbf{Polinomio mónico:} Polinomio cuyo término principal tiene como coeficiente a 1.

\textbf{Polinomio completo:} Polinomio que tiene todos sus términos. Desde el término con exponente $n$ hasta el término con exponente 0.

\textbf{Polinomio iguales:} Diremos que dos polinomios $P(x)$ y $Q(x)$ son iguales si y solo si $\deg{(P)} = \deg{(Q)} = n$ y $a_i = b_i,$ con $0\leq i\leq n$. Donde $a_i, b_i$ son los coeficientes de $P(x)$ y $Q(x)$, respectivamente.

\textbf{Polinomio recíproco:} Un polinomio $P(x)$ es recíproco si cumple que $a_i = a_{n-i}$, con $0\leq i\leq n$, donde $\deg{(P)} = n$.

\textbf{Polinomio de varias variables:} Un polinomio que dependen de más de una variable es llamado multivariante o multivariable, y es denotado por $P(x_1, x_2, \dots, x_k)$.

\textbf{Polinomio ordenado:} Un polinomio es ordenado cuando los exponente de la variable de referencia, guardan cierto orden, ya sea ascendente o descendente.

\textbf{Polinomio homogéneo:} Un polinomio multivariable es homogéneo si todos usus términos tienen el mismo grado absoluto.


\subsection{Operaciones con polinomios}

Sean dos polinomios $P(x)$ y $Q(x)$ de grado 3,
\begin{gather*}
    P(x) = a_3 x^3 + a_2 x^2 + a_1 x + a_0\\
    Q(x) = b_3 x^3 + b_2 x^2 + b_1 x + b_0
\end{gather*}
se definen:

\textbf{Suma:} $P(x) + Q(x) = (a_3+b_3)x^3+(a_2+b_2)x^2+(a_1+b_1)x+(a_0+b_0)$

\textbf{Resta:} $P(x) - Q(x) = (a_3-b_3)x^3+(a_2-b_2)x^2+(a_1-b_1)x+(a_0-b_0)$

\textbf{Multiplicación:} $P(x)\times Q(x) = P(x)Q(x) = a_3 b_3 x^6 + $
$(a_2 b_3 + a_3 b_2)x^5 + $
$(a_1 b_3 + a_2 b_2 + a_3 b_1)x^4 + $\\
$(a_0 b_3 + a_1 b_2 + a_2 b_1 + a_3 b_0)x^3 + $
$(a_0 b_2 + a_1 b_1 + a_2 b_0)x^2 + $
$(a_0 b_1 + a_1 b_0)x + $
$a_0 b_0$

\textbf{Composición:} $(P \circ Q)(x) = P(Q(x)) = a_3 Q(x)^3 + a_2 Q(x)^2 + a_1 Q(x) + a_0$. Se lee $P$ compuesto de $Q$ y consiste en remplazar la variable $x$ por $Q$ en $P$.

En general si $P(x)$ y $Q(x)$ son polinomios no nulos, entonces se verifica que
\begin{gather*}
    \deg{(P\pm Q)} \leq m\acute ax\{\deg{(P)}, \deg{(Q)}\}\\
    \deg{(P\times Q)} = \deg{(P)} + \deg{(Q)}
\end{gather*}


La composición de polinomios es asociativa, es decir, $(P \circ Q) \circ R = P \circ (Q \circ R)$, mas no conmutativa, salvo casos especiales, así que generalmente supondremos que $P \circ Q \neq  Q \circ P$.

\textbf{Una pequeña aclaración:} $P(x)^n \neq P^n(x)$. Denotaremos como $P(x)^n$ al polinomio elevado a la $n$-ésima potencia y a $P^n(x)$ como la composición $n$-ésima de $P(x)$ con si mismo, es decir $P(P(P\dots P(P(x))\dots))$ $n$ veces.


\subsection{Ejercicios y problemas}

\textbf{Ejercicio 1.1.} Sean $P(x) = 5x^2 - 33x + 59$ y $Q(x) = 3 - 2x$. Determine
\begin{multicols}{2}
    \begin{enumerate}
        \item $P(x) + Q(x)$
        \item $Q(x) - P(x)$
        \item $P(x) + Q((x+1)^2)$
        \item $P(1-x) + Q(x-1)$
        \item $(P\circ Q)(x)$
        \item $Q(P(x))$
        \item $P(Q(3x - 4))$
        \item $Q^3(x) + P(x)^2$
    \end{enumerate}
\end{multicols}

\begin{section-problem}
    Sea $P(x)$ un polinomio mónico de grado 3. Halle la suma de coeficientes del término cuadrático y lineal, siendo su término independiente igual a 5. Además, $P(x + 1) = P(x) + nx + 2$.
\end{section-problem}

\begin{section-problem}
    Hallar un polinomio cuadrático cuyo coeficiente de $x$ y término independiente son iguales y se cumple que $P(1) = 7$ y $P(2) = 18$.
\end{section-problem}

\begin{section-problem}
    Dado que
        \begin{gather*}
            Q(x) = 2x + 3 \\
            Q( F(x) + G(x) ) = 4x + 3 \\
            Q( F(x) \times G(x) ) = 5
        \end{gather*}
    Calcular $F(G(F(G(\dots F(G(1))\dots))))$.
\end{section-problem}