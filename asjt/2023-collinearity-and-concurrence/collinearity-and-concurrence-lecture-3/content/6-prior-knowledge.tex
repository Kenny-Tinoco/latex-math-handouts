\section{Conocimiento previo}

\begin{section-theorem.tcb}[Ley de los Senos]\label{law-of-sines}
    Sea el triángulo $ABC$ con circunradio $R$.
    Entonces
    \[
        \frac{a}{\sen(\angle A)} = \frac{b}{\sen(\angle B)} = \frac{c}{\sen(\angle C)} = 2R.
    \]
\end{section-theorem.tcb}

\textbf{Pista:} Considerar las antipodas de los vertices en el circuncírculo de $\triangle ABC$, notar ciertos cuadriláteros cíclicos y luego utilzar la definción del seno para el triángulo rectángulo.

\begin{section-theorem.tcb}[Teorema de bisectriz generalizada]\label{ratio-lemma}
    Dado el triángulo $ABC$, sea el punto $X$ en $BC$.
    Entonces se cumple que
    \[
        \frac{BX}{XC} = \frac{AB}{AC} \cdot \frac{\sen(\angle BAX)}{\sen(\angle XAC)}.
    \]
\end{section-theorem.tcb}

\textbf{Pista:} Utilizar el~\refTheorem{\ref{law-of-sines}} y la siguiente identidad trigonométrica:
\[
    \text{Si}\ \alpha + \beta = 180^\circ\text{,}\ \text{entonces}\ \sen(\alpha) = \sen(\beta).
\]