\section{Problemas propuestos}

Recordar que los problemas de esta sección son los asignados como \textbf{tarea}.
Es el deber del estudiante resolverlos y entregarlos de manera clara y ordenada el próximo encuentro
(de ser necesario, también se pueden entregar borradores).

\begin{section-problem}
    Sea el $\triangle ABC$ con puntos $D$ y $E$ sobre las rectas $AB$ y $AC$, respectivamente, tal que $(E \neq B, D \neq C)$.
    Entonces la $A$-simediana de $\triangle ABC$ biseca al segmento $DE$ si y solo si los puntos $D$, $B$, $C$ y $E$ con concíclicos.
\end{section-problem}

\begin{section-problem}
    Sea el $\triangle ABC$ y $M$ un punto sobre el segmento $BC$.
    Se escoge un punto $D$ sobre la recta $AB$ tal que el circuncírculo del triángulo $BCD$ es tangente a $AC$.
    Entonces, $AM$ es la simediana del triángulo $ACD$ si y solo si $M$ es el punto medio de $BC$.
\end{section-problem}