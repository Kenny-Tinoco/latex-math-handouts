\documentclass[12pt]{article}

\usepackage[charter]{mathdesign}
\usepackage[spanish]{babel}
\let\circledS\undefined % here - PS
\spanishdecimal{.}


\usepackage{cite}
\usepackage{fancyhdr}
\usepackage{multicol}
\usepackage[margin=2.4cm]{geometry}


%math packages
\usepackage{amsmath}
\usepackage{amssymb}
\usepackage{amsthm}
\usepackage{amsfonts}
\allowdisplaybreaks



%Figures dependences
\usepackage{pgf}
\usepackage{tikz}
\usepackage{float}
\usepackage{graphicx}
\usepackage{subcaption}
\usetikzlibrary{arrows}


%Text color box dependences
\usepackage{tcolorbox}
\usepackage{varwidth}
\tcbuselibrary{theorems}
\tcbuselibrary{breakable}
\tcbuselibrary{skins}


\setlength{\parskip}{2mm}
\setlength{\parindent}{0pt}

\renewcommand{\qedsymbol}{$\blacksquare$}
\addto\captionsspanish{\renewcommand{\proofname}{\textnormal{\textbf{Demostración}}}}
\DeclareSymbolFont{yhlargesymbols}{OMX}{yhex}{m}{n}
\DeclareMathAccent{\wideparen}{\mathord}{yhlargesymbols}{"F3}

%\renewcommand{\theenumi}{\alph{enumi}}

%Useful text commands
\newcommand{\ie}{\ensuremath{\text{i.e.}\ }}
\newcommand{\emphtext}[1]{\textbf{\emph{#1}}}


%Number sets
\newcommand{\N}{\ensuremath{\mathbb{N}}}
\newcommand{\Z}{\ensuremath{\mathbb{Z}}}
\newcommand{\Q}{\ensuremath{\mathbb{Q}}}
\newcommand{\R}{\ensuremath{\mathbb{R}}}
\newcommand{\C}{\ensuremath{\mathbb{C}}}

\newcommand{\positiveSet}[1]{\ensuremath{#1^+}}
\newcommand{\negativeSet}[1]{\ensuremath{#1^-}}
\newcommand{\nonnegativeSet}[1]{\ensuremath{#1^{\geq 0}}}

%Useful math commands
\newcommand{\ds}{\displaystyle}
\renewcommand{\max}[1]{\ensuremath{máx #1}}
\newcommand{\inverseOf}[1]{\frac{1}{#1}}
\newcommand{\fullMod}[2]{\equiv #1 \pmod{#2}}
\newcommand{\refTheorem}[1]{\textbf{Teorema #1}}
\newcommand{\refDefinition}[1]{\textbf{Definición #1}}
\newcommand{\inverseOfD}[1]{\ensuremath{\dfrac{1}{#1}}}

\newcommand{\showLine}{
    \setlength{\columnsep}{0.9cm}
    \setlength{\columnseprule}{0.2pt}
}
\newcommand{\hideLine}{
    \setlength{\columnsep}{0.4cm}
    \setlength{\columnseprule}{0pt}
}


% -- Geometry commads --
\renewcommand{\emptyset}{\varnothing}
\newcommand{\theTriangle}[1]{\ensuremath{\triangle #1}}
\newcommand{\homothety}[4]{\ensuremath{H\hspace{-1mm}\left(#1, #2\right) : #3 \to #4}}
\newcommand{\ratioCM}[6]{\ensuremath{\frac{#1 #4}{#4 #2} \cdot \frac{#2 #5}{#5 #3} \cdot \frac{#3 #6}{#6 #1}}}
\newcommand{\problemImage}[2]{
    \begin{center}
        \includegraphics[width=#2]{#1}
    \end{center}}

\newenvironment{solution}[1][]{
    \ifstrempty{#1}{
        \begin{proof}[\textnormal{\textbf{Solución}}]}
    {
        \begin{proof}[\textnormal{\textbf{Solución #1}}]}
    }{
    \end{proof}}
\theoremstyle{definition}

%With section
\newtheorem{section-lemma}{Lema}[section]
\newtheorem{section-example}{Ejemplo}[section]
\newtheorem{section-theorem}{Teorema}[section]
\newtheorem{section-problem}{Problema}[section]
\newtheorem{section-exercise}{Ejercicio}[section]
\newtheorem{section-corollary}{Corolario}[section]
\newtheorem{section-definition}{Definición}[section]

%Without section
\newtheorem{case}{Caso}
\newtheorem{example}{Ejemplo}
\newtheorem{problem}{Problema}
\newtheorem{remark}{Observación}
\newtheorem{corollary}{Corolario}

%Without numeration
\newtheorem*{note}{Nota}
\newtheorem*{definition}{Definición}


%Text Color Box styles
\newtcbtheorem[number within=section]{tcb-theorem-style}{Teorema}
{
    enhanced,
    frame empty,
    interior empty,
    coltitle = black,
    colbacktitle = gray!15!white,
    fonttitle = \bfseries,
    extras broken = {frame empty, interior empty},
    borderline = {0.3mm}{0mm}{black},
    breakable = true,
    top = 4mm,
    before skip = 3.5mm,
    attach boxed title to top left = {yshift = -3mm, xshift = 3mm},
    boxed title style = {boxrule = 0mm, borderline = {0.1mm}{0mm}{black}},
    varwidth boxed title,
    separator sign none, description delimiters parenthesis,
    description font=\bfseries,
    terminator sign={.\hspace{1mm}}
}
{theorem.tcb-label}

\newtcbtheorem[number within=section]{tcb-problem-style}{Problema}
{
    enhanced,
    frame empty,
    interior empty,
    coltitle = black,
    colbacktitle = white,
    fonttitle = \bfseries,
    extras broken = {frame empty, interior empty},
    borderline = {0.4mm}{0mm}{black},
    breakable = true,
    top = 4mm,
    before skip = 3.5mm,
    attach boxed title to top left = {yshift = -3mm, xshift = 3mm},
    boxed title style = {boxrule = 0.3mm, borderline = {0.3mm}{0mm}{black}},
    varwidth boxed title,
    terminator sign={.\hspace{1mm}}
}
{problem.tcb-label}

\newtcbtheorem[number within=section]{tcb-example-style}{Ejemplo}
{
    enhanced,
    frame empty,
    interior empty,
    coltitle = black,
    colback = white,
    colbacktitle = gray!15!white,
    fonttitle = \bfseries,
    borderline = {0.2mm}{0mm}{black},
    breakable = true,
    before skip = 3.5mm,
    after skip = 0mm,
    top = 1mm,
    bottom = 1mm,
    left = 1mm,
    right = 1mm,
    separator sign none, description delimiters parenthesis,
    description font=\bfseries,
    terminator sign={.\hspace{1mm}}
}
{example.tcb-label}

\newtcbtheorem[number within=section]{tcb-definition-style}{Definición}
{
    enhanced,
    frame empty,
    interior empty,
    coltitle = black,
    colbacktitle = gray!15!white,
    fonttitle = \bfseries,
    extras broken = {frame empty, interior empty},
    borderline = {0.3mm}{0mm}{black},
    breakable = true,
    top = 4mm,
    before skip = 3.5mm,
    attach boxed title to top left = {yshift = -3mm, xshift = 3mm},
    boxed title style = {boxrule = 0mm, borderline = {0.1mm}{0mm}{black}},
    varwidth boxed title,
    separator sign none, description delimiters parenthesis,
    description font=\bfseries,
    terminator sign={.\hspace{1mm}}
}
{definition.tcb-label}

\newtcbtheorem[number within=section]{tcb-lemma-style}{Lema}
{
    enhanced,
    frame empty,
    interior empty,
    coltitle = black,
    colbacktitle = white,
    fonttitle = \bfseries,
    extras broken = {frame empty, interior empty},
    borderline = {0.4mm}{0mm}{black},
    breakable = true,
    top = 4mm,
    before skip = 3.5mm,
    attach boxed title to top left = {yshift = -3mm, xshift = 3mm},
    boxed title style = {boxrule = 0.3mm, borderline = {0.3mm}{0mm}{black}},
    varwidth boxed title,
    separator sign none, description delimiters parenthesis,
    description font=\bfseries,
    terminator sign={.\hspace{1mm}}
}
{lemma.tcb-label}

\newtcolorbox[auto counter]{remark.tcb}[1][]
{
    breakable,
    title = Observación~\thetcbcounter.,
    colback = white,
    %colbacktitle = green!20!white,
    colbacktitle = gray!15!white,
    coltitle = black,
    fonttitle = \bfseries,
    bottomrule = 0pt,
    toprule = 0pt,
    leftrule = 2.5pt,
    rightrule = 0pt,
    titlerule = 0pt,
    arc = 2pt,
    outer arc = 2pt,
    colframe = black
}


%Enviroments section-x.tcb
\newenvironment{section-theorem.tcb}[1][]
{
    \ifstrempty{#1}
    {
        \begin{tcb-theorem-style}{}{}
    }
    {
        \begin{tcb-theorem-style}{#1}{}
    }
    }{
    \end{tcb-theorem-style}
}

\newenvironment{section-problem.tcb}
{
    \begin{tcb-problem-style}{}{}
    }{
    \end{tcb-problem-style}
}

\newenvironment{section-example.tcb}[1][]
{
    \ifstrempty{#1}
    {
        \begin{tcb-example-style}{}{}
    }
    {
        \begin{tcb-example-style}{#1}{}
    }
    }{
    \end{tcb-example-style}
}

\newenvironment{section-definition.tcb}[1][]
{
    \ifstrempty{#1}
    {
        \begin{tcb-definition-style}{}{}
    }
    {
        \begin{tcb-definition-style}{#1}{}
    }
    }{
    \end{tcb-definition-style}
}

\newenvironment{section-lemma.tcb}[1][]
{
    \ifstrempty{#1}
    {
        \begin{tcb-lemma-style}{}{}
    }
    {
        \begin{tcb-lemma-style}{#1}{}
    }
    }{
    \end{tcb-lemma-style}
}

\title{Introducción a las ecuaciones funcionales\\Propiedades de funciones}
\author{Kenny J. Tinoco}
\date{Septiembre de 2024}

\begin{document}
   \maketitle
   Con la técnica de sustitución vista en las sesiones pasadas se pueden resolver los siguientes problemas, con ciertas
   evaluaciones y manipuleos es posible hallar sus soluciones fácilmente, además estos han aparecido en olimpiadas internacionales.
   Por favor, intentá resolver estos problemas por tu cuenta.

   \begin{prob-without-section}[India, 2010]
      Encontrar todas las funciones $f: \R \to \R$ que satisfacen
      \[
         f(x + y) + xy = f(x)f(y),\ \text{para todo}\ x,y \in \R.
      \]
   \end{prob-without-section}

   \begin{prob-without-section}[IMO, 2002]
      Hallar todas las funciones $f: \R \to \R$ tales que, para cualesquiera $x,y,u,v$ reales, se cumple
      \[
         \left[f(x) + f(y)\right]\left[f(u) + f(v)\right] = f(xu - yv).
      \]
   \end{prob-without-section}

   \begin{prob-without-section}[Korea, 2000]
      Hallar todas las funciones $f : \R \to \R$ que satisfacen
      \[
         f(x^2 - y^2) = (x - y)\left(f(x) + f(y)\right),\ \text{para todo}\ x,y \in \R.
      \]
   \end{prob-without-section}

   \section{Propieades de funciones}

   Veamos algunas propiedades importantes sobre las funciones.
   \begin{definition.box}{Inyectiva}{}
      Una función $f : A \to B$ es \textbf{inyectiva} si para los números $x_1, x_2 \in A$ con $f(x_1) = f(x_2)$, entonces se tiene $x_1 = x_2$.
      Es decir, valores de entrada distintos tienen salidas distintas.
   \end{definition.box}

   \begin{definition.box}{Sobreyectiva}{}
      Una función $f : A \to B$ es \textbf{sobreyectiva} si para todo $y \in B$ existe un $x \in A$ tal que $f(x) = y$.
   \end{definition.box}

   \begin{definition.box}{Biyectiva}{}
      Una función $f : A \to B$ es \textbf{biyectiva} si es inyectiva y sobreyectiva a la vez.
   \end{definition.box}

   \begin{definition.box}{Creciente}{}
      Una función $f : A \to B$ es \textbf{creciente} si para todo $(x_1 \leq x_2) \in A$ se tiene que $f(x_1) \leq f(x_2)$.
      En el caso de que la igualdad no se cumpla se dice que es estrictamente creciente.
   \end{definition.box}

   \begin{definition.box}{Decreciente}{}
      Una función $f : A \to B$ es \textbf{decreciente} si para todo $(x_1 \leq x_2) \in A$ se tiene que $f(x_2) \leq f(x_1)$.
      En el caso de que la igualdad no se cumpla se dice que es estrictamente decreciente.
   \end{definition.box}

   \begin{definition.box}{Monótona}{}
      Una función $f : A \to B$ es \textbf{monótona} si es creciente o decreciente.
      Además, se dice que es \textbf{estrictamente monótona} si es estrictamente creciente o estrictamente decreciente.
   \end{definition.box}

   \begin{exercise}
      Hallar todas las funciones $f : \Z \to \Z$ con $f(0) = 1$ que satisfacen
      \[
         f(f(n)) = f(f(n + 2) + 2) = n,
      \]
      para todo entero $n$.
   \end{exercise}

   \begin{exercise}
      Encontrar todas las funciones $f : \Z \to \Z$ tales que
      \[
         f(x + f(y)) = f(x) + y,\ \text{para todo}\ x,y \in \Z
      \]
   \end{exercise}

   \begin{exercise}
      Hallar todas las funciones $f,g : \R \to \R$ tales que $g$ es inyectiva y
      \[
         f(g(x) + y) = g(x + f(y)),\ \text{para todo}\ x,y \in \R.
      \]
   \end{exercise}

   \begin{exercise}
      Hallar todas las funciones $f$ de reales a reales tales que
      \[
         f(f(x) + y) = 2x + f(f(y) - x)
      \]
      para todo $x,y \in \R$.
   \end{exercise}

   \begin{exercise}
      Hallar todas las funciones $f : \R \to \R$ tal que para cualesquieras $x,y \in \R$,
      \[
         (y + 1)f(x)  f(x f(y) + f(x + y)) = y.
      \]
   \end{exercise}

   \begin{exercise}
      Sea $f: (0, \infty) \to \R$ una función tal que
      \begin{enumerate}
         \item[i)] $f$ es estrictamente decreciente,
         \item[ii)] $f(x) > - \frac{1}{x}$ para todo $x > 0$ y
         \item[iii)] $f(x)f\left(f(x) + \frac{1}{x}\right) = 1$ para todo $x > 0$.
      \end{enumerate}
      Hallar $f(1)$.
   \end{exercise}

   \begin{exercise}
      Hallar todas las funciones $f : \R \to \R$ tal que
      \[
         f\left(f(x)^2 + f(y)\right) = xf(x) + y
      \]
      para todo $x,y \in \R$.
   \end{exercise}


   \section{Problemas}

   \begin{prob-without-section}[Lista corta IMO, 1988]
      Sea $f: \N \to \N$ una función que cumple
      \[
         f(f(m) + f(n)) = m + n,\ \text{para todos}\ m,n.
      \]
      Hallar los posibles valores de $f(1988)$.
   \end{prob-without-section}

   \begin{prob-without-section}[Lista corta IMO, 2002]
      Encontrar todas las funciones $f : \R \to \R$ tales que
      \[
         f(f(x) + y) = 2x + f(f(y) - x),\ \text{para todos}\ x,y \in \R.
      \]
   \end{prob-without-section}

   \begin{prob-without-section}[Ibero, 1993]
      Encontrar todas las funciones estrictamente crecientes $f : \N \to \N$ que satisfacen
      \[
         f(n f(m)) = m^2 f(mn),\ \text{para todos}\ m,n \in \R.
      \]
   \end{prob-without-section}

   \begin{prob-without-section}[Italia, 1999]
      Encontrar todas las funciones estrictamente monótonas $f : \R \to \R$ tal que
      \[
         f(x + f(y)) = f(x) + y,\ \text{para}\ x,y \in \R.
      \]
   \end{prob-without-section}
\end{document}